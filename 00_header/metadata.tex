%=============================================================================%
%                            Doctoral Dissertation                            %
%                            (c) by Manuel Schlund                            %
%=============================================================================%

%=============================================================================%
% This work is licensed under a
% Creative Commons Attribution 4.0 International License.
%
% You should have received a copy of the license along with this
% work. If not, see <http://creativecommons.org/licenses/by/4.0/>.
%=============================================================================%



% Titles
\newcommand*{\Prof}{Prof.\@}
\newcommand*{\Dr}{Dr.\@}
\newcommand*{\PhD}{Ph.D.\@}

% Dissertation information
\newcommand*{\TheAuthor}{Manuel Schlund}
\newcommand*{\TheColloquiumDate}{2 August 2021}
\newcommand*{\TheDegree}{\foreignlanguage{ngerman}{%
    Doktor der Naturwissenschaften (\Dr{} rer.\@ nat.\@)}%
}
\newcommand*{\TheInstitute}{Institute of Environmental Physics (IUP)}
\newcommand*{\TheKeywords}{%
  Climate Modeling\sep Climate Model Weighting\sep Climate Projections\sep %
  Climate Sensitivity\sep Emergent Constraints\sep Machine Learning%
}
\newcommand*{\TheLanguage}{en-US}
\newcommand*{\TheLicense}{%
  Creative Commons Attribution 4.0 International License%
}
\newcommand*{\TheLicenseURL}{https://creativecommons.org/licenses/by/4.0/}
\newcommand*{\TheMonth}{August 2021}
\newcommand*{\ThePrimaryExaminer}{\Prof{} \Dr{} Veronika Eyring}
\newcommand*{\TheSecondaryExaminer}{\Prof{} \Dr{} Pierre Gentine}
\newcommand*{\TheSubmissionDate}{13 April 2021}
\newcommand*{\TheThesis}{Doctoral Dissertation}
\newcommand*{\TheTitle}{%
  Constraining Uncertainties in Multi-Model Projections of the Future Climate
  with Observations%
}
\newcommand*{\TheUniversity}{University of Bremen}

% Plain text of the abstract
% Note: needs to be changed if abstract.tex changes!
\newcommand*{\TheAbstract}{%
  A precise quantification of climate change is crucial to assess optimal
  mitigation and adaptation strategies. Earth system models (ESMs), which are
  state-of-the-art climate models that allow numerical simulations of the
  complex physical, biological, and chemical processes of the Earth system, are
  common tools to understand and project climate change. Due to the chaotic
  nature of the climate system, unknowns in future emission pathways, and
  uncertainties in the climate models, projections of the future climate are
  associated with large uncertainties. The main focus of this thesis is the
  analysis of future climate projections from ESMs participating in the Coupled
  Model Intercomparison Project (CMIP) with the aim to reduce uncertainties in
  climate projections with observations. In a first step, climate sensitivity
  (i.e., the temperature response of the climate system to an external forcing)
  is evaluated in the latest generation of ESMs from CMIP6. For the effective
  climate sensitivity (ECS), which is an estimation of the equilibrium
  temperature response that follows a doubling of the atmospheric carbon
  dioxide (CO2) concentration, a multimodel mean (MMM) of 3.74 K and a
  multi-model range of 1.8–5.6 K are found. These values are higher than in any
  previous CMIP ensemble before. Moreover, a third of the analyzed CMIP6 models
  exceed the upper bound of the likely ECS range of 1.5–4.5 K assessed by the
  Intergovernmental Panel on Climate Change (IPCC) Fifth Assessment Report
  (AR5) from 2013. Similarly, the transient response of the climate system to a
  doubling of CO2, also known as transient climate response (TCR), shows an
  inter-model range of 1.3–3.0 K with an upper bound again higher than the
  likely range assessed in AR5 of 1.0–2.5 K. Possible reasons for the increased
  climate sensitivity in many CMIP6 models are the addition of new prognostic
  aerosol schemes that include aerosol-cloud interactions and changes in the
  microphysical representation of mixed-phase clouds. The changes in the
  mixed-cloud representation reduce the strong negative shortwave cloud phase
  change feedback over the Southern Ocean that is present in climate models
  from previous CMIP generations. To reduce uncertainties in ECS projected by
  the CMIP6 models, 11 published emergent constraints on ECS (mostly derived
  from models participating in CMIP5, the predecessor generation of CMIP6) are
  systematically analyzed. Emergent constraints are potentially promising
  approaches to reduce uncertainties in climate model projections by combining
  observations and output from ESMs. The focus of this analysis is on testing
  if these emergent constraints hold for ESMs participating in CMIP6. Since
  none of the emergent constraints considered here have been derived from the
  CMIP6 ensemble, the CMIP6 models can be used for crosschecking the emergent
  constraints on a new model ensemble. The application of the emergent
  constraints to CMIP6 data shows a decrease in skill and statistical
  significance of the emergent relationships for nearly all constraints, with
  this decrease being large in many cases. Consequently, the sizes of the
  constrained ECS ranges (66 \% confidence intervals) widen by 51 \% on average
  in CMIP6 compared to CMIP5. This is likely related to the increased
  multi-model spread of ECS in CMIP6, but may in some cases also be due to
  spurious statistical relationships or a too-small number of models in the
  ensemble that the emergent constraint was originally derived from. The
  corresponding best estimates of ECS given by the emergent constraints also
  increase from CMIP5 to CMIP6 by 12 \% on average. This can at least be partly
  explained by the increased number of high-ECS models in CMIP6 without a
  corresponding change in the constraint predictors, suggesting the emergence
  of new feedback processes rather than changes in the strength of those
  previously dominant. The results support previous studies concluding that
  emergent constraints should be based on an independently verifiable physical
  mechanism and that process-based emergent constraints on ECS should rather be
  thought of as constraints for the process or feedback they are actually
  targeting. To overcome these issues of single-process-oriented emergent
  constraints, an alternative approach based on machine learning (ML) is
  introduced. Since this new technique relies on a large number of data points
  to train the ML algorithm, the scalar climate sensitivity expressed as ECS or
  TCR is not an appropriate target variable. Therefore, gross primary
  production (GPP) as a process that contributes to climate sensitivity is
  studied as an alternative. GPP is the largest flux of the terrestrial carbon
  uptake and slows down global warming by removing CO2 from the atmosphere. In
  this analysis, an existing emergent constraint on CO2 fertilization is
  combined with an ML approach to constrain the spatial variations of
  multi-model GPP projections. In the first step of the two-step approach,
  observed changes in the CO2 seasonal cycle at Cape Kumukahi, Hawaii are used
  to constrain the global mean GPP at the end of the 21st century (2091–2100)
  in Representative Concentration Pathway (RCP) 8.5 simulations with ESMs
  participating in CMIP5 to (171 ± 12) GtC yr−1 , compared to the unconstrained
  model range of 156–247 GtC yr−1 . In a second step, an ML model is used to
  constrain gridded future absolute GPP and gridded fractional GPP change in
  two independent approaches. For this, observational data is fed into the ML
  algorithm that has been trained on CMIP5 data to learn relationships between
  present-day physically relevant diagnostics and the target variable. In a
  leave-one-model-out cross-validation approach, the ML model shows superior
  performance to the CMIP5 MMM. The new approach predicts a higher GPP increase
  in high latitudes and a lower GPP increase in regions closer to the equator.%
}
