%=============================================================================%
%                                Dissertation                                 %
%                               Manuel Schlund                                %
%                                  (c) 2020                                   %
%=============================================================================%
%                                   Makros                                    %
%=============================================================================%



% Dissertation information
\newcommand{\TheUniversity}{University of Bremen}
\newcommand{\TheInstitute}{Institute of Environmental Physics (IUP)}
\newcommand{\TheTitle}{%
  Constraining Uncertainties in Multi-Model Projections of the Future Climate
  with Observations}
\newcommand{\TheThesis}{Doctoral Dissertation}
\newcommand{\TheDegree}{\foreignlanguage{ngerman}{%
  Doktor der Naturwissenschaften (Dr.\@ rer.\@ nat.\@)}}
\newcommand{\TheAuthor}{Manuel \textsc{Schlund}}
\newcommand{\TheFirstSupervisor}{Prof.\@ Dr.\@ Veronika \textsc{Eyring}}
\newcommand{\TheSecondSupervisor}{Prof.\@ Dr.\@ Pierre \textsc{Gentine}}
\newcommand{\ThePlace}{Oberpfaffenhofen}
\newcommand{\TheMonth}{March 2021}
\newcommand{\TheSubmissionDate}{March 2021}
\newcommand{\TheLicense}{%
  Creative Commons Attribution 4.0 International License}
\newcommand{\TheLicenseLink}{https://creativecommons.org/licenses/by/4.0/}

% Key science questions
\newcommand{\KeyScienceQuestionOne}{%
  What is the range of climate sensitivity in the latest generation of
  \acp{ESM} from \acs{CMIP}6 compared to previous multi-model ensembles, and
  do we understand the processes that determine this uncertainty range?}
\newcommand{\KeyScienceQuestionTwo}{%
  Can uncertainties in climate sensitivity be reduced with observations using
  the emergent constraint approach?}
\newcommand{\KeyScienceQuestionThree}{%
  Can uncertainties in multi-dimensional (gridded) climate projections be
  reduced with \ac{ML} techniques and observations?}

% Colors
\colorlet{BrownRed}{red!50!black}

% Figure scales
\newcommand{\FigureWidth}{0.70\columnwidth}
\newcommand{\LargerFigureWidth}{0.74\columnwidth}
\newcommand{\LargeFigureWidth}{0.97\columnwidth}
\newcommand{\SmallSubfigureWidth}{0.47\columnwidth}
\newcommand{\SubfigureWidth}{0.48\columnwidth}

% Abbreviations
\newcommand{\eg}{\mbox{e.g.,}}
\newcommand{\etc}{\mbox{etc.\@}}
\newcommand{\ie}{\mbox{i.e.,}}
\newcommand{\stddev}{\mbox{std.\@ dev.\@}}
\newcommand{\onepctcotwo}{1\%\acs{CO2}}
\newcommand{\nxcotwo}[1]{#1x\acs{CO2}}
\newcommand{\xaxis}{$x$-axis}
\newcommand{\yaxis}{$y$-axis}
\newcommand{\xintercept}{$x$-intercept}
\newcommand{\yintercept}{$y$-intercept}

% Copyright statements
\newcommand{\AdaptedFrom}[1]{Adapted with permission from \textcite{#1}}
\newcommand{\IPCCReproducedFrom}[2]{%
  Reproduced with permission from \textcite{#1} (#2)}
\newcommand{\SpringerAdaptedFrom}[1]{%
  Adapted by permission from Springer Nature Customer Service Centre GmbH:
  \textcite{#1}}

% Other commands
\newcommand{\range}[2]{#1\mathtxt{--}#2}
\newcommand{\commentcite}[2]{(\cite{#1}; #2)}
\newcommand{\tabitem}{~~\llap{\textbullet}~~}

% Math commands
\newcommand{\GtCyr}{\ensuremath{GtC \, yr^{-1}}}
\newcommand{\mmday}{\ensuremath{mm \, day^{-1}}}
\newcommand{\pctK}{\ensuremath{\% \, K^{-1}}}
\newcommand{\dd}{\ensuremath{\, d}}
\newcommand{\unit}[1]{\ensuremath{\, \si{#1}}}
\newcommand{\pmrangeunit}[3]{\ensuremath{\left( #1 \pm #2 \right) \unit{#3}}}
\newcommand{\rangeunit}[3]{\range{#1}{#2} \unit{#3}}
\newcommand{\cond}[2]{\ensuremath{P(#1|#2)}}
\newcommand{\mathtxt}[1]{\textnormal{#1}}

% Tables
\newcommand{\ignorecolumntype}[2]{\multicolumn{1}{#1}{#2}}
\newcommand{\echead}[2]{\makecell[tc]{#1 \\ {[$\si{#2}$]}}}
\newcommand{\ecreshead}[2]{\ignorecolumntype{c}{\makecell[tc]{#1 \\ #2}}}
\newcommand{\predictor}[1]{\,\!\Acf{#1}}
\newcolumntype{y}[1]{>{\raggedleft \arraybackslash $}p{#1\columnwidth}<{$}}

% CSV table styles
\csvstyle{ECSAndTCRTable}{
  head to column names,
  tabular=l l r r,
  table head=\toprule Model & Index used in plots & \acs{ECS} {[$\si{K}$]} &
  \acs{TCR} {[$\si{K}$]} \\ \midrule,
  table foot=\bottomrule,
  before line=\hspace{-0.5\tabcolsep}, % Avoid indentation for first two rows
  late after line=\ifthenelse{\equal{\csvcoli}{Multi-model mean}}{\\
    \midrule}{\\},
}
\csvstyle{EmergentConstraintsPart1Table}{
  head to column names,
  tabular=l l r r r r r r r,
  table head=\toprule Model & Index & \echead{\acs{ECS}}{K} &
  \echead{BRI}{\pctK{}} & \echead{COX}{K} & \echead{LIP}{\degree} &
  \echead{SHD}{1} & \echead{SHL}{1} & \echead{SHS}{1} \\ \midrule,
  table foot=\bottomrule,
  before line=\hspace{-0.5\tabcolsep}, % Avoid indentation for first two rows
}
\csvstyle{EmergentConstraintsPart2Table}{
  head to column names,
  tabular=l l r r r r r r,
  table head=\toprule Model & Index &\echead{\acs{ECS}}{K} & \echead{SU}{1} &
  \echead{TIH}{\%} & \echead{TII}{\mmday{}} & \echead{VOL}{\%} &
  \echead{ZHA}{\pctK{}} \\ \midrule,
  table foot=\bottomrule,
  before line=\hspace{-0.5\tabcolsep}, % Avoid indentation for first two rows
}
