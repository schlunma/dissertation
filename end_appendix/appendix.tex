%=============================================================================%
%                                Dissertation                                 %
%                               Manuel Schlund                                %
%                                  (c) 2020                                   %
%=============================================================================%
%                                  Appendix                                   %
%=============================================================================%



\chapter{Appendix}

\renewcommand{\thesection}{\Alph{section}}

%\begin{figure}[t]
%  \centering
%  \begin{subfigure}[t]{\SubfigureWidth{}}
%    \includegraphics[width=\linewidth]{ch2_entropy_based_tuning/fig/two_peaks.pdf}
%    \caption{Discrete spectrum $p(f_i)$.}
%    \label{fig:2:two_peaks_shannon_entropy:a}
%  \end{subfigure}
%  ~
%  \begin{subfigure}[t]{\SubfigureWidth{}}
%    \includegraphics[width=\linewidth]{ch2_entropy_based_tuning/fig/two_peaks_shannon_entropy.pdf}
%    \caption{Shannon entropy of the spectrum dependent on the peak distance $\Delta f$.}
%    \label{fig:2:two_peaks_shannon_entropy:b}
%  \end{subfigure}
%  \caption{Shannon entropy $H$ of a discrete spectrum $p(f_i)$ composed of two peaks with distance $\Delta f$. When both peaks coincide ($\Delta f = 0\unit{Hz}$), the entropy reaches its global minimum.}
%  \label{fig:2:two_peaks_shannon_entropy}
%\end{figure}


\section{TBA}

\subsection{test}

\subsubsection{test}

\paragraph{hi}

The \ac{ECS} is really cool. I like it very much!

This is e.g. without an "at" and this is it with an "at" e.g.\@ difference?
Test space. Real dot!

\Eg blaa. \Eg{} blaaaa. \ie blaaaa, \ie{} blaa.

These are really cool papers: \autocite{Schlund2020, Schlund2020a}

autocite:
\autocite{Lauer2018}

cite:
\cite{Lauer2010}
\autocite{Anav2015}
\autocite{Anav2013}
\autocite{Allen2002}

textcite:
\textcite{Lauer2010}

And this one, too: \autocite{Lauer2020}

This is a reference to the equation: \cref{eq:2:cent_def}

Three authors: \autocite{Bao2020}

Many many authors: \autocite{Eyring2020}

\texttt{input <iostream>}

\begin{equation}
  c_{k_1,k_2} := 1200 \, \log_2 \left( \frac{f_1^{(k_2)}}{f_1^{(k_1)}} \right) \unit{cents}. \label{eq:2:cent_def}
\end{equation}

\begin{table}
  \caption{The effects of treatments X and Y on the four groups studied.}
  \label{tab:treatments}
  \centering
  \begin{tabular}{l l l}
    \toprule
    \tabhead{Groups} & \tabhead{Treatment X} & \tabhead{Treatment Y} \\
    \midrule
    1 & 0.2 & 0.8\\
    2 & 0.17 & 0.7\\
    3 & 0.24 & 0.75\\
    4 & 0.68 & 0.3\\
    \bottomrule\\
  \end{tabular}
\end{table}

\begin{table}[t]
  \centering
  \begin{tabular}{c l r r}
    \toprule
    Semitones & Interval & $c/\unit{cents}$ (ET) & $c/\unit{cents}$ (JI) \\
    \midrule
    $0$ & Perfect unison & $0$ & $0$ \\
    $1$ & Minor second & $100$ & $112$ \\
    $2$ & Major second & $200$ & $204$ \\
    $3$ & Minor third & $300$ & $316$ \\
    $4$ & Major third & $400$ & $386$ \\
    $5$ & Perfect fourth & $500$ & $498$ \\
    $6$ & Augmented fourth & $600$ & $590$ \\
    $7$ & Perfect fifth & $700$ & $702$ \\
    $8$ & Minor sixth & $800$ & $814$ \\
    $9$ & Major sixth & $900$ & $884$ \\
    $10$ & Minor seventh & $1000$ & $996$ \\
    $11$ & Major seventh & $1100$ & $1088$ \\
    $12$ & Perfect octave & $1200$ & $1200$ \\
    \bottomrule
  \end{tabular}
  \caption{Logarithmic frequency ratios $c$ of certain intervals in the equal temperament (ET) and the just intonation (JI). $x$ cents correspond to a frequency ratio of $2^{\, x/1200}$.}
  \label{tab:2:cents}
\end{table}


\section{TBA}

TBA.