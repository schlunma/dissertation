%=============================================================================%
%                                Dissertation                                 %
%                               Manuel Schlund                                %
%                                  (c) 2020                                   %
%=============================================================================%
%                                  Appendix                                   %
%=============================================================================%



\chapter{Appendix}

\begingroup


\crefalias{section}{appendix}
\numberwithin{equation}{section}
\numberwithin{figure}{section}
\numberwithin{table}{section}
\renewcommand{\thesection}{\Alph{section}}


\section{Supplementary Materials for
  \texorpdfstring{\Cref{ch:05:paper_ecs}}{Chapter \ref{ch:05:paper_ecs}}}
\label{sec:app:si_for_paper_ecs}

\vspace{\fill}

\begin{table}[!h]
  \centering
  \begin{tabulary}{\columnwidth}{>{\em}L L}
    \toprule
    \ignorecolumntype{l}{Variable short name} & Description \\
    \midrule
    cl & Cloud area fraction \\
    clt & Total cloud area fraction \\
    hur & Relative humidity \\
    hus & Specific humidity \\
    pr & Precipitation \\
    rsdt & \Acf{TOA} incoming shortwave radiation \\
    rsut & \acs{TOA} outgoing shortwave radiation \\
    rsutcs & Clear-sky \acs{TOA} outgoing shortwave radiation \\
    ta & Air temperature \\
    tas & Surface air temperature \\
    tos & Sea surface temperature \\
    ts & Surface temperature \\
    va & Northward wind speed \\
    wap & Vertical velocity \\
    \bottomrule
  \end{tabulary}
  \caption{Overview of the variables used in \cref{ch:05:paper_ecs}
    (\nameref*{ch:05:paper_ecs}). More details are given in
    \cref{tab:05:overview_emergent_constraints}, which lists the variables
    used for each emergent constraint.}
  \label{tab:app:a:overview_variables}
\end{table}

\vspace{\fill}

\begin{table}[!b]
  \centering
  \begin{tabulary}{\columnwidth}{L >{\em}L L}
    \toprule
    Observational dataset & \ignorecolumntype{l}{Corresponding variables} &
    Reference \\
    \midrule
    AIRS & hur, hus & \textcite{Aumann2003} \\
    AMSRE \ac{SST} & tos & \textcite{AMSRE2011} \\
    CERES-EBAF & rsdt, rsut, rsutcs & \textcite{Loeb2018} \\
    Cloudsat/CALIPSO & cl & \textcite{Mace2009} \\
    ERA-Interim & hur, ta, va, wap & \textcite{Dee2011} \\
    GPCP & pr & \textcite{Adler2003} \\
    HadCRUT4 & tas & \textcite{Morice2012} \\
    HadISST & ts & \textcite{Rayner2003} \\
    ISCCP D-2 & clt & \textcite{Rossow1991} \\
    MLS-Aura & hur & \textcite{Beer2006} \\
    \bottomrule
  \end{tabulary}
  \caption{References for all observational datasets used in
    \cref{ch:05:paper_ecs} (\nameref*{ch:05:paper_ecs}). More details are
    given in \cref{tab:05:overview_emergent_constraints}, which lists the
    observational datasets used for each emergent constraint.}
  \label{tab:app:a:observations}
\end{table}

\begin{table}[p]
  \centering
  \begin{tabulary}{\columnwidth}{L >{$}L<{$} L}
    \toprule
    Model & \ignorecolumntype{l}{Index} & Main reference \\
    \midrule
    ACCESS1-0 & 1 & \textcite{Dix2013} \\
    ACCESS1-3 & 2 & \textcite{Dix2013} \\
    BNU-ESM & 3 & \textcite{Ji2014} \\
    CCSM4 & 4 & \textcite{Gent2011, Meehl2012} \\
    CNRM-CM5 & 5 & \textcite{Voldoire2013} \\
    CNRM-CM5-2 & 6 & \textcite{Voldoire2013} \\
    CSIRO-Mk3-6-0 & 7 & \textcite{Rotstayn2012} \\
    CanESM2 & 8 & \textcite{Arora2011} \\
    FGOALS-g2 & 9 & \textcite{Li2013} \\
    GFDL-CM3 & 10 & \textcite{Donner2011} \\
    GFDL-ESM2G & 11 & \textcite{Dunne2012} \\
    GFDL-ESM2M & 12 & \textcite{Dunne2012} \\
    GISS-E2-H & 13 & \textcite{Schmidt2006} \\
    GISS-E2-R & 14 & \textcite{Schmidt2006} \\
    HadGEM2-ES & 15 & \textcite{Collins2011} \\
    IPSL-CM5A-LR & 16 & \textcite{Dufresne2013} \\
    IPSL-CM5A-MR & 17 & \textcite{Dufresne2013} \\
    IPSL-CM5B-LR & 18 & \textcite{Dufresne2013} \\
    MIROC-ESM & 19 & \textcite{Watanabe2011} \\
    MIROC5 & 20 & \textcite{Watanabe2010} \\
    MPI-ESM-LR & 21 & \textcite{Giorgetta2013} \\
    MPI-ESM-MR & 22 & \textcite{Giorgetta2013} \\
    MPI-ESM-P & 23 & \textcite{Giorgetta2013} \\
    MRI-CGCM3 & 24 & \textcite{Yukimoto2012} \\
    NorESM1-M & 25 & \textcite{Bentsen2013, Iversen2013} \\
    bcc-csm1-1 & 26 & \textcite{Wu2014} \\
    bcc-csm1-1-m & 27 & \textcite{Wu2014} \\
    inmcm4 & 28 & \textcite{Volodin2010} \\
    \bottomrule
  \end{tabulary}
  \caption{List of \acs{CMIP}5 models used in \cref{ch:05:paper_ecs}
    (\nameref*{ch:05:paper_ecs}) alongside the main reference and the index
    used in the corresponding figures. \AdaptedFrom{Schlund2020a}.}
  \label{tab:app:a:cmip5_models}
\end{table}

\begin{table}[p]
  \centering
  \begin{tabulary}{\columnwidth}{L >{$}L<{$} L}
    \toprule
    Model & \ignorecolumntype{l}{Index} & Main reference \\
    \midrule
    ACCESS-CM2 & 29 & \textcite{Bi2013} \\
    ACCESS-ESM1-5 & 30 & \textcite{Law2017, Ziehn2017} \\
    AWI-CM-1-1-MR & 31 & \textcite{Rackow2018, Sidorenko2015} \\
    BCC-CSM2-MR & 32 & \textcite{Wu2019} \\
    BCC-ESM1 & 33 & \textcite{Wu2019} \\
    CAMS-CSM1-0 & 34 & \textcite{Rong2018} \\
    CAS-ESM2-0 & 35 & \textcite{Wang2020} \\
    CESM2 & 36 & \textcite{Danabasoglu2020} \\
    CESM2-FV2 & 37 & \textcite{Danabasoglu2020} \\
    CESM2-WACCM & 38 & \textcite{Danabasoglu2020, Gettelman2019a} \\
    CESM2-WACCM-FV2 & 39 & \textcite{Danabasoglu2020, Gettelman2019a} \\
    CMCC-CM2-SR5 & 40 & \textcite{Cherchi2019} \\
    CNRM-CM6-1 & 41 & \textcite{Voldoire2019} \\
    CNRM-CM6-1-HR & 42 & \textcite{Voldoire2019} \\
    CNRM-ESM2-1 & 43 & \textcite{Seferian2019} \\
    CanESM5 & 44 & \textcite{Swart2019} \\
    E3SM-1-0 & 45 & \textcite{Golaz2019} \\
    EC-Earth3-Veg & 46 & \textcite{Wyser2020} \\
    FGOALS-f3-L & 47 & \textcite{Guo2020, He2019, He2020} \\
    FGOALS-g3 & 48 & \textcite{Li2020} \\
    GISS-E2-1-G & 49 & \textcite{Rind2020} \\
    GISS-E2-1-H & 50 & \textcite{Rind2020} \\
    HadGEM3-GC31-LL & 51 & \textcite{Kuhlbrodt2018} \\
    HadGEM3-GC31-MM & 52 & \textcite{Williams2018} \\
    INM-CM4-8 & 53 & \textcite{Volodin2017,Volodin2017a} \\
    INM-CM5-0 & 54 & \textcite{Volodin2017,Volodin2017a} \\
    IPSL-CM6A-LR & 55 & \textcite{Boucher2020} \\
    KACE-1-0-G & 56 & \textcite{Lee2020} \\
    MCM-UA-1-0 & 57 & \textcite{Delworth2002} \\
    MIROC-ES2L & 58 & \textcite{Hajima2020} \\
    MIROC6 & 59 & \textcite{Tatebe2019} \\
    MPI-ESM-1-2-HAM & 60 & \textcite{Mauritsen2019} \\
    MPI-ESM1-2-HR & 61 & \textcite{Muller2018} \\
    MPI-ESM1-2-LR & 62 & \textcite{Mauritsen2019} \\
    MRI-ESM2-0 & 63 & \textcite{Yukimoto2019} \\
    NESM3 & 64 & \textcite{Cao2018} \\
    NorCPM1 & 65 & \textcite{Counillon2016} \\
    NorESM2-LM & 66 & \textcite{Seland2020} \\
    NorESM2-MM & 67 & \textcite{Seland2020} \\
    SAM0-UNICON & 68 & \textcite{Park2019} \\
    TaiESM1 & 69 & \textcite{Lee2020a} \\
    UKESM1-0-LL & 70 & \textcite{Sellar2019} \\
    \bottomrule
  \end{tabulary}
  \caption{As in \cref{tab:app:a:cmip5_models} but for the \acs{CMIP}6 models.
    \AdaptedFrom{Schlund2020a}.}
  \label{tab:app:a:cmip6_models}
\end{table}

\begin{table}[p]
  \centering
  \csvreader[EmergentConstraintsPart1Table]{
    ch05_paper_ecs/data/cmip5_emergent_constraints.csv}{}{
    \dataset & $\idx$ & $\ECS$ & $\BRI$ & $\COX$ & $\LIP$ & $\SHD$ & $\SHL$ &
    $\SHS$}
  \caption{All \acs{CMIP}5 models used in \cref{ch:05:paper_ecs}
    (\nameref*{ch:05:paper_ecs}) including their \acf{ECS} and \xaxis{} values
    for the emergent constraints BRI, COX, LIP, SHD, SHL and SHS. More details
    on the emergent constraints are given in
    \cref{tab:05:overview_emergent_constraints}. The specified index
    corresponds to the index used in the associated figures.
    \AdaptedFrom{Schlund2020a}.}
  \label{tab:app:a:cmip5_emergent_constraints_part1}
\end{table}

\begin{table}[p]
  \centering
  \csvreader[EmergentConstraintsPart2Table]{
    ch05_paper_ecs/data/cmip5_emergent_constraints.csv}{}{
    \dataset & $\idx$ & $\ECS$ & $\SU$ & $\TIH$ & $\TII$ & $\VOL$ & $\ZHA$}
  \caption{As in \cref{tab:app:a:cmip5_emergent_constraints_part1} but for the
    emergent constraints SU, TIH, TII, VOL and ZHA.
    \AdaptedFrom{Schlund2020a}.}
  \label{tab:app:a:cmip5_emergent_constraints_part2}
\end{table}

\begin{table}[p]
  \centering
  \csvreader[EmergentConstraintsPart1Table]{
    ch05_paper_ecs/data/cmip6_emergent_constraints.csv}{}{
    \dataset & $\idx$ & $\ECS$ & $\BRI$ & $\COX$ & $\LIP$ & $\SHD$ & $\SHL$ &
    $\SHS$}
  \caption{As in \cref{tab:app:a:cmip5_emergent_constraints_part1} but for the
    \acs{CMIP}6 models. \AdaptedFrom{Schlund2020a}.}
  \label{tab:app:a:cmip6_emergent_constraints_part1}
\end{table}

\begin{table}[p]
  \centering
  \csvreader[EmergentConstraintsPart2Table]{
    ch05_paper_ecs/data/cmip6_emergent_constraints.csv}{}{
    \dataset & $\idx$ & $\ECS$ & $\SU$ & $\TIH$ & $\TII$ & $\VOL$ & $\ZHA$}
  \caption{As in \cref{tab:app:a:cmip5_emergent_constraints_part2} but for the
  \acs{CMIP}6 models. \AdaptedFrom{Schlund2020a}.}
  \label{tab:app:a:cmip6_emergent_constraints_part2}
\end{table}


\section{Supplementary Materials for
  \texorpdfstring{\Cref{ch:06:paper_gpp}}{Chapter \ref{ch:06:paper_gpp}}}
\label{sec:app:si_for_paper_gpp}

\begin{table}[p]
  \centering
  \begin{tabulary}{\columnwidth}{L L L}
    \toprule
    Climate model & Land model & Main reference \\
    \midrule
    CESM1-BGC & CLM4 & \textcite{Gent2011} \\
    CanESM2 & CLASS2.7 + CTEM1 & \textcite{Arora2011} \\
    GFDL-ESM2M & LM3 & \textcite{Dunne2012} \\
    HadGEM2-ES & JULES + TRIFFID & \textcite{Collins2011} \\
    MIROC-ESM & MATSIRO + SEIB-DGVM & \textcite{Watanabe2011} \\
    MPI-ESM-LR & JSBACH + BETHY & \textcite{Giorgetta2013} \\
    NorESM1-ME & CLM4 & \textcite{Iversen2013} \\
    \bottomrule
  \end{tabulary}
  \caption{List of the seven \acs{CMIP}5 models used in \cref{ch:06:paper_gpp}
    (\nameref*{ch:06:paper_gpp}) alongside the main reference. More details are
    given by \textcite{Anav2013}. We chose all \acs{CMIP}5 models which provide
    all necessary variables (\emph{co2}, \emph{gpp}, \emph{lai}, \emph{pr},
    \emph{rsds} and \emph{tas}) for all used experiments (\emph{esmHistorical},
    \emph{esmrcp85} and \emph{esmFixClim1}). For all models, we only use the
    first ensemble member available. \AdaptedFrom{Schlund2020}.}
  \label{tab:app:b:cmip5_models}
\end{table}

\begin{figure}[p]
  \centering
  \begin{subfigure}[b]{0.39\columnwidth}
    \includegraphics[width=\columnwidth]{ch06_paper_gpp/figs/s1a.pdf}
    \caption{}
    \label{fig:app:b:co2:a}
  \end{subfigure}
  \begin{subfigure}[b]{0.39\columnwidth}
    \includegraphics[width=\columnwidth]{ch06_paper_gpp/figs/s1b.pdf}
    \caption{}
    \label{fig:app:b:co2:b}
  \end{subfigure}
  \begin{subfigure}[b]{0.2\columnwidth}
    \raisebox{14.5mm}{\includegraphics[width=\columnwidth]{
      ch06_paper_gpp/figs/s1c.pdf}}
  \end{subfigure}
  \caption{(a) Monthly mean atmospheric \acs{CO2} concentration at \acf{KUM}
    from 1979 to 2019. The thin colored lines show the individual \acs{CMIP}5
    models (emission-driven historical simulations for the years
    \range{1979}{2005} and emission-driven \acs{RCP}8.5 simulations for the
    years \range{2006}{2019}; the latter is not available for HadGEM2-ES). The
    thick black line shows the observations. For the \acs{CMIP}5 models, the
    grid cell closest to \acs{KUM} is considered. The curves show an increase
    of the atmospheric \acs{CO2} concentration superimposed by a pronounced
    seasonal cycle (see \cref{subsec:02:carbon_cycle_perturbations}). (b)
    Annual amplitude of the seasonal cycle of \acs{CO2} (defined as the
    difference between the maximum and the minimum monthly mean atmospheric
    \acs{CO2} concentration for each year) against the annual mean atmospheric
    \acs{CO2} concentration at \acs{KUM}. Colored points show the \acs{CMIP}5
    models (similar time ranges as in (a)) and thick black points show the
    observations. The lines show the corresponding linear regression fits for
    each dataset. The slopes of these linear fits define the sensitivity of
    the seasonal \acs{CO2} cycle amplitude to atmospheric \acs{CO2}
    concentrations, which is used as predictor for the emergent constraint
    step of our approach (step 1). \AdaptedFrom{Schlund2020}.}
  \label{fig:app:b:co2}
\end{figure}

\begin{figure}[p]
  \centering
  \begin{subfigure}[b]{\SubfigureWidth{}}
    \includegraphics[width=\columnwidth]{ch06_paper_gpp/figs/s2a.pdf}
    \caption{}
    \label{fig:app:b:residuals:a}
  \end{subfigure}
  ~
  \begin{subfigure}[b]{\SubfigureWidth{}}
    \includegraphics[width=\columnwidth]{ch06_paper_gpp/figs/s2b.pdf}
    \caption{}
    \label{fig:app:b:residuals:b}
  \end{subfigure}
  \caption{Distribution of the residuals of the \acf{GBRT} model for the two
    different target variables used in step 2a (absolute \acf{GPP} at the end
    of the \nth{21} century) and step 2b (fractional \acs{GPP} change over the
    \nth{21} century). The distributions are derived by kernel density
    estimation using training (blue) and test (green) data. The plots show
    approximately unbiased distributions for the training and the test
    datasets, which are very similar to each other. This indicates that the
    \acl{ML} model does not overfit the data. \AdaptedFrom{Schlund2020}.}
  \label{fig:app:b:residuals}
\end{figure}

\begin{figure}[p]
  \centering
  \begin{subfigure}[b]{\SubfigureWidth{}}
    \includegraphics[width=\columnwidth]{ch06_paper_gpp/figs/s3a.pdf}
    \caption{}
    \label{fig:app:b:cmip5_hist_mte:a}
  \end{subfigure}
  ~
  \begin{subfigure}[b]{\SubfigureWidth{}}
    \includegraphics[width=\columnwidth]{ch06_paper_gpp/figs/s3b.pdf}
    \caption{}
    \label{fig:app:b:cmip5_hist_mte:b}
  \end{subfigure}
  \caption{Geographical distributions of the historical \acf{GPP} averaged
    between 1991 and 2000. (a) C\acs{MIP}5 \acf{MMM}. (b) FLUXNET-MTE product
    \autocite{Jung2011}. \AdaptedFrom{Schlund2020}.}
  \label{fig:app:b:cmip5_hist_mte}
\end{figure}

\begin{figure}[p]
  \centering
  \begin{subfigure}[b]{\SubfigureWidth{}}
    \includegraphics[width=\columnwidth]{ch06_paper_gpp/figs/s4a.pdf}
    \caption{}
    \label{fig:app:b:step2a_results:a}
  \end{subfigure}
  ~
  \begin{subfigure}[b]{\SubfigureWidth{}}
    \includegraphics[width=\columnwidth]{ch06_paper_gpp/figs/s4b.pdf}
    \caption{}
    \label{fig:app:b:step2a_results:b}
  \end{subfigure}
  \\
    \begin{subfigure}[b]{\SubfigureWidth{}}
    \includegraphics[width=\columnwidth]{ch06_paper_gpp/figs/s4c.pdf}
    \caption{}
    \label{fig:app:b:step2a_results:c}
  \end{subfigure}
  ~
  \begin{subfigure}[b]{\SubfigureWidth{}}
    \includegraphics[width=\columnwidth]{ch06_paper_gpp/figs/s4d.pdf}
    \caption{}
    \label{fig:app:b:step2a_results:d}
  \end{subfigure}
  \\
    \begin{subfigure}[b]{\SubfigureWidth{}}
    \includegraphics[width=\columnwidth]{ch06_paper_gpp/figs/s4e.pdf}
    \caption{}
    \label{fig:app:b:step2a_results:e}
  \end{subfigure}
  ~
  \begin{subfigure}[b]{\SubfigureWidth{}}
    \includegraphics[width=\columnwidth]{ch06_paper_gpp/figs/s4f.pdf}
    \caption{}
    \label{fig:app:b:step2a_results:f}
  \end{subfigure}
  \caption{Geographical distributions of the absolute \acf{GPP} at the end of
    the \nth{21} century in the \acs{RCP}8.5 scenario (step 2a) for different
    statistical models. (a) \acs{CMIP}5 \acf{MMM}. (b) Rescaled \acs{CMIP}5
    \acs{MMM} using \cref{eq:06:xxx}. (c) \Acf{LASSO} model using only the
    historical \acs{GPP} as single predictor. (d) \Acf{GBRT} model using only
    the historical \acs{GPP} as single predictor. (e) \acs{LASSO} model using
    all predictors. (f) \acs{GBRT} model using all predictors.
    \AdaptedFrom{Schlund2020}.}
  \label{fig:app:b:step2a_results}
\end{figure}

\begin{figure}[p]
  \centering
  \begin{subfigure}[b]{\SubfigureWidth{}}
    \includegraphics[width=\columnwidth]{ch06_paper_gpp/figs/s5a.pdf}
    \caption{}
    \label{fig:app:b:step2a_results_errors:a}
  \end{subfigure}
  ~
  \begin{subfigure}[b]{\SubfigureWidth{}}
    \includegraphics[width=\columnwidth]{ch06_paper_gpp/figs/s5b.pdf}
    \caption{}
    \label{fig:app:b:step2a_results_errors:b}
  \end{subfigure}
  \\
  \begin{subfigure}[b]{\SubfigureWidth{}}
    \includegraphics[width=\columnwidth]{ch06_paper_gpp/figs/s5c.pdf}
    \caption{}
    \label{fig:app:b:step2a_results_errors:c}
  \end{subfigure}
  ~
  \begin{subfigure}[b]{\SubfigureWidth{}}
    \includegraphics[width=\columnwidth]{ch06_paper_gpp/figs/s5d.pdf}
    \caption{}
    \label{fig:app:b:step2a_results_errors:d}
  \end{subfigure}
  \\
  \begin{subfigure}[b]{\SubfigureWidth{}}
    \includegraphics[width=\columnwidth]{ch06_paper_gpp/figs/s5e.pdf}
    \caption{}
    \label{fig:app:b:step2a_results_errors:e}
  \end{subfigure}
  ~
  \begin{subfigure}[b]{\SubfigureWidth{}}
    \includegraphics[width=\columnwidth]{ch06_paper_gpp/figs/s5f.pdf}
    \caption{}
    \label{fig:app:b:step2a_results_errors:f}
  \end{subfigure}
  \caption{Geographical distributions of the \acfp{SPE} of the absolute
    \acf{GPP} at the end of the \nth{21} century in the \acs{RCP}8.5 scenario
    (step 2a) for different statistical models. Details on the calculation of
    the \acs{SPE} are given in \cref{subsec:app:b:xxx}. (a) \acs{CMIP}5
    \acf{MMM}. (b) Rescaled \acs{CMIP}5 \acs{MMM} using \cref{eq:06:xxx}. (c)
    \Acf{LASSO} model using only the historical \acs{GPP} as single predictor.
    (d) \Acf{GBRT} model using only the historical \acs{GPP} as single
    predictor. (e) \acs{LASSO} model using all predictors. (f) \acs{GBRT}
    model using all predictors. The \acs{SPE} is minimal for the \acs{GBRT}
    model using all predictors. \AdaptedFrom{Schlund2020}.}
  \label{fig:app:b:step2a_results_errors}
\end{figure}

\begin{figure}[p]
  \centering
  \begin{subfigure}[b]{\SubfigureWidth{}}
    \includegraphics[width=\columnwidth]{ch06_paper_gpp/figs/s6a.pdf}
    \caption{}
    \label{fig:app:b:step2b_results_errors:a}
  \end{subfigure}
  ~
  \begin{subfigure}[b]{\SubfigureWidth{}}
    \includegraphics[width=\columnwidth]{ch06_paper_gpp/figs/s6b.pdf}
    \caption{}
    \label{fig:app:b:step2b_results_errors:b}
  \end{subfigure}
  \\
  \begin{subfigure}[b]{\SubfigureWidth{}}
    \includegraphics[width=\columnwidth]{ch06_paper_gpp/figs/s6c.pdf}
    \caption{}
    \label{fig:app:b:step2b_results_errors:c}
  \end{subfigure}
  ~
  \begin{subfigure}[b]{\SubfigureWidth{}}
    \includegraphics[width=\columnwidth]{ch06_paper_gpp/figs/s6d.pdf}
    \caption{}
    \label{fig:app:b:step2b_results_errors:d}
  \end{subfigure}
  \caption{Geographical distributions of the \acfp{SPE} of the fractional
    change in \acf{GPP} over the \nth{21} century in the \acs{RCP}8.5 scenario
    (step 2b) for different statistical models. Details on the calculation of
    the \acs{SPE} are given in \cref{subsec:app:b:xxx}. (a) \acs{CMIP}5
    \acf{MMM}. (b) Rescaled \acs{CMIP}5 \acs{MMM} using \cref{eq:06:xxx}. (c)
    \Acf{LASSO} model. (d) \Acf{GBRT} model. The \acs{SPE} is minimal for the
    \acs{GBRT} model. \AdaptedFrom{Schlund2020}.}
  \label{fig:app:b:step2b_results_errors}
\end{figure}


\endgroup
