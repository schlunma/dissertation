%=============================================================================%
%                                Dissertation                                 %
%                               Manuel Schlund                                %
%                                  (c) 2020                                   %
%=============================================================================%
%                        Papers ECS and TCR assessment                        %
%=============================================================================%



\chapter{Assessment of Climate Sensitivity in the \acs{CMIP}6 Ensemble}
\label{ch:04:papers_ecs_tcr_assessment}

In order to reduce uncertainties in multi-model climate projections, a first
important step is the assessment of the desired target variables in the
corresponding climate model ensemble. This is particularly relevant when a new
generation of \acp{ESM} is published that includes considerable modifications
compared to the respective predecessor model versions. Currently, we are in a
situation like this with the \acs{CMIP}6 ensemble, for which new data is still
released every day (as of \TheMonth{}). In light of the upcoming international
climate assessment of the \ac{IPCC}'s \acs{AR}6, the evaluation of
policy-relevant metrics like \ac{ECS} or \ac{TCR} for these new \acs{CMIP}6
models is crucial since they provide vital information about the future climate
of planet Earth. As the \ac{ESMValTool} is an excellent tool that allows for a
quick and robust evaluation of new \ac{CMIP} data as soon as it gets published
on the \ac{ESGF} servers (see \cref{sec:03:esmvaltool}), we use it to assess
and analyze climate sensitivity (expressed by \ac{ECS} and \ac{TCR}) of the
\acs{CMIP}6 models and compare it to corresponding results of predecessor model
generations. This work is already published in two scientific papers:
\textcite{Bock2020} and \textcite{Meehl2020}. For \textcite{Bock2020}, the
author of this thesis contributed their figure 8 (a bar chart showing the
\ac{ECS} for the three climate model generations \acs{CMIP}3, \acs{CMIP}5 and
\acs{CMIP}6), their figure 10 (map plots showing cloud-related feedback
parameters for the \acs{CMIP}5 and \acs{CMIP}6 \acp{MMM}; see
\cref{fig:04:cloud_feedback_parameters}) and discussion about \ac{ECS} of the
\acs{CMIP}6 models. For \textcite{Meehl2020}, the author of this thesis
contributed all figures and tables of the study and details on the calculations
of \ac{ECS} and \ac{TCR}.


\section{Evaluation of \acs{ECS} and \acs{TCR} in \acs{CMIP}6}
\label{sec:04:evaluation_ecs_and_tcr}

Following \cref{subsec:02:ecs}, \ac{ECS} is calculated with the Gregory
regression method using \ac{GSAT} and \ac{TOA} net radiation data for 150 years
of a \nxcotwo{4} simulation \autocite{Gregory2004}. This calculation is
illustrated in \cref{fig:04:gregory_regression_cmip6_mmm} for the \acs{CMIP}6
\ac{MMM}, which yields an \ac{ECS} of $3.74 \unit{K}$ when all 150 years of the
run are used. However, similar to the \acs{CMIP}5 ensemble, the exact value of
\ac{ECS} depends on the years considered in the Gregory regression (see
\cref{fig:02:gregory_regression_different_years}). Using only the first 20
years of the simulation gives a significantly lower \ac{ECS} of $3.31 \unit{K}$
than using only the last 130 years of the simulation, which gives an \ac{ECS}
of $4.05 \unit{K}$. As thoroughly describes in \cref{subsec:02:ecs}, the reason
for this is the state and time dependence of the climate feedback parameter,
which is given by the slope of the Gregory regression line. Due to non-linear
effects in the feedbacks, this slope changes over time, resulting in lower
(higher) values of \ac{ECS} when only early (late) years of the simulation are
considered.

\begin{figure}[t]
  \centering
  \includegraphics[width=\FigureWidth{}]{
    ch04_papers_ecs_tcr_assessment/figs/gregory_regression_cmip6_mmm.pdf}
  \caption{Gregory regression for the \acs{CMIP}6 \acf{MMM} following
    \textcite{Gregory2004}: Global and annual mean net \acf{TOA} radiation $N$
    versus the change in global and annual mean near-surface air temperature
    $\Delta T$ for 150 years of a simulation with an abrupt quadrupling of the
    atmospheric \acs{CO2} concentration (\nxcotwo{4}) for the \acs{CMIP}6
    \acs{MMM} (circles). To account for energy leakage and model drift, a
    linear fit of the corresponding pre-industrial control run is subtracted
    from the \nxcotwo{4} simulation. A linear regression can be used to
    calculate the \acf{ECS} according to \cref{eq:02:ecs} with the radiative
    forcing $F_\text{4x}$ (\yintercept) and the climate feedback parameter
    $\lambda$ (slope) for all 150 years (black line), which results in
    $\text{ECS} = 3.74 \unit{K}$. Due to non-linear effects, the exact value
    of \acs{ECS} depends on the years considered for the Gregory regression
    (see \cref{subsec:02:ecs} and
    \cref{fig:02:gregory_regression_different_years}): Using only the first 20
    years of the simulation (blue circles and line) yields a considerably
    smaller \acs{ECS} than using only the last 130 years of the simulation
    (orange circles and line). Updated from \textcite{Meehl2020}.}
  \label{fig:04:gregory_regression_cmip6_mmm}
\end{figure}

In contrast to that, the transient response of the climate system \ac{TCR} is
calculated from \ac{GSAT} data at the time of \ac{CO2} doubling in a
\onepctcotwo{} run \commentcite{Bindoff2013}{see \cref{subsec:02:tcr}}. Using
these two definitions and the \ac{ESMValTool}, we assess \ac{ECS} and \ac{TCR}
for all \acs{CMIP}5 and \acs{CMIP}6 models where the necessary temperature and
radiation data is available (as of \TheMonth{}).
\Cref{tab:04:ecs_tcr_cmip5,tab:04:ecs_tcr_cmip6} give an overview over the
results for the \acs{CMIP}5 and \acs{CMIP}6 models, respectively.

\begin{table}[!t]
  \centering
  \csvreader[ECSAndTCRTable]{
    ch04_papers_ecs_tcr_assessment/data/CMIP5_ecs-tcr.csv}{}{
    \dataset & \idx & \ecs & \tcr}
  \caption{\acf{ECS} and \acf{TCR} evaluated for the \acs{CMIP}5 models.
    Details on the calculation of \acs{ECS} and \acs{TCR} are given in
    \cref{subsec:02:ecs} and \cref{subsec:02:tcr}, respectively. The \acf{MMM}
    is calculated from the Gregory regression method using the \acs{MMM} net
    \acf{TOA} radiation and the \acs{MMM} change in \acf{GSAT} similar to
    \cref{fig:04:gregory_regression_cmip6_mmm}. The multi-model standard
    deviation is given by the sample standard deviation of \acs{ECS} evaluated
    over all climate models (using the normalization $1 / M$, where $M$ is the
    number of models).}
  \label{tab:04:ecs_tcr_cmip5}
\end{table}

\begin{table}[p]
  \centering
  \csvreader[ECSAndTCRTable]{
    ch04_papers_ecs_tcr_assessment/data/CMIP6_ecs-tcr.csv}{}{
    \dataset & \idx & \ecs & \tcr}
  \caption{As in \cref{tab:04:ecs_tcr_cmip5} but for the \acs{CMIP}6 models.}
  \label{tab:04:ecs_tcr_cmip6}
\end{table}

The first striking feature of these two tables is the increased \ac{MMM} of
\ac{ECS} and \ac{TCR} in \acs{CMIP}6. For \ac{ECS}, the \acs{CMIP}6 \ac{MMM} is
about $16 \unit{\%}$ ($0.51 \unit{K}$) higher than the corresponding
\acs{CMIP}5 \ac{MMM}. For \ac{TCR}, the relative difference between the two
model ensembles is notably smaller with about $10 \unit{\%}$ ($0.19 \unit{K}$).
The spread in the multi-model ensembles (expressed as the multi-model standard
deviation) shows an even larger increase in the \acs{CMIP}6 ensemble: For
\ac{ECS}, the relative difference is about $49 \unit{\%}$ ($0.72 \unit{K}$ in
\acs{CMIP}5 to $1.07 \unit{K}$ in \acs{CMIP}6) and for \ac{TCR}, the relative
difference is about $17 \unit{\%}$ ($0.36 \unit{K}$ in \acs{CMIP}5 to $0.42
\unit{K}$ in \acs{CMIP}6). The main reason for the increased \ac{MMM} and
spread in the \acs{CMIP}6 ensemble is the existence of several models with very
high values of \ac{ECS} and \ac{TCR}. In addition to that, there also exists a
number of models with very low values of \ac{ECS}. Therefore, the \ac{CMIP}6
model ranges of \ac{ECS} and \ac{TCR} are well outside the corresponding
assessed ranges given by the latest published \ac{AR} of the \ac{IPCC} from
2013 \autocite{Stocker2013} with $\range{1.8}{5.6} \unit{K}$ for \ac{ECS}
(\acs{AR}5: $\range{1.5}{4.5} \unit{K}$) and  $\range{1.3}{3.0} \unit{K}$
(\acs{AR}5: $\range{1.0}{2.5} \unit{K}$) for \ac{TCR}.

\begin{figure}[t]
  \centering
  \includegraphics[width=\LargeFigureWidth{}]{
    ch04_papers_ecs_tcr_assessment/figs/historical_ecs_tcr_ranges.pdf}
  \caption{Assessed values of the \acf{ECS} (blue bars) and the \acf{TCR} (red
    bars) over the years in the Charney report from 1979 \autocite{Charney1979}
    and the subsequent \acp{AR} of the \acf{IPCC} \autocite{Mitchell1990,
      Kattenberg1996, Albritton2001, Solomon2007, Stocker2013}. Orange and
    green bars represent the modeled ranges of \acs{ECS} and \acs{TCR},
    respectively, from the different climate model ensembles of the \acf{CMIP}
    with the corresponding \acf{MMM} illustrated with the horizontal black
    lines (the green dot represents output from a single climate model). The
    numbers correspond to individual \acs{CMIP}5 and \acs{CMIP}6 models (see
    \cref{tab:04:ecs_tcr_cmip5,tab:04:ecs_tcr_cmip6}). Updated from
    \textcite{Meehl2020}.}
  \label{fig:04:historical_ecs_tcr_ranges}
\end{figure}

To illustrate this and put it into historical context,
\cref{fig:04:historical_ecs_tcr_ranges} shows the assessed \ac{ECS} and
\ac{TCR} ranges over the years from the Charney report \autocite{Charney1979}
and the different \acp{AR} of the \ac{IPCC} in combination with the
corresponding modeled ranges from the different \ac{CMIP} generations. Since
the Charney report in 1979, the assessed range of \ac{ECS} of $\range{1.5}{4.5}
\unit{K}$ has almost remained unchanged during the last 40 years
\autocite{Charney1979, Mitchell1990, Kattenberg1996, Albritton2001,
  Stocker2013} with the exception of \acs{AR}4, where the lower bound was
temporarily increased to $2.0 \unit{K}$ \autocite{Solomon2007}. For \acs{AR}2,
\acs{AR}4 and \acs{AR}5, the corresponding climate model generations from
\acs{CMIP}1, \acs{CMIP}3 and \acs{CMIP}5 more or less agree with this assessed
range \autocite{Kattenberg1996, Randall2007, Flato2013}. In contrast to that,
some of the early climate models used in \acs{AR}1 and some \acs{CMIP}2 models
used in \acs{AR}3 exhibit values of \acs{ECS} above $4.5 \unit{K}$, resulting
in a upper model range of about $0.6 \unit{K}$ higher than the assessed range
\autocite{Mitchell1990, Cubasch2001}. However, these deviations from the
assessed range are insignificant to the ones present in the \acs{CMIP}6
ensemble, which have become evident over the course of 2020. As shown in
\cref{fig:04:historical_ecs_tcr_ranges}, both the lower and upper bounds are
more extreme than in any previous climate model generation, resulting in a
\acs{CMIP}6  \ac{ECS} model range of $\range{1.8}{5.6} \unit{K}$ (as of
\TheMonth{}). This increase in range is particularly relevant for the upper
bound, which is about $1.1 \unit{K}$ higher than the assessed upper bound of
$4.5 \unit{K}$. Moreover, it is not only a small fraction of models that
exceeds this upper bound: out of the 42 analyzed \ac{CMIP}6 models, a third (14
models) has an \ac{ECS} above $4.5 \unit{K}$. Because of the massive political
and societal relevance of \ac{ECS}, its apparent increase in the \ac{CMIP}6
ensemble is currently one of the most important questions for the entire
climate modeling community. A list of possible reasons for it is given in
\cref{sec:04:possible_reasons_high_ecs_cmip6}.

In contrast to \ac{ECS}, \ac{TCR} has only been evaluated since \acs{AR}1 in
1990, in which it is estimated with $2.3 \unit{K}$ using a single climate model
\commentcite{Bretherton1990}{see single green circle in
  \cref{fig:04:historical_ecs_tcr_ranges}}. Over the years, the modeled range
of \ac{TCR} has decreased to about $\range{1.1}{2.6} \unit{K}$ in the
\acs{CMIP}5 ensemble \autocite{Flato2013}. Similarly, the assessed range of
\ac{TCR}, which was first quoted in \acs{AR}4 with $\range{1.0}{3.0} \unit{K}$
\autocite{Solomon2007}, has been reduced to $\range{1.0}{2.5} \unit{K}$ in
\acs{AR}5 \autocite{Stocker2013}. While the corresponding climate model ranges
from \acs{CMIP}3 and \acs{CMIP}5 agree with these assessed ranges
\autocite{Randall2007, Flato2013}, the upper bound of the \acs{CMIP}6 model
range ($\range{1.3}{3.0} \unit{K}$) is well above the assessed upper bound of
$2.5 \unit{K}$ given by \acs{AR}5. However, due to the correlation of \ac{TCR}
and \ac{ECS} given by \cref{eq:02:tcr_vs_ecs}, this result is not surprising in
light of the large \ac{ECS} range combined with the many high sensitivity
models of the \acs{CMIP}6 ensemble.


\section{Possible Reasons for High Climate Sensitivity in \acs{CMIP}6}
\label{sec:04:possible_reasons_high_ecs_cmip6}
