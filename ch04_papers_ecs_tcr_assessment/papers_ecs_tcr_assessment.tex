%=============================================================================%
%                            Doctoral Dissertation                            %
%                               Manuel Schlund                                %
%                                  (c) 2021                                   %
%=============================================================================%
%                        Papers ECS and TCR assessment                        %
%=============================================================================%



\chapter{Assessment of Climate Sensitivity in the \acs{CMIP}6 Ensemble}
\label{ch:04:papers_ecs_tcr_assessment}

In order to reduce uncertainties in multi-model climate projections, a first
important step is the assessment of the desired target variables in the
corresponding climate model ensemble. This is particularly relevant when a new
generation of \acp{ESM} is published that includes considerable modifications
compared to the respective predecessor model versions. Currently, we are in a
situation like this with the \acs{CMIP}6 ensemble, for which new data is still
released every day (as of \TheMonth{}). In light of the upcoming international
climate assessment of the \ac{IPCC}'s \acs{AR}6, the evaluation of
policy-relevant metrics like \ac{ECS} or \ac{TCR} for these new \acs{CMIP}6
models is crucial since they provide vital information about the future climate
of planet Earth. As the \ac{ESMValTool} is an excellent tool that allows for a
quick and robust evaluation of new \ac{CMIP} data as soon as it gets published
on the \ac{ESGF} servers (see \cref{sec:03:esmvaltool}), it is used here to
assess and analyze \ac{ECS} and \ac{TCR} of the \acs{CMIP}6 models and compare
it to corresponding results of predecessor model generations and international
climate assessments. This work is already published in two scientific papers:
\textcite{Bock2020} and \textcite{Meehl2020}. For \textcite{Bock2020}, the
author of this thesis contributed their figure 8 (a bar chart showing the
\ac{ECS} for several \acs{CMIP}3, \acs{CMIP}5 and \acs{CMIP}6 models), their
figure 10 (map plots showing cloud-related feedback parameters for the
\acs{CMIP}5 and \acs{CMIP}6 \acp{MMM}; see
\cref{fig:04:cloud_feedback_parameters}), code to create these figures with the
\ac{ESMValTool} and text for the manuscript (in particular a section about the
evaluation of \ac{ECS}). For \textcite{Meehl2020}, the author of this thesis
contributed all figures and tables, code to produce these with the
\ac{ESMValTool} and text for the manuscript. This
\namecref{ch:04:papers_ecs_tcr_assessment} first presents the evaluation of
climate sensitivity for the \acs{CMIP}5 and \acs{CMIP}6 ensembles
(\cref{sec:04:evaluation_ecs_and_tcr}) and puts the corresponding \ac{ECS} and
\ac{TCR} values into historical context (\cref{sec:04:historical_context}).
Finally, possible reasons for the apparent increase of climate sensitivity in
the \acs{CMIP}6 models are discussed in detail
(\cref{sec:04:possible_reasons_high_ecs_cmip6}).


\section{Evaluation of \acs{ECS} and \acs{TCR} in \acs{CMIP}5 and \acs{CMIP}6}
\label{sec:04:evaluation_ecs_and_tcr}

Following \cref{subsec:02:ecs}, \ac{ECS} is calculated with the Gregory
regression method using \ac{GSAT} and \ac{TOA} net radiation data for 150 years
of a \nxcotwo{4} simulation \autocite{Gregory2004}. This calculation is
illustrated in \cref{fig:04:gregory_regression_cmip6_mmm} for the \acs{CMIP}6
\ac{MMM}, which yields an \ac{ECS} of $3.74 \unit{K}$ when all 150 years of the
run are used. However, similar to the \acs{CMIP}5 ensemble, the exact value of
\ac{ECS} depends on the years considered in the Gregory regression (see
\cref{fig:02:gregory_regression_different_years}). Using only the first 20
years of the simulation gives a significantly lower \ac{ECS} of $3.31 \unit{K}$
than using only the last 130 years of the simulation, which gives an \ac{ECS}
of $4.05 \unit{K}$. As thoroughly describes in \cref{subsec:02:ecs}, the reason
for this is the state and time dependence of the climate feedback parameter,
which is given by the slope of the Gregory regression line. Due to non-linear
effects in the feedbacks, this slope changes over time, resulting in lower
(higher) values of \ac{ECS} when only early (late) years of the simulation are
considered. In contrast to that, the transient response of the climate system
\ac{TCR} is calculated from \ac{GSAT} data at the time of \ac{CO2} doubling in
a \onepctcotwo{} run \commentcite{Bindoff2013}{see \cref{subsec:02:tcr}}. Using
these two definitions and the \ac{ESMValTool}, \ac{ECS} and \ac{TCR} are
assessed for all \acs{CMIP}5 and \acs{CMIP}6 models where the necessary
temperature and radiation data is available (as of \TheMonth{}).
\Cref{tab:04:ecs_tcr_cmip5,tab:04:ecs_tcr_cmip6} give an overview over the
results for the \acs{CMIP}5 and \acs{CMIP}6 models, respectively.

\begin{figure}[t]
  \centering
  \includegraphics[width=\FigureWidth{}]{
    ch04_papers_ecs_tcr_assessment/figs/gregory_regression_cmip6_mmm.pdf}
  \caption[
    Gregory regression for the \acs{CMIP}6 \acl{MMM}.
  ]{
    Gregory regression for the \acs{CMIP}6 \acf{MMM} following
    \textcite{Gregory2004}: Global and annual mean net \acl{TOA} radiation $N$
    versus the change in global and annual mean \acl{T} $\Delta T$ for 150
    years of a simulation with an abrupt quadrupling of the atmospheric
    \acs{CO2} concentration (\nxcotwo{4}) for the \acs{CMIP}6 \acs{MMM}
    (circles). To account for energy leakage and model drift, a linear fit of
    the corresponding pre-industrial control run is subtracted from the
    \nxcotwo{4} simulation. A linear regression can be used to calculate the
    \acf{ECS} according to \cref{eq:02:ecs} with the radiative forcing
    $F_\mathtxt{4x}$ (\yintercept) and the climate feedback parameter $\lambda$
    (slope) for all 150 years (black line), which results in $\mathtxt{ECS} =
    3.74 \unit{K}$. Due to non-linear effects, the exact value of \acs{ECS}
    depends on the years considered for the Gregory regression (see
    \cref{subsec:02:ecs} and \cref{fig:02:gregory_regression_different_years}):
    Using only the first 20 years of the simulation (blue circles and line)
    yields a considerably smaller \acs{ECS} than using only the last 130 years
    of the simulation (orange circles and line). \AdaptedFrom{Meehl2020}.
  }
  \label{fig:04:gregory_regression_cmip6_mmm}
\end{figure}

\begin{table}[!t]
  \centering
  \csvreader[ECSAndTCRTable]{
    ch04_papers_ecs_tcr_assessment/data/cmip5_ecs-tcr.csv}{}{
    \dataset & $\idx$ & $\ecs$ & $\tcr$}
  \caption[
    \Acf{ECS} and \acf{TCR} calculated for the \acs{CMIP}5 models.
  ]{
    \Acf{ECS} and \acf{TCR} calculated for the \acs{CMIP}5 models. Details on
    this are given in \cref{subsec:02:ecs} and \cref{subsec:02:tcr},
    respectively. The \acf{MMM} is calculated from the Gregory regression
    method using the \acs{MMM} net \acf{TOA} radiation and the \acs{MMM} change
    in \acf{GSAT} similar to \cref{fig:04:gregory_regression_cmip6_mmm}. The
    multi-model standard deviation is given by the sample standard deviation of
    \acs{ECS} evaluated over all climate models (using the normalization $1 /
    M$, where $M$ is the number of models). Corresponding references for each
    model are given in \cref{tab:app:a:cmip5_models}. \AdaptedFrom{Meehl2020}.
  }
  \label{tab:04:ecs_tcr_cmip5}
\end{table}

\begin{table}[p]
  \centering
  \csvreader[ECSAndTCRTable]{
    ch04_papers_ecs_tcr_assessment/data/cmip6_ecs-tcr.csv}{}{
    \dataset & $\idx$ & $\ecs$ & $\tcr$}
  \caption[
    As in \cref{tab:04:ecs_tcr_cmip5} but for the \acs{CMIP}6 models.
  ]{
    As in \cref{tab:04:ecs_tcr_cmip5} but for the \acs{CMIP}6 models.
    Corresponding references for each model are given in
    \cref{tab:app:a:cmip6_models}. \AdaptedFrom{Meehl2020}.
  }
  \label{tab:04:ecs_tcr_cmip6}
\end{table}

The first striking feature of these two tables is the increased \ac{MMM} of
\ac{ECS} and \ac{TCR} in \acs{CMIP}6. For \ac{ECS}, the \acs{CMIP}6 \ac{MMM} is
about $16 \unit{\%}$ ($0.51 \unit{K}$) higher than the corresponding
\acs{CMIP}5 \ac{MMM}. For \ac{TCR}, the relative difference between the two
model ensembles is notably smaller with about $10 \unit{\%}$ ($0.19 \unit{K}$).
The spread in the multi-model ensembles (expressed as the multi-model standard
deviation) shows an even larger increase in the \acs{CMIP}6 ensemble: For
\ac{ECS}, the relative difference is about $49 \unit{\%}$ ($0.72 \unit{K}$ in
\acs{CMIP}5 to $1.07 \unit{K}$ in \acs{CMIP}6) and for \ac{TCR}, the relative
difference is about $17 \unit{\%}$ ($0.36 \unit{K}$ in \acs{CMIP}5 to $0.42
\unit{K}$ in \acs{CMIP}6). The main reason for the increased \ac{MMM} and
spread in the \acs{CMIP}6 ensemble is the existence of several models with very
high values of \ac{ECS} and \ac{TCR}. In addition to that, there also exists a
number of models with very low values of \ac{ECS}. Therefore, the \acs{CMIP}6
model ranges of \ac{ECS} and \ac{TCR} are well outside the corresponding
assessed ranges given by the latest published \ac{AR} of the \ac{IPCC} from
2013 \autocite{Stocker2013} with $\rangeunit{1.8}{5.6}{K}$ for \ac{ECS}
(\acs{AR}5: $\rangeunit{1.5}{4.5}{K}$) and $\rangeunit{1.3}{3.0}{K}$
(\acs{AR}5: $\rangeunit{1.0}{2.5}{K}$) for \ac{TCR}.


\section{Comparison to Previous \acs{CMIP} Generations and International
  Climate Assessments}
\label{sec:04:historical_context}

To illustrate the results of the previous
\namecref{sec:04:evaluation_ecs_and_tcr} and put them into historical context,
\cref{fig:04:historical_ecs_tcr_ranges} shows the assessed \ac{ECS} and
\ac{TCR} ranges over the years from the Charney report \autocite{Charney1979}
and the different \acp{AR} of the \ac{IPCC} in combination with the
corresponding modeled ranges from the different \ac{CMIP} generations. Since
the Charney report in 1979, the assessed range of \ac{ECS} of
$\rangeunit{1.5}{4.5}{K}$ has almost remained unchanged during the last 40
years \autocite{Charney1979, Mitchell1990, Kattenberg1996, Albritton2001,
  Stocker2013} with the exception of \acs{AR}4, where the lower bound was
temporarily increased to $2.0 \unit{K}$ \autocite{Solomon2007}. For \acs{AR}2,
\acs{AR}4 and \acs{AR}5, the corresponding climate model generations from
\acs{CMIP}1, \acs{CMIP}3 and \acs{CMIP}5 more or less agree with this assessed
range \autocite{Kattenberg1996, Randall2007, Flato2013}. In contrast to that,
some of the early climate models used in \acs{AR}1 and some \acs{CMIP}2 models
used in \acs{AR}3 exhibit values of \acs{ECS} well above $4.5 \unit{K}$,
resulting in a upper model range of about $0.6 \unit{K}$ higher than the
assessed range \autocite{Mitchell1990, Cubasch2001}. However, these deviations
from the assessed range are small compared to the ones present in the
\acs{CMIP}6 ensemble, which have become evident over the course of 2020. As
shown in \cref{fig:04:historical_ecs_tcr_ranges}, both the lower and upper
bound are more extreme than in any previous climate model generation, resulting
in a \acs{CMIP}6 \ac{ECS} model range of $\rangeunit{1.8}{5.6}{K}$ (as of
\TheMonth{}). This increase in range is particularly relevant for the upper
bound, which is about $1.1 \unit{K}$ higher than the assessed upper bound of
$4.5 \unit{K}$. Moreover, it is not only a small fraction of models that
exceeds this upper bound: out of the 42 analyzed \acs{CMIP}6 models, a third
(14 models) has an \ac{ECS} above $4.5 \unit{K}$. In addition to the analysis
presented here, several modeling centers independently confirmed high \ac{ECS}
values in their models \autocite{Andrews2019, BodasSalcedo2019, Gettelman2019,
  Wyser2020}. A list of possible reasons for the increased climate sensitivity
in the \acs{CMIP}6 ensemble including aerosol-cloud interactions and changes in
the shortwave cloud feedback over the Southern Ocean is given in
\cref{sec:04:possible_reasons_high_ecs_cmip6}.

\begin{figure}[t]
  \centering
  \includegraphics[width=\LargeFigureWidth{}]{
    ch04_papers_ecs_tcr_assessment/figs/historical_ecs_tcr_ranges.pdf}
  \caption[
    Historical values of the \acf{ECS} and the \acf{TCR} for different
    \aclp{AR} and climate model ensembles.
  ]{
    Assessed values of the \acf{ECS} (blue bars) and the \acf{TCR} (red bars)
    over the years in the Charney report from 1979 \autocite{Charney1979} and
    the subsequent \acfp{AR} of the \acf{IPCC} \autocite{Mitchell1990,
      Kattenberg1996, Albritton2001, Solomon2007, Stocker2013}. Orange and
    green bars represent the modeled ranges of \acs{ECS} and \acs{TCR},
    respectively, from the different climate model ensembles of the \acf{CMIP}
    with the corresponding \acf{MMM} illustrated with the horizontal black
    lines (the green circle represents output from a single climate model). The
    numbers correspond to individual \acs{CMIP}5 and \acs{CMIP}6 models (see
    \cref{tab:04:ecs_tcr_cmip5,tab:04:ecs_tcr_cmip6}). \AdaptedFrom{Meehl2020}.
  }
  \label{fig:04:historical_ecs_tcr_ranges}
\end{figure}

In contrast to \ac{ECS}, \ac{TCR} has only been evaluated since \acs{AR}1 in
1990, in which it is estimated with $2.3 \unit{K}$ using only a single climate
model \commentcite{Bretherton1990}{see single green circle in
  \cref{fig:04:historical_ecs_tcr_ranges}}. Over the years, the modeled range
of \ac{TCR} has decreased from about $\rangeunit{1.3}{3.8}{K}$ in the
\acs{CMIP}1 models \autocite{Kattenberg1996} to about $\rangeunit{1.1}{2.6}{K}$
in the \acs{CMIP}5 ensemble \autocite{Flato2013}. Similarly, the assessed range
of \ac{TCR}, which was first quoted in \acs{AR}4 with $\rangeunit{1.0}{3.0}{K}$
\autocite{Solomon2007}, has been reduced to $\rangeunit{1.0}{2.5}{K}$ in the
subsequent \acs{AR}5 \autocite{Stocker2013}. While the corresponding climate
model ranges from \acs{CMIP}3 and \acs{CMIP}5 agree with these assessed ranges
\autocite{Randall2007, Flato2013}, the upper bound of the \acs{CMIP}6 model
range ($\rangeunit{1.3}{3.0}{K}$) is well above the assessed upper bound of
$2.5 \unit{K}$ given by \acs{AR}5. However, unlike \ac{ECS}, also the lower
bound of the \acs{CMIP}6 model range increased compared to the \acs{CMIP}5
ensemble, leading to a similar \ac{TCR} model spread in both climate model
generations. Due to the correlation of \ac{TCR} and \ac{ECS} given by
\cref{eq:02:tcr_vs_ecs}, the increase of \ac{TCR} in many \acs{CMIP}6 models is
not surprising in light of the many high \ac{ECS} models that are present the
\acs{CMIP}6 ensemble.


\section{Possible Reasons for High Climate Sensitivity in \acs{CMIP}6}
\label{sec:04:possible_reasons_high_ecs_cmip6}

Because of the massive political and societal relevance of \ac{ECS}, its
apparent increase in the \acs{CMIP}6 ensemble is currently one of the most
important questions for the entire climate modeling community. For this reason,
this \namecref{sec:04:possible_reasons_high_ecs_cmip6} discusses possible
reasons for the increased climate sensitivity in \acs{CMIP}6.

Various improvements of the underlying physical, biological and chemical
processes have been introduced to the \acs{CMIP}6 \acp{ESM} in order to
represent the coupled Earth system in more detail. Since many of these
processes directly influence the models' \ac{ECS} \autocite{Forster2020},
determining reasons for the increased \ac{ECS} in the \acs{CMIP}6 ensemble is
highly non-trivial. As discussed in detail by \textcite{Meehl2020}, one
possible reason is the consideration of aerosol-cloud interactions. New
prognostic aerosol schemes added to some \acs{CMIP}6 models that include
aerosol-cloud interactions could have produced overly large negative radiative
forcing, which then required a stronger model response to \ac{GHG} forcing in
order to reproduce the observed historical temperature trend. In fact,
\textcite{Meehl2020} find a weak inter-model relationship between the aerosol
forcing and \ac{ECS} for some \acs{CMIP}6 models, with larger negative
present-day aerosol forcing associated with higher \ac{ECS}. Such relationships
have also been found for previous climate model generations
\autocite{Kiehl2007, Forster2013}. However, due to the varying aerosol forcing
over time, these relations cannot be used to tune \ac{ECS} based on the models'
responses to aerosols over the different periods of the \nth{20} century
\autocite{Dittus2020}. A further type of aerosol-cloud interactions is for
example present in the \ac{HadGEM} version 3. The new aerosol scheme in this
model suppresses a strong negative cloud feedback over the Southern Ocean that
is found in the model's predecessor versions \autocite{BodasSalcedo2019}. The
Southern Ocean is one of the most pristine aerosol regions in the world
\autocite{Hamilton2014}, where aerosols predominantly originate from natural
sources like marine emissions of gaseous dimethyl sulfide. Earlier \ac{HadGEM}
versions show a large reduction in cloud droplet size with warming over the
Southern Ocean, which increases the cloud albedo and thus produces a negative
shortwave cloud feedback. On the contrary, in \acs{HadGEM}3, this negative
feedback is completely suppressed or even positive due to a small increase of
the cloud droplet size with warming, which eventually increases \ac{ECS} due to
an increased net cloud feedback \autocite{BodasSalcedo2019}.

Another possible reason for the high climate sensitivity in the \acs{CMIP}6
ensemble is related to cloud feedbacks \autocite{Bock2020}. As already
discussed in \cref{subsec:02:climate_feedbacks}, uncertainties in cloud
feedbacks are a major source of uncertainty of \ac{ECS} in modern \acp{ESM}
\autocite{Boucher2013, Flato2013}. Thus, changes in processes that are related
to cloud feedbacks immediately impact the climate models' climate sensitivity.
The particular feedback mechanism that is thought to be relevant for the high
\ac{ECS} in \acs{CMIP}6 is connected to cloud phase changes over the Southern
Ocean that are present in earlier \ac{CMIP} generations (\eg{} \acs{CMIP}5).
This so-called \emph{cloud phase change feedback} is illustrated in
\cref{fig:04:cloud_phase_change_feedback} (green arrows). As the climate warms,
the predominantly ice clouds over the Southern Ocean in these climate models
become liquid clouds \autocite{McCoy2015}. As a result, the cloud will get more
reflective to the incoming solar radiation since a cloud consisting of smaller
liquid droplets reflects more sunlight than a cloud consisting of larger ice
crystals (for a fixed water content), which has a cooling effect on the climate
system. In addition, predominantly liquid clouds tend to precipitate less than
mixed-phase clouds (composed of liquid water and ice). This leads to a higher
liquid water content of the cloud and a further amplification of the cooling
effect. Overall, this forms a strong negative shortwave cloud feedback over the
Southern Ocean in climate models from previous \ac{CMIP} generations.

\begin{figure}[t]
  \centering
  \includegraphics[width=\LargeFigureWidth{}]{
    ch04_papers_ecs_tcr_assessment/figs/cloud_phase_change_feedback.pdf}
  \caption[
    Schematic illustration of the strong negative shortwave cloud phase change
    feedback over the Southern Ocean.
  ]{
    Schematic illustration of the strong negative shortwave cloud phase change
    feedback over the Southern Ocean, which is present in earlier \acs{CMIP}
    model generations like \acs{CMIP}5 (green arrows). Due to substantial
    improvements in the microphysical representation of mixed-phase clouds in
    many \acs{CMIP}6 models, the predominantly ice clouds over the Southern
    Ocean in the present-day climate of previous climate model generations have
    been replaced with clouds that predominantly consist of (supercooled)
    liquid clouds \autocite{BodasSalcedo2019, Gettelman2019}. Thus, in the
    affected \acs{CMIP}6 models, this cloud phase change due to warming is
    reduced by allowing for supercooled cloud liquid, which substantially
    reduces the strong negative cloud phase change feedback (orange arrows).
    This leads to an increase in the global net cloud feedback
    \autocite{BodasSalcedo2019, Tan2016} and the \acf{ECS} in the corresponding
    \acs{CMIP}6 models \autocite{Andrews2019, Gettelman2019}, which is a
    possible explanation for the high \acs{ECS} values found in many
    \acs{CMIP}6 models.
  }
  \label{fig:04:cloud_phase_change_feedback}
\end{figure}

However, it is well known that the low-level mixed-phase clouds over the
Southern Ocean in these earlier climate models are biased towards too large
amounts of ice crystals and too little amounts of supercooled liquid water when
compared to satellite observations \autocite{BodasSalcedo2016}. Thus, the size
of the resulting strong negative cloud phase change feedback in these models
has long been questioned \autocite{McCoy2015, Tan2016}. Improvements in the
representation of mixed-phase clouds are known to reduce this long-standing
cloud phase bias over the Southern Ocean \autocite{BodasSalcedo2016, McCoy2016}
and to improve the representation of both cloud microphysical structure and
cloud radiative impacts in this region \autocite{Hyder2018, Kay2016}. For this
reason, the microphysical representation of mixed-phase clouds has been updated
in many \acs{CMIP}6 models. For example, in \acs{CESM}2, a new mixed-phase ice
nucleation scheme by \textcite{Hoose2010} replaces the \textcite{Meyers1992}
empirical scheme used in predecessor models \autocite{Gettelman2019}.
Simulations with the \textcite{Meyers1992} scheme show almost no supercooled
liquid water in high-latitude clouds and a strong negative cloud phase feedback
when ice turns to liquid. On the contrary, simulations with the new ice
nucleation scheme by \textcite{Hoose2010} show a strong reduction in this
feedback and are more consistent with observations of the current climate
\autocite{Gettelman2019}. Another example is the \acs{HadGEM}3 model, for which
a new cloud scheme that accounts for turbulent production of liquid water in
mixed-phase clouds has been implemented \autocite{BodasSalcedo2019}. The new
scheme parameterizes the role of subgrid-scale turbulence in the production and
maintenance of supercooled liquid water \autocite{Furtado2016}. As a
consequence, the cloud liquid water content over the Southern Ocean is
increased, which is in closer agreement to observations compared to earlier
\ac{HadGEM} versions. Moreover, the increase in the cloud liquid water content
with warming is reduced in \acs{HadGEM}3, leading to a reduction in the
negative cloud phase change feedback over the Southern Ocean
\autocite{BodasSalcedo2019}. By evaluating traditional metrics used for
numerical weather prediction, \textcite{Williams2020} additionally demonstrate
that short-term forecasts created with \acs{HadGEM}3 show a higher performance
than comparable simulations of predecessor model versions, which indicates an
improvement of the physical processes in the new climate model.

Overall, these changes in many of the \acs{CMIP}6 models lead to a better
agreement of the \acs{CMIP}6 \ac{MMM} of the \ac{TOA} shortwave \ac{CRE} with
corresponding observations \autocite{Bock2020}. Moreover, the strong negative
cloud feedback described above that results from a cloud phase change from ice
clouds in the present-day to liquid clouds in the future is substantially
reduced in those \acs{CMIP}6 models that simulate predominantly liquid clouds
over the present-day Southern Ocean. This is illustrated in
\cref{fig:04:cloud_phase_change_feedback} (orange arrows). As a consequence,
this reduction of the negative shortwave cloud phase change feedback increases
the global net cloud feedback \autocite{BodasSalcedo2019, Tan2016} and
eventually the \ac{ECS} of the affected \acs{CMIP}6 models
\autocite{Andrews2019, Gettelman2019}, offering a possible explanation for the
high \ac{ECS} values found in many \acs{CMIP}6 models.

\begin{figure}[p]
  \centering
  \begin{subfigure}[b]{\SubfigureWidth{}}
    \includegraphics[width=\columnwidth]{
      ch04_papers_ecs_tcr_assessment/figs/abs_cmip6_netcre.pdf}
    \caption{}
    \label{fig:04:cloud_feedback_parameters:a}
  \end{subfigure}
  ~
  \begin{subfigure}[b]{\SubfigureWidth{}}
    \includegraphics[width=\columnwidth]{
      ch04_papers_ecs_tcr_assessment/figs/bias_cmip6_netcre-cmip5_netcre.pdf}
    \caption{}
    \label{fig:04:cloud_feedback_parameters:b}
  \end{subfigure}
  \\
  \begin{subfigure}[b]{\SubfigureWidth{}}
    \includegraphics[width=\columnwidth]{
      ch04_papers_ecs_tcr_assessment/figs/abs_cmip6_swcre.pdf}
    \caption{}
    \label{fig:04:cloud_feedback_parameters:c}
  \end{subfigure}
  ~
  \begin{subfigure}[b]{\SubfigureWidth{}}
    \includegraphics[width=\columnwidth]{
      ch04_papers_ecs_tcr_assessment/figs/bias_cmip6_swcre-cmip5_swcre.pdf}
    \caption{}
    \label{fig:04:cloud_feedback_parameters:d}
  \end{subfigure}
  \\
  \begin{subfigure}[b]{\SubfigureWidth{}}
    \includegraphics[width=\columnwidth]{
      ch04_papers_ecs_tcr_assessment/figs/abs_cmip6_lwcre.pdf}
    \caption{}
    \label{fig:04:cloud_feedback_parameters:e}
  \end{subfigure}
  ~
  \begin{subfigure}[b]{\SubfigureWidth{}}
    \includegraphics[width=\columnwidth]{
      ch04_papers_ecs_tcr_assessment/figs/bias_cmip6_lwcre-cmip5_lwcre.pdf}
    \caption{}
    \label{fig:04:cloud_feedback_parameters:f}
  \end{subfigure}
  \caption[
    Geographical distributions of the \acl{CRE} feedback parameters for the
    \acs{CMIP}5 and \acs{CMIP}6 \aclp{MMM}.
  ]{
    Geographical distributions of the net (a), shortwave (c) and longwave (e)
    \acf{CRE} feedback parameters for the \acs{CMIP}6 \acf{MMM} and the
    corresponding differences to the \acs{CMIP}5 \acs{MMM}, again for the net
    (b), shortwave (d) and longwave (f) components. The \acs{CRE} feedback
    parameters for each grid cell are calculated with a Gregory regression of
    the grid cell's \acs{CRE} versus the change in the \acl{GSAT} (see
    \cref{subsec:02:cloud_feedback_parameters}). \AdaptedFrom{Bock2020}.
  }
  \label{fig:04:cloud_feedback_parameters}
\end{figure}

To further confirm this hypothesis, the global distributions of relevant cloud
feedback parameters (expressed by the net, shortwave and longwave \ac{CRE}
feedback parameters; see \cref{subsec:02:cloud_feedback_parameters}) are shown
for the \acs{CMIP}6 \ac{MMM} and its difference to the \acs{CMIP}5 \ac{MMM} in
\cref{fig:04:cloud_feedback_parameters}. First of all, the comparison of
\cref{fig:04:cloud_feedback_parameters:a,fig:04:cloud_feedback_parameters:c,%
  fig:04:cloud_feedback_parameters:e} demonstrates that the shortwave component
dominates the net \ac{CRE} feedback parameter over large swaths of the globe
in the \acs{CMIP}6 \ac{MMM}. The sign change at approximately $60
\unit{\degree S}$ in the shortwave \ac{CRE} feedback parameter (see
\cref{fig:04:cloud_feedback_parameters:c}) roughly shows where the climate
models simulate a high fraction of ice clouds (south of $60 \unit{\degree
  S}$) for which possible phase changes under global warming are potentially
important. Such possible phase changes (from ice to liquid) contribute to a
negative shortwave cloud feedback in this region as liquid clouds tend to be
more reflective than ice clouds with a similar cloud water content.
\Cref{fig:04:cloud_feedback_parameters:b,fig:04:cloud_feedback_parameters:d}
show a higher shortwave and net \ac{CRE} feedback parameter over the Southern
Ocean in \acs{CMIP}6 compared to \acs{CMIP}5. This is consistent to the
hypothesis presented in the paragraph above, which claims that the reduced
negative cloud phase change feedback over the Southern Ocean in \acs{CMIP}6
leads to an increased net cloud feedback parameter in comparison to older
\ac{CMIP} generations. This result is further supported by
\textcite{Zelinka2020}, who find a larger positive zonal mean net cloud
feedback parameter in \acs{CMIP}6 compared to \acs{CMIP}5 due to a more
positive (reduced negative) low-level cloud feedback, primarily in the
extratropics. In comparison to earlier \acs{CMIP} generations, the
\acs{CMIP}6 models exhibit weaker increases in the extratropical low-level
cloud cover and the corresponding liquid water content as a result of global
warming. Moreover, \textcite{Zelinka2020} show that this mainly results from
an increase in the liquid condensate fraction in the \acs{CMIP}6 clouds for
the pre-industrial and present-day periods, leading to the aforementioned
reduction in the negative cloud phase change feedback on warming and
eventually to higher \ac{ECS} values.
