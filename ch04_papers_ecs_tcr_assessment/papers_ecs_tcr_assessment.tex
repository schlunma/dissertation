%=============================================================================%
%                                Dissertation                                 %
%                               Manuel Schlund                                %
%                                  (c) 2020                                   %
%=============================================================================%
%                        Papers ECS and TCR assessment                        %
%=============================================================================%



\chapter{Assessment of Climate Sensitivity in the \acs{CMIP}6 Ensemble}
\label{ch:04:papers_ecs_tcr_assessment}

In order to reduce uncertainties in multi-model climate projections, a first
important step is the assessment of the desired target variables in the
corresponding climate model ensemble. This is particularly relevant when a new
generation of \acp{ESM} is published that includes considerable modifications
compared to the respective predecessor model versions. Currently, we are in a
situation like this with the \acs{CMIP}6 ensemble, for which new data is still
released every day (as of \TheMonth{}). In light of the upcoming international
climate assessment of the \ac{IPCC} (\ac{AR6}), the evaluation of
policy-relevant metrics like \ac{ECS} or \ac{TCR} for these new \acs{CMIP}6
models is crucial since they provide vital information about the future climate
of planet Earth. As the \ac{ESMValTool} is an excellent tool that allows for a
quick and robust evaluation of new \ac{CMIP} data as soon as it gets published
on the \ac{ESGF} servers (see \cref{sec:03:esmvaltool}), we assess and analyze
climate sensitivity (expressed by \ac{ECS} and \ac{TCR}) of the \acs{CMIP}6
models and compare it to corresponding results of predecessor model
generations. This work is already published in two scientific papers:
\textcite{Bock2020} and \textcite{Meehl2020}.


\section{Evaluation of \acs{ECS} and \acs{TCR} in \acs{CMIP}6}
\label{sec:04:evaluation_ecs_and_tcr}


