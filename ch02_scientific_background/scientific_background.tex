%=============================================================================%
%                            Doctoral Dissertation                            %
%                            (c) by Manuel Schlund                            %
%=============================================================================%

%=============================================================================%
% This work is licensed under a
% Creative Commons Attribution 4.0 International License.
%
% You should have received a copy of the license along with this
% work. If not, see <http://creativecommons.org/licenses/by/4.0/>.
%=============================================================================%



\chapter{Scientific Background}
\label{ch:02:scientific_background}

This \namecref{ch:02:scientific_background} introduces the scientific
background of this thesis. First, basic concepts of climate model simulations
and associated uncertainties are introduced (\cref{sec:02:esms}). Next,
important metrics describing climate sensitivity
(\cref{sec:02:climate_sensitivity}) and fundamental biogeochemical processes of
the global carbon cycle (\cref{sec:02:esms}) are presented. Finally,
state-of-the art techniques that can be used to reduce uncertainties in
projections of the future climate with observations are shown
(\cref{sec:02:reducing_uncertainties}). These methods form the basis for new
techniques developed in this thesis.


\section{Earth System Models: Simulations and Analysis}
\label{sec:02:esms}


\subsection{Numerical Climate Modeling}
\label{subsec:02:numerical_climate_modeling}

In contrast to many other fields of science, researching the future evolution
of the Earth's climate cannot be purely done by performing experiments in a
laboratory. Due to the immense complexity of the Earth system (including
physical, biological, and chemical processes on various temporal and spatial
scales and their mutual interactions), we do not have access to a tiny replica
of the Earth that we can analyze when exposed to different external conditions
\autocite{Flato2011}. While observing the current state of the Earth System is
(relatively) straightforward, gaining evidence about the future evolution of
the climate by only considering present-day observations is rather difficult.

A possible way out is given by numerical climate models, which offer the
possibility to simulate the Earth's climate on a computer. To efficiently
replicate the vastly complex Earth system with finite computational resources,
climate models divide the Earth into a set of \emph{grid cells}. The typical
size of an atmospheric grid cell in modern-day global climate models is about
$100 \unit{km}$ along the horizontal dimensions (latitude and longitude) and $1
\unit{km}$ along the vertical dimension (pressure level or height). In
addition, also the temporal evolution of the Earth system is discretized using
time steps that are typically about $30 \unit{min}$ long in modern-day global
climate models. For each grid cell and time step, usually a single value per
model variable is given. Examples for such variables provided by a climate
model are the atmospheric prognostic variables velocity (horizontal and
vertical), temperature, specific humidity, pressure, and density. These
prognostic variables are related to each other via the \emph{primitive
  equations}, a set of partial differential equations that can be derived from
the conservation of momentum, mass, energy, and moisture \autocite{Holton2004}.
To progress further in time, these primitive equations are solved numerically
by the dynamical core of the climate models, which eventually describes the
Earth's large-scale atmospheric motions. Similar to this, many other processes
of the Earth system are simulated by the climate models using other fundamental
physical laws and principles. As opposed to these calculated variables that
form the output of climate models, the corresponding input is mainly given by
the radiative forcing.

\begin{figure}[t]
  \centering
  \includegraphics[width=\FigureWidth{}]{
    ch02_scientific_background/figs/esms_historical_evolution.pdf}
  \caption[
    Historical evolution of coupled climate models over the last 45 years.
  ]{
    Historical evolution of coupled climate models over the last 45 years. In
    early days, these models were so-called \acfp{AOGCM}, which only included
    three components: the atmosphere, the land surface, and the ocean. Over the
    time, the individual components grew in complexity and included a wider
    range of processes (illustrated by the growing cylinders). Eventually, more
    and more components (aerosols, carbon cycle, \etc{}) were added to the
    coupled system, forming the modern \acfp{ESM}.
    \IPCCReproducedFrom{Cubasch2013}{their figure 1.13}.
  }
  \label{fig:02:esms_historical_evolution}
\end{figure}

Many processes in the Earth system occur at spatial scales much smaller than
the size of a typical grid cell. Illustrative examples for this are clouds,
which are usually smaller than a $100 \unit{km} \times 100 \unit{km}$ grid
cell, but still play an important role in the overall climate system by
reflecting incoming and outgoing radiation \autocite{Boucher2013}. In order to
reasonably approximate these subgrid-scale processes, a concept called
\emph{parameterization} is used. Instead of simulating a process exactly,
parameterizations aim to represent the effect of that process at the grid scale
of the climate model by generating the appropriate forcing terms for the rest
of the system and the rest of the processes \autocite{Gettelman2016}.
Parameterizations are necessary to simulate many processes of the Earth system,
for example surface heat and moisture fluxes, moist convection, turbulent
mixing, and radiation \autocite{Holton2004}.

The first numerical climate models came up in the 1960s and were based on
weather prediction models \autocite{Flato2011}. Early models from the 1970s
simulated only the physical components of the climate system: atmosphere, land
surface, ocean, and sea ice (see \cref{fig:02:esms_historical_evolution}).
These models are called \emph{\acfp{AOGCM}} \autocite{Flato2013}. Over the
course of the years, climate models became more and more complex by including a
wider range of processes within the components, but also by introducing new
components to the coupled system. Examples of these are aerosols, the carbon
cycle, a dynamic vegetation, atmospheric chemistry, and land ice (see
\cref{fig:02:esms_historical_evolution}). \Acp{AOGCM} coupled to these
additional components are called \emph{\acfp{ESM}}, which are the current
state-of-the-art models that allow the most sophisticated simulations of the
Earth's climate. In contrast to \acp{AOGCM}, \acp{ESM} enable the simulation of
biological and chemical processes in addition to the dynamics of the physical
components of the Earth system. Especially in the context of anthropogenic
climate change, these additional processes are of uttermost importance for
realistic climate model simulations, since the anthropogenic interference with
the Earth system directly influences the various biogeochemical cycles of the
Earth. For example, the emission of the most prominent \ac{GHG}, \ac{CO2},
immediately impacts the global carbon cycle by inserting additional carbon into
the system (see \cref{sec:02:carbon_cycle} for details). Further examples
include land use changes like the deforestation of tropical rainforests, which
also directly influences several biogeochemical cycles (\eg{} carbon cycle,
nitrogen cycle, phosphorus cycle, \etc{}) by altering respective sinks and
sources.

Due to the complex interactions between the different components of the Earth
system, these changes in the biogeochemical processes also affect the physical
properties of the climate system. For example, due to the global carbon cycle,
only about $50 \unit{\%}$ of the emitted \ac{CO2} by humankind remains in the
atmosphere \autocite{Friedlingstein2020}. The residual part is absorbed by the
two other main carbon sinks of the planet, the terrestrial biosphere and the
ocean. Since only atmospheric \ac{CO2} can act as \ac{GHG} by introducing an
additional radiative forcing to the Earth System leading to increasing surface
temperatures, this uptake of \ac{CO2} by the carbon cycle slows down global
warming.


\subsection{The \acfAbstract{CMIP}}
\label{subsec:02:cmip}

Due to the complex nature of the Earth system itself, numerical models of it
consist of hundreds of thousands of lines of computer code. Thus, a
standardization of the experimental setup and model output to a certain degree
is crucial for the various research groups developing \acp{ESM} all around the
world in order to obtain comparable output and to facilitate model analysis.
For this reason, the \ac{WGCM} of the \ac{WCRP} initiated the \acf{CMIP} in
1995, with the objective to \enquote{better understand past, present and future
  climate changes arising from natural, unforced variability or in response to
  changes in radiative forcing in a multi-model context} \autocite{WCRP2020}.
One major element of \ac{CMIP} is to establish common standards, coordination,
infrastructure, and documentation in order to facilitate the distribution of
climate model output \autocite{Eyring2016, Juckes2020}.

A further main aspect is to provide a set of standardized experiments for
global climate model simulations. To participate in the latest phase of
\ac{CMIP}, \acs{CMIP}6, climate models need to run a \emph{historical}
simulation of the period \range{1850}{2014} and the so-called \ac{DECK}
experiments, which include a pre-industrial control run (\emph{piControl}), a
historical Atmospheric \ac{MIP} simulation (\emph{amip}), a simulation forced
with an abrupt quadrupling of \ac{CO2} (\emph{abrupt-4xCO2}), and a simulation
forced with a $1 \unit{\%}$ per year increase of the atmospheric \ac{CO2}
concentration (\emph{1pctCO2}) \autocite{Eyring2016}. This is shown in the
center of \cref{fig:02:cmip6}, which illustrates the experimental design of
\acs{CMIP}6.

\begin{figure}[t]
  \centering
  \includegraphics[width=\FigureWidth{}]{
    ch02_scientific_background/figs/cmip6.pdf}
  \caption[
    Schematic of the experiment design of Phase 6 of the \acl{CMIP}
    (\acs{CMIP}6).
  ]{
    Schematic of the experiment design of Phase 6 of the \acl{CMIP}
    (\acs{CMIP}6). The center of the circle illustrates the four \acs{DECK}
    (\acl{DECK}) experiments and the \acs{CMIP}6 historical simulation. The
    circular sectors show additional science themes that can be explored
    through the 23 \acs{CMIP}6-Endorsed \acfp{MIP}.
    \SpringerAdaptedFrom{Simpkins2017}.
  }
  \label{fig:02:cmip6}
\end{figure}

To increase diversity and answer more scientific questions, \acs{CMIP}6 models
can participate in the so-called \acs{CMIP}6-Endorsed \acp{MIP}, of which
\acs{CMIP}6 offers 23 (see circular sectors in \cref{fig:02:cmip6}). Some
\acp{MIP} offer additional experiments to explore specific aspects of the Earth
system, like the \ac{C4MIP} which focuses on the carbon cycle
\autocite{Jones2016} or the \ac{AerChemMIP} which focuses on aerosol chemistry
\autocite{Collins2017}. Other \acp{MIP} allow the assessment of future climate
change. An example is the \ac{ScenarioMIP}, which provides common experiments
that simulate different possible futures \autocite{ONeill2016}. These
experiments are based on the so-called \acp{SSP}, a set of alternative pathways
of future societal development \autocite{ONeill2013, ONeill2017}. For each
experiment, a set of emissions and land use changes is calculated from the
\acp{SSP} \autocite{Riahi2017} which are then used to force the global climate
models. For \ac{ScenarioMIP}, five different \acp{SSP} are considered, ranging
from \acs{SSP}1 (sustainability) to \acs{SSP}5 (fossil fuel--based
development). Each \ac{SSP} is combined with a climate outcome (measured as
radiative forcing in the year 2100) based on a particular forcing pathway that
\acp{IAM} have shown to be feasible. For example, \acs{SSP}5-8.5 represents a
scenario based on a fossil fuel--based development with a radiative forcing of
$8.5 \unit{W m^{-2}}$ in 2100 while \acs{SSP}1-2.6 represents a sustainable
future with a radiative forcing of $2.6 \unit{W m^{-2}}$ in the year 2100. The
two scenarios in the main category of \ac{ScenarioMIP}, the \emph{Tier 1}
experiments, are the \acs{SSP}2-4.5 and \acs{SSP}3-7.0 scenarios. In contrast
to the \ac{ScenarioMIP} experiments, the corresponding \acs{CMIP}5 counterparts
\autocite{Taylor2012}, the so-called \acp{RCP}, only used the radiative forcing
in 2100 as only dimension to describe the possible futures (\eg{} \acs{RCP}8.5,
\acs{RCP}4.5, \acs{RCP}2.6, \etc{}).

In this thesis, climate model data from the two most recent \ac{CMIP}
generations is used, \acs{CMIP}5 and \acs{CMIP}6. More detailed information
about the specific variables and experiments analyzed is given in the
corresponding \namecrefs{ch:04:papers_ecs_tcr_assessment}.


\subsection{Sources of Uncertainties in Climate Model Projections}
\label{subsec:02:source_of_uncertainties}

Simulations from climate model ensembles of \ac{CMIP} allow us to assess future
climate change in a consistent and transparent way. Especially the
\ac{ScenarioMIP} experiments can give valuable insights into possible
developments of the Earth system by providing \emph{projections} of the future
climate. In contrast to climate predictions, climate projections run over
multiple decades and depend upon the future scenario considered, which are
based on assumptions that may or may not turn out to be correct. On the
contrary, climate predictions are attempts to predict the actual evolution of
the climate on much shorter time scales from seasons to years. Similar to any
other scientific experiment, climate model projections suffer from associated
uncertainties. There are three major sources of uncertainties in climate model
projections we can distinguish: natural variability, climate response
uncertainty, and emission uncertainty \autocite{Hawkins2009, Hawkins2010}.
\Cref{fig:02:sources_of_uncertainty} shows these three sources for the
projected global mean surface temperature anomaly over the \nth{21} century.

\begin{figure}[t]
  \centering
  \includegraphics[width=\LargeFigureWidth{}]{
    ch02_scientific_background/figs/sources_of_uncertainty.pdf}
  \caption[
    Schematic illustration of the importance of different sources of
    uncertainties in climate model projections and their evolution in time.
  ]{
    Schematic illustration of the importance of different sources of
    uncertainties in climate model projections and their evolution in time. (a)
    Time series of the anomaly of the decadal and global mean surface
    temperature relative to the period \range{1961}{1980}. The black line shows
    the historical observations with estimates of uncertainty from climate
    models (gray). The remaining colors show different sources of uncertainty
    in future climate projections: Natural variability (orange), climate
    response uncertainty (blue), and emission uncertainty (green)
    \autocite{Hawkins2009, Hawkins2010}. Climate response uncertainty can
    increase (b) in newer generations of climate models when a new process is
    discovered to be relevant or decrease (c) with additional model
    improvements and observational constraints.
    \IPCCReproducedFrom{Cubasch2013}{their FAQ 1.1, figure 1}.
  }
  \label{fig:02:sources_of_uncertainty}
\end{figure}

\emph{Natural variability} is connected to the chaotic nature of the Earth
system that arises from complex interactions between the ocean, atmosphere,
land, biosphere, and cryosphere \autocite{Cubasch2013}. It constitutes a
fundamental limit of how precisely we can project the future climate since it
is inherent in the Earth system and cannot be eliminated by more knowledge and
more advanced climate models. Natural variability is more relevant on regional
and local scales than on continental or global scales. Further contributions to
natural variability on longer time scales come from phenomena like the
\ac{ENSO} or the \ac{NAO} and from other events like volcanic eruptions and
variations in the solar activity. Natural variability can be seen as the
\enquote{noise} in the climate record as opposed to the anthropogenic
\enquote{signal} \autocite{Cubasch2013}.

The second source of uncertainty in climate model projections is \emph{emission
  uncertainty}. This arises from the different possible trajectories in terms
of future forcing (\acp{GHG}, aerosols, land use changes, \etc{}) humankind
might take. Examples for these are the \ac{SSP}-based experiments given by
\ac{ScenarioMIP} that include a variety of different scenarios from a
sustainable future to a full fossil fuel--based development (see
\cref{subsec:02:cmip}). A possible approach to quantify emission uncertainty is
to assess the climate impact of these different trajectories. Since the
emission uncertainty strongly depends on the future development of the human
society, it cannot be reduced by improving climate models. In contrast to
natural variability, the emission uncertainty increases over time in climate
projections, since estimating forcings for the near future is simpler than for
the far future.

Finally, the third source of uncertainty in climate model projections is the
\emph{climate response uncertainty}, which comes from our imperfect knowledge
of how the climate system will respond to anthropogenic forcings. Due to the
complexity of the Earth system, the future climate could develop in many
different ways that are all consistent with our current knowledge and models
\autocite{Cubasch2013}. In the context of climate model ensembles, the climate
response uncertainty is often also called \emph{model uncertainty} and reflects
the different responses of the different climate models to a given forcing.
Even though all climate models are built on the same physical principles, they
differ in terms of spatial resolution, processes included, and
parameterizations of unresolved processes (see
\cref{subsec:02:numerical_climate_modeling}).

These differences in the climate models also give rise to different intensities
of \emph{climate feedbacks} (or even their presence/absence) in the models. A
climate feedback is a mechanism that either amplifies (\emph{positive
  feedback}) or diminishes (\emph{negative feedback}) the effect of an external
forcing. An example of a strong positive feedback is the water vapor feedback,
in which the increased surface temperature (caused by anthropogenic forcing)
leads to an enhanced evaporation of water which increases the amount of water
vapor in the atmosphere. Since water vapor itself is a powerful \ac{GHG}, this
amplifies the effect of the anthropogenic forcing by further increasing the
surface temperatures \autocite{Cubasch2013}. Further examples and a
mathematical framework for the analysis of feedbacks are given in
\cref{subsec:02:climate_feedbacks}.

As sciences evolves, representations of already included processes can be
improved in climate models. Moreover, new geophysical and biogeochemical
processes can be added to them. On the one hand, this can increase the climate
response uncertainty when a new process is discovered to be relevant
\commentcite{Cubasch2013}{see
  \cref{fig:02:sources_of_uncertainty}{\color{BrownRed} b}}. However, such an
increase corresponds to a previously unmeasured uncertainty. An example for
this has recently happened in \acs{CMIP}6: most likely due to changes in the
cloud representation of the models the spread in the projected \ac{GSAT} caused
by a doubling of the atmospheric \ac{CO2} concentration has substantially
increased in \acs{CMIP}6 compared to older \ac{CMIP} generations
\autocite{Zelinka2020}. On the other hand, the climate response uncertainty can
decrease with additional model improvements and better understanding of the
Earth system (see \cref{fig:02:sources_of_uncertainty}{\color{BrownRed} c}).
Moreover, it can also be reduced by observational constraints, which is the
main topic of this thesis.


\section{Climate Sensitivity}
\label{sec:02:climate_sensitivity}

An important policy-relevant metric of the Earth system that can be assessed
with numerical climate model simulations is the climate sensitivity. Climate
sensitivity refers to the change in \ac{GSAT} that results from a change in the
radiative forcing. In other words, it describes how sensitive the climate
system is to an external forcing. The source of this forcing might either be
natural (changes in the solar activity, volcanic eruptions, \etc{}) or
anthropogenic (emissions of \acp{GHG}, land use changes, \etc{}). Thus, the
assessment of climate sensitivity is essential for a precise quantification of
the human-made climate change in order to determine optimal mitigation and
adaptation strategies.


\subsection{Climate Feedbacks}
\label{subsec:02:climate_feedbacks}

\begin{figure}[!t]
  \centering
  \includegraphics[width=\LargeFigureWidth{}]{
    ch02_scientific_background/figs/feedbacks_overview.pdf}
  \caption[
    Climate feedbacks and corresponding time scales.
  ]{
    Climate feedbacks and corresponding time scales. The different sings refer
    to the signs of the different feedbacks. Positive feedbacks amplify the
    effect of the external forcing (\eg{} the water vapor feedback) and
    negative feedbacks diminish the effect of the external forcing (\eg{} the
    longwave radiation feedback). An example for a feedback that might be
    either positive or negative is the cloud feedback. The smaller box
    highlights the large differences in time scales for the various feedbacks.
    \IPCCReproducedFrom{Cubasch2013}{their figure 1.2}.
  }
  \label{fig:02:feedbacks_overview}
\end{figure}

As already described in \cref{subsec:02:source_of_uncertainties}, the effects
of an external forcing acting on the climate system can additionally be
amplified or diminished by climate feedbacks. Thus, feedback processes play a
crucial role determining the magnitude of the climate sensitivity.
\Cref{fig:02:feedbacks_overview} shows an overview of important feedbacks in
the Earth system with their corresponding time scales on which they operate.

An example for a positive feedback is the already mentioned \emph{water vapor
  feedback}. Being the primary \ac{GHG} in the Earth's atmosphere, water vapor
is the largest contributor to the natural greenhouse effect. Since its amount
in the atmosphere is mainly controlled by the air temperature, and
anthropogenic emissions of water vapor are negligible, the influence of water
vapor on the climate system is described as a feedback mechanism and not as an
external forcing \autocite{Myhre2013}. Basis of this feedback is the enhanced
evaporation of water with increasing air temperatures. Each degree of warming
allows the atmosphere to retain about $7 \unit{\%}$ more water vapor
\autocite{Myhre2013}, which closes the positive feedback loop by further
increasing air temperatures through the greenhouse effect. With a typical
residence time of water vapor in the atmosphere of several days, the water
vapor feedback operates on relatively short time scales. As the largest
positive feedback in the Earth system \autocite{Soden2006}, the water vapor
feedback amplifies any initial forcing (\eg{} caused by anthropogenic \ac{CO2}
emissions) by a typical factor between $2$ and $3$, rendering water vapor a
fundamental agent of climate change \autocite{Myhre2013}. An example for a
positive feedback that operates on longer time scales (several years) is the
\emph{snow/ice albedo feedback}, in which the surface albedo decreases as
highly reflective ice and snow surfaces melt with global warming, exposing the
darker and more absorbing surfaces below \autocite{Cubasch2013}.

In contrast to positive feedbacks, negative feedbacks diminish the effect of an
external forcing. An example for this is the \emph{blackbody feedback} (also
known as \emph{Planck feedback} or \emph{longwave radiation feedback}), which
is the strongest negative feedback in the Earth system \autocite{Cubasch2013}.
It is based on the thermal electromagnetic radiation that any object with a
non-zero temperature emits (the so-called \emph{blackbody radiation}). Since
the power of this radiation strongly depends on the temperature of the object,
higher surface temperatures of the Earth increase the outgoing longwave
radiation flux from the surface which reduces the effect of the external
forcing and cools the planet.

For some domains of the Earth system, feedbacks can be positive and/or
negative, since a variety of different mechanisms is involved. An example for
this is the cloud feedback. Changes in the clouds induced by climate change can
cause changes in their longwave (greenhouse warming) and shortwave (reflective
cooling) effects on the Earth's radiation budget, which both need to be
considered for the overall cloud feedback \autocite{Boucher2013}. Relevant
cloud properties that may change as a response to an external forcing and that
may alter the Earth's radiative budget are cloud cover, cloud optical
thickness, cloud top and cloud base height, vertical extent, and the
geographical distribution of clouds. Examples for robust cloud feedback
processes are the increase in cloud top height of high-level clouds in a
warming climate which traps longwave radiation and enhances global warming, and
the reduction in mid- and low-level cloud cover which diminishes the reflection
of incoming solar radiation and also increases the surface warming
\autocite{Boucher2013}. In global climate model ensembles, the overall cloud
feedback shows a large range with positive and negative values, but tends to be
slightly positive on average \autocite{Soden2006, Dufresne2008, Vial2013,
  Zelinka2020}. This large uncertainty in the cloud feedback is a major reason
for uncertainties in the climate sensitivity of climate models
\autocite{Boucher2013, Flato2013}.

Further examples of feedbacks with positive and negative contributions are
biogeochemical feedbacks. Negative contributions come from increased \ac{CO2}
fluxes into the land and ocean carbon reservoirs due to increased
photosynthesis rates and \ac{CO2} dissolution in the sea, respectively, which
decrease the atmospheric \ac{CO2} content and diminish global warming. An
example for a positive contribution is the decreased solubility of \ac{CO2} in
water in a warmer climate, which reduces the atmosphere-ocean \ac{CO2} flux and
enhances climate change. More details on this are given in
\cref{subsec:02:carbon_cycle_perturbations}.


\subsection{Mathematical Framework for Feedbacks Analysis}
\label{subsec:02:mathematical_framework_feedbacks}

For a precise quantification of climate feedbacks and thus climate sensitivity,
a mathematical framework for climate feedback analysis necessary. One possible
approach for this is based on a simple energy balance model
\autocite{Gregory2009, Roe2009}. Anthropogenic activities in the Earth system
like the emissions of \acp{GHG} or aerosols introduce an external forcing to
the climate system, which is quantified with a radiative forcing $F$ measured
in $\si{W m^{-2}}$. To restore a stable state, the climate system opposes this
forcing with a climate response $R$, leading to a net energy flux of
\begin{equation}
  N = F + R
  \label{eq:02:N}
\end{equation}
into the system. Positive values of $N$, $F$, and $R$ indicate incoming fluxes;
usually $F > 0$ and $R < 0$. On long time scales (multiple years), the net
incoming radiative flux at the \ac{TOA} and the net heat flux into the ocean
are basically equal definitions of $N$ since nearly all of the Earth's heat
capacity resides in the ocean \autocite{Gregory2009}. While $N \neq 0$, the
climate system evolves; when $N = 0$ a new steady state has been reached.

To quantify the effects of different feedbacks, a reference system with a basic
response needs to be defined, which is a crucial aspect of feedback analysis
\autocite{Roe2009}. Usually, the idealization of a blackbody Earth without an
atmosphere is used for that: In equilibrium, the incoming solar irradiance is
balanced with an outgoing thermal irradiance $J_0$ that solely depends on the
global mean surface temperature $T_0$ following the Stefan-Boltzmann law
\begin{equation}
  J_0 = -\sigma T_0^4.
  \label{eq:02:stefan_boltzmann_law_eq}
\end{equation}
$\sigma \approx 5.67 \unit{W m^{-2} K^{-4}}$ is the Stefan-Boltzmann constant.
To answer an external forcing $F$, the climate system reacts with a response
$R$ expressed by a change in the global mean surface temperature $\Delta T$:
\begin{equation}
  J_0 + R = -\sigma \left( T_0 + \Delta T \right)^4.
  \label{eq:02:stefan_boltzmann_law_non_eq}
\end{equation}
Since the temperature change caused by an anthropogenic forcing is much smaller
than the equilibrium temperature $\Delta T \ll T_0 \approx 255 \unit{K}$, a
simple first-order Taylor expansion can be used to linearize the blackbody
response:
\begin{equation}
  -\sigma \left( T_0 + \Delta T \right)^4 \approx J_0 - 4 \sigma T_0^3 \cdot
  \Delta T.
  \label{eq:02:stefan_boltzmann_law_lin}
\end{equation}
Thus, by comparing
\cref{eq:02:stefan_boltzmann_law_non_eq,eq:02:stefan_boltzmann_law_lin} the
climate response $R$ can be expressed as
\begin{equation}
  R = -4 \sigma T_0^3 \cdot \Delta T \coloneq \lambda_\mathtxt{BB} \cdot
  \Delta T
  \label{eq:02:linear_blackbody_response}
\end{equation}
with the blackbody feedback parameter $\lambda_\mathtxt{BB} \approx -3.8
\unit{W m^{-2} K^{-1}}$. Results from climate models and analyses of
observations confirm this linear relationship between $R$ and $\Delta T$
\autocite{Gregory2004}. However, the value of this linear constant $\lambda$,
the \emph{climate feedback parameter}, is found to be considerably larger than
the blackbody response ($\lambda \approx -1.0 \unit{W m^{-2} K^{-1}}$),
indicating that additional processes affect the Earth's radiative balance: the
climate feedbacks \autocite{Flato2013, Gregory2009}. Since climate models
suggest that the radiative effects of these additional feedbacks are also
proportional to $\Delta T$ \autocite{Gregory2008a},
\cref{eq:02:linear_blackbody_response} can be adjusted to
\begin{equation}
  R = \lambda \cdot \Delta T = \left( \lambda_\mathtxt{BB} +
  \lambda_\mathtxt{WV} + \lambda_\mathtxt{Albedo} + \lambda_\mathtxt{Cloud} +
  \ldots \right) \cdot \Delta T.
  \label{eq:02:linear_response}
\end{equation}
$\lambda_\mathtxt{WV}$ refers to the water vapor feedback,
$\lambda_\mathtxt{Albedo}$ to the snow/ice albedo feedback, and
$\lambda_\mathtxt{Cloud}$ to the cloud feedback. Thus, the overall climate
feedback parameter $\lambda$ can be written as the sum of the individual
feedback parameters $\lambda_i$:
\begin{equation}
  \lambda = \sum_i \lambda_i.
  \label{eq:02:lambda_as_sum_of_lambdas}
\end{equation}
Positive values of $\lambda_i$ indicate positive feedbacks (\eg{} the water
vapor feedback) and negative values indicate negative feedbacks (\eg{} the
blackbody feedback). This equation assumes that the individual radiative
responses from the different feedbacks are independent, which is a reasonable
first-order approximation but not entirely true \autocite{Soden2008}.

\subsection{Equilibrium and Effective Climate Sensitivity}
\label{subsec:02:ecs}

An important metric describing climate sensitivity is the \emph{equilibrium
  climate sensitivity}. It is defined as the change in \ac{GSAT} after an
instantaneous doubling of the atmospheric \ac{CO2} concentration from
pre-industrial conditions once the climate system reaches radiative equilibrium
\autocite{Bindoff2013}. Being already used in one of the first assessments of
the anthropogenic climate change, the \emph{Charney Report} from 1979
\autocite{Charney1979}, the equilibrium climate sensitivity is one of the
oldest metrics describing climate change. However, in practice this traditional
definition of is not always useful. Running fully-coupled \acp{ESM} into
equilibrium is computationally expensive as it would require thousands of model
years \autocite{Rugenstein2020}.

\begin{figure}[t]
  \centering
  \includegraphics[width=\FigureWidth{}]{
    ch02_scientific_background/figs/gregory_regression.pdf}
  \caption[
    Gregory regression for the \acs{CMIP}5 \acl{MMM} illustrating the
    definition of the \acf{ECS}.
  ]{
    Global and annual mean net \acf{TOA} radiation $N$ versus the change in
    global and annual mean \acl{T} $\Delta T$ for 150 years of a simulation
    with an abrupt quadrupling of the atmospheric \acs{CO2} concentration
    (\nxcotwo{4}) for the \acs{CMIP}5 \acl{MMM} (blue circles). To account for
    energy leakage and model drift, a linear fit of the corresponding
    pre-industrial control run is subtracted from the \nxcotwo{4} simulation.
    As given by \cref{eq:02:N_vs_delta_t}, the slope of the linear regression
    (black line) corresponds to the climate feedback parameter $\lambda$, and
    the \yintercept{} corresponds to the radiative forcing $F_\mathtxt{4x}$.
    These can be used to calculate the \acf{ECS} with the Gregory regression
    method according to \cref{eq:02:ecs} \autocite{Gregory2004}. \acs{ECS} is
    equivalently given by the \xintercept{} of the linear regression line
    divided by $2$. Here, $\mathtxt{ECS} = 3.14 \unit{K}$.
  }
  \label{fig:02:gregory_regression}
\end{figure}

For this reason, the equilibrium climate sensitivity is commonly approximated
with the \emph{\acf{ECS}}, which can be derived from only 150 model years of a
simulation with an abrupt quadrupling of the atmospheric \ac{CO2} concentration
(\nxcotwo{4}) \autocite{Gregory2004}. The basis of the definition of \ac{ECS}
is the simple energy balance model introduced in
\cref{subsec:02:mathematical_framework_feedbacks}. Assuming radiative
equilibrium ($N = 0$), \cref{eq:02:N,eq:02:linear_response} imply
\begin{equation}
  \Delta T = -\frac{F}{\lambda}.
  \label{eq:02:delta_t_eq}
\end{equation} Thus, the change in \ac{GSAT} in radiative equilibrium can be
easily calculated from the external forcing $F$ and the climate feedback
parameter $\lambda$. The steady state values for $F$ and $\lambda$ can be
estimated from a \nxcotwo{4} simulation that is not in equilibrium by
extrapolation with a linear regression \autocite{Gregory2004}. For this
so-called \emph{Gregory regression} (see \cref{fig:02:gregory_regression}), the
global and annual mean net \ac{TOA} radiation $N$ versus the change in the
annual mean \ac{GSAT} $\Delta T$ for all 150 years of the \nxcotwo{4} run are
plotted. To account for energy leakage and remove any model drift present in
the control climate, a linear fit of the corresponding pre-industrial control
run is subtracted from the \nxcotwo{4} simulation beforehand
\autocite{Andrews2012}. Since the combination of
\cref{eq:02:N,eq:02:linear_response} yields
\begin{equation}
  N = F + \lambda \cdot \Delta T,
  \label{eq:02:N_vs_delta_t}
\end{equation}
$F$ is now given by the \yintercept{} of this linear regression
($F_\mathtxt{4x}$) and $\lambda$ by its slope. Thus, the \ac{ECS} is given by
\begin{equation}
  \mathtxt{ECS} = -\frac{F_\mathtxt{4x}}{2 \lambda}.
  \label{eq:02:ecs}
\end{equation}
The factor of $2$ in the denominator accounts for the fact the traditional
equilibrium climate sensitivity is defined for an abrupt \ac{CO2} doubling,
whereas here a simulation with an abrupt quadrupling is considered.

\begin{figure}[t]
  \centering
  \includegraphics[width=\FigureWidth{}]{
    ch02_scientific_background/figs/gregory_regression_different_years.png}
  \caption[
  Gregory regression for different time periods.
  ]{
    As in \cref{fig:02:gregory_regression} but for different time periods
    considered in the Gregory regression resulting in different values for the
    \acf{ECS} and the climate feedback parameter $\lambda$. Dark gray, blue,
    red, and orange colors correspond to the \acf{CESM} of the \acf{NCAR} for
    different time periods of a long running simulation (see legend). The
    initial years are simulated many times for different initial conditions.
    Light gray colors correspond to the \acs{CMIP}5 ensemble (150 years each).
    \SpringerAdaptedFrom{Knutti2017}.
  }
  \label{fig:02:gregory_regression_different_years}
\end{figure}

Although commonly used in the literature, the \ac{ECS} is known to be only an
approximation of the true equilibrium climate sensitivity. One major reason for
this is the state and time dependence of the global feedbacks
\autocite{Knutti2015, Knutti2017}. As
\cref{fig:02:gregory_regression_different_years} shows, the slope in the
Gregory regression is not constant, but rather decreases over time when a long
running \nxcotwo{4} simulation with more than 1000 model years is considered.
As a result, the climate feedback parameter $\lambda$ decreases over time,
resulting in a higher \ac{ECS}. Major reasons for this are temperature
dependencies of the feedbacks, atmospheric and oceanic adjustments over time,
changing warming patterns over time, non-additive feedbacks, and dependencies
on the type and magnitude of the external forcings \autocite{Knutti2017}. All
in all, this demonstrates the limits of the linear feedback framework
introduced in \cref{subsec:02:mathematical_framework_feedbacks} that is not
capable of describing non-linear effects. A second major reason for the
discrepancies between the equilibrium climate sensitivity and \ac{ECS} is the
use of a \nxcotwo{4} instead of a \nxcotwo{2} simulation. The factor of $2$ in
the denominator of \cref{eq:02:ecs} only partly compensates this difference
since the radiative forcing logarithmically depends on the atmospheric \ac{CO2}
concentration \autocite{Huang2014}.

However, despite these deficiencies the \ac{ECS} is still a practical estimate
of the equilibrium climate sensitivity. With the help of climate models,
\textcite{Sherwood2020} showed that the \ac{ECS} is only about $6 \unit{\%}$
lower than the best estimate of the true equilibrium warming obtained from
integrating climate models until a new steady state is reached. Nevertheless,
for \acs{CMIP}6 long running simulations from the \ac{LongRunMIP}
\autocite{Rugenstein2019} can be a promising way forward to estimate the true
equilibrium climate sensitivity for \acp{ESM}.


\subsection{Cloud-Related Feedback Parameters}
\label{subsec:02:cloud_feedback_parameters}

In addition to the calculation of the external forcing $F$, the overall climate
feedback parameter $\lambda$ and the \ac{ECS}, the Gregory regression can also
be used to estimate cloud-related feedback parameters. For this, the net
\ac{TOA} radiation $N$ on the \yaxis{} in \cref{fig:02:gregory_regression} is
replaced with the \ac{CRE}, which is defined as the difference between the
all-sky (\ie{} with clouds if present) net \ac{TOA} radiation and the clear-sky
(\ie{} clouds artificially removed) net \ac{TOA} radiation
\autocite{Andrews2012}. This can be done for the shortwave $N_\mathtxt{SWCRE}$
and longwave $N_\mathtxt{LWCRE}$ components separately, but also for the
combined effect $N_\mathtxt{CRE} = N_\mathtxt{SWCRE} + N_\mathtxt{LWCRE}$. The
slopes in the corresponding Gregory regressions are the so-called
\emph{\ac{CRE} feedback parameters} $\lambda_\mathtxt{SWCRE}$,
$\lambda_\mathtxt{LWCRE}$, and $\lambda_\mathtxt{CRE}$, which quantify the
change in \ac{CRE} as a response to increasing \acp{GSAT}.


\subsection{Transient Climate Response}
\label{subsec:02:tcr}

A further metric describing the climate sensitivity is the \emph{\ac{TCR}}. In
contrast to the \ac{ECS}, this metric does not assume radiative equilibrium of
the Earth system but describes the transient response of an evolving climate.
Following \textcite{Bindoff2013}, \ac{TCR} is defined as the change in the
\ac{GSAT} at the time of \ac{CO2} doubling in a simulation with a $1 \unit{\%}$
per year increase of the atmospheric \ac{CO2} concentration (\onepctcotwo{}).
For this, the annual mean \acp{GSAT} are averaged over a 20-year period
centered at the time of the \ac{CO2} doubling (years \range{61}{80} when the
first year is indexed with $1$). To account for model drift, the annual mean
changes in \ac{GSAT} are calculated relative to a corresponding pre-industrial
control simulation smoothed with a linear fit that considers 140 model years
(length of the \onepctcotwo{} simulation). An illustration of that calculation
is shown in \cref{fig:02:tcr}.

\begin{figure}[t]
  \centering
  \includegraphics[width=\FigureWidth{}]{
    ch02_scientific_background/figs/tcr_calculation.pdf}
  \caption[
    Illustration of the definition of the \acf{TCR}.
  ]{
    Global and annual mean \acl{T} change $\Delta T$ for 140 years of a
    simulation with a $1 \unit{\%}$ per year increase of the atmospheric
    \acs{CO2} concentration (\onepctcotwo{}) for the \acf{CESM} of the
    \acf{NCAR} (blue circles). The temperature change is calculated relative to
    a corresponding pre-industrial control simulation smoothed with a linear
    fit over all 140 years. The \acf{TCR} is defined as the temperature change
    $\Delta T$ at the time of \acs{CO2} doubling averaged over a 20-year period
    (illustrated by the vertical dashed lines). Here, $\mathtxt{TCR} = 2.35
    \unit{K}$ (horizontal dashed line).
  }
  \label{fig:02:tcr}
\end{figure}

Similar to \ac{ECS}, \ac{TCR} can also be defined in terms of an external
forcing and climate feedbacks. For this, the energy balance
\cref{eq:02:N_vs_delta_t} can be slightly adjusted. Since over $90 \unit{\%}$
of the excess energy introduced into the climate system by the radiative
forcing $F$ is taken up by the ocean due to its large heat capacity, $N$ can be
taken equal to the global ocean heat uptake \autocite{Knutti2017}. In
experiments with a steadily increasing radiative forcing, which is the case for
the \onepctcotwo{} simulation, this ocean heat uptake can be approximated with
\begin{equation}
  N = \kappa \cdot \Delta T,
  \label{eq:02:ocean_heat_uptake}
\end{equation}
where $\kappa$ is the ocean heat uptake efficiency \autocite{Gregory2008}.
Since there is a net energy flux into the climate system ($N > 0$) due to the
external forcing $F$, $\kappa$ is positive. This approximation becomes less
accurate as the deeper ocean warms up and cannot be applied to simulations with
a steady state climate change in which $N \to 0$ \autocite{Gregory2009}. By
applying the definition of \ac{TCR} (transient \ac{GSAT} change $\Delta T$ at
the time of \ac{CO2} doubling) and combining
\cref{eq:02:N_vs_delta_t,eq:02:ocean_heat_uptake}, \ac{TCR} can be estimated as
\begin{equation}
  \mathtxt{TCR} = \frac{F_\mathtxt{2x}}{\kappa - \lambda},
  \label{eq:02:tcr}
\end{equation}
where $F_\mathtxt{2x}$ is the radiative forcing induced by a doubling of the
atmospheric \ac{CO2} concentration.

This equation can be used to derive a relationship between \ac{TCR} and
\ac{ECS}. Writing $\mathtxt{ECS} = -F_\mathtxt{2x} / \lambda$ and assuming path
independence of the forcing (\ie{} the resulting radiative forcing from the
\onepctcotwo{} and \nxcotwo{2} runs are comparable) gives
\begin{equation}
  \mathtxt{TCR} = \frac{F_\mathtxt{2x} \cdot \mathtxt{ECS}}{F_\mathtxt{2x} +
    \kappa \cdot \mathtxt{ECS}},
  \label{eq:02:tcr_vs_ecs}
\end{equation}
which demonstrates the non-linear connection between \ac{TCR} and \ac{ECS}
\autocite{Gregory2008, Nijsse2020}. Since $\lambda < 0$ and $\kappa > 0$,
\cref{eq:02:tcr} implies that the equilibrium response \ac{ECS} is (as
expected) larger than transient response \ac{TCR}, \ie{} $\mathtxt{ECS} >
\mathtxt{TCR}$.


\section{The Global Carbon Cycle}
\label{sec:02:carbon_cycle}

Currently, only about $50 \unit{\%}$ of the emitted \ac{CO2} by humankind
remains in the atmosphere \autocite{Friedlingstein2020}. The residual part is
absorbed by the two other main carbon sinks of the planet, the terrestrial
biosphere and the ocean. These exchange fluxes of carbon are part of the global
carbon cycle of planet Earth. Since only atmospheric \ac{CO2} can act as
\ac{GHG}, this removal of \ac{CO2} from the atmosphere substantially slows down
global warming. Moreover, since the emission of carbon-based \acp{GHG}
introduces additional carbon into the Earth system, the carbon exchange fluxes
might change under global warming, which in turn directly influences the
magnitude of the climate sensitivity. However, the two idealized climate
sensitivity metrics \ac{ECS} and \ac{TCR} introduced in the previous
\namecref{sec:02:climate_sensitivity} are by definition independent from carbon
cycle--related effects, since both of them are defined in terms of a doubling
of the atmospheric \ac{CO2} concentration, not in terms of \ac{CO2} emissions.
Nevertheless, in order to assess policy-relevant metrics like the allowable
fossil fuel emissions to meet particular warming targets, for example the $1.5
\unit{\degreeCelsius}$ of the Paris Agreement \autocite{UNFCCC2015}, it is
crucial to take the global carbon cycle into account. This
\namecref{sec:02:carbon_cycle} introduces the scientific background of the
global carbon cycle and its current anthropogenic perturbations.


\subsection{Overview}
\label{subsec:02:carbon_cycle_overview}

A schematic overview of the global carbon cycle is shown in
\cref{fig:02:carbon_cycle_schematic}. To quantify the carbon cycle, common
units are \ac{ppm} for the atmospheric trace gas concentrations (dry-air mole
fraction) and \ac{GtC} or $\si{\GtCyr{}}$ for the reservoirs masses or exchange
fluxes, respectively. The carbon exchange processes between the different
carbon reservoirs run on a wide range of time scales. Conceptually, one can
distinguish between two domains of the global carbon cycle: a slow and a fast
domain. The slow domain with turnover times (reservoir mass of carbon divided
by exchange flux) of more than 10000 years consists of the large carbon stores
in rocks and sediments which are connected to the rapid domain of the carbon
cycle through volcanic emissions of \ac{CO2}, chemical weathering, erosion, and
sediment formation on the sea floor. These natural exchange fluxes between the
slow and the fast domain are comparatively small ($< 0.3 \unit{\GtCyr{}}$) and
can be assumed as approximately constant in time over the last few centuries
\autocite{Ciais2013}.

The fast domain of the global carbon cycle consists of three main carbon
reservoirs: the atmosphere, the terrestrial biosphere, and the ocean. In the
atmosphere, carbon is mainly stored in trace gases, with \ac{CO2} as the major
component with a current (2019) concentration of about $410 \unit{ppm}$
\autocite{Friedlingstein2020}. Additional contributors to the atmospheric
carbon content are the trace gas \ac{CH4}, the trace gas \ac{CO}, hydrocarbons,
black carbon aerosols, and organic compounds \autocite{Ciais2013}. Carbon in
the terrestrial biosphere is mainly stored as organic compounds, with about
$\rangeunit{450}{650}{GtC}$ in the living vegetation biomass,
$\rangeunit{1500}{2400}{GtC}$ in dead organic matter in litter and soils, and
about $1700 \unit{GtC}$ in permafrost soils \autocite{Ciais2013}. The main
component of the oceanic carbon reservoir is dissolved inorganic carbon
(carbonic acid, bicarbonate ions, and carbonate ions) with about $38000
\unit{GtC}$. Further carbon is stored as dissolved organic carbon (about $700
\unit{GtC}$), in surface sediments (about $1750 \unit{GtC}$), and in marine
biota (about $3 \unit{GtC}$, predominantly phytoplankton and other
microorganisms) \autocite{Ciais2013, Friedlingstein2020}.

\begin{figure}[t]
  \centering
  \includegraphics[width=\FigureWidth{}]{
    ch02_scientific_background/figs/carbon_cycle_schematic.png}
  \caption[
    Simplified schematic of the global carbon cycle.
  ]{
    Simplified schematic of the global carbon cycle including the typical
    turnover time scales for carbon transfers through the major reservoirs
    (atmosphere, land surface, and ocean). \IPCCReproducedFrom{Ciais2013}{their
      FAQ 6.2, figure 1}.
  }
  \label{fig:02:carbon_cycle_schematic}
\end{figure}

In the fast domain of the global carbon cycle, reservoir turnover times range
from seconds to millennia. In contrast to the slow domain, the carbon exchange
fluxes within the fast domain of the carbon cycle are much higher. One major
group of exchange processes in the fast domain connects the atmosphere and the
terrestrial biosphere. \Ac{CO2} is removed from the atmosphere by plant
photosynthesis with about $120 \unit{\GtCyr{}}$ \autocite{Ciais2013}. This
process is also known as \emph{\ac{GPP}}. The carbon fixed into plants can be
released back into the atmosphere by autotrophic (plant) and heterotrophic
(soil microbial and animal) respiration and additional disturbance processes
like fires \autocite{Ciais2013}. Since the land \ac{CO2} uptake by
photosynthesis occurs only during the growing season, whereas respiration
occurs nearly all year, the larger amount of vegetation in the Northern
Hemisphere (due to the larger land mass) gives rise to a seasonal cycle of the
atmospheric \ac{CO2} concentration \autocite{Keeling1995}. This seasonal cycle
reflects the phase of the global carbon cycle and shows a maximum of the
atmospheric \ac{CO2} concentration in the Northern Hemisphere winter (net
\ac{CO2} flux into atmosphere due to respiration) and a minimum during the
Northern Hemisphere summer (net \ac{CO2} flux into the land due to
photosynthesis). Another major carbon exchange process connects the atmosphere
and the ocean. Atmospheric \ac{CO2} is exchanged with the surface ocean through
gas exchange, which is driven by the partial \ac{CO2} pressure difference
between the air and the sea \autocite{Ciais2013}.


\subsection{Anthropogenic Perturbations}
\label{subsec:02:carbon_cycle_perturbations}

Before the Industrial Era, the global carbon cycle was roughly in a dynamic
equilibrium, which means that exchange fluxes balanced each other and the
amount of carbon in the different reservoirs did neither increase nor decrease.
This can be inferred from ice core measurements, which show an almost constant
atmospheric \ac{CO2} concentration over the last several thousand years before
the Industrial Revolution in the \nth{19} century \autocite{Ciais2013}. Since
the beginning of the Industrial Era, humanity is constantly emitting
carbon-based \acp{GHG} (\eg{} \ac{CO2} and \ac{CH4}) into the atmosphere.
Especially the atmospheric \ac{CO2} concentration has substantially increased,
which has already been shown by Charles D. Keeling in 1976 by his continuous
\ac{CO2} measurements at Mauna Loa, Hawaii that started in 1958
\commentcite{Keeling1976}{see \cref{fig:02:keeling_curve}}. From 1958, the
atmospheric \ac{CO2} concentration at Mauna Loa has steadily increased by about
$100 \unit{ppm}$ to $410 \unit{ppm}$ in the year 2019 \autocite{Keeling2005}.
In addition to the steady increase, the so-called \emph{Keeling Curve} is
further superimposed with the seasonal \ac{CO2} cycle, which gives rise to
local maxima of the atmospheric \ac{CO2} concentration in the Northern
Hemisphere winter and local minima in the Northern Hemisphere summer
\commentcite{Keeling1995}{see \cref{subsec:02:carbon_cycle_overview}}. Due to
its location in the middle of the Pacific Ocean, the Mauna Loa Observatory
offers perfect conditions for \ac{CO2} measurements by being far away from big
population centers. Moreover, its elevation of more than $3000 \unit{m}$
provides access to the free troposphere where \ac{CO2} is well mixed, which
prevents any interference from the vegetation present on the Hawaiian Islands.

\begin{figure}[t]
  \centering
  \includegraphics[width=\EvenLargerFigureWidth{}]{
    ch02_scientific_background/figs/keeling_curve.pdf}
  \caption[
    The Keeling Curve.
  ]{
    The Keeling Curve: monthly mean atmospheric \acs{CO2} concentration at the
    Mauna Loa Observatory, Hawaii ($19.5 \unit{\degree N}$, $155.6
    \unit{\degree W}$; elevation: $3397 \unit{m}$) from 1958 to 2019
    \autocite{Keeling2005}. The steady increase of the atmospheric \acs{CO2}
    concentration is superimposed with a pronounced seasonal oscillation caused
    by the seasonal \acs{CO2} cycle (see
    \cref{subsec:02:carbon_cycle_overview}).
  }
  \label{fig:02:keeling_curve}
\end{figure}

Apart from warming the Earth by altering its radiation budget, the
anthropogenically emitted \ac{CO2} directly influences the carbon exchange
fluxes of the global carbon cycle. Due to the excess carbon in the atmosphere,
there is now a net carbon flux from the atmosphere into the land and ocean
reservoirs (see \cref{fig:02:carbon_cycle_perturbation}). Thus, the carbon
cycle is not in a steady state anymore. In the decade \range{2010}{2019},
anthropogenic activities caused net carbon fluxes of $3.4 \unit{\GtCyr{}}$ from
the atmosphere into the terrestrial biosphere due to increased plant
photosynthesis and $2.5 \unit{\GtCyr{}}$ from the atmosphere into the ocean due
to increased dissolution of \ac{CO2} into the sea
\autocite{Friedlingstein2020}. In the same time, the amount of carbon in the
atmosphere reservoir increased with a rate of $5.1 \unit{\GtCyr{}}$, indicating
that only about half of the anthropogenic \ac{CO2} emissions in the last decade
remained in the atmosphere \autocite{Friedlingstein2020} where they can act as
\ac{GHG}.

\begin{figure}[t]
  \centering
  \includegraphics[width=\LargeFigureWidth{}]{
    ch02_scientific_background/figs/carbon_cycle_perturbation.png}
  \caption[
    Schematic representation of the overall perturbation of the global carbon
    cycle caused by anthropogenic activities.
  ]{
    Schematic representation of the overall perturbation of the global carbon
    cycle caused by anthropogenic activities, averaged globally for the decade
    \range{2010}{2019}. Arrows represent carbon exchange fluxes; circles carbon
    reservoirs. More details are given in the legend of this figure.
    \AdaptedFrom{Friedlingstein2020}.
  }
  \label{fig:02:carbon_cycle_perturbation}
\end{figure}

Thus, this removal of \ac{CO2} from the atmosphere actively slows down global
warming. However, whether this benefit will persist in the future remains
unclear, which is primarily linked to two feedback processes connecting the
physical climate system and the global carbon cycle: the
\emph{concentration-carbon feedback} and the \emph{climate-carbon feedback}
\autocite{Friedlingstein2006, Gregory2009, Collins2013}. For the terrestrial
biosphere, the concentration-carbon feedback is connected to the \emph{\ac{CO2}
  fertilization effect} \autocite{Walker2020}, that causes an increase of
photosynthesis rates when the atmospheric \ac{CO2} concentration increases,
which in turns removes \ac{CO2} from the atmosphere, forming a negative
feedback. For the ocean, the concentration-carbon feedback is negative as well.
In this case, an elevated atmospheric \ac{CO2} concentration causes an
increased dissolution of \ac{CO2} into the sea, which increases the ocean
carbon uptake. On the other hand, the climate-carbon feedback is thought to be
positive for both the terrestrial biosphere and the ocean
\autocite{Gregory2009}. In the first case, temperature and \acl{PR} changes due
to anthropogenic activities decrease the land carbon uptake because of
increased temperature and water stress on photosynthesis and higher ecosystem
respiration costs, which accelerates global warming due to more \ac{CO2} that
remains in the atmosphere. For the ocean, increased temperatures lead to a
reduction of vertical transport in the ocean resulting from increased stability
and reduced solubility of \ac{CO2} in the sea, which reduces the ocean carbon
uptake and enhances climate change \autocite{Gregory2009}.


\subsection{Representation in Earth System Models}
\label{subsec:02:carbon_cycle_representation}

In modern \acp{ESM}, the carbon cycle is usually represented by a land carbon
cycle model and an ocean carbon cycle model that are both coupled to the other
components of the \ac{ESM}. An overarching principle is the conservation of the
total carbon mass in the Earth system \autocite{Gregory2009}, \ie{}
\begin{equation}
  \CE{} = \CA{} + \CL{} + \CO{}.
  \label{eq:02:carbon_balance}
\end{equation}
Here, $\CE{}$ represents the cumulative anthropogenic carbon emissions that are
distributed among the three main carbon reservoirs of the Earth system.
$\CA{}$, $\CL{}$, and $\CO{}$ describe the corresponding changes in these
carbon stores, namely the atmosphere, the terrestrial biosphere (land), and the
ocean, respectively.

Similar to many cloud processes, most carbon cycle--related processes need to
be parameterized in \acp{ESM} since they occur on spatial scales much smaller
than the typical grid cell sizes (see
\cref{subsec:02:numerical_climate_modeling}). For the terrestrial carbon cycle,
the major carbon exchange processes that need to be considered are
photosynthesis, respiration, and disturbances, which connect the atmospheric
carbon reservoir and the land carbon reservoir. The latter mainly consists of
the vegetation biomass, litter, and soil. The terrestrial carbon exchange
processes are commonly described with \ac{GPP}, \ac{NPP}, \ac{NEP}, and
\ac{NBP}, which are all illustrated in \cref{fig:02:carbon_cycle_processes}.
\Ac{GPP} describes the total carbon uptake of the vegetation by photosynthesis,
while \ac{NPP} refers to the net carbon gain of the plants after accounting for
autotrophic respiration ($\Ra{}$). \Ac{NEP} additionally considers
heterotrophic respiration ($\Rh{}$) and describes the carbon uptake/release of
the entire ecosystem. Finally, \ac{NBP} describes changes in the long-term
carbon storage of the terrestrial biosphere by including disturbances like
fires ($\Ld{}$).

The different land carbon cycle models integrated into modern \acp{ESM} use a
variety of different parameterizations for the different processes
\autocite{Arora2020}. Examples for the photosynthesis (\ac{GPP}) are given by
\textcite{Farquhar1980} for \ch{C3} plants and \textcite{Collatz1992} for
\ch{C4} plants. The terms \enquote{\ch{C3}} and \enquote{\ch{C4}} refer to the
different metabolic pathways used in the carbon fixation during photosynthesis.
In both cases, the enzyme \ac{RuBisCO} is an important catalyst for the
corresponding chemical reactions. Both parameterizations utilize variables that
are known to affect photosynthesis, such as the availability of light, the
\ac{CO2} concentration, the soil moisture, and the temperature.

\begin{figure}[t]
  \centering
  \includegraphics[width=\LargeFigureWidth{}]{
    ch02_scientific_background/figs/carbon_cycle_processes.pdf}
  \caption[
  Schematic illustration of the processes of the terrestrial carbon uptake.
  ]{
    Schematic illustration of the processes of the terrestrial carbon uptake.
    The \acf{GPP} describes the total carbon uptake of the vegetation by
    photosynthesis, while the \acf{NPP} refers to the net carbon gain of the
    plants after accounting for autotrophic respiration ($\Ra{}$). The
    \acf{NEP} additionally considers heterotrophic respiration ($\Rh{}$) and
    describes the carbon uptake/release of the entire ecosystem. Finally, the
    \acf{NBP} describes changes in the long-term carbon storage of the
    terrestrial biosphere by including disturbance losses ($\Ld{}$).
  }
  \label{fig:02:carbon_cycle_processes}
\end{figure}


\section{Reducing Uncertainties in Multi-Model Climate Projections
  with Observations}
\label{sec:02:reducing_uncertainties}

As shown in \cref{subsec:02:source_of_uncertainties}, projections of the future
climate are always associated with uncertainties. In the context of this
thesis, the most relevant source of uncertainty is the climate response
uncertainty. It originates from necessary simplifications that have to be
implemented into the climate models due to limited computational resources and
from our imperfect knowledge on how the climate system will respond to external
forcing. In a multi-model ensemble (\eg{} from \ac{CMIP}), the climate response
uncertainty is expressed in the different responses of the different climate
models to a given forcing. A common approach to distill information about a
projected quantity from multi-model ensembles is to treat the arithmetic
\ac{MMM} and the multi-model range as best estimate and uncertainty measure of
this quantity \autocite{Collins2013}. This \emph{model democracy} approach
basically assumes that all climate models are independent, equally plausible,
distributed around reality, and that the projected multi-model range is
representative for the uncertainty in the projected quantity
\autocite{Knutti2017a}. However, since the \ac{CMIP} ensembles, sometimes
referred to as \emph{ensembles of opportunity}, have not been designed to
represent a true statistical sample of the reality composed of independent
climate models \autocite{Tebaldi2007}, these assumptions do not hold in
practice. The main reasons for this are that different climate models (even for
different modeling institutions) share parts of their code
\autocite{Abramowitz2019, Knutti2013}, that models do not equally well
represent the observed past and present-day climate \autocite{Gleckler2008,
  Knutti2013}, and that models might suffer from common structural limitations
like missing processes \autocite{Knutti2017a}.

Thus, more sophisticated techniques are necessary to evaluate multi-model
climate projections. This \namecref{sec:02:reducing_uncertainties} introduces
three state-of-the-art methods to assess multi-model projections and reduce
associated uncertainties with observations. These techniques form the baseline
of the work presented in the following \namecrefs{ch:05:paper_ecs}.


\subsection{Emergent Constraints}
\label{subsec:02:emergent_constraints}

As indicated at the beginning of this \namecref{sec:02:reducing_uncertainties},
one main issue of multi-model ensembles is that not all participating climate
models are equally plausible. Usually, this is quantified with some kind of
measure of the models' \emph{performance}, \ie{} their agreement with
observations of the real climate system. However, this model performance can
only be evaluated against observations of the past and present-day climate,
which does not necessarily provide insights into the quality of model
projections of the future climate.

The \emph{emergent constraint} approach tackles this problem by
\enquote{identifying robust, physically interpretable relationships between
  Earth system feedback behaviors on short, well-observed time scales and on
  time scales that span the twenty-first century and beyond}
\autocite{Eyring2019}. An illustration of the concept of emergent constraints
is shown in \cref{fig:02:emergent_constraint}. Each emergent constraint
requires two key components: an \emph{emergent relationship} and a
corresponding observation of the real world \autocite{Eyring2019}. The emergent
relationship (red line in \cref{fig:02:emergent_constraint}) is a robust and
physically-interpretable inter-model relationship between a target variable $y$
related to the future climate and an observable $x$ of the past or present-day
climate. Basis for the relationship is output of the different climate models
of a multi-model ensemble (blue circles in \cref{fig:02:emergent_constraint}).
Using an observation of $x$, this emergent relationship can then be used to
derive a emergent constraint on $y$ (gray shaded area in
\cref{fig:02:emergent_constraint}) that considers uncertainties in the emergent
relationship itself (red shaded area in \cref{fig:02:emergent_constraint}) and
uncertainties in the observation (blue shaded area in
\cref{fig:02:emergent_constraint}).

\begin{figure}[t]
  \centering
  \includegraphics[width=\FigureWidth{}]{
    ch02_scientific_background/figs/emergent_constraint.pdf}
  \caption[
    Schematic illustration of the emergent constraint approach.
  ]{
    Schematic illustration of the emergent constraint approach. Basis of every
    emergent constraint is a robust and physically-interpretable emergent
    relationship (red line) between a target variable $y$ (\eg{} Earth system
    sensitivity or projection of future climate change) and an observable $x$
    (\eg{} past or present-day trend or variation) for the climate models of a
    multi-model ensemble (blue circles). With an observation of $x$,
    uncertainties in $y$ in the multi-model ensemble, illustrated by the yellow
    \acf{PDF}, can be reduced (gray \acs{PDF}). Uncertainties in the target
    variable $y$ arise from two sources: uncertainties in the observation (blue
    shaded area) and uncertainties in the emergent relationship (red shaded
    area). \SpringerAdaptedFrom{Eyring2019}.
  }
  \label{fig:02:emergent_constraint}
\end{figure}

One possible mathematical framework for the evaluation of emergent constraints
is based on linear regression and Gaussian probability densities
\autocite{Cox2013, Cox2018}. Let $x_m$ be the observable predictor variable for
climate model $m$ and $y_m$ the corresponding target variable. To find the
linear emergent relationship for a climate model ensemble with $M$ climate
models and data $\left\{ \left( x_m, y_m \right) \mid m \in I_M \right\}$ with
index set $I_M = \left\{ 1, 2, \ldots, M \right\}$, a linear
regression model
\begin{equation}
  \hat{y}(x) = \hat{b}_0 + \hat{b}_1 x
  \label{eq:02:linear_regression_y}
\end{equation}
for the predicted target variable $\hat{y}$ with estimated intercept
$\hat{b}_0$ and slope $\hat{b}_1$ is used (see \cref{eq:02:linear_coefficients}
for details). Fitting this regression line with ordinary least squares includes
minimizing the standard error $s$ of the estimate
\begin{equation}
  s^2 = \frac{1}{M - 2} \sum_{m=1}^M \left( y_m - \hat{y}_m \right)^2,
  \label{eq:02:sse}
\end{equation}
where $\hat{y}_m \coloneq \hat{y}(x_m)$ is the predicted target variable for
climate model $m$ and $M$ is the total number of climate models. The
uncertainty of the emergent relationship for a value $x$ that has not been used
to fit the regression line is given by the \ac{SPE} $\sigma_{\hat{y}}(x)$:
\begin{equation}
  \sigma_{\hat{y}}^2(x) = s^2 \left[ 1 + \frac{1}{M} + \frac{\left( x - \bar{x}
    \right)^2}{\sum_{m=1}^M \left( x_m - \bar{x} \right)^2} \right].
  \label{eq:02:spe}
\end{equation}
Here, $\bar{x}$ indicates the arithmetic mean of $x$ over all climate models.
Assuming Gaussian errors and a mean of $\hat{y}(x)$ (\ie{} the best estimate of
the target variable $y$ is given by the regression line), \cref{eq:02:spe} can
be used to define a conditional \acf{PDF} for predicting a target variable of
$y$ given $x$:
\begin{equation}
  \cond{y}{x} = \frac{1}{\sqrt{2 \pi \sigma_{\hat{y}}^2(x)}} \exp \left[
    -\frac{\left( y - \hat{y}(x) \right)^2}{2 \sigma_{\hat{y}}^2(x)} \right].
  \label{eq:02:pdf_y_given_x}
\end{equation}
This distribution describes the uncertainty in the emergent relationship itself
introduced by the imperfect alignment of the climate model data (red shaded
area in \cref{fig:02:emergent_constraint}). Its maximum is given by the
emergent relationship itself (red line in \cref{fig:02:emergent_constraint}).
The conditional \ac{PDF} can be interpreted as the posterior distribution of
the regression model based on the climate model output but constrained on the
observable $x$. However, the observed value of $x$, called $x_0$, also has
uncertainties associated with it (blue shaded are in
\cref{fig:02:emergent_constraint}). Assuming again a Gaussian distribution, the
observational \ac{PDF} for observing $x_0$ given the true value $x$ can be
written as
\begin{equation}
  \cond{x_0}{x} = \frac{1}{\sqrt{2 \pi \sigma_x^2}} \exp \left[ -\frac{\left(
    x_0 - x \right)^2}{2 \sigma_x^2} \right],
  \label{eq:02:pdf_obs}
\end{equation}
where $\sigma_x$ is the standard deviation of the observation around the true
value. Assuming an imperfect uniform prior $P(x) \propto 1$ with cut-offs at
$-\infty$ and $+\infty$ and using Bayes' theorem implies $\cond{x_0}{x} =
\cond{x}{x_0}$. In a final step, this can be used to calculate the posterior
\ac{PDF} for the constrained prediction of the target variable $y$ given the
observation $x_0$ (gray \ac{PDF} in \cref{fig:02:emergent_constraint}) with
numerical integration:
\begin{equation}
  \cond{y}{x_0} = \int_{-\infty}^{+\infty} \cond{y}{x} \, \cond{x}{x_0} \dd{}x.
  \label{eq:02:pdf_y_given_x0}
\end{equation}
Posterior estimates of the target variable are influenced by the way the
statistical inference has been performed. Alternative methods that can be used
include Bayesian frameworks \autocite{Renoult2020}, information theoretic
approaches based on the Kullback-Leibler divergence between the models'
\acp{PDF} of $x$ and the observational \ac{PDF} \autocite{Brient2016}, and
linear regression models based on hierarchical Bayesian models
\autocite{Nijsse2020}. However, no consensus has yet been found for this
statistical inference \autocite{Brient2020}.

A convenient metric to quantify the skill of an emergent relationship is the
coefficient of determination $R^2$ of its underlying statistical model. In the
presented framework which is based on univariate ordinary least squares
regression, $R^2$ is given by the squared Pearson correlation coefficient $r$
evaluated on the climate model ensemble data $\left\{ \left( x_m, y_m \right)
\mid m \in I_M \right\}$, \ie{} $R^2 = r^2$. A further quantity describing the
skill of an emergent relationship is its statistical significance. In the
introduced framework, a two-sided $t$-test can be used to determine how likely
the correlation found between the target variable $y$ and the predictor $x$
would be to appear by chance. The null hypothesis for this test is that the
predictor and the target variable are not linearly correlated, \ie{} that the
true underlying Pearson correlation coefficient of the population is zero. If
this null hypothesis is true, the probability distribution of the variable
\begin{equation}
  t = \frac{r \sqrt{M - 2}}{\sqrt{1 - r^2}}
  \label{eq:02:t}
\end{equation}
is a Student's $t$-distribution with $M - 2$ degrees of freedom. The
statistical significance can then be measured with the $p$-value of this
two-sided $t$-test, which describes the probability of obtaining an absolute
sample Pearson correlation coefficient greater than $\abs*{r}$ if the null
hypothesis is true. Smaller $p$-values indicate a higher statistical
significance and vice versa.

One limitation of the presented framework is the assumption that the individual
data points from the different climate models are independent. As already noted
in the beginning of \cref{sec:02:reducing_uncertainties}, this is not the case
for typical climate models ensembles, as some modeling groups provide output
for multiple climate models and some climate models from different modeling
institutions share components and code \autocite{Knutti2013}. The duplicated
code in the different climate models leads to an overestimation of the sample
size and may result in spurious correlations \autocite{Sanderson2015}. Possible
approaches to tackle this problem are presented in
\cref{subsec:02:model_weighting} and include a weighting of the climate models
based on their degree of interdependence \autocite{Knutti2017a, Sanderson2015,
  Sanderson2017}. A further limitation of this approach is the use of an
ordinary least squares linear regression model. This is not always appropriate,
for example when a non-linear emergent relationship is expected
\autocite{Nijsse2020} or when additional physical considerations further
constrain the regression model, \eg{} by demanding a zero intercept ($\hat{b}_0
= 0$) \autocite{Annan2020, JimenezdelaCuesta2019}. Moreover, using only a
single observational dataset to estimate $x_0$ and $\sigma_x$ when different
datasets are available might lead to an underestimation of the observational
uncertainty, as different observational datasets might lead to different
emergent constraints.

A crucial aspect for every emergent constraint is a verifiable physical process
explaining the correlation between $x$ and $y$ \autocite{Hall2019}. Only if the
underlying emergent relationship can be derived from a robust and plausible
physical mechanism, an emergent constraint can be considered credible. The
reason for this are spurious relationships: Due to the large number of possible
observables provided by modern \acp{ESM} and the comparatively small number of
climate models, spurious relationships are possible just by chance
\autocite{Caldwell2014}. Furthermore, out-of-sample tests offer an important
tool to evaluate the credibility of emergent constraints \autocite{Hall2019}.
These ensure that the existence of an emergent relationship is not limited to a
certain climate model ensemble and might indicate that the relationship is also
valid for the true climate system. Testing emergent constraints in different
\ac{CMIP} generations offers a straightforward setup for out-of-sample testing
\autocite{Caldwell2018}, which is discussed in more detail in
\cref{ch:05:paper_ecs}, where eleven emergent constraints on \ac{ECS} are
evaluated on the new \acs{CMIP}6 ensemble.

In the last two decades, many emergent constraints on various aspects of the
Earth system have been published. Early studies tackled the hydrological cycle
\autocite{Allen2002} and the snow-albedo feedback \autocite{Hall2006}. Over the
years, the climate sensitivity expressed by the \ac{ECS} has been a prominent
target variable. Since cloud feedbacks are a major source of uncertainty for
climate sensitivity, a variety of scientific papers focus on constraining
\ac{ECS} with cloud-related processes \autocite{Brient2015, Brient2016,
  Fasullo2012, Lipat2017, Sherwood2014, Su2014, Tian2015, Volodin2008, Qu2013,
  Zhai2015}, which are discussed in detail in
\cref{sec:05:comparison_of_emergent_constraints}. More recent studies aim to
constrain \ac{ECS} with the historical temperature variability
\autocite{Cox2018} or the historical warming trend
\autocite{JimenezdelaCuesta2019, Nijsse2020, Tokarska2020}. Emergent
constraints are not only limited to physical processes, but can also be applied
to other domains, like the global carbon cycle \autocite{Cox2013,
  Kwiatkowski2017, Wenzel2014, Wenzel2016, Winkler2019}. An extensive
discussion on the emergent constraint by \textcite{Wenzel2016}, which focuses
on the concentration-carbon feedback, is given in \cref{subsec:06:step_1}.


\subsection{Performance- and Interdependence-based Weighting of Climate Models}
\label{subsec:02:model_weighting}

A further technique to reduce uncertainties in climate model projections with
observations are model weighting schemes. Their basic idea is to abandon model
democracy by replacing the arithmetic mean used to calculate the \acp{MMM} by a
weighted mean of the form
\begin{equation}
  y = \sum_{m=1}^{M} w_m y_m
  \label{eq:02:weighted_mean}
\end{equation}
with normalized weights $w_m$. Similar to the notation introduced in the
previous \namecref{subsec:02:emergent_constraints}, $y$ is a target variable
(\eg{} a projection of the future climate) and $m$ indexes the $M$ different
climate models. To address two major issues of model democracy (different
climate models are not equally plausible and not independent; see beginning of
\cref{sec:02:reducing_uncertainties}), \textcite{Knutti2017a} propose a
weighting scheme based on climate model performance and interdependence with
the following weights:
\begin{equation}
  w_m \propto \frac{\exp \left( -\frac{D_m^2}{\sigma_D^2} \right)}{1 + \sum_{n
    \neq m}^{M} \exp \left( -\frac{S_{mn}^2}{\sigma_S^2} \right)}.
  \label{eq:02:knutti_weights}
\end{equation}
The metric $D_m$ describes the distance between climate model $m$ and
observations (= model performance) and the metric $S_{mn}$ describes the
distance between climate model $m$ and $n$ (= model interdependence).
$\sigma_D$ and $\sigma_S$ are constants that determine the individual strength
of the performance and interdependence weighting, respectively.

A commonly used distance metric to measure model performance $D_m$ and model
interdependence $S_{mn}$ is the \ac{RMSE}, but others are possible
\autocite{Knutti2017a}. The metrics are evaluated on a set of past or
present-day diagnostics and variables, whose choice is crucial for the
weighting scheme. A helpful strategy for this is to focus on addressing the
question \enquote{which climate model is adequate for predicting the target
  variable $y$?} instead of trying to answer the question \enquote{which
  climate model is the best?} \autocite{Parker2009}. Thus, diagnostics and
variables are chosen that are relevant for the projection of the target
variable \autocite{Knutti2017a}. In practice, this choice is either based on
expert judgment about relevant processes, on emergent relationships (see
\cref{subsec:02:emergent_constraints}) or on multivariate regression models
\commentcite{Karpechko2013, Sanderson2015a}{see \cref{subsec:02:mder}}. It
might also be beneficial to use different diagnostics for the calculations of
the performance and interdependence metrics \autocite{Merrifield2020} and/or to
remove selected diagnostics based on their mutual correlation
\autocite{Lorenz2018}.

The constants $\sigma_D$ and $\sigma_S$ determine how strongly the climate
models' performance and interdependence are weighted \autocite{Knutti2017a}.
Small values of the performance parameter $\sigma_D$ lead to an aggressive
weighting with only a few climate models receiving a majority of the weight,
while large values of $\sigma_D$ result in an equal weighting. For the
interdependence parameter $\sigma_S$, this is slightly different: Here, small
(all climate models are independent) and large (all climate models are
dependent) values lead to an almost equal weighting. Thus, an optimal choice
for $\sigma_D$ and $\sigma_S$ is crucial. A useful tool to estimate these
optimal parameters is the \emph{leave-one-model-out \ac{CV}} approach, which is
also known as \emph{pseudo-reality}, \emph{model-as-truth} or \emph{perfect
  model} setup \autocite{Elia2002, Karpechko2013}. For this, a single climate
model is removed from the multi-model ensemble and treated as observation
(\emph{pseudo-observation}). Then, a weighted \ac{MMM} with weights computed
from the updated model ensemble is calculated, which gives a prediction for the
\enquote{true} climate model. This allows a simple quantitative assessment of
the weighting scheme by calculating the \ac{RMSE} between the prediction and
the known ground truth of the pseudo-observation, which is also known as
\emph{\ac{RMSEP}}. The whole process is repeated for every climate model of the
ensemble to get a statistical distribution of \acp{RMSEP}. Finally, different
\ac{RMSEP} distributions calculated from different parameters $\sigma_D$ and
$\sigma_S$ can be assessed using specific criteria to find optimal values for
$\sigma_D$ and $\sigma_S$ \autocite{Knutti2017a}. Furthermore, the
leave-one-model-out \ac{CV} approach can be used to evaluate different climate
model weightings schemes (including the unweighted \ac{MMM}) and compare them
against each other.

The definition of the weights according to \cref{eq:02:knutti_weights} is based
on reasonable and comprehensible principles. However, the exact form of the
equation is purely subjective. Moreover, the additional freedom in choosing a
suitable metric and optimal values for the parameters $\sigma_D$ and $\sigma_S$
adds another level of subjectivity to the weighting scheme, which can partly be
addressed with the introduced leave-one-model-out \ac{CV} setup. Nevertheless,
due to its flexibility, the climate model weighting scheme of
\textcite{Knutti2017a} has already been used for various target variables:
Arctic sea ice \autocite{Knutti2017a}, Antarctic ozone concentrations
\autocite{Amos2020}, North American maximum temperature \autocite{Lorenz2018},
European temperature and \acl{PR} \autocite{Brunner2019, Merrifield2020}, and
global warming over the \nth{21} century \autocite{Brunner2020, Liang2020a}.


\subsection{Multiple Diagnostic Ensemble Regression}
\label{subsec:02:mder}

An alternative climate model weighting scheme is the \emph{\ac{MDER}} approach
\autocite{Karpechko2013}. Similar to all methods presented in
\cref{sec:02:reducing_uncertainties}, it can be used to reduce uncertainties in
climate model projections with observations. The basis of \ac{MDER} is a set of
$K$ predictor diagnostics $\left\{ x^{(1)}, x^{(2)}, \ldots, x^{(K)} \right\}$
which are relevant for the projection of the target variable $y$. The reasoning
for this choice of the diagnostics is similar to the one presented in the
previous \namecref{subsec:02:model_weighting}: Weighting schemes should address
the question \enquote{which climate model is adequate for predicting the target
  variable?} and not \enquote{which climate model is the best?}.

The concept of \ac{MDER} is mathematically similar to the concept of emergent
constraints. In a first step, an inter-model relationship between the target
variable and the process-relevant diagnostics is used to fit a multivariate
linear regression model. Let $\bm{y} = \left( y_1, y_2, \ldots, y_M \right)^T
\in \mathbb{R}^M$ be the vector of target variables of the $M$ climate models
($T$ denotes the transpose) and $\bm{X} \in \mathbb{R}^{M \times (K + 1)}$ the
design matrix representing the predictors:
\begin{equation}
  \bm{X} =
  \begin{pmatrix}
    1 & x_1^{(1)} & x_1^{(2)} & \cdots & x_1^{(K)} \\
    1 & x_2^{(1)} & x_2^{(2)} & \cdots & x_2^{(K)} \\
    \vdots & \vdots & \vdots & \ddots & \vdots \\
    1 & x_M^{(1)} & x_M^{(2)} & \cdots & x_M^{(K)} \\
  \end{pmatrix}.
  \label{eq:02:design_matrix}
\end{equation}
The entry $x_m^{(k)}$ of this matrix refers to the diagnostic variable of
diagnostic $k$ and climate model $m$. With this notation, the linear
inter-model relationship can be written as
\begin{equation}
  \bm{y} = \bm{X} \bm{b} + \bm{\varepsilon},
  \label{eq:02:multivariate_linear_model}
\end{equation}
where $\bm{b} = \left( b_0, b_1, \ldots, b_K \right)^T \in \mathbb{R}^{(K +
  1)}$ is the vector of linear coefficients (with intercept $b_0$) and
$\bm{\varepsilon} = \left( \varepsilon_1, \varepsilon_2, \ldots, \varepsilon_M
\right)^T \in \mathbb{R}^M$ a vector of independent random variables
representing the noise in the target variable. \Cref{fig:02:mder} shows a
schematic that illustrates this linear relationship (gray surface) for the
different climate models (colored circles) for two diagnostics ($K = 2$). Using
ordinary least squares regression, the estimated linear coefficients
$\hat{\bm{b}}$ are given by
\begin{equation}
  \hat{\bm{b}} = \left( \bm{X}^T \bm{X} \right)^{-1} \bm{X}^T \bm{y},
  \label{eq:02:linear_coefficients}
\end{equation}
where the exponent \enquote{$-1$} denotes the inverse matrix. Since this
definition works for any number of diagnostics $K$, it can also be used to
calculate the linear coefficients $\hat{b}_0$ (intercept) and $\hat{b}_1$
(slope) that define emergent relationships (see
\cref{eq:02:linear_regression_y}).

\begin{figure}[t]
  \centering
  \includegraphics[width=\FigureWidth{}]{
    ch02_scientific_background/figs/mder_schematic.pdf}
  \caption[
    Schematic illustration of the \acf{MDER} approach.
  ]{
    Schematic illustration of the \acf{MDER} approach \autocite{Karpechko2013}.
    First, inter-model relationships between a target variable $y$ and multiple
    process-based predictors $x^{(k)}$ (here: two predictors $x^{(1)}$ and
    $x^{(2)}$) are used to fit a multivariate linear regression model (gray
    surface). Second, observations of the predictors (horizontal black lines)
    are fed into the regression model to calculate an observation-based best
    estimate of the target variable $\hat{y}_0$ following
    \cref{eq:02:mder_best_estimate}. The black circle indicates the best
    estimate for the target variable $y$ given by the observed values of
    $x^{(1)}$ and $x^{(2)}$. Each of the remaining colored circles represents a
    single climate model of the multi-model ensemble. The vertical dashed lines
    visualize the distance between the climate model data and the linear
    regression surface, which represents the noise term $\bm{\varepsilon}$ in
    \cref{eq:02:multivariate_linear_model}.
  }
  \label{fig:02:mder}
\end{figure}

In the second step of the algorithm, observed data of the process-based
diagnostics $\bm{x}_0 = \left( 1, x_0^{(1)}, x_0^{(2)}, \ldots, x_0^{(K)}
\right)^T \in \mathbb{R}^{(K + 1)}$ is fed into the multivariate linear
regression model to get an observation-based prediction of the target variable
$\hat{y}_0$
\autocite{Karpechko2013}:
\begin{equation}
  \hat{y}_0 = \bm{x}_0^T \hat{\bm{b}}
  \label{eq:02:mder_best_estimate}
\end{equation}
This is mathematically similar to the calculation of the best estimate target
variable $y$ for emergent constraints. In \cref{fig:02:mder}, the observations
of the predictors $\bm{x}_0$ are illustrated with horizontal black lines, and
the best estimate $\hat{y}_0$ is shown as a black circle. By combining
\cref{eq:02:linear_coefficients,eq:02:mder_best_estimate} and comparing this to
the definition of weighted means in \cref{eq:02:weighted_mean}, climate model
weights $\bm{w} = \left( w_1, w_2, \ldots, w_M \right)^T \in \mathbb{R}^M$ can
be defined by
\begin{equation}
  \bm{w} = \left[ \bm{x}_0^T \left( \bm{X}^T \bm{X} \right)^{-1} \bm{X}^T
  \right]^T,
  \label{eq:02:mder_weights}
\end{equation}
which can be used to calculate the weighted target variable by
\begin{equation}
  \hat{y}_0 = \hat{\bm{w}}^T \bm{y}.
  \label{eq:02:weighted_mean_vector}
\end{equation}

A crucial aspect for the success of the \ac{MDER} approach is the choice of the
process-relevant diagnostics. In addition to the pre-selection based on expert
judgment, an additional selection based on statistical criteria is necessary
for two reasons: First, predictors which only show a weak correlation with the
target variable should not be included in the regression model since they
introduce additional noise and might lead to overconfident results. Second,
multicollinearity (\ie{} mutually correlated predictors) should be avoided
since this reduces the robustness of the linear regression. A common technique
to deal with these problems is a stepwise feature selection algorithm based on
statistical tests of the correlations between the involved variables
\autocite{Karpechko2013}.

The basic assumption of the \ac{MDER} algorithm is that the inter-model
relationship between the process-based predictors and the target variable also
holds for the true climate. This may seem weak at first glance, especially
since it explicitly requires climate models that deviate from the observed
climate to span the desired relationship (similar to emergent constraints).
However, a much weaker assumption is made traditional in weighting approaches
which assume that climate models that are better in simulating the past or
present-day climate are necessarily better in simulating the future climate. In
contrast to these other weighting approaches, \ac{MDER} explicitly establishes
the relationship between past/present and future within the climate model
ensemble \autocite{Karpechko2013}.

Drawbacks of the \ac{MDER} approach are the missing consideration of errors in
the observational data, the limitation to linear relationships between the
process-relevant diagnostics and the target variable, and the limitation to a
single data point per climate model. Despite these, \ac{MDER} has been
successfully used to constrain uncertainties in Antarctic total ozone
projections \autocite{Karpechko2013}, in the projected change of the austral
jet position \autocite{Wenzel2016a}, and in projections of the Arctic sea ice
extent \autocite{Senftleben2020}.
