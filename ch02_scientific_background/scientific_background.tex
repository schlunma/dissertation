%=============================================================================%
%                                Dissertation                                 %
%                               Manuel Schlund                                %
%                                  (c) 2020                                   %
%=============================================================================%
%                            Scientific Background                            %
%=============================================================================%



\chapter{Scientific Background}
\label{ch:02:scientific_background}

This chapter introduces the scientific background of this thesis. First, basic
concepts of climate model simulations and associated uncertainties are
introduced. Next, the fundamental biogeochemical processes of the global carbon
cycle and important metrics describing climate change are presented. Finally,
state-of-the art techniques used reduce uncertainties in projections of the
future climate are shown. These methods form the basis for the new techniques
developed in this thesis.


\section{\aclp{ESM}: Simulations and Analysis}
\label{sec:02:esms}


\subsection{Numerical Climate Modeling}
\label{subsec:02:climate_modeling}

In contrast to other fields of science, researching the future evolution of the
Earth's climate cannot be purely done by performing experiments in a
laboratory. Due to the immense complexity of the Earth system (including
physical, biological and chemical processes on various temporal and spatial
scales and their mutual interactions), we do not have access to a tiny replica
of the Earth that we can analyze when exposed to different external conditions
\autocite{Flato2011}. While observing the current state of the Earth System is
(relatively) straightforward, gaining evidence about the future evolution of
the climate by only considering present-day observations is rather difficult.

\begin{figure}[t]
  \centering
  \includegraphics[width=\FigureWidth{}]{
    ch02_scientific_background/figs/esms_historical_evolution.pdf}
  \caption{Historical evolution of coupled climate models over the last 45
    years. In early days, these models were so-called \acfp{AOGCM} and only
    included three components: the atmosphere, the land surface and the ocean.
    Over the time, the individual components grew in complexity and included a
    wider range of processes (illustrated by the growing cylinders).
    Eventually, more and more components (aerosols, carbon cycle, \etc{}) were
    added to the coupled system, forming the modern \acfp{ESM}. Taken from
    \textcite{Cubasch2013}.}
  \label{fig:02:esms_historical_evolution}
\end{figure}

A possible way out is given by numerical climate models, which offer the
possibility to simulate the Earth's climate on a computer. The first numerical
climate models came up in the 1960s and were based on weather prediction models
\autocite{Flato2011}. Early models from the 1970s simulated only the physical
components of the climate system: atmosphere, land surface, ocean and sea ice
(see \cref{fig:02:esms_historical_evolution}). The basis of these so-called
\acp{AOGCM} \autocite{Flato2013} is the numerical solving of the differential
equations describing the exchange of energy and matter between these physical
components.

Over the course of the years, climate models became more and more complex by
including a wider range of processes within the components, but also by
introducing new components to the coupled system. Examples of these are
aerosols, the carbon cycle, a dynamic vegetation, atmospheric chemistry and
land ice (see \cref{fig:02:esms_historical_evolution}). \acp{AOGCM} coupled to
these additional components are called \acfp{ESM}, which are the current
state-of-the-art models that allow the most sophisticated simulations of the
Earth's climate. In contrast to \acp{AOGCM}, \acp{ESM} enable the simulation of
biological and chemical processes in addition to the dynamics of the physical
components of the Earth system. Especially in the context of anthropogenic
climate change, these additional processes are of uttermost importance for
realistic climate model simulations, since the anthropogenic interference with
the Earth system directly influences the various biogeochemical cycles of the
Earth. For example, the emission of the most prominent \ac{GHG}, \ac{CO2},
immediately impacts the global carbon cycle by inserting additional carbon into
the system (for details see \cref{sec:02:carbon_cycle}). Further examples
include land use changes like the deforestation of tropical rainforests, which
also directly influences several biogeochemical cycles (\eg{} carbon cycle,
nitrogen cycle, phosphorus cycle, \etc{}) by altering respective sinks and
sources.

Due to the complex interactions between the different components of the Earth
system, these changes in the biogeochemical processes also affect the physical
properties of the climate system. For example, due to the global carbon cycle,
only about \pct{50} of the emitted \ac{CO2} by humankind remains in the
atmosphere \autocite{Friedlingstein2019}. The residual part is absorbed by the
two other main carbon sinks of the planet, the terrestrial biosphere and the
ocean. Since only atmospheric \ac{CO2} can act as \ac{GHG} by introducing an
additional radiative forcing to the Earth System leading to increasing surface
temperatures, this uptake of \ac{CO2} by the carbon cycle slows down global
warming.


\subsection{\acs{CMIP}}
\label{subsec:02:cmip}

Due to the complex nature of the Earth system itself, numerical models of it
consist of hundreds of thousands of lines of computer code. Thus, a
standardization to a certain degree is crucial for the various research groups
developing \acp{ESM} all around the world in order to obtain comparable output
and to facilitate model analysis. For this reason, the \ac{WGCM} of the
\ac{WCRP} initiated the \ac{CMIP} in 1995, with the objective to
\enquote{better understand past, present and future climate changes arising
  from natural, unforced variability or in response to changes in radiative
  forcing in a multi-model context} \autocite{WCRP2020}. One major element of
\ac{CMIP} is to establish common standards, coordination, infrastructure, and
documentation in order to facilitate the distribution of climate model output
\autocite{Eyring2016}.

\begin{figure}[t]
  \centering
  \includegraphics[width=\FigureWidth{}]{
    ch02_scientific_background/figs/cmip6.pdf}
  \caption{Schematic of the experiment design of Phase 6 of the \acl{CMIP}
    (\acs{CMIP}6). The center of the circle illustrates the four \acs{DECK}
    (\acl{DECK}) experiments and the \acs{CMIP}6 historical simulation. The
    circular sectors show additional science themes that can be explored
    through the 21 \acs{CMIP}6-Endorsed \acfp{MIP}. Taken from
    \textcite{Simpkins2017}.}
  \label{fig:02:cmip6}
\end{figure}

A further main aspect is to provide a set of standardized experiments for
global climate model simulations. To participate in the latest phase of
\ac{CMIP}, \acs{CMIP}6, climate models need to run a \emph{historical}
simulation of the period \range{1850}{2014} and the so-called \ac{DECK}
experiments, which include a pre-industrial control run (\emph{piControl}), a
historical Atmospheric \ac{MIP} simulation (\emph{amip}), a simulation forced
with an abrupt quadrupling of \ac{CO2} (\emph{abrupt-4xCO2}) and a simulation
forced with a \pct{1} per year increase of the atmospheric \ac{CO2}
concentration (\emph{1pctCO2}) \autocite{Eyring2016}. This is shown in the
center of \cref{fig:02:cmip6}, which illustrates the experimental design of
\acs{CMIP}6.

To increase diversity and answer more scientific questions, \acs{CMIP}6 models
can participate in the so-called \acs{CMIP}6-Endorsed \acp{MIP}, of which
\acs{CMIP}6 offers 21 (see circular sectors in \cref{fig:02:cmip6}). Some
\acp{MIP} offer additional experiments to explore specific aspects of the Earth
system, like the \ac{C4MIP} which focuses on the carbon cycle
\autocite{Jones2016} or the \ac{CFMIP} which focuses on the evaluation of cloud
feedbacks \autocite{Webb2017}. Other \acp{MIP} allow the assessment of future
climate change. An example is the \ac{ScenarioMIP}, which provides common
experiments that simulate different possible futures \autocite{ONeill2016}.
These experiments are based on the so-called \acp{SSP}, a set of alternative
pathways of future societal development \autocite{ONeill2013, ONeill2017}. For
each experiment, a set of emissions and land use changes is calculated from the
\acp{SSP} \autocite{Riahi2017} which are then used to force the global climate
models. For \ac{ScenarioMIP}, five different \acp{SSP} are considered, ranging
from \ac{SSP}1 (sustainability) to \ac{SSP}5 (fossil-fuel development). Each
\ac{SSP} is combined with a climate outcome (measured as radiative forcing in
the year 2100) based on a particular forcing pathway that \acp{IAM} have shown
to be feasible. For example, \ac{SSP}5-8.5 represents a scenario based on a
fossil-fuel development with a radiative forcing of $8.5 \unit{W m^{-2}}$ in
2100 while \ac{SSP}1-2.6 represents a sustainable future with a radiative
forcing of $2.6 \unit{W m^{-2}}$ in the year 2100. The two other main scenarios
(called \emph{Tier 1} experiments in \ac{ScenarioMIP}) are the \ac{SSP}2-4.5
and \ac{SSP}3-7.0 scenarios. In contrast to the \ac{ScenarioMIP} experiments,
the corresponding \acs{CMIP}5 counterparts \autocite{Taylor2012}, the so-called
\acp{RCP}, only used the radiative forcing in 2100 as only dimension to
describe the possible futures (\eg{} \ac{RCP}8.5, \ac{RCP}4.5, \ac{RCP}2.6,
\etc{}).

In this thesis, climate model data from the two most recent \ac{CMIP}
generations is used, \acs{CMIP}5 and \acs{CMIP}6. More detailed information
about the specific variables and experiments analyzed is given in the
corresponding chapters.


\subsection{Sources of Uncertainties in Climate Model Projections}
\label{subsec:02:source_of_uncertainties}

- anthropogenic forcing vs. natural forcing (volcanoes, solar cycle, etc.)

- basic physics of greenhouse gases (vibration modes, etc.)

- historical vs. future: in historical (future), uncertainty in forcing low
(high)

- definition of feedbacks, examples


\section{Climate Sensitivity}
\label{sec:02:climate_sensitivity}

- Physical Feedbacks (ECS)

- Equations for feedbacks + decomposition

- ocean heat uptake (TCR)

- different between effective climate sensitivity and equilibrium version

- equations for ECS and TCR


\section{The Global Carbon Cycle}
\label{sec:02:carbon_cycle}

Since one study presented in this thesis aims to reduce uncertainties in carbon
cycle-related processes \commentcite{Schlund2020}{see \cref{ch:06:paper_gpp}},
this chapter introduces the scientific background of the global carbon cycle.


\subsection{Overview}
\label{subsec:02:carbon_cycle_overview}

\begin{figure}[t]
  \centering
  \includegraphics[width=\FigureWidth{}]{
    ch02_scientific_background/figs/carbon_cycle_schematic.png}
  \caption{Simplified schematic of the global carbon cycle including the
    typical turnover time scales for carbon transfers through the major
    reservoirs (atmosphere, land surface and ocean). Taken from
    \textcite{Ciais2013}.}
  \label{fig:02:carbon_cycle_schematic}
\end{figure}

A schematic overview of the global carbon cycle is shown in
\cref{fig:02:carbon_cycle_schematic}. To quantify the carbon cycle, common
units are \ac{ppm} for the atmospheric trace gas concentrations (dry-air mole
fraction) and \ac{GtC} or $\si{\GtCyr{}}$ for the reservoirs masses or exchange
fluxes, respectively. The carbon exchange processes between the different
carbon reservoirs run on a wide range of time scales. Conceptually, one can
distinguish between two domains of the global carbon cycle: a slow and a fast
domain. The slow domain with turnover times (reservoir mass of carbon divided
by exchange flux) of more than 10000 years consists of the large carbon stores
in rocks and sediments which are connected to the rapid domain of the carbon
cycle through volcanic emissions of \ac{CO2}, chemical weathering, erosion and
sediment formation on the sea floor. These natural exchange fluxes between the
slow and the fast domain are comparatively small ($< 0.3 \unit{\GtCyr{}}$) and
can be assumed as approximately constant in time over the last few centuries
\autocite{Ciais2013}.

The fast domain of the global carbon cycle consists of three main carbon
reservoirs: the atmosphere, the terrestrial biosphere and the ocean. In the
atmosphere, carbon is mainly stored in trace gases, with \ac{CO2} as the major
component with a current (2019) concentration of about $410 \unit{ppm}$
\autocite{Friedlingstein2019}. Additional contributors to the atmospheric
carbon content are the trace gas \ac{CH4}, the trace gas \ac{CO}, hydrocarbons,
black carbon aerosols and organic compounds \autocite{Ciais2013}. Carbon in the
terrestrial biosphere is mainly stored as organic compounds, with about
$\range{450}{650} \unit{GtC}$ in the living vegetation biomass,
$\range{1500}{2400} \unit{GtC}$ in dead organic matter in litter and soils and
about $1700 \unit{GtC}$ in permafrost soils \autocite{Ciais2013}. The main
component of the oceanic carbon reservoir is dissolved inorganic carbon
(carbonic acid, bicarbonate ions and carbonate ions) with about $38000
\unit{GtC}$. Further carbon is stored as dissolved organic carbon (about $700
\unit{GtC}$), in surface sediments (about $1750 \unit{GtC}$) and in marine
biota (about $3 \unit{GtC}$, predominantly phytoplankton and other
microorganisms) \autocite{Ciais2013, Friedlingstein2019}.

In the fast domain of the global carbon cycle, reservoir turnover times range
from seconds to millennia. In contrast to the slow domain, the carbon exchange
fluxes within the fast domain of the carbon cycle are much higher. One major
group of exchange processes in the fast domain connects the atmosphere and the
terrestrial biosphere. \ac{CO2} is removed from the atmosphere by plant
photosynthesis with about $120 \unit{\GtCyr{}}$ \autocite{Ciais2013}. This
process is also known as \ac{GPP}. The carbon fixed into plants can be released
back into the atmosphere by autotrophic (plant) and heterotrophic (soil
microbial and animal) respiration and additional disturbance processes like
fires \autocite{Ciais2013}. Since the land \ac{CO2} uptake by photosynthesis
occurs only during the growing season, whereas respiration occurs nearly all
year, the larger amount of vegetation in the Northern hemisphere (due to the
larger land mass) gives rise to a seasonal cycle of the atmospheric \ac{CO2}
concentration \autocite{Keeling1995}. This seasonal cycle reflects the phase of
the global carbon cycle and shows a maximum of the atmospheric \ac{CO2}
concentration in the Northern hemisphere winter (net \ac{CO2} flux into
atmosphere due to respiration) and a minimum during the Northern hemisphere
summer (net \ac{CO2} flux into the land due to photosynthesis). Another major
carbon exchange process connects the atmosphere and the ocean. Atmospheric
\ac{CO2} is exchanged with the surface ocean through gas exchange, which is
driven by the partial \ac{CO2} pressure difference between the air and the sea
\autocite{Ciais2013}.


\subsection{Anthropogenic Perturbations}
\label{subsec:02:carbon_cycle_perturbations}

\begin{figure}[t]
  \centering
  \includegraphics[width=\FigureWidth{}]{
    ch02_scientific_background/figs/keeling_curve.pdf}
  \caption{The Keeling Curve: monthly-mean atmospheric \acs{CO2} concentration
    at the Mauna Loa Observatory, Hawaii ($19.5 \unit{\degree N}$, $155.6
    \unit{\degree W}$; elevation: $3397 \unit{m}$) from 1958 to 2019
    \autocite{Keeling2005}. The steady increase of the atmospheric \acs{CO2}
    concentration is superimposed with a seasonal oscillation caused by the
    seasonal \acs{CO2} cycle (see \cref{subsec:02:carbon_cycle_overview}).}
  \label{fig:02:keeling_curve}
\end{figure}

Before the Industrial Era, the global carbon cycle was roughly in a dynamic
equilibrium, which means that exchange fluxes balanced each other and the
amount of carbon in the different reservoirs did neither increase nor decrease.
This can be inferred from ice core measurements, which show an almost constant
atmospheric \ac{CO2} concentration over the last several thousand years before
the Industrial Revolution in the \nth{19} century \autocite{Ciais2013}. Since
the beginning of the Industrial Era, humanity is constantly emitting
carbon-based \acp{GHG} (\eg{} \ac{CO2} and \ac{CH4}) into the atmosphere.
Especially the atmospheric \ac{CO2} concentration has substantially increased,
which has already been shown by Charles D. Keeling in 1976 by his continuous
\ac{CO2} measurements at Mauna Loa, Hawaii that started in 1958
\commentcite{Keeling1976}{see \cref{fig:02:keeling_curve}}. From 1958, the
atmospheric \ac{CO2} concentration at Mauna Loa has steadily increased by about
$100 \unit{ppm}$ to $410 \unit{ppm}$ in the year 2019 \autocite{Keeling2005}.
In addition to the steady increase, the so-called \emph{Keeling Curve} is
further superimposed with the seasonal \ac{CO2} cycle, which gives rise to
local maxima of the atmospheric \ac{CO2} concentration in the Northern
hemisphere winter and local minima in the Northern hemisphere summer
\commentcite{Keeling1995}{see \cref{subsec:02:carbon_cycle_overview}}. Due to
its location in the middle of the Pacific Ocean, the Mauna Loa Observatory
offers perfect conditions for \ac{CO2} measurements by being far away from big
population centers. Moreover, its elevation of more than $3000 \unit{m}$
provides access to the well-mixed air of the Pacific Ocean in high altitudes,
which prevents any interference from the vegetation present on the Hawaiian
Islands.

\begin{figure}[t]
  \centering
  \includegraphics[width=\LargeFigureWidth{}]{
    ch02_scientific_background/figs/carbon_cycle_perturbation.png}
  \caption{Schematic representation of the overall perturbation of the global
    carbon cycle caused by anthropogenic activities, averaged globally for the
    decade \range{2009}{2018}. Arrows represent carbon exchange fluxes; circles
    carbon reservoirs. More details are given in the legend of this figure.
    Taken from \textcite{Friedlingstein2019}.}
  \label{fig:02:carbon_cycle_perturbation}
\end{figure}

Apart from warming the Earth by altering its radiation budget, the
anthropogenically emitted \ac{CO2} directly influences the carbon exchange
fluxes of the global carbon cycle. Due to the excessive carbon in the
atmosphere, there is now a net carbon flux from the atmosphere into the land
and ocean reservoirs (see \cref{fig:02:carbon_cycle_perturbation}). Thus, the
carbon cycle is not in a steady state anymore. In the decade
\range{2009}{2018}, anthropogenic activities caused net carbon fluxes of $3.2
\unit{\GtCyr{}}$ from the atmosphere into the terrestrial biosphere due to
increased plant photosynthesis and $2.5 \unit{\GtCyr{}}$ from the atmosphere
into the ocean due to increase dissolution of \ac{CO2} into the sea
\autocite{Friedlingstein2019}. In the same time, the amount of carbon in the
atmosphere reservoir increased with a rate of $4.9 \unit{\GtCyr{}}$, indicating
that only about half of the anthropogenic \ac{CO2} emissions in the last decade
remained in the atmosphere \autocite{Friedlingstein2019} where they can act as
\ac{GHG}.

Thus, this removal of \ac{CO2} from the atmosphere actively slows down global
warming. However, whether this benefit will persist in the future remains
unclear, which is primarily linked to two feedback processes connecting the
climate system and the global carbon cycle: the \emph{concentration-carbon
  feedback} and the \emph{climate-carbon feedback}
\autocite{Friedlingstein2006, Gregory2009, Collins2013}. For the terrestrial
biosphere, the concentration-carbon feedback is connected to the \emph{\ac{CO2}
  fertilization effect} \autocite{Walker2020}, that causes an increase of
photosynthesis rates when the atmospheric \ac{CO2} concentration increases,
which in turns removes \ac{CO2} from the atmosphere, forming a negative
feedback. For the ocean, the concentration-carbon feedback is negative as well.
In this case, an elevated atmospheric \ac{CO2} concentration causes an
increased dissolution of \ac{CO2} into the sea, which increases the ocean
carbon uptake. On the other hand, the climate-carbon feedback is thought to be
positive for both the terrestrial biosphere and the ocean
\autocite{Gregory2009}. In the first case, temperature and precipitation
changes due to anthropogenic activities decrease the land carbon uptake because
of increased temperature and water stress on photosynthesis and higher
ecosystem respiration costs, which accelerates global warming due to more
\ac{CO2} that remains in the atmosphere. For the ocean, increased temperatures
lead to a reduction of vertical transport in the ocean resulting from increased
stability and reduced solubility of \ac{CO2} in the sea, which reduces the
ocean carbon uptake and enhances climate change \autocite{Gregory2009}.


\section{Techniques to reduce Uncertainties in Climate Model Projections}
\label{sec:02:techniques}

- Process-based weighting (Knutti + MDER)

- emergent constraints

- Wenzel et al., Nature (2016) as example (refer to carbon cycle chapter)

- Review of ECS emergent constraints is given in chapter on Schlund et al., ESD
(2020)