%=============================================================================%
%                                Dissertation                                 %
%                               Manuel Schlund                                %
%                                  (c) 2020                                   %
%=============================================================================%
%                            Scientific Background                            %
%=============================================================================%



\chapter{Scientific Background}
\label{ch:02:scientific_background}

This chapter introduces the scientific background of this thesis. First, basic
concepts of climate model simulations and associated uncertainties are
introduced. Next, the fundamental biogeochemical processes of the global carbon
cycle and important metrics describing climate change are presented. Finally,
state-of-the art techniques used reduce uncertainties in projections of the
future climate are shown. These methods form the basis for the new techniques
developed in this thesis.


\section{\aclp{ESM}: Simulations and Analysis}
\label{sec:02:esms}


\subsection{Numerical Climate Modeling}
\label{subsec:02:climate_modeling}

In contrast to other fields of science, researching the future evolution of the
Earth's climate cannot be purely done by performing experiments in a
laboratory. Due to the immense complexity of the Earth system (including
physical, biological and chemical processes on various temporal and spatial
scales and their mutual interactions), we do not have access to a tiny replica
of the Earth that we can analyze when exposed to different external conditions
\autocite{Flato2011}. While observing the current state of the Earth System is
(relatively) straightforward, gaining evidence about the future evolution of
the climate by only considering present-day observations is rather difficult.

A possible way out is given by numerical climate models, which offer the
possibility to simulate the Earth's climate on a computer. The first numerical
climate models came up in the 1960s and were based on weather prediction models
\autocite{Flato2011}. Early models from the 1970s simulated only the physical
components of the climate system: atmosphere, land surface, ocean and sea ice
(see \cref{fig:02:esms_historical_evolution}). The basis of these so-called
\acp{AOGCM} \autocite{Flato2013} is the numerical solving of the differential
equations describing the exchange of energy and matter between these physical
components.

\begin{figure}[t]
  \centering
  \includegraphics[width=\figureWidth{}]{
    ch02_scientific_background/figs/esms_historical_evolution.pdf}
  \caption{Historical evolution of coupled climate models over the last 45
    years. In early days, these models were so-called \acfp{AOGCM} and only
    included three components: the atmosphere, the land surface and the ocean.
    Over the time, the individual components grew in complexity and included a
    wider range of processes (illustrated by the growing cylinders).
    Eventually, more and more components (aerosols, carbon cycle, \etc{}) were
    added to the coupled system, forming the modern \acfp{ESM}. Taken from
    \textcite{Cubasch2013}.}
  \label{fig:02:esms_historical_evolution}
\end{figure}

Over the course of the years, climate models became more and more complex by
including a wider range of processes within the components, but also by
introducing new components to the coupled system. Examples of these are
aerosols, the carbon cycle, a dynamic vegetation, atmospheric chemistry and
land ice (see \cref{fig:02:esms_historical_evolution}). \acp{AOGCM} coupled to
these additional components are called \acfp{ESM}, which are the current
state-of-the-art models that allow the most sophisticated simulations of the
Earth's climate. In contrast to \acp{AOGCM}, \acp{ESM} enable the simulation of
biological and chemical processes in addition to the dynamics of the physical
components of the Earth system. Especially in the context of anthropogenic
climate change, these additional processes are of uttermost importance for
realistic climate model simulations, since the anthropogenic interference with
the Earth system directly influences the various biogeochemical cycles of the
Earth. For example, the emission of the most prominent \ac{GHG}, \ac{CO2},
immediately impacts the global carbon cycle by inserting additional carbon into
the system (for details see \cref{sec:02:carbon_cycle}). Further examples
include land use changes like the deforestation of tropical rainforests, which
also directly influences several biogeochemical cycles (\eg{} carbon cycle,
nitrogen cycle, phosphorus cycle, \etc{}) by altering respective sinks and
sources.

Due to the complex interactions between the different components of the Earth
system, these changes in the biogeochemical processes also affect the physical
properties of the climate system. For example, due to the global carbon cycle,
only about \pct{50} of the emitted \ac{CO2} by humankind remains in the
atmosphere \autocite{Friedlingstein2019}. The residual part is absorbed by the
two main carbon sinks of the planet, the terrestrial biosphere and the ocean.
Since only atmospheric \ac{CO2} can act as \ac{GHG} by introducing an
additional radiative forcing to the Earth System leading to increasing surface
temperatures, this uptake of \ac{CO2} by the carbon cycle slows down global
warming.


\subsection{\acs{CMIP}}
\label{subsec:02:cmip}

Due to the complex nature of the Earth system itself, numerical models of it
consist of hundreds of thousands of lines of computer code. Thus, a
standardization to a certain degree is crucial for the various research groups
developing \acp{ESM} all around the world in order to obtain comparable output
and to facilitate analysis. For this reason, the \ac{WGCM} of the \ac{WCRP}
initiated the \ac{CMIP} in 1995, with the objective to \enquote{better
understand past, present and future climate changes arising from natural,
unforced variability or in response to changes in radiative forcing in a
multi-model context} \autocite{WCRP2020}. One major element of \ac{CMIP} is to
establish common standards, coordination, infrastructure, and documentation in
order to facilitate the distribution of climate model output
\autocite{Eyring2016}.

A further main aspect is to provide a set of standardized experiments for
global climate model simulations. To participate in the latest phase of
\ac{CMIP}, \acs{CMIP}6, climate models need to run a \emph{historical}
simulation of the period \range{1850}{2014} and the so-called \ac{DECK}
experiments, which include a pre-industrial control run (\emph{piControl}), a
historical Atmospheric \ac{MIP} simulation (\emph{amip}), a simulation forced
with an abrupt quadrupling of \ac{CO2} (\emph{abrupt-4xCO2}) and a simulation
forced with a \pct{1} per year increase of the atmospheric \ac{CO2}
concentration (\emph{1pctCO2}) \autocite{Eyring2016}. This is shown in the
center of \cref{fig:02:cmip6}, which illustrates the experimental design of
\acs{CMIP}6.

\begin{figure}[t]
  \centering
  \includegraphics[width=\figureWidth{}]{
    ch02_scientific_background/figs/cmip6.pdf}
  \caption{Schematic of the experiment design of Phase 6 of the \acl{CMIP}
    (\acs{CMIP}6). The center of the circle illustrates the four \acs{DECK}
    (\acl{DECK}) experiments and the \acs{CMIP}6 historical simulation. The
    circular sectors show additional science themes that can be explored
    through the 21 \acs{CMIP}6-Endorsed \acfp{MIP}. Taken from
    \textcite{Simpkins2017}.}
  \label{fig:02:cmip6}
\end{figure}

To increase diversity and answer more scientific questions, \acs{CMIP}6 models
can participate in the so-called \acs{CMIP}6-Endorsed \acp{MIP}, of which
\acs{CMIP}6 offers 21 (see circular sectors in \cref{fig:02:cmip6}). Some
\acp{MIP} offer additional experiments to explore specific aspects of the Earth
system, like the \ac{C4MIP} which focuses on the carbon cycle
\autocite{Jones2016} or the \ac{CFMIP} which focuses on the evaluation of cloud
feedbacks \autocite{Webb2017}. Other \acp{MIP} allow the assessment of future
climate change. An example is the \ac{ScenarioMIP}, which provides common
experiments that simulate different possible futures \autocite{ONeill2016}.
These experiments are based on the so-called \acp{SSP}, a set of alternative
pathways of future societal development \autocite{ONeill2013, ONeill2017}. For
each experiment, a set of emissions and land use changes is calculated from the
\acp{SSP} \autocite{Riahi2017} which are then used to force the global climate
models. For \ac{ScenarioMIP}, five different \acp{SSP} are considered, ranging
from \ac{SSP}1 (sustainability) to \ac{SSP}5 (fossil-fuel development). Each
\ac{SSP} is combined with a climate outcome (measured as radiative forcing in
the year 2100) based on a particular forcing pathway that \acp{IAM} have shown
to be feasible. For example, \ac{SSP}5-8.5 represents a scenario based on a
fossil-fuel development with a radiative forcing of $8.5 \unit{W m^{-2}}$ in
2100 while \ac{SSP}1-2.6 represents a sustainable future with a radiative
forcing of $2.6 \unit{W m^{-2}}$ in the year 2100. The two other main scenarios
(called \enquote{Tier 1} experiments in \ac{ScenarioMIP}) are the \ac{SSP}2-4.5
and \ac{SSP}3-7.0 scenarios. In contrast to the \ac{ScenarioMIP} experiments,
the corresponding \acs{CMIP}5 counterparts \autocite{Taylor2012}, the so-called
\ac{RCP}, only used the radiative forcing in 2100 as only dimension to describe
the possible futures (\eg{} \ac{RCP}8.5, \ac{RCP}4.5, \ac{RCP}2.6, \etc{}).

In this thesis, climate model data from the two most recent \ac{CMIP}
generations is used, \acs{CMIP}5 and \acs{CMIP}6. More detailed information
about the specific variables and experiments analyzed is given in the
corresponding chapters.


\subsection{Sources of Uncertainties in Climate Model Projections}
\label{subsec:02:source_of_uncertainties}


\section{The Global Carbon Cycle}
\label{sec:02:carbon_cycle}

I like \ac{GPP}!


\section{Techniques to reduce Uncertainties in Climate Model Projections}
\label{sec:02:techniques}