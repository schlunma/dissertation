%=============================================================================%
%                                Dissertation                                 %
%                               Manuel Schlund                                %
%                                  (c) 2020                                   %
%=============================================================================%
%                            Scientific Background                            %
%=============================================================================%



\chapter{Scientific Background}
\label{ch:02:scientific_background}


\section{Earth System Models: Simulations and Analysis}
\label{sec:02:esms}

An \ac{ESM} is really cool! Yes, \acp{ESM}!!

\begin{figure}[t]
  \centering
  \includegraphics[width=\figureWidth{}]{
    ch02_scientific_background/figs/esms_historical_evolution.pdf}
  \caption{Historical evolution of coupled climate models over the last 45
  years. In early days, these models were \acfp{AOGCM} and only included three
  components: the atmosphere, the land surface and the ocean. Over the time,
  the individual components grew in complexity and included a wider range of
  processes (illustrated by the growing cylinders). Eventually, more and more
  components (aerosols, carbon cycle, \etc{}) were added to the coupled system,
  forming the modern \acfp{ESM}. Taken from \textcite{Cubasch2013}.}
  \label{fig:02:esms_historical_evolution}
\end{figure}

TBA.


\section{Techniques to reduce uncertainties in climate model projections}
\label{sec:02:techniques}