%=============================================================================%
%                                Dissertation                                 %
%                               Manuel Schlund                                %
%                                  (c) 2020                                   %
%=============================================================================%
%                            Scientific Background                            %
%=============================================================================%



\chapter{Scientific Background}
\label{ch:02:scientific_background}

This chapter introduces the scientific background of this thesis. First, basic
concepts of climate model simulations and associated uncertainties are
introduced. Next, the fundamental biogeochemical processes of the global carbon
cycle and important metrics describing climate change are presented. Finally,
state-of-the art techniques used reduce uncertainties in projections of the
future climate are shown. These methods form the basis for the new techniques
developed in this thesis.


\section{\aclp{ESM}: Simulations and Analysis}
\label{sec:02:esms}


\subsection{Numerical Climate Modeling}
\label{subsec:02:climate_modeling}

In contrast to other fields of science, researching the future evolution of the
Earth's climate cannot be purely done by performing experiments in a
laboratory. Due to the immense complexity of the Earth system (including
physical, biological and chemical processes on various temporal and spatial
scales and their mutual interactions), we do not have access to a tiny replica
of the Earth that we can analyze when exposed to different external conditions
\autocite{Flato2011}. While observing the current state of the Earth System is
(relatively) straightforward, gaining evidence about the future evolution of
the climate by only considering present-day observations is rather difficult.

A possible way out is given by numerical climate models, which offer the
possibility to simulate the Earth's climate on a computer. The first numerical
climate models came up in the 1960s and were based on weather prediction models
\autocite{Flato2011}. Early models from the 1970s simulated only the physical
components of the climate system: atmosphere, land surface, ocean and sea ice
(see \cref{fig:02:esms_historical_evolution}). The basis of these so-called
\acp{AOGCM} \autocite{Flato2013} is the numerical solving of the differential
equations describing the exchange of energy and matter between these physical
components.

\begin{figure}[t]
  \centering
  \includegraphics[width=\FigureWidth{}]{
    ch02_scientific_background/figs/esms_historical_evolution.pdf}
  \caption{Historical evolution of coupled climate models over the last 45
    years. In early days, these models were so-called \acfp{AOGCM} and only
    included three components: the atmosphere, the land surface and the ocean.
    Over the time, the individual components grew in complexity and included a
    wider range of processes (illustrated by the growing cylinders).
    Eventually, more and more components (aerosols, carbon cycle, \etc{}) were
    added to the coupled system, forming the modern \acfp{ESM}. Taken from
    \textcite{Cubasch2013}.}
  \label{fig:02:esms_historical_evolution}
\end{figure}

Over the course of the years, climate models became more and more complex by
including a wider range of processes within the components, but also by
introducing new components to the coupled system. Examples of these are
aerosols, the carbon cycle, a dynamic vegetation, atmospheric chemistry and
land ice (see \cref{fig:02:esms_historical_evolution}). \acp{AOGCM} coupled to
these additional components are called \acfp{ESM}, which are the current
state-of-the-art models that allow the most sophisticated simulations of the
Earth's climate. In contrast to \acp{AOGCM}, \acp{ESM} enable the simulation of
biological and chemical processes in addition to the dynamics of the physical
components of the Earth system. Especially in the context of anthropogenic
climate change, these additional processes are of uttermost importance for
realistic climate model simulations, since the anthropogenic interference with
the Earth system directly influences the various biogeochemical cycles of the
Earth. For example, the emission of the most prominent \ac{GHG}, \ac{CO2},
immediately impacts the global carbon cycle by inserting additional carbon into
the system (for details see \cref{sec:02:carbon_cycle}). Further examples
include land use changes like the deforestation of tropical rainforests, which
also directly influences several biogeochemical cycles (\eg{} carbon cycle,
nitrogen cycle, phosphorus cycle, \etc{}) by altering respective sinks and
sources.

Due to the complex interactions between the different components of the Earth
system, these changes in the biogeochemical processes also affect the physical
properties of the climate system. For example, due to the global carbon cycle,
only about \pct{50} of the emitted \ac{CO2} by humankind remains in the
atmosphere \autocite{Friedlingstein2019}. The residual part is absorbed by the
two other main carbon sinks of the planet, the terrestrial biosphere and the
ocean. Since only atmospheric \ac{CO2} can act as \ac{GHG} by introducing an
additional radiative forcing to the Earth System leading to increasing surface
temperatures, this uptake of \ac{CO2} by the carbon cycle slows down global
warming.


\subsection{\acs{CMIP}}
\label{subsec:02:cmip}

Due to the complex nature of the Earth system itself, numerical models of it
consist of hundreds of thousands of lines of computer code. Thus, a
standardization to a certain degree is crucial for the various research groups
developing \acp{ESM} all around the world in order to obtain comparable output
and to facilitate model analysis. For this reason, the \ac{WGCM} of the
\ac{WCRP} initiated the \ac{CMIP} in 1995, with the objective to
\enquote{better understand past, present and future climate changes arising
  from natural, unforced variability or in response to changes in radiative
  forcing in a multi-model context} \autocite{WCRP2020}. One major element of
\ac{CMIP} is to establish common standards, coordination, infrastructure, and
documentation in order to facilitate the distribution of climate model output
\autocite{Eyring2016}.

A further main aspect is to provide a set of standardized experiments for
global climate model simulations. To participate in the latest phase of
\ac{CMIP}, \acs{CMIP}6, climate models need to run a \emph{historical}
simulation of the period \range{1850}{2014} and the so-called \ac{DECK}
experiments, which include a pre-industrial control run (\emph{piControl}), a
historical Atmospheric \ac{MIP} simulation (\emph{amip}), a simulation forced
with an abrupt quadrupling of \ac{CO2} (\emph{abrupt-4xCO2}) and a simulation
forced with a \pct{1} per year increase of the atmospheric \ac{CO2}
concentration (\emph{1pctCO2}) \autocite{Eyring2016}. This is shown in the
center of \cref{fig:02:cmip6}, which illustrates the experimental design of
\acs{CMIP}6.

\begin{figure}[t]
  \centering
  \includegraphics[width=\LargeFigureWidth{}]{
    ch02_scientific_background/figs/cmip6.pdf}
  \caption{Schematic of the experiment design of Phase 6 of the \acl{CMIP}
    (\acs{CMIP}6). The center of the circle illustrates the four \acs{DECK}
    (\acl{DECK}) experiments and the \acs{CMIP}6 historical simulation. The
    circular sectors show additional science themes that can be explored
    through the 21 \acs{CMIP}6-Endorsed \acfp{MIP}. Taken from
    \textcite{Simpkins2017}.}
  \label{fig:02:cmip6}
\end{figure}

To increase diversity and answer more scientific questions, \acs{CMIP}6 models
can participate in the so-called \acs{CMIP}6-Endorsed \acp{MIP}, of which
\acs{CMIP}6 offers 21 (see circular sectors in \cref{fig:02:cmip6}). Some
\acp{MIP} offer additional experiments to explore specific aspects of the Earth
system, like the \ac{C4MIP} which focuses on the carbon cycle
\autocite{Jones2016} or the \ac{AerChemMIP} which focuses on aerosol chemistry
\autocite{Collins2017}. Other \acp{MIP} allow the assessment of future climate
change. An example is the \ac{ScenarioMIP}, which provides common experiments
that simulate different possible futures \autocite{ONeill2016}. These
experiments are based on the so-called \acp{SSP}, a set of alternative pathways
of future societal development \autocite{ONeill2013, ONeill2017}. For each
experiment, a set of emissions and land use changes is calculated from the
\acp{SSP} \autocite{Riahi2017} which are then used to force the global climate
models. For \ac{ScenarioMIP}, five different \acp{SSP} are considered, ranging
from \ac{SSP}1 (sustainability) to \ac{SSP}5 (fossil-fuel development). Each
\ac{SSP} is combined with a climate outcome (measured as radiative forcing in
the year 2100) based on a particular forcing pathway that \acp{IAM} have shown
to be feasible. For example, \ac{SSP}5-8.5 represents a scenario based on a
fossil-fuel development with a radiative forcing of $8.5 \unit{W m^{-2}}$ in
2100 while \ac{SSP}1-2.6 represents a sustainable future with a radiative
forcing of $2.6 \unit{W m^{-2}}$ in the year 2100. The two other main scenarios
(called \emph{Tier 1} experiments in \ac{ScenarioMIP}) are the \ac{SSP}2-4.5
and \ac{SSP}3-7.0 scenarios. In contrast to the \ac{ScenarioMIP} experiments,
the corresponding \acs{CMIP}5 counterparts \autocite{Taylor2012}, the so-called
\acp{RCP}, only used the radiative forcing in 2100 as only dimension to
describe the possible futures (\eg{} \ac{RCP}8.5, \ac{RCP}4.5, \ac{RCP}2.6,
\etc{}).

In this thesis, climate model data from the two most recent \ac{CMIP}
generations is used, \acs{CMIP}5 and \acs{CMIP}6. More detailed information
about the specific variables and experiments analyzed is given in the
corresponding chapters.


\subsection{Sources of Uncertainties in Climate Model Projections}
\label{subsec:02:source_of_uncertainties}

Simulations from climate model ensembles of \ac{CMIP} allow us to assess the
impact of future climate change in a consistent and transparent way. Especially
the \ac{ScenarioMIP} experiments can give valuable insights into the upcoming
development of the Earth system by providing \emph{projections} of the future
climate. In contrast to climate predictions, climate projections run over
multiple decades and depend upon the future scenario considered, which are
based on assumptions that may or may not turn out to be correct. On the
contrary, climate predictions are attempts to predict the actual evolution of
the climate on much shorter time scales from seasons to years. Similar to any
other scientific experiment, climate model projections suffer from associated
uncertainties. There are three major sources of climate model projections we
can distinguish: natural variability, climate response uncertainty and emission
uncertainty \autocite{Hawkins2009, Hawkins2010}.
\Cref{fig:02:sources_of_uncertainty} shows these three sources for the
projected global mean surface temperature anomaly over the \nth{21} century.

\begin{figure}[t]
  \centering
  \includegraphics[width=\LargeFigureWidth{}]{
    ch02_scientific_background/figs/sources_of_uncertainty.pdf}
  \caption{Schematic illustrating the importance of different sources of
    uncertainties in climate model projections and their evolution in time. (a)
    Time series of the anomaly of the decadal and global mean surface
    temperature relative to the period \range{1961}{1980}. The black line
    shows the historical observations with estimates of uncertainty from
    climate models (gray). The remaining colors show different sources of
    uncertainty in future climate projections: Natural variability (orange),
    climate response uncertainty (blue) and emission uncertainty (green)
    \autocite{Hawkins2009, Hawkins2010}. Climate response uncertainty can (b)
    increase in newer generations of climate models when a new process is
    discovered to be relevant or (c) decrease with additional model
    improvements and observational constraints. Taken from
    \textcite{Cubasch2013}.}
  \label{fig:02:sources_of_uncertainty}
\end{figure}

\emph{Natural variability} is connected to the chaotic nature of the Earth
system that arises from complex interactions between the ocean, atmosphere,
land, biosphere and cryosphere \autocite{Cubasch2013}. It constitutes a
fundamental limit of how precisely we can project the future climate since it
is inherent in the Earth system and cannot be eliminated by more knowledge and
more advanced climate models. Natural variability is more relevant on regional
and local scales than on continental or global scales. Further contributions to
natural variability on longer time scales come from phenomena like the
\ac{ENSO} or the \ac{NAO} and from externally (and thus explainable) events
like volcanic eruptions and variations in the solar activity. Natural
variability can be seen as the \emph{noise} in the climate record as opposed to
the anthropogenic \emph{signal} \autocite{Cubasch2013}. As illustrated by
\cref{fig:02:sources_of_uncertainty}, the uncertainty associated with natural
variability is constant over time.

The second source of uncertainty in climate model projections is \emph{emission
  uncertainty}. This arises from the different possible trajectories in terms
of future forcing (\acp{GHG}, aerosols, land use changes, \etc{}) humankind
might take. Examples for these are the \ac{SSP}-based experiments given by
\ac{ScenarioMIP} that include a variety of different scenarios from a
sustainable future to a full fossil fuel-based development (see
\cref{subsec:02:cmip}). A possible approach to quantify emission uncertainty is
to assess the climate impact of these different trajectories. Since the
emission uncertainty strongly depends on the future development of the human
society, it cannot be reduced by improving climate models. In contrast to
natural variability, the emission uncertainty increases over time in climate
projections, since estimating forcings for the near future is easier than for
the far future.

Finally, the third source of uncertainty in climate model projections is the
\emph{climate response uncertainty}, which comes from our imperfect knowledge
of how the climate system will respond to future anthropogenic forcings. Due to
the complexity of the Earth system, the future climate could develop in many
different ways that are all consistent with our current knowledge and models
\autocite{Cubasch2013}. In the context of climate model ensembles, the climate
response uncertainty is often also called \emph{model uncertainty} and reflects
the different responses of the different climate models to a given forcing.
Even though all climate models are built on the same physical principles, they
differ in terms of spatial resolution, processes included and parametrizations
of unresolved processes. The latter can be thought of as different
approximations that are necessary to represent processes that take place on
scales smaller than the common size of grid boxes in modern global climate
models (about $\ang{1} \times \ang{1}$ horizontally), which is for example the
case for some cloud processes and vegetation processes.

These differences in the climate models also give rise to different intensities
of \emph{climate feedbacks} (or even their presence/absence) in the models. A
climate feedback is a mechanism that either amplifies (\emph{positive
  feedback}) or diminishes (\emph{negative feedback}) the effect of an external
forcing. An example of a strong positive feedback is the water vapor feedback,
in which the increased surface temperature (caused by anthropogenic forcing)
leads to an enhanced evaporation of water which enhances the amount of water
vapor in the atmosphere. Since water vapor itself is a powerful \ac{GHG}, this
amplifies the effect of the anthropogenic forcing by further increasing the
surface temperatures \autocite{Cubasch2013}. Further examples and a
mathematical framework for the analysis of feedbacks are given in
\cref{subsec:02:climate_feedbacks}.

As sciences evolves, representations of already included processes can be
improved in climate models. Moreover, new geophysical and biogeochemical
processes can be added to them. On the one hand, this can increase the climate
response uncertainty when a new process is discovered to be relevant
\commentcite{Cubasch2013}{see \cref{fig:02:sources_of_uncertainty}b}. However,
such an increase corresponds to a previously unmeasured uncertainty. An example
for this has recently happened in \acs{CMIP}6: most likely due to changes in
the cloud representation of the models the spread in the projected \ac{GSAT}
caused by a doubling of the atmospheric \ac{CO2} concentration has
substantially increased in \ac{CMIP}6 compared to older \ac{CMIP} generations
\autocite{Zelinka2020}. On the other hand, the climate response uncertainty can
decrease with additional model improvements and better understanding of the
Earth system (see \cref{fig:02:sources_of_uncertainty}c). Moreover, it can also
be reduced by observational constraints, which is the main topic of this
thesis.


\section{Climate Sensitivity}
\label{sec:02:climate_sensitivity}

Climate sensitivity refers to the change in the \ac{GSAT} that results from a
change in the radiative forcing. In other words, it describes how sensitive the
climate system is to an external forcing. The source of this forcing might
either be natural (changes in the solar activity, volcanic eruptions, \etc{})
or anthropogenic (emissions of \acp{GHG}, land use changes, \etc{}).


\subsection{Climate Feedbacks}
\label{subsec:02:climate_feedbacks}

As already described in \cref{subsec:02:source_of_uncertainties}, the effects
of an external forcing acting on the climate system can additionally be
amplified or diminished by climate feedbacks. Thus, feedback processes play a
crucial role in the evaluation of climate sensitivity.
\Cref{fig:02:feedbacks_overview} shows an overview of important feedbacks in
the Earth system with their corresponding time scales on which they operate.

\begin{figure}[t]
  \centering
  \includegraphics[width=\LargeFigureWidth{}]{
    ch02_scientific_background/figs/feedbacks_overview.pdf}
  \caption{Climate feedbacks and corresponding time scales. \enquote{+} refers
    to positive feedbacks, which amplify the effect of the external forcing
    (\eg{} the water vapor feedback). \enquote{-} refers to negative feedbacks,
    which diminish the effect of the external forcing (\eg{} the longwave
    radiation feedback). \enquote{+/-} refers to feedbacks which might be
    either positive or negative (\eg{} the cloud feedback). The smaller box
    highlights the large differences in time scales for the various feedbacks.
    Taken from \textcite{Cubasch2013}.}
  \label{fig:02:feedbacks_overview}
\end{figure}

An example for a positive feedback is the already mentioned \emph{water vapor
  feedback}. Being the primary \ac{GHG} in the Earth's atmosphere, water vapor
is the largest contributor to the natural greenhouse effect. Since its amount
in the atmosphere is mainly controlled by the air temperature and anthropogenic
emissions of water vapor are negligible, the influence of water vapor on the
climate system is described as a feedback mechanism and not as an external
forcing \autocite{Myhre2013}. Basis of this feedback is the enhanced
evaporation of water with increasing air temperatures. Each degree of warming
allows the atmosphere to retain about \pct{7} more water vapor
\autocite{Myhre2013}, which closes the positive feedback loop by further
increasing air temperatures through the greenhouse effect. With a typical
residence time of water vapor in the atmosphere of several days, the water
vapor feedback operates on relatively short time scales. As the largest
positive feedback in the Earth system \autocite{Soden2006}, the water vapor
feedback amplifies any initial forcing (\eg{} caused by anthropogenic \ac{CO2}
emissions) by a typical factor between two and three, rendering water vapor a
fundamental agent of climate change \autocite{Myhre2013}. An example for a
positive feedback that operates on longer time scales (several years) is the
\emph{snow/ice albedo feedback}, in which the surface albedo decreases as
highly reflective ice and snow surfaces melt with global warming, exposing the
darker and more absorbing surfaces below \autocite{Cubasch2013}.

In contrast to positive feedbacks, negative feedbacks diminish the effect of an
external forcing. An example for this is the \emph{blackbody feedback} (also
known as \emph{Planck feedback} or \emph{longwave radiation feedback}), which
is the strongest negative feedback in the Earth system \autocite{Cubasch2013}.
It is based on the thermal electromagnetic radiation that any object with a
non-zero temperature emits (the so-called \emph{blackbody radiation}). Since
the power of this radiation strongly depends on the temperature of the object,
higher surface temperatures of Earth increase the outgoing longwave radiation
flux from the surface which reduces the effect of the external forcing and
cools the planet.

For some domains of the Earth system, feedbacks can be positive and/or
negative, since a variety of different mechanisms is involved. An example for
this is the cloud feedback. Changes in clouds induced by climate change can
cause both longwave (greenhouse warming) and shortwave (reflective cooling)
effects, which both need to be considered for the overall cloud feedback
\autocite{Boucher2013}. Relevant cloud properties that may change as a response
to an external forcing and that may alter the Earth's radiative budget are
cloud cover, cloud optical thickness, cloud altitude and the geographical
distribution of clouds. Examples for robust cloud feedback processes are the
rise of high clouds in a warming climate which traps longwave radiation and
enhances global warming and the reduction in mid- and low-level cloud cover
which diminishes the reflection of incoming solar radiation and also increases
the surface warming \autocite{Boucher2013}. In global climate model ensembles,
the overall cloud feedback shows a large range with positive and negative
values, but tends to be slightly positive on average \autocite{Soden2006,
  Dufresne2008, Vial2013, Zelinka2020}. This large uncertainty in the cloud
feedback is a major reason for uncertainties in the climate sensitivity of
climate models \autocite{Boucher2013, Flato2013}.

Further examples of feedbacks with positive and negative contributions are
biogeochemical feedbacks. Negative contributions come from increased \ac{CO2}
fluxes into the land and ocean carbon reservoirs due to increased
photosynthesis rates and \ac{CO2} dissolution in the sea, respectively, which
decrease the atmospheric \ac{CO2} content and diminish global warming. An
example for a positive contribution is the decreased solubility of \ac{CO2} in
water in a warmer climate, which reduces the atmosphere-ocean \ac{CO2} flux and
enhances climate change. More details on this are given in
\cref{subsec:02:carbon_cycle_perturbations}.


\subsection{Mathematical Framework for Feedbacks Analysis}
\label{subsec:02:mathematical_framework_feedbacks}

The foundation for a basic mathematical framework for the analysis of climate
feedbacks is a simple energy balance model \autocite{Gregory2009, Roe2009}.
Anthropogenic activities in the Earth system like the emissions of \acp{GHG} or
aerosols introduce an external forcing to the climate system, which is
quantified with a radiative forcing $F$ measured in $\si{W m^{-2}}$. To restore
a stable state, the climate system opposes this forcing with a climate response
$R$, leading to a net energy flux of
\begin{equation}
  N = F + R
  \label{eq:02:N}
\end{equation}
into the system. Positive values of $N$, $F$ and $R$ indicate incoming fluxes;
usually $F > 0$ and $R < 0$. On long time scales (multiple years), the net
incoming radiative flux at the \ac{TOA} and the net heat flux into the ocean
are basically equal definitions of $N$ since nearly all of the Earth's heat
capacity resides in the ocean \autocite{Gregory2009}. While $N \neq 0$, the
climate system evolves; when $N = 0$ a new steady state has been reached.

To quantify the effects of different feedbacks, a reference system with a basic
response needs to be defined, which is a crucial aspect of feedback analysis
\autocite{Roe2009}. Usually, the idealization of a blackbody Earth (without an
atmosphere) is used for that: In equilibrium, the incoming solar irradiance is
balanced with an outgoing thermal irradiance $J_0$ that solely depends on the
global mean surface temperature $T_0$ following the Stefan–Boltzmann law
\begin{equation}
  J_0 = -\sigma T_0^4.
  \label{eq:02:stefan_boltzmann_law_eq}
\end{equation}
$\sigma \approx 5.67 \unit{W m^{-2} K^{-4}}$ is the Stefan–Boltzmann constant.
To answer an external forcing $F$, the climate system reacts with a response
$R$ expressed by a change in the global mean surface temperature $\Delta T$:
\begin{equation}
  J_0 + R = -\sigma \left( T_0 + \Delta T \right)^4
  \label{eq:02:stefan_boltzmann_law_non_eq}
\end{equation}
Since the temperature change caused by an anthropogenic forcing is much smaller
than the equilibrium temperature $\Delta T \ll T_0 \approx 255 \unit{K}$, a
simple \nth{1}-order Taylor expansion can be used to linearize the blackbody
response:
\begin{equation}
  -\sigma \left( T_0 + \Delta T \right)^4 \approx J_0 - 4 \sigma T_0^3 \cdot
  \Delta T
  \label{eq:02:stefan_boltzmann_law_lin}
\end{equation}
Thus, by comparing
\cref{eq:02:stefan_boltzmann_law_non_eq,eq:02:stefan_boltzmann_law_lin} the
climate response $R$ can be expressed as
\begin{equation}
  R = -4 \sigma T_0^3 \cdot \Delta T \coloneqq \lambda_\text{BB} \cdot \Delta T
  \label{eq:02:linear_blackbody_response}
\end{equation}
with the blackbody feedback parameter $\lambda_\text{BB} \approx -3.8 \unit{W
  m^{-2} K^{-1}}$. Results from climate models and analyses of observations
confirm this linear relationship between $R$ and $\Delta T$
\autocite{Gregory2004}. However, the value of this linear constant $\lambda$,
the \emph{climate feedback parameter}, is found to be considerably larger than
the blackbody response ($\lambda \approx -1.0 \unit{W m^{-2} K^{-1}}$),
indicating that additional processes affect the Earth's radiative balance: the
climate feedbacks \autocite{Flato2013, Gregory2009}. Since climate models
suggest that the radiative effects of these additional feedbacks are also
proportional to $\Delta T$ \autocite{Gregory2008a},
\cref{eq:02:linear_blackbody_response} can be adapted to
\begin{equation}
  R = \lambda \cdot \Delta T = \left( \lambda_\text{BB} + \lambda_\text{WV} +
  \lambda_\text{Albedo} + \lambda_\text{Cloud} + \ldots \right) \cdot \Delta T.
  \label{eq:02:linear_response}
\end{equation}
$\lambda_\text{WV}$ refers to the water vapor feedback, $\lambda_\text{Albedo}$
to the snow/ice albedo feedback and $\lambda_\text{Cloud}$ to the cloud
feedback. Thus, the overall climate feedback parameter $\lambda$ can be written
as the sum of the individual feedback parameters $\lambda_i$:
\begin{equation}
  \lambda = \sum_i \lambda_i
  \label{eq:02:lambda_as_sum_of_lambdas}
\end{equation}
Positive values of $\lambda_i$ indicate positive feedbacks (\eg{} the water
vapor feedback) and negative values indicate negative feedbacks (\eg{} the
blackbody feedback). This equation assumes that the individual radiative
responses from the different feedbacks are independent, which is a reasonable
first-order approximation but not entirely true \autocite{Soden2008}.

\subsection{Equilibrium and \acl{ECS}}
\label{subsec:02:ecs}

- Physical Feedbacks (ECS)

- ocean heat uptake (TCR)

- different between effective climate sensitivity and equilibrium version

- equations for ECS and TCR


\section{The Global Carbon Cycle}
\label{sec:02:carbon_cycle}

Since one study presented in this thesis aims to reduce uncertainties in carbon
cycle-related processes \commentcite{Schlund2020}{see \cref{ch:06:paper_gpp}},
this chapter introduces the scientific background of the global carbon cycle.


\subsection{Overview}
\label{subsec:02:carbon_cycle_overview}

A schematic overview of the global carbon cycle is shown in
\cref{fig:02:carbon_cycle_schematic}. To quantify the carbon cycle, common
units are \ac{ppm} for the atmospheric trace gas concentrations (dry-air mole
fraction) and \ac{GtC} or $\si{\GtCyr{}}$ for the reservoirs masses or exchange
fluxes, respectively. The carbon exchange processes between the different
carbon reservoirs run on a wide range of time scales. Conceptually, one can
distinguish between two domains of the global carbon cycle: a slow and a fast
domain. The slow domain with turnover times (reservoir mass of carbon divided
by exchange flux) of more than 10000 years consists of the large carbon stores
in rocks and sediments which are connected to the rapid domain of the carbon
cycle through volcanic emissions of \ac{CO2}, chemical weathering, erosion and
sediment formation on the sea floor. These natural exchange fluxes between the
slow and the fast domain are comparatively small ($< 0.3 \unit{\GtCyr{}}$) and
can be assumed as approximately constant in time over the last few centuries
\autocite{Ciais2013}.

\begin{figure}[t]
  \centering
  \includegraphics[width=\FigureWidth{}]{
    ch02_scientific_background/figs/carbon_cycle_schematic.png}
  \caption{Simplified schematic of the global carbon cycle including the
    typical turnover time scales for carbon transfers through the major
    reservoirs (atmosphere, land surface and ocean). Taken from
    \textcite{Ciais2013}.}
  \label{fig:02:carbon_cycle_schematic}
\end{figure}

The fast domain of the global carbon cycle consists of three main carbon
reservoirs: the atmosphere, the terrestrial biosphere and the ocean. In the
atmosphere, carbon is mainly stored in trace gases, with \ac{CO2} as the major
component with a current (2019) concentration of about $410 \unit{ppm}$
\autocite{Friedlingstein2019}. Additional contributors to the atmospheric
carbon content are the trace gas \ac{CH4}, the trace gas \ac{CO}, hydrocarbons,
black carbon aerosols and organic compounds \autocite{Ciais2013}. Carbon in the
terrestrial biosphere is mainly stored as organic compounds, with about
$\range{450}{650} \unit{GtC}$ in the living vegetation biomass,
$\range{1500}{2400} \unit{GtC}$ in dead organic matter in litter and soils and
about $1700 \unit{GtC}$ in permafrost soils \autocite{Ciais2013}. The main
component of the oceanic carbon reservoir is dissolved inorganic carbon
(carbonic acid, bicarbonate ions and carbonate ions) with about $38000
\unit{GtC}$. Further carbon is stored as dissolved organic carbon (about $700
\unit{GtC}$), in surface sediments (about $1750 \unit{GtC}$) and in marine
biota (about $3 \unit{GtC}$, predominantly phytoplankton and other
microorganisms) \autocite{Ciais2013, Friedlingstein2019}.

In the fast domain of the global carbon cycle, reservoir turnover times range
from seconds to millennia. In contrast to the slow domain, the carbon exchange
fluxes within the fast domain of the carbon cycle are much higher. One major
group of exchange processes in the fast domain connects the atmosphere and the
terrestrial biosphere. \ac{CO2} is removed from the atmosphere by plant
photosynthesis with about $120 \unit{\GtCyr{}}$ \autocite{Ciais2013}. This
process is also known as \ac{GPP}. The carbon fixed into plants can be released
back into the atmosphere by autotrophic (plant) and heterotrophic (soil
microbial and animal) respiration and additional disturbance processes like
fires \autocite{Ciais2013}. Since the land \ac{CO2} uptake by photosynthesis
occurs only during the growing season, whereas respiration occurs nearly all
year, the larger amount of vegetation in the Northern hemisphere (due to the
larger land mass) gives rise to a seasonal cycle of the atmospheric \ac{CO2}
concentration \autocite{Keeling1995}. This seasonal cycle reflects the phase of
the global carbon cycle and shows a maximum of the atmospheric \ac{CO2}
concentration in the Northern hemisphere winter (net \ac{CO2} flux into
atmosphere due to respiration) and a minimum during the Northern hemisphere
summer (net \ac{CO2} flux into the land due to photosynthesis). Another major
carbon exchange process connects the atmosphere and the ocean. Atmospheric
\ac{CO2} is exchanged with the surface ocean through gas exchange, which is
driven by the partial \ac{CO2} pressure difference between the air and the sea
\autocite{Ciais2013}.


\subsection{Anthropogenic Perturbations}
\label{subsec:02:carbon_cycle_perturbations}

Before the Industrial Era, the global carbon cycle was roughly in a dynamic
equilibrium, which means that exchange fluxes balanced each other and the
amount of carbon in the different reservoirs did neither increase nor decrease.
This can be inferred from ice core measurements, which show an almost constant
atmospheric \ac{CO2} concentration over the last several thousand years before
the Industrial Revolution in the \nth{19} century \autocite{Ciais2013}. Since
the beginning of the Industrial Era, humanity is constantly emitting
carbon-based \acp{GHG} (\eg{} \ac{CO2} and \ac{CH4}) into the atmosphere.
Especially the atmospheric \ac{CO2} concentration has substantially increased,
which has already been shown by Charles D. Keeling in 1976 by his continuous
\ac{CO2} measurements at Mauna Loa, Hawaii that started in 1958
\commentcite{Keeling1976}{see \cref{fig:02:keeling_curve}}. From 1958, the
atmospheric \ac{CO2} concentration at Mauna Loa has steadily increased by about
$100 \unit{ppm}$ to $410 \unit{ppm}$ in the year 2019 \autocite{Keeling2005}.
In addition to the steady increase, the so-called \emph{Keeling Curve} is
further superimposed with the seasonal \ac{CO2} cycle, which gives rise to
local maxima of the atmospheric \ac{CO2} concentration in the Northern
hemisphere winter and local minima in the Northern hemisphere summer
\commentcite{Keeling1995}{see \cref{subsec:02:carbon_cycle_overview}}. Due to
its location in the middle of the Pacific Ocean, the Mauna Loa Observatory
offers perfect conditions for \ac{CO2} measurements by being far away from big
population centers. Moreover, its elevation of more than $3000 \unit{m}$
provides access to the well-mixed air of the Pacific Ocean in high altitudes,
which prevents any interference from the vegetation present on the Hawaiian
Islands.

\begin{figure}[t]
  \centering
  \includegraphics[width=\FigureWidth{}]{
    ch02_scientific_background/figs/keeling_curve.pdf}
  \caption{The Keeling Curve: monthly-mean atmospheric \acs{CO2} concentration
    at the Mauna Loa Observatory, Hawaii ($19.5 \unit{\degree N}$, $155.6
    \unit{\degree W}$; elevation: $3397 \unit{m}$) from 1958 to 2019
    \autocite{Keeling2005}. The steady increase of the atmospheric \acs{CO2}
    concentration is superimposed with a seasonal oscillation caused by the
    seasonal \acs{CO2} cycle (see \cref{subsec:02:carbon_cycle_overview}).}
  \label{fig:02:keeling_curve}
\end{figure}

Apart from warming the Earth by altering its radiation budget, the
anthropogenically emitted \ac{CO2} directly influences the carbon exchange
fluxes of the global carbon cycle. Due to the excessive carbon in the
atmosphere, there is now a net carbon flux from the atmosphere into the land
and ocean reservoirs (see \cref{fig:02:carbon_cycle_perturbation}). Thus, the
carbon cycle is not in a steady state anymore. In the decade
\range{2009}{2018}, anthropogenic activities caused net carbon fluxes of $3.2
\unit{\GtCyr{}}$ from the atmosphere into the terrestrial biosphere due to
increased plant photosynthesis and $2.5 \unit{\GtCyr{}}$ from the atmosphere
into the ocean due to increase dissolution of \ac{CO2} into the sea
\autocite{Friedlingstein2019}. In the same time, the amount of carbon in the
atmosphere reservoir increased with a rate of $4.9 \unit{\GtCyr{}}$, indicating
that only about half of the anthropogenic \ac{CO2} emissions in the last decade
remained in the atmosphere \autocite{Friedlingstein2019} where they can act as
\ac{GHG}.

\begin{figure}[t]
  \centering
  \includegraphics[width=\LargeFigureWidth{}]{
    ch02_scientific_background/figs/carbon_cycle_perturbation.png}
  \caption{Schematic representation of the overall perturbation of the global
    carbon cycle caused by anthropogenic activities, averaged globally for the
    decade \range{2009}{2018}. Arrows represent carbon exchange fluxes; circles
    carbon reservoirs. More details are given in the legend of this figure.
    Taken from \textcite{Friedlingstein2019}.}
  \label{fig:02:carbon_cycle_perturbation}
\end{figure}

Thus, this removal of \ac{CO2} from the atmosphere actively slows down global
warming. However, whether this benefit will persist in the future remains
unclear, which is primarily linked to two feedback processes connecting the
physical climate system and the global carbon cycle: the
\emph{concentration-carbon feedback} and the \emph{climate-carbon feedback}
\autocite{Friedlingstein2006, Gregory2009, Collins2013}. For the terrestrial
biosphere, the concentration-carbon feedback is connected to the \emph{\ac{CO2}
  fertilization effect} \autocite{Walker2020}, that causes an increase of
photosynthesis rates when the atmospheric \ac{CO2} concentration increases,
which in turns removes \ac{CO2} from the atmosphere, forming a negative
feedback. For the ocean, the concentration-carbon feedback is negative as well.
In this case, an elevated atmospheric \ac{CO2} concentration causes an
increased dissolution of \ac{CO2} into the sea, which increases the ocean
carbon uptake. On the other hand, the climate-carbon feedback is thought to be
positive for both the terrestrial biosphere and the ocean
\autocite{Gregory2009}. In the first case, temperature and precipitation
changes due to anthropogenic activities decrease the land carbon uptake because
of increased temperature and water stress on photosynthesis and higher
ecosystem respiration costs, which accelerates global warming due to more
\ac{CO2} that remains in the atmosphere. For the ocean, increased temperatures
lead to a reduction of vertical transport in the ocean resulting from increased
stability and reduced solubility of \ac{CO2} in the sea, which reduces the
ocean carbon uptake and enhances climate change \autocite{Gregory2009}.


\section{Techniques to Reduce Uncertainties in Climate Model Projections}
\label{sec:02:techniques}

- Problems with multi-model analysis (search for refs in Schlund et al., ESD,
2020) -> models are not independent, multi-model ensemble is not "real"
statistical sample

- Process-based weighting (Knutti + MDER)

- emergent constraints

- Wenzel et al., Nature (2016) as example (refer to carbon cycle chapter)

- Review of ECS emergent constraints is given in chapter on Schlund et al., ESD
(2020)