%=============================================================================%
%                                Dissertation                                 %
%                               Manuel Schlund                                %
%                                  (c) 2020                                   %
%=============================================================================%
%                            Scientific Background                            %
%=============================================================================%



\chapter{Scientific Background}
\label{ch:02:scientific_background}

This chapter introduces the scientific background necessary for the work
presented in this thesis. After a brief introduction to climate model
simulations and associated uncertainties, we present state-of-the art
techniques used to evaluate climate model simulations and reduce uncertainties
in projections of the future climate. These methods form the basis for the new
techniques developed in this thesis.


\section{\aclp{ESM}: Simulations and Analysis}
\label{sec:02:esms}

In contrast to other fields of science, researching the future evolution of the
Earth's climate cannot be purely done by performing experiments in a
laboratory. Due to the immense complexity of the Earth system (including
physical, biological and chemical processes on various temporal and spatial
scales), we do not have access to a mini version of the Earth that we can
expose to varying external conditions and analyze its response to it
\autocite{Flato2011}. While observing the current state of the Earth System is
(relatively) straightforward, gaining evidence about the future climate by only
considering present-day observations is rather difficult. A possible way out is
given by numerical climate models, which offer the possibility to simulate the
Earth's climate on a computer. The first numerical climate models came up in
the 1960s and were based on weather prediction models \autocite{Flato2011}.
Early models from the 1970s simulated only the physical components of the
climate system: atmosphere, land surface, ocean and sea ice (see
\cref{fig:02:esms_historical_evolution}). The main aim of these so-called
\acp{AOGCM} \autocite{Flato2013} is to numerically solve the differential
equations that describe the exchange of energy and matter between these
physical components.

\begin{figure}[t]
  \centering
  \includegraphics[width=\figureWidth{}]{
    ch02_scientific_background/figs/esms_historical_evolution.pdf}
  \caption{Historical evolution of coupled climate models over the last 45
    years. In early days, these models were so-called \acfp{AOGCM} and only
    included three components: the atmosphere, the land surface and the ocean.
    Over the time, the individual components grew in complexity and included a
    wider range of processes (illustrated by the growing cylinders). Eventually,
    more and more components (aerosols, carbon cycle, \etc{}) were added to the
    coupled system, forming the modern \acfp{ESM}. Taken from
    \textcite{Cubasch2013}.}
  \label{fig:02:esms_historical_evolution}
\end{figure}

Over the course of the years, climate models became more and more complex by
including a wider range of processes within the components, but also by
introducing new components to the coupled system. Examples of these are
aerosols, the carbon cycle, a dynamic vegetation, atmospheric chemistry and
land ice (see \cref{fig:02:esms_historical_evolution}). \acp{AOGCM} coupled to
these additional components are called \acfp{ESM}, which are the current
state-of-the-art models that allow the most sophisticated simulations of the
Earth's climate. In contrast to \acp{AOGCM}, \acp{ESM} enable the simulation of
biological and chemical processes in addition to the dynamics of the physical
components of the Earth system. Especially in the context of anthropogenic
climate change, these additional processes are of uttermost importance for
realistic climate model simulations, since the anthropogenic interference with
the Earth system directly influences the various biogeochemical cycles of the
Earth. For example, the emission of the most prominent \ac{GHG}, \ac{CO2},
immediately impacts the global carbon cycle by introducing an additional carbon
source to the cycle (for details see \cref{sec:02:carbon_cycle}). Due to the
global carbon cycle, only about \pct{50} of the emitted \ac{CO2} remains in the
atmosphere, where it can act as \ac{GHG} by introducing an additional radiative
forcing to the Earth System, which eventually leads to increasing surface
temperatures. Thus, this uptake of \ac{CO2} by the global carbon cycle slows
down global warming. Further examples include land use changes like the
deforestation of tropical rainforests, which also directly influences several
biogeochemical cycles (\eg{} carbon cycle, nitrogen cycle, phosphorus cycle,
\etc{}) by altering respective sinks and sources.


\section{Techniques to reduce uncertainties in climate model projections}
\label{sec:02:techniques}


\section{The global Carbon Cycle}
\label{sec:02:carbon_cycle}