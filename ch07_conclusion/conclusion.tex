%=============================================================================%
%                                Dissertation                                 %
%                               Manuel Schlund                                %
%                                  (c) 2020                                   %
%=============================================================================%
%                                 Conclusion                                  %
%=============================================================================%



\chapter{Conclusion}
\label{ch:07:conclusion}


\section{Overall Summary}
\label{sec:07:overall_summary}

The analysis of future climate projections from numerical climate model
simulations is of paramount importance for an accurate quantification of the
future climate change. Since this involves metrics of public interest like the
allowable fossil fuel emissions to meet particular warming targets, for example
the $1.5 \unit{\degreeCelsius}$ of the Paris Agreement \autocite{UNFCCC2015},
this scientific question is not only relevant to climate scientists but to
policymakers and the whole human society. In the light of the large spread of
climate sensitivity in the most recent generation of climate models from
\acs{CMIP}6 \autocite{Meehl2020}, a careful statistical evaluation and
refinement of the output of multi-model climate projections is as relevant as
ever. This thesis quantifies the associated uncertainties, presents the
evaluation of established methods to reduce these uncertainties in the new
climate model ensemble and finally introduces a novel alternative technique
based on supervised \ac{ML}.

To answer the \namecrefs{enum:01:question_1} posed in
\cref{sec:01:key_science_questions} and to ensure a consistent evaluation of
the participating climate model ensembles in this thesis, the \ac{ESMValTool}
is used, an open-source community diagnostics and performance metrics tool for
the routine evaluation of \acp{ESM}. All analyses shown in this thesis are
performed with the \ac{ESMValTool}. Apart from that, further substantial
changes and additions to the code base of the tool have been implemented (see
\cref{ch:03:esmvaltool}), which lead to co-authorship in the scientific
documentation of the \ac{ESMValTool} \autocite{Eyring2020, Lauer2020,
  Righi2020, Weigel2020}. Since the open-source tool is freely available for
everyone, these improvements are beneficial for the entire scientific
community.

In a first analysis, we evaluate the climate sensitivity metrics \ac{ECS} and
\ac{TCR} for the latest generation of \acp{ESM} from \acs{CMIP}6. This work,
which is presented in \cref{ch:04:papers_ecs_tcr_assessment} and already
published in \textcite{Bock2020, Meehl2020}, directly addresses
\cref{enum:01:question_1} (\enquote{\emph{\KeyScienceQuestionOne{}}}). For
\ac{ECS}, we find a \acs{CMIP}6 model range of $\rangeunit{1.8}{5.6}{K}$, which
is higher on the upper and lower end than any model range from previous
\ac{CMIP} generations before. In comparison to \acs{CMIP}5, the \acs{CMIP}6
\ac{MMM} of \ac{ECS} is about $16 \unit{\%}$ higher ($3.74 \unit{K}$ in
\acs{CMIP}6 versus $3.23 \unit{K}$ in \acs{CMIP}5). Moreover, especially the
upper bound of the \acs{CMIP}6 models range is considerably larger than in the
assessed range of $\rangeunit{1.5}{4.5}{K}$ given by the latest published
\ac{AR} of the \ac{IPCC} from 2013 \autocite{Stocker2013}. The assessed upper
bound of $4.5 \unit{K}$ is exceeded by about a third of the \acs{CMIP}6 models,
with many models showing \ac{ECS} values above $5 \unit{K}$. For \ac{TCR}, the
model range of \acs{CMIP}6 is $\rangeunit{1.3}{3.0}{K}$, which also exceeds the
\acs{CMIP}5 range of $\rangeunit{1.1}{2.5}{K}$ and the assessed range from
\acs{AR}5 of $\rangeunit{1.0}{2.5}{K}$. A possible reason for the increased
climate sensitivity in many \acs{CMIP}6 models is a change in the microphysical
representation of mixed-phase clouds above the Southern Ocean. For one thing,
this change leads to an improved shortwave \ac{CRE} in these models when
compared to observations. However, this change also substantially reduces the
strong negative shortwave cloud feedback over the Southern Ocean that is
present in previous \ac{CMIP} generations and that results from a cloud phase
change from ice clouds in the present-day to liquid clouds in the future. In
the affected \acs{CMIP}6 models, this cloud phase change is not possible
anymore since the models predominantly simulate liquid clouds over the
present-day Southern Ocean.

In order to reduce this large range of \ac{ECS} in the latest generation of
climate  models, we evaluate suitable emergent constraints on the \acs{CMIP}6
ensemble. Emergent constraints use a physically-based inter-model relationship
between an observable quantity of the Earth system and a target variable to
reduce uncertainties in the target variable with observations. In total we
assess eleven emergent constraints on \ac{ECS}, which are mostly related to
cloud feedbacks since these constitute the most important source of uncertainty
for \ac{ECS} \autocite{Boucher2013, Flato2013}. Since all of the evaluated
emergent constraints have been derived on the \acs{CMIP}3 or \acs{CMIP}5
ensemble, we can perform out-of-sample tests on the emergent constraints by
assessing whether they still hold for the \acs{CMIP}6 models. In this study,
which is shown in \cref{ch:05:paper_ecs} and already published in
\textcite{Schlund2020a}, we find a substantial reduction of skill for the
majority of emergent constraints when applied to the \acs{CMIP}6 ensemble in
comparison to the \acs{CMIP}5 ensemble. This drop in skill is expressed as a
decrease of the coefficient of determination $R^2$ of the emergent relationship
and a decrease of the statistical significance using the null hypothesis that
there is no correlation between the predictor and \ac{ECS}. Moreover, the
corresponding \acp{PDF} for the emergent constraints show higher $66 \unit{\%}$
\ac{ECS} ranges ($\rangeunit{17}{83}{\%}$ confidence) for almost all emergent
constraints, resulting in values of $\rangeunit{1.32}{2.70}{K}$ for \acs{CMIP}6
(\acs{CMIP}5: $\rangeunit{1.16}{1.75}{K}$). Averaged over all emergent
constraints, this is an increase in the $66 \unit{\%}$ \ac{ECS} range of $51
\unit{\%}$. Similarly, the best estimates for \ac{ECS} show values of
$\rangeunit{2.97}{3.88}{K}$ in \acs{CMIP}5 and $\rangeunit{3.48}{4.32}{K}$ in
\acs{CMIP}6, resulting in an increase of about $12 \unit{\%}$ averaged over all
emergent constraints. Thus, \cref{enum:01:question_2}
(\enquote{\emph{\KeyScienceQuestionTwo{}}}) needs to be answered with a
\enquote{no} for the \acs{CMIP}6 ensemble. The increased best estimates and
spreads resulting from the emergent constraints in \acs{CMIP}6 are likely
related to the increased \ac{MMM} and multi-model spread of \ac{ECS} in
\acs{CMIP}6. A possible reason for the reduced skill of the emergent
constraints when applied to the \acs{CMIP}6 ensemble is the increased
complexity of the \acs{CMIP}6 models: A basic assumption for these
single-process-oriented emergent constraints is that a single observable
process dominates the uncertainty in \ac{ECS}, which might not be valid anymore
due to an increased number of processes that are included in the \acs{CMIP}6
models.

To overcome these issues of single-process-oriented emergent constraints,
\cref{ch:06:paper_gpp} introduces an alternative approach based on \ac{ML}.
This work is already published in \textcite{Schlund2020}. Since the new
technique relies on a large number of data points in order to train the \ac{ML}
algorithm, the scalar climate sensitivity expressed as \ac{ECS} or \ac{TCR} is
not an appropriate target variable. As an alternative, we study projections of
the future \ac{GPP}, which is the largest flux of the terrestrial carbon uptake
and slows down global warming by removing \ac{CO2} from the atmosphere. In the
first step of our new two-step approach, we constrain the global mean \ac{GPP}
at the end of the \nth{21} century in the \acs{CMIP}5 \acs{RCP}8.5 scenario to
$\pmrangeunit{171}{12}{\GtCyr{}}$ using a published emergent constraint by
\textcite{Wenzel2016}. This first step corrects the \acs{CMIP}5 models' GPP
response to \ac{CO2}, which shows a range of $\rangeunit{156}{248}{\GtCyr{}}$
in the raw multi-model ensemble. In the second step, a \ac{ML}-based climate
model weighting approach is used to further constrain the gridded \ac{GPP}
based on present-day predictors that are relevant for the simulation of
\ac{GPP} in the \acp{ESM}. Our \ac{ML} approach is mathematically similar to
the \ac{MDER} approach \autocite{Karpechko2013, Senftleben2020, Wenzel2016a},
but additionally considers multi-dimensional (gridded) target variables and
non-linear relationships between the predictors and the target variable. We
establish a relationship between process‐oriented predictors and future
projections of \ac{GPP} and then utilize this to project today's observed
conditions into the future. The prediction phase of our method can be
interpreted as an implicit performance weighting. However, due to the complex
structure of the used \ac{ML} algorithm (\ac{GBRT}) it is not possible to
extract specific values for the individual weights. We consider two target
variables: the gridded monthly climatologies of absolute \ac{GPP}
(\range{2091}{2100}) and the gridded fractional \ac{GPP} change over the
\nth{21} century (2100 versus 2000). The latter quantity shows an increased
\acs{GPP} change in the high latitudes compared to regions closer to the
equator. The results of both approaches are consistent with each other and with
the global constraint of the first step. We validate our approach by comparing
it to other statistical models (the \acs{CMIP}5 \ac{MMM} and a linear
\ac{LASSO} model) in a leave‐one‐model‐out \ac{CV} setup. Compared to \ac{MMM}
(\ac{LASSO}), we find a reduction of the resulting mean \ac{RMSEP} of up to $48
\unit{\%}$ ($3 \unit{\%}$) when using our \ac{ML} approach. Moreover, the
evaluation of the global and local feature importance allows us to get further
insights into the \ac{ML} model. For the first target variable (absolute
\ac{GPP}), historical \ac{GPP} is by far the most important predictor, which
can be explained with a correction of the historical bias in \ac{GPP} by our
approach. For the second variable, \ac{T} and \ac{LAI} are the dominant
features. This study directly addresses \cref{enum:01:question_3}
(\enquote{\emph{\KeyScienceQuestionThree{}}}), which we can answer with
\enquote{yes} based on the results presented here.


\section{Outlook}
\label{sec:07:outlook}

- Why GBRT not used for ECS?

- GBRT used for temperature anomaly based on other studies (emergent
constraints, historical T trend, weighting following Brunner et al.)

- Missing feedback process relevant for high ECS (Grise et al. 2016)?

- Causal model evaluation (Nowack et al., Runge et al.)?
