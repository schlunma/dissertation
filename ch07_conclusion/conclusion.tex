%=============================================================================%
%                            Doctoral Dissertation                            %
%                            (c) by Manuel Schlund                            %
%=============================================================================%

%=============================================================================%
% This work is licensed under a
% Creative Commons Attribution 4.0 International License.
%
% You should have received a copy of the license along with this
% work. If not, see <http://creativecommons.org/licenses/by/4.0/>.
%=============================================================================%



\chapter{Conclusion}
\label{ch:07:conclusion}


\section{Overall Summary}
\label{sec:07:overall_summary}

The analysis of future climate projections from numerical climate model
simulations is of paramount importance to assess future climate change under
different forcing scenarios. Since this involves metrics of public interest
like the allowable fossil fuel emissions to meet particular warming targets,
for example the $1.5 \unit{\degreeCelsius}$ of the Paris Agreement
\autocite{UNFCCC2015}, research in this field of science is not only relevant
to climate scientists but to policymakers and human society as a whole. In the
light of the large spread of climate sensitivity in the most recent generation
of climate models from \acs{CMIP}6, that has been demonstrated in
\cref{ch:04:papers_ecs_tcr_assessment}, a careful statistical evaluation and
refinement of the output of multi-model climate projections is as relevant as
ever. This thesis quantifies uncertainties in climate model projections,
presents the evaluation of established methods to reduce these uncertainties in
a new climate model ensemble, and finally introduces a novel alternative
technique based on supervised \ac{ML}.

To answer the \lcnamecrefs{enum:01:question_1} posed in
\cref{sec:01:key_science_questions} and to ensure a consistent evaluation of
the participating climate model ensembles in this thesis, the \ac{ESMValTool}
is used, an open-source community diagnostics and performance metrics tool for
the routine evaluation of \acp{ESM}. All analyses shown in this thesis are
performed with the \ac{ESMValTool}. Apart from that, further substantial
changes and additions to the code base of the tool have been implemented as
part of this thesis (see \cref{ch:03:esmvaltool}), which led to co-authorship
in the technical and scientific documentation of the \ac{ESMValTool}
\autocite{Eyring2020, Lauer2020, Righi2020, Weigel2020}. Since the open-source
tool is freely available, the code that has been implemented as part of this
thesis is beneficial for the entire scientific community.

In the first study of this thesis, the climate sensitivity metrics \ac{ECS} and
\ac{TCR} are evaluated for the latest generation of \acp{ESM} from \acs{CMIP}6.
This work, which is presented in \cref{ch:04:papers_ecs_tcr_assessment} and
already published in \textcite{Bock2020, Meehl2020}, directly addresses
\Cref{enum:01:question_1} (\enquote{\KeyScienceQuestionOne{}}). For \ac{ECS}, a
\acs{CMIP}6 model range of $\rangeunit{1.8}{5.6}{K}$ is found, which is higher
on the upper and lower bound than any model range from previous \ac{CMIP}
generations. In comparison to \acs{CMIP}5, the \acs{CMIP}6 \ac{MMM} of \ac{ECS}
is about $16 \unit{\%}$ higher ($3.74 \unit{K}$ in \acs{CMIP}6 versus $3.23
\unit{K}$ in \acs{CMIP}5). Moreover, especially the upper bound of the
\acs{CMIP}6 model range is considerably larger than in the assessed range of
$\rangeunit{1.5}{4.5}{K}$ given by the latest published \ac{AR} of the
\ac{IPCC} from 2013 \autocite{Stocker2013}. The assessed upper bound of $4.5
\unit{K}$ is exceeded by a third of the \acs{CMIP}6 models, with many models
showing \ac{ECS} values above $5 \unit{K}$. For \ac{TCR}, the model range of
\acs{CMIP}6 is $\rangeunit{1.3}{3.0}{K}$, which also exceeds the \acs{CMIP}5
range of $\rangeunit{1.1}{2.5}{K}$ and the assessed range from \acs{AR}5 of
$\rangeunit{1.0}{2.5}{K}$. One possible reason for the increased climate
sensitivity in some \acs{CMIP}6 models is the addition of prognostic aerosol
schemes that include aerosol-cloud interactions, which might result in overly
large negative radiative forcing. In this case, a stronger response to \ac{GHG}
forcing is required to correctly reproduce the observed historical temperature
trend, resulting in a higher climate sensitivity in the corresponding models. A
further reason for the high \ac{ECS} in \acs{CMIP}6 is a change in the
microphysical representation of mixed-phase clouds in some \acs{CMIP}6 models.
This change was implemented to improve the representation of the fractioning of
cloud ice and cloud liquid in those models, \eg{} by allowing for (more)
supercooled cloud liquid water. In addition, the \acs{CMIP}6 \ac{MMM} shows an
improved simulation of the shortwave \ac{CRE} when compared to satellite
observations. The change in the representation of mixed-phase clouds in some
\acs{CMIP}6 models reduces the strong negative shortwave cloud feedback over
the Southern Ocean that is present in predecessor versions of these models and
that results from a cloud phase change from ice clouds in the present day to
liquid clouds in the future. In the affected \acs{CMIP}6 models, this cloud
phase change due to warming is reduced since these models simulate less cloud
ice over the present-day Southern Ocean than their predecessor versions with no
supercooled cloud liquid.

To reduce this large range of \ac{ECS} in the latest generation of climate
models, already-published emergent constraints that have been derived from
models from the previous \ac{CMIP} generations \acs{CMIP}3 and \acs{CMIP}5 are
evaluated for their skill in the \acs{CMIP}6 ensemble. Emergent constraints use
a physically based inter-model relationship between an observable quantity of
the Earth system and a target variable to reduce uncertainties in the target
variable with observations \autocite{Allen2002}. In total 11 emergent
constraints on \ac{ECS} are assessed, which are mostly related to cloud
feedbacks since these constitute the most important source of uncertainty for
\ac{ECS} \autocite{Boucher2013, Flato2013}. Since all of the evaluated emergent
constraints have been derived from the \acs{CMIP}3 or \acs{CMIP}5 ensemble,
out-of-sample tests on the emergent constraints can be performed by assessing
whether they still hold for the \acs{CMIP}6 models. In this study, which is
shown in \cref{ch:05:paper_ecs} and already published in
\textcite{Schlund2020a}, a substantial reduction of skill for the majority of
emergent constraints is found when applied to the \acs{CMIP}6 ensemble in
comparison to the \acs{CMIP}5 ensemble. This drop in skill is expressed as a
decrease of the coefficient of determination $R^2$ of the emergent relationship
and a decrease of the statistical significance using the null hypothesis that
there is no correlation between the predictor and \ac{ECS}. Moreover, the
corresponding \acp{PDF} for the emergent constraints show higher $66 \unit{\%}$
\ac{ECS} ranges ($\rangeunit{17}{83}{\%}$ confidence) for almost all emergent
constraints, resulting in values of $\rangeunit{1.32}{2.70}{K}$ for \acs{CMIP}6
(\acs{CMIP}5: $\rangeunit{1.16}{1.75}{K}$). Averaged over all emergent
constraints, this is an increase in the $66 \unit{\%}$ \ac{ECS} range of $51
\unit{\%}$. Similarly, the best estimates for \ac{ECS} show values of
$\rangeunit{2.97}{3.88}{K}$ in \acs{CMIP}5 and $\rangeunit{3.48}{4.32}{K}$ in
\acs{CMIP}6, resulting in an increase of about $12 \unit{\%}$ averaged over all
emergent constraints. Thus, \Cref{enum:01:question_2}
(\enquote{\KeyScienceQuestionTwo{}}) needs to be answered with a \enquote{not
  very well} for the \acs{CMIP}6 ensemble. The increased best estimates and
spreads resulting from the emergent constraints in \acs{CMIP}6 are likely
related to the increased \ac{MMM} and multi-model spread of \ac{ECS} in
\acs{CMIP}6. A possible reason for the reduced skill of the emergent
constraints when applied to the \acs{CMIP}6 ensemble is the increased
complexity of the \acs{CMIP}6 models: A basic assumption for these
single-process-oriented emergent constraints is that a single observable
process dominates the uncertainty in \ac{ECS}, which might not be valid anymore
for the latest generation of \acp{ESM} from \acs{CMIP}6 due to an increased
number of processes that are included in these models.

To overcome these issues of single-process-oriented emergent constraints,
\cref{ch:06:paper_gpp} introduces an alternative approach based on \ac{ML}.
This work is already published in \textcite{Schlund2020}. Since the new
technique relies on a large number of data points to train the \ac{ML}
algorithm, the scalar climate sensitivity expressed as \ac{ECS} or \ac{TCR} is
not an appropriate target variable. Therefore, this analysis does not focus on
reducing uncertainties in climate sensitivity itself but rather on a selected
process that contributes to it: \ac{GPP}. \Ac{GPP} is the largest flux of the
terrestrial carbon uptake and slows down global warming by removing \ac{CO2}
from the atmosphere. In the first step of the new two-step approach, a
published emergent constraint by \textcite{Wenzel2016} is used to constrain the
global mean \ac{GPP} at the end of the \nth{21} century in the \acs{CMIP}5
\acs{RCP}8.5 scenario to $\pmrangeunit{171}{12}{\GtCyr{}}$, compared to
unconstrained \acs{CMIP}5 model range of $\rangeunit{156}{247}{\GtCyr{}}$. This
first step corrects the \acs{CMIP}5 models' \ac{GPP} response to \ac{CO2},
which is primarily driven by the \ac{CO2} fertilization. In the second step, an
\acs{ML}-based climate model weighting approach is used to further constrain
the gridded \ac{GPP} based on present-day predictors that are relevant for the
simulation of \ac{GPP} in the \acp{ESM}. The \ac{ML} approach is mathematically
similar to the \ac{MDER} approach \autocite{Karpechko2013, Senftleben2020,
  Wenzel2016a}, but additionally considers multi-dimensional (gridded) target
variables and non-linear relationships between the predictors and the target
variables. A relationship between process-oriented predictors and future
projections of \ac{GPP} is established and then utilized to project today's
observed conditions into the future. The prediction phase of the new method can
be interpreted as an implicit performance weighting. However, due to the
complex structure of the used \ac{ML} algorithm (\ac{GBRT}) it is not possible
to extract specific values for the individual weights. Two target variables are
considered: the gridded monthly climatologies of absolute \ac{GPP}
(\range{2091}{2100}) and the gridded fractional \ac{GPP} change over the
\nth{21} century (2100 versus 2000). The latter quantity shows an increased
\acs{GPP} change in the high latitudes compared to regions closer to the
equator. The results of both approaches are consistent with each other and with
the global constraint of the first step. The new approach is validated by
comparing it to other statistical models (the \acs{CMIP}5 \ac{MMM} and a linear
\ac{LASSO} model) in a leave-one-model-out \ac{CV} setup. Compared to \ac{MMM}
(\ac{LASSO}), a reduction of the resulting mean \ac{RMSEP} of up to $48
\unit{\%}$ ($3 \unit{\%}$) is found when using the \ac{ML} approach. Moreover,
the evaluation of the global and local feature importance allows further
insights into the \ac{ML} model. For the first target variable (absolute
\ac{GPP}), historical \ac{GPP} is by far the most important predictor, which
can be explained with a correction of the historical bias in \ac{GPP} by the
new approach. For the second variable (fractional change in \ac{GPP}), \ac{T}
and \ac{LAI} are the most important features. A corresponding feature map shows
that \ac{T} dominates in the tropics and \ac{LAI} dominates in the northern
high latitudes, which might be attributed to an additional greening trend in
the high latitudes that enhances the \ac{GPP} increase in this region compared
to regions closer to the equator through an extension of the growing season.
This study directly addresses \Cref{enum:01:question_3}
(\enquote{\KeyScienceQuestionThree{}}), which can be confidently answered with
\enquote{yes} based on the results found.


\section{Outlook}
\label{sec:07:outlook}

Climate sensitivity is a policy-relevant and easy-to-use metric to assess the
strength of climate change. The range of model results for this important
climate metric, however, remains large and could not be narrowed down over the
last decades. In the latest generation of climate models contributing to
\acs{CMIP}6, the range has even increased \autocite{Meehl2020}.
\Cref{ch:04:papers_ecs_tcr_assessment} presents two possible reasons for this:
changes in the aerosol-cloud interaction and changes in the shortwave cloud
phase change feedback over the Southern Ocean. While these are likely
explanations, there might be additional relevant aspects that have not been
analyzed in detail yet. This could include additional feedback processes that
are not present in older \ac{CMIP} model generations. Obvious candidates for
these feedback processes are cloud-related feedbacks, which pose a major source
of uncertainty in climate sensitivity in modern-day \acp{ESM}
\autocite{Boucher2013}. Identifying and quantifying such feedback mechanisms
can further help to gain a better understanding of the \acs{CMIP}6 models and
potentially to reduce associated uncertainties in the entire multi-model
ensemble. A potentially interesting process relevant in this context is the
influence of the midlatitude jet position on clouds and \acp{CRE}
\autocite{Grise2016}.

A promising way forward to reduce uncertainties in multi-model climate
projections is to apply the new flexible \acs{ML}-based climate model weighting
approach introduced in \cref{ch:06:paper_gpp} to other target variables. While
the scalar metrics \ac{ECS} or \ac{TCR} with one value per climate model do not
offer enough training points for this technique, a possible alternative could
be the gridded \acl{T} over the \nth{21} century in the \ac{SSP} scenarios. A
relevant study in this context is given by \textcite{Brunner2020}, who use
performance- and interdependence-based climate model weighting to constrain the
\acl{T} over the \nth{21} century based on observable process-based predictors
of today's climate and which could serve as a possible baseline for the new
\acs{ML}-based weighting approach. An exciting and valuable addition to the
method could be causal inference \autocite{Nowack2020, Runge2019}. In the
current approach, the relationships between the different predictors and the
target variable are based on statistical correlation alone. Moreover, the local
feature importance maps do not reveal true causal relationships between the
predictors and the target variable, but only show the relative weight that is
given to a specific predictor at a specific location for the prediction of the
target variable by the \ac{ML} model. By integrating causal networks into the
new approach, it might be possible to explore and utilize true causal
connections instead.

The massive progress in \ac{AI} and \ac{ML} in recent years in combination with
ever increasing computational power and resources provided by modern
supercomputers has the potential for enormous improvements in climate modeling
and analysis \autocite{Reichstein2019}. Apart from the presented \acs{ML}-based
weighting approach, further possible applications of \ac{AI} and \ac{ML} in
climate science involve for example \acs{ML}-based parameterizations that are
learned from high-resolution climate models \autocite{Gentine2018, Rasp2018},
\acs{ML}-based analysis and prediction of forcing patterns
\autocite{Barnes2019, Mansfield2020}, or learning of entire \acp{ESM} from
observational products \autocite{Geer2021}. Therefore, the foundation for
innovative and groundbreaking research in climate science for the near future
has been laid, which will help to further understand the Earth system and fight
one of the greatest challenges for humankind today: climate change.
