%=============================================================================%
%                                Dissertation                                 %
%                               Manuel Schlund                                %
%                                  (c) 2020                                   %
%=============================================================================%
%                                 ESMValTool                                  %
%=============================================================================%



\chapter{Improving Routine Climate Model Evaluation}
\label{ch:03:esmvaltool}

In order to answer the research questions of this thesis posed in
\cref{sec:01:research_questions}, a reliable and efficient tool to read,
process and evaluate climate model output and observational data is necessary.
A valuable software that fits these criteria is the \ac{ESMValTool}. The
\ac{ESMValTool} is an open-source community diagnostics and performance metrics
tool for the routine evaluation of \ac{ESM} output, which notably facilitates
the analysis of \ac{CMIP} models (\url{https://www.esmvaltool.org}). For this
reason, all analyses presented in
\cref{ch:04:assessment_climate_metrics,ch:05:paper_ecs,ch:06:paper_gpp} of this
thesis were implemented into the \ac{ESMValTool}. Apart from that, further
substantial changes and additions to the code base of the \ac{ESMValTool} have
been contributed that improve the routine evaluation of climate models, which
is beneficial for the entire scientific community. This lead to co-authorship
in the scientific documentation of the \ac{ESMValTool}, which is published in
four peer-reviewed studies \autocite{Eyring2020, Lauer2020, Righi2020,
  Weigel2020}. After a brief overview over the \ac{ESMValTool}'s structure,
this chapter presents all these contributions that are not documented in other
chapters of this thesis.


\section{The \acf{ESMValTool}}
\label{sec:03:esmvaltool}

As shown in \cref{subsec:02:climate_modeling}, climate models have been
continuously improved and extended over the last decades from the relatively
simple \acp{AOGCM} to the complex state-of-the-art \acp{ESM} that include an
immense number of variables and processes. In \acs{CMIP}6, more modeling
institutes provide data for more versions of these complex models running
simulations for more experiments. Consequently, the data volume of the entire
\acs{CMIP}6 archive is expected to reach up to $80 \unit{PB}$
\autocite{Balaji2018}, which is a vast increase in comparison to the $2
\unit{PB}$ of \acs{CMIP}5. The increasing complexity and data volume of the
climate models pose a major challenge for the evaluation and analysis of the
model output. To address this big data challenge and support the scientific
community, the \ac{ESMValTool} has been developed to provide an
\enquote{open-source, standardized, community-based software package for the
  systematic, efficient, and well-documented analysis of \ac{ESM} results}
\autocite{Righi2020}. Moreover, the \ac{ESMValTool} allows a routine comparison
of single or multiple climate models against predecessor versions and/or
observations.

\begin{figure}[t]
  \centering
  \includegraphics[width=\LargeFigureWidth{}]{
    ch03_esmvaltool/figs/esmvaltool_schematic.pdf}
  \caption{Schematic representation of the \acf{ESMValTool}. First, input data
    is preprocessed by the \acf{ESMValCore}. Second, the preprocessed data is
    read by diagnostic scripts, which create the final output (\eg{} plots).
    The user can specify the desired input datasets, preprocessing operations
    and diagnostic scripts in the recipe, which is a configuration file that
    controls the main workflow of the \acs{ESMValTool}. Taken and modified from
    \textcite{Righi2020}.}
  \label{fig:03:esmvaltool_schematic}
\end{figure}

Since its first release in 2016 \autocite{Eyring2016a}, the \ac{ESMValTool} has
been greatly extended and improved. A schematic of the current structure of the
\ac{ESMValTool} is illustrated in \cref{fig:03:esmvaltool_schematic}. The first
major element of this workflow is an extensive preprocessing of the input data.
For this, the Python-based \acf{ESMValCore} provide a set of operators
(\emph{preprocessors}) that are applied to the input data. There are two
classes of preprocessors: non-optional and optional ones. Examples for
non-optional preprocessors are the \ac{I/O} operations \emph{load} (loading
input data from climate models and observations) and \emph{save} (saving the
preprocessed data), which utilize the \ac{netCDF} format (a self-describing,
machine-independent binary file format used for the storage of array-based
scientific data). Further non-optional preprocessors include \emph{checks} and
\emph{fixes} of the input data, which test whether the input data adheres to
standards given by the \ac{CMOR} format and fixes the data if necessary. The
\ac{CMOR} format ensures that the output from the many different modeling
institutes within a generation of \ac{CMIP} follows common standards. The
optional preprocessors include commonly used operations on spatiotemporal
datasets, like spatial and temporal \emph{subsetting}, the calculation of
\emph{spatial and temporal statistics}, horizontal and vertical
\emph{interpolation}, land/sea/ice \emph{masking}, \emph{unit conversion}, or
the calculation of \emph{multi-model statistics}. A further example is the
\emph{variable derivation}, which can be used to derive non-\ac{CMOR} variables
from the input data. The aim of the preprocessor is to facilitate the routine
evaluation of climate models by providing a set of commonly used data
operations. To reduce computation times, \ac{ESMValCore} allows parallel
preprocessing of different datasets.

The second main element of the \ac{ESMValTool} workflow is the calculation of
\emph{diagnostics}. This task is performed by the diagnostic scripts, which can
be written in multiple programming languages. Currently, the languages Python,
NCL, R and Julia are supported. The diagnostic scripts, which contain the code
that runs the actual scientific evaluation, read the preprocessed datasets from
\ac{ESMValCore} and use these to create the final output of the tool. Apart
from \ac{netCDF} files and plots, this includes log files with provenance
information that ensure the reproducibility and transparency of the results.
The whole workflow of the \ac{ESMValTool} is controlled with the \emph{recipe},
which is a configuration file in which the user can specify the desired input
datasets, preprocessing operations and diagnostic scripts. Implementing a new
scientific assessment into the \ac{ESMValTool} usually includes writing a new
recipe and one or more diagnostic scripts. A third major element of the
\ac{ESMValTool} is the \emph{\ac{CMOR}ization} (\ie{} the reformatting of data
so it adheres to \ac{CMOR} standards) of raw observational datasets (not shown
in \cref{fig:03:esmvaltool_schematic}). This ensures that the \ac{ESMValTool}
can process arbitrary observational datasets, which can for example be used to
assess the skill of climate model simulations. More details on the structure of
the \ac{ESMValTool} are given in its extensive documentation
(\url{https://esmvaltool.readthedocs.io}).

To ensure a rapid evaluation of the \acs{CMIP}6 models, the \ac{ESMValTool} is
fully integrated into the infrastructure of the \ac{ESGF}, which provides the
\ac{CMIP} model output for the general public \autocite{Eyring2016b}. As soon
as new model data is published on the \ac{ESGF} servers, it can be accessed
with the \ac{ESMValTool} and analyzed. This instantaneous evaluation of the
\ac{CMIP} models is urgently needed since there is a growing dependency on
\ac{CMIP} products by a broad research community and by national and
international climate assessments. For this reason, the \ac{ESMValTool} is for
example used to evaluate climate model output and observational data in several
chapters of the upcoming Sixth \ac{AR} of the \ac{IPCC}.

The \ac{ESMValTool} is developed open-source on GitHub
(\url{https://github.com/ESMValGroup}). It is released under the Apache
License, version 2.0. The source code of the latest released version of the
\ac{ESMValCore} package, which includes the core functionalities of the
\ac{ESMValTool}, is publicly available at Zenodo \autocite{Andela2020a}.
Similarly, the source code of the latest released version of the
\ac{ESMValTool} package, which includes the recipe and diagnostic scripts that
can for example be used to reproduce scientific assessments, is also available
at Zenodo \autocite{Andela2020}.


\section{Contributions to \acs{ESMValCore}}
\label{sec:03:contributions_to_esmvalcore}

As of December 2020, $40408$ lines of code have been added and $21050$ lines of
code have been removed from to the GitHub repository of \ac{ESMValCore} by the
author of this thesis. Apart from general improvements of the code base, these
changes and additions mainly include new preprocessor functions that can be
applied to the input data and derivation scripts for new non-\ac{CMOR}
variables. \Cref{tab:03:changes_to_esmvalcore} shows a summary of these main
contributions, which are partly published in the scientific documentation of
\ac{ESMValCore} \autocite{Righi2020}.

\begin{table}[t]
  \centering
  \begin{tabulary}{\columnwidth}{L L L}
    \toprule
    Type & Name & Description \\
    \midrule
    Preprocessor & \emph{amplitude} & Amplitude of periodic phenomena (\eg{}
    diurnal or seasonal cycles) \\
    \cmidrule(l){2-3}
    & \emph{land/sea fraction weighting} & Weighting of fields based on
    the land or sea fraction of the respective grid cells \\
    \cmidrule(l){2-3}
    & \emph{trend} & Linear trend (slope of ordinary least squares regression)
    \\
    \midrule
    Derived variable & \emph{asr} & Absorbed shortwave radiation \\
    \cmidrule(l){2-3}
    & \emph{co2s} & Atmospheric \ac{CO2} concentration at
    surface \\
    \cmidrule(l){2-3}
    & \emph{et} & Evapotranspiration \\
    \cmidrule(l){2-3}
    & \emph{rlntcs} & Clear-sky net \acf{TOA} longwave radiation \\
    \cmidrule(l){2-3}
    & \emph{rsntcs} & Clear-sky net \ac{TOA} shortwave radiation \\
    \cmidrule(l){2-3}
    & \emph{uajet} & Position of austral jet stream \\
    \bottomrule
  \end{tabulary}
  \caption{Summary of changes and features contributed to the
    \acf{ESMValCore}.}
  \label{tab:03:changes_to_esmvalcore}
\end{table}

In total, three new preprocessor functions have been added: \emph{amplitude},
\emph{land/sea fraction weighting} and \emph{trend}. The \emph{amplitude}
preprocessor calculates the peak-to-peak amplitude of periodic phenomena along
an arbitrary coordinate. Usually, this coordinate is a temporal dimension. A
common application of this preprocessor is the calculation of a variable's
diurnal or seasonal cycle amplitude. The \emph{land/sea fraction weighting}
preprocessor weights fields with the land or sea fraction of the respective
grid cells. For example, this is necessary for the spatial integration of
flux-related variables which are reported in units of \enquote{per square meter
  of land/sea} and not in \enquote{per square meter of grid cell}. After the
weighting, the grid cell areas can be used to integrate the flux-related
variable over a desired region. The \emph{trend} preprocessor calculates the
linear trend of a variable along an arbitrary coordinate. The linear trend is
defined as the slope of an ordinary least squares linear regression of the
variable against the selected coordinate. For example, this can be used to
calculate the temporal trend of the \ac{GSAT} over the \nth{20} century.

\begin{figure}[t]
  \centering
  \begin{subfigure}[t]{\SubfigureWidth{}}
    \includegraphics[width=\columnwidth]{ch03_esmvaltool/figs/co2s_map.pdf}
    \caption{}
    \label{fig:03:co2s:a}
  \end{subfigure}
  ~
  \begin{subfigure}[t]{\SubfigureWidth{}}
    \includegraphics[width=\columnwidth]{
      ch03_esmvaltool/figs/co2s_mlo_cycle.pdf}
    \caption{}
    \label{fig:03:co2s:b}
  \end{subfigure}
  \caption{(a) \acs{CMIP}6 \acf{MMM} of the global atmospheric \acs{CO2}
    concentration at surface (\emph{co2s}) averaged over the months \acf{JJA}
    of the year 2014 in the emission-driven historical simulation. The
    \acs{CMIP}6 \ac{MMM} includes the four climate models that provide all
    data needed for the calculation of \emph{co2s} (see legend in (b)). (b)
    Monthly-mean \emph{co2s} at Mauna Loa, Hawaii ($19.5 \unit{\degree N}$,
    $155.6 \unit{\degree W}$; elevation: $3397 \unit{m}$) from 1958 to 2014.
    The thick black line shows observations from the Mauna Loa Observatory
    \autocite{Keeling2005}; the remaining colored lines show emission-driven
    historical simulations from individual \acs{CMIP}6 models.}
  \label{fig:03:co2s}
\end{figure}

Apart from new preprocessor functions, six derivation scripts for the
non-\ac{CMOR} variables \emph{asr}, \emph{co2s}, \emph{et}, \emph{rlntcs},
\emph{rsntcs} and \emph{uajet} have been added. The absorbed shortwave
radiation (\emph{asr}) is defined as the difference in the incoming \ac{TOA}
shortwave radiation and the outgoing \ac{TOA} shortwave radiation. The
atmospheric \ac{CO2} concentration at surface (\emph{co2s}) can be calculated
from the pressure level-dependent atmospheric \ac{CO2} concentration and the
surface air pressure using interpolation. An example illustrating \emph{co2s}
is given in \cref{fig:03:co2s}. \Cref{fig:03:co2s:a} shows the \acs{CMIP}6
\ac{MMM} of the global \emph{co2s} for the months \acf{JJA} of the year 2014 in
the emission-driven historical simulation. As expected, the map shows high
atmospheric \ac{CO2} concentrations over large metropolitan areas with high
\ac{CO2} emissions (\eg{} over North America, the Arabian Peninsula and East
Asia). Moreover, regions with high photosynthetic activity (\eg{} the boreal
forests in the Northern hemisphere summer and the tropical rainforests in South
America) exhibit smaller \ac{CO2} concentrations. \Cref{fig:03:co2s:b} shows
the monthly-mean \emph{co2s} at Mauna Loa, Hawaii from 1958 to 2014 for
observations performed at the Mauna Loa Observatory \autocite{Keeling2005}
(thick black line) and emission-driven historical simulations from four
\acs{CMIP}6 models (remaining colored lines). Apart from the model CNRM-ESM2-1,
the simulated atmospheric \ac{CO2} concentrations show the expected seasonal
cycle present in the Keeling Curve (see \cref{fig:02:keeling_curve}). All
climate models correctly simulate the increase in \emph{co2s} over the years.
However, their are some differences in the absolute values with the model
BCC-CSM2-MR showing the largest deviations.

The evapotranspiration (\emph{et}) defined as the sum of evaporation and plant
transpiration can be calculated from the surface latent heat flux and the
latent heat vaporization constant. The longwave/shortwave clear-sky net
\acf{TOA} radiations (\emph{rlntcs}/\emph{rsntcs}) is defined as the difference
between the longwave/shortwave incoming \ac{TOA} radiation assuming clear-sky
and the longwave/shortwave outgoing \ac{TOA} radiation assuming clear-sky.
Finally, the position of the austral jet stream (\emph{uajet}) is given by the
latitude with maximum zonal-mean eastward wind speed at $850 \unit{hPa}$ on the
Southern hemisphere between $80 \unit{\degree S}$ and $30 \unit{\degree S}$.


\section{Contributions to \acs{ESMValTool}}
\label{sec:03:contributions_to_esmvaltool}

- total number of changed lines of code

- Added recipes and diagnostics for ESMValTool + table

* Anav13

* MDER

* TCR

* ECS

* Schlund20jgr

* Schlund20ESD

* Flato13
