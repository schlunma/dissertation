%=============================================================================%
%                            Doctoral Dissertation                            %
%                            (c) by Manuel Schlund                            %
%=============================================================================%

%=============================================================================%
% This work is licensed under a
% Creative Commons Attribution 4.0 International License.
%
% You should have received a copy of the license along with this
% work. If not, see <http://creativecommons.org/licenses/by/4.0/>.
%=============================================================================%



\chapter{Improving Routine Climate Model Evaluation}
\label{ch:03:esmvaltool}

To address the \lcnamecrefs{enum:01:question_1} of this thesis posed in
\cref{sec:01:key_science_questions}, a reliable and efficient tool to read,
process, and evaluate climate model output and observational data is necessary.
A valuable software package that fits these criteria is the \ac{ESMValTool}.
The \ac{ESMValTool} is an open-source community diagnostics and performance
metrics tool for the routine evaluation of \ac{ESM} output, which notably
facilitates the analysis of \ac{CMIP} models
(\url{https://www.esmvaltool.org}). For this reason, all analyses presented in
\cref{ch:04:papers_ecs_tcr_assessment,ch:05:paper_ecs,ch:06:paper_gpp} of this
thesis have been implemented into the \ac{ESMValTool}. Apart from that, further
substantial changes and additions to the code base of the \ac{ESMValTool} have
been contributed that improve the routine evaluation of climate models, which
is beneficial for the entire scientific community. This led to co-authorship in
the scientific documentation of the \ac{ESMValTool}, which is published in four
peer-reviewed studies \autocite{Eyring2020, Lauer2020, Righi2020, Weigel2020}.
After a brief overview of the \ac{ESMValTool}'s structure
(\cref{sec:03:esmvaltool}), this \lcnamecref{ch:03:esmvaltool} presents all
these contributions that are not documented in other
\lcnamecrefs{ch:04:papers_ecs_tcr_assessment} of this thesis and puts them into
context
(\cref{sec:03:contributions_to_esmvalcore,sec:03:contributions_to_esmvaltool}).


\section{The \acfAbstract{ESMValTool}}
\label{sec:03:esmvaltool}

As outlined in \cref{subsec:02:numerical_climate_modeling}, climate models have
been continuously improved and extended over the last decades from the
relatively simple \acp{AOGCM} to the complex state-of-the-art \acp{ESM} that
include an immense number of variables and processes. In \acs{CMIP}6, more
modeling institutes provide data for more versions of these complex models
running simulations for more experiments. Consequently, the data volume of the
entire \acs{CMIP}6 archive is expected to reach up to $80 \unit{PB}$
\autocite{Balaji2018}, which is a vast increase in comparison to the $2
\unit{PB}$ of \acs{CMIP}5. The increasing complexity and data volume of the
climate models pose a major challenge for the evaluation and analysis of the
model output. To address this big data challenge and support the scientific
community, the \ac{ESMValTool} has been developed to provide an
\enquote{open-source, standardized, community-based software package for the
  systematic, efficient, and well-documented analysis of \ac{ESM} results}
\autocite{Righi2020}. Moreover, the \ac{ESMValTool} allows a routine comparison
of single or multiple climate models against predecessor versions and/or
observations.

\begin{figure}[t]
  \centering
  \includegraphics[width=\LargeFigureWidth{}]{
    ch03_esmvaltool/figs/esmvaltool_schematic.pdf}
  \caption[
    Schematic representation of the structure of the \acf{ESMValTool}.
  ]{
    Schematic representation of the structure of the \acf{ESMValTool}. First,
    input data is preprocessed by the \acf{ESMValCore}. Second, the
    preprocessed data is read by diagnostic scripts, which create the final
    output (\eg{} plots). The user can specify the desired input datasets,
    preprocessing operations, and diagnostic scripts in the recipe, which is a
    configuration file that controls the main workflow of the \acs{ESMValTool}.
    \AdaptedFrom{Righi2020}.
  }
  \label{fig:03:esmvaltool_schematic}
\end{figure}

Since its first release in 2016 \autocite{Eyring2016a}, the \ac{ESMValTool} has
been greatly extended and improved. A schematic representation of the current
structure of the \ac{ESMValTool} is illustrated in
\cref{fig:03:esmvaltool_schematic}. The first major element of this workflow is
an extensive preprocessing of the input data. For this, the Python-based
\acf{ESMValCore} provide a set of operators (\emph{preprocessors}) that are
applied to the input data. There are two classes of preprocessors: non-optional
and optional preprocessors. Examples for non-optional preprocessors are the
input/output operations \emph{load} (loading input data from climate models and
observations) and \emph{save} (saving the preprocessed data), which utilize the
\ac{NetCDF} format (a self-describing, machine-independent binary file format
used for the storage of array-based scientific data). Further non-optional
preprocessors include \emph{checks} and \emph{fixes} of the input data, which
test whether the input data adheres to standards given by the \ac{CMOR} format
and fixes the data if necessary. The \ac{CMOR} format ensures that the output
from the many different modeling institutes within a generation of \ac{CMIP}
follows common standards. The optional preprocessors include commonly used
operations on spatiotemporal datasets, like spatial and temporal
\emph{subsetting}, the calculation of \emph{spatial and temporal statistics},
horizontal and vertical \emph{interpolation}, land/sea/ice \emph{masking},
\emph{unit conversion}, and the calculation of \emph{multi-model statistics}. A
further example is \emph{variable derivation}, which can be used to derive
non-\ac{CMOR} variables from the input data. The preprocessing functions of
\ac{ESMValCore} aim to facilitate the routine evaluation of climate models by
providing a set of commonly used data operations. To reduce computation times,
\ac{ESMValCore} allows parallel preprocessing of different datasets. However,
since this leads to higher memory usage, a compromise has to be found in
practice to optimize the use of time and memory resources.

The second main element of the \ac{ESMValTool} workflow is the calculation of
\emph{diagnostics}. This task is performed by the diagnostic scripts, which can
be written in multiple programming languages. Currently, the languages Python,
NCL, R, and Julia are supported. The diagnostic scripts, which contain the code
that runs the actual scientific evaluation, read the preprocessed datasets from
\ac{ESMValCore} and use these to create the final output of the tool. Apart
from \ac{NetCDF} files and plots, this includes log files with provenance
information that ensure the reproducibility and transparency of the results.
The whole workflow of the \ac{ESMValTool} is controlled with the \emph{recipe},
which is a configuration file in which the user can specify the desired input
datasets, preprocessing operations, and diagnostic scripts. Implementing a new
scientific assessment into the \ac{ESMValTool} usually includes writing a new
recipe and one or more diagnostic scripts. A third major element of the
\ac{ESMValTool} is the \emph{\acs{CMOR}ization} (\ie{} the reformatting of data
so it adheres to \ac{CMOR} standards) of raw observational datasets (not shown
in \cref{fig:03:esmvaltool_schematic}). This ensures that the \ac{ESMValTool}
can process arbitrary observational datasets, which can be used, for example,
to assess the skill of climate model simulations. More details on the structure
of the \ac{ESMValTool} are given in its extensive documentation
(\url{https://docs.esmvaltool.org}).

To ensure a rapid evaluation of the \acs{CMIP}6 models, the \ac{ESMValTool} is
fully integrated into the infrastructure of the \ac{ESGF}, which provides the
\ac{CMIP} model output for the general public \autocite{Eyring2016b}. As soon
as new model data is published on the \ac{ESGF} servers, it can be accessed
with the \ac{ESMValTool} and analyzed. This instantaneous evaluation of the
\ac{CMIP} models is urgently needed since there is a growing dependency on
\ac{CMIP} products by a broad research community and by national and
international climate assessments. For this reason, the \ac{ESMValTool} is used
to evaluate climate model output and observational data in several chapters of
the upcoming Sixth \acl{AR} (\acs{AR}6) of the \ac{IPCC}.

The \ac{ESMValTool} is developed open-source on GitHub, a web service that
provides hosting for version control with \emph{git}
(\url{https://github.com/ESMValGroup}). It is released under the Apache
License, Version 2.0. The source code of the latest released version of the
\ac{ESMValCore} package, which includes the core functionalities of the
\ac{ESMValTool}, is publicly available at Zenodo \autocite{Andela2020a}.
Similarly, the source code of the latest released version of the
\ac{ESMValTool} package, which includes all publicly available recipes and
diagnostic scripts, is also available at Zenodo \autocite{Andela2020}.


\section{Contributions to \acs{ESMValCore}}
\label{sec:03:contributions_to_esmvalcore}

As of \TheMonth{}, 40408 lines of code have been added and 21050 lines of code
have been removed from the GitHub repository of \ac{ESMValCore} by the author
of this thesis. Apart from general improvements of the code base, these changes
and additions mainly include new preprocessor functions that can be applied to
the input data and derivation scripts for new non-\ac{CMOR} variables.
\Cref{tab:03:changes_to_esmvalcore} shows a summary of these main
contributions, which are partly published in the scientific documentation of
\ac{ESMValCore} \autocite{Righi2020}.

\begin{table}[!b]
  \centering
  \begin{tabulary}{\columnwidth}{L >{\em}L L}
    \toprule
    Type & \ignorecolumntype{l}{Name} & Description \\
    \midrule
    Preprocessor & amplitude & Amplitude of periodic phenomena (\eg{} cycles)
    \\
    \cmidrule(l){2-3}
    & land/sea fraction weighting & Weighting of fields based on the land or
    sea fraction of the respective grid cells \\
    \cmidrule(l){2-3}
    & trend & Linear trend (slope of linear regression) \\
    \midrule
    Derived variable & asr & Absorbed shortwave radiation \\
    \cmidrule(l){2-3}
    & co2s & Atmospheric \acs{CO2} concentration at the surface \\
    \cmidrule(l){2-3}
    & et & Evapotranspiration \\
    \cmidrule(l){2-3}
    & rlntcs & Clear-sky net \acf{TOA} longwave radiation \\
    \cmidrule(l){2-3}
    & rsntcs & Clear-sky net \acs{TOA} shortwave radiation \\
    \cmidrule(l){2-3}
    & uajet & Position of austral jet stream \\
    \bottomrule
  \end{tabulary}
  \caption[
    Summary of new preprocessor functions and variable derivation scripts
    contributed to the \acf{ESMValCore} by the author of this thesis.
  ]{
    Summary of new preprocessor functions and variable derivation scripts
    contributed to the \acf{ESMValCore} by the author of this thesis.
  }
  \label{tab:03:changes_to_esmvalcore}
\end{table}

In total, three new preprocessor functions have been added: \emph{amplitude},
\emph{land/sea fraction weighting}, and \emph{trend}. The \emph{amplitude}
preprocessor calculates the peak-to-peak amplitude of periodic phenomena along
an arbitrary coordinate, which is usually time. A common application of this
preprocessor is the computation of a variable's diurnal or seasonal cycle
amplitude. For instance, this is necessary to calculate the predictor variable
(\ac{CO2} seasonal cycle amplitude) of the emergent constraint on \ac{CO2}
fertilization by \textcite{Wenzel2016} that is also used in
\cref{ch:06:paper_gpp} to constrain the global mean \ac{GPP} at the end of the
\nth{21} century. The \emph{land/sea fraction weighting} preprocessor weights
fields with the land or sea fraction of the respective grid cells. For example,
this is necessary for the spatial integration of flux-related variables which
are reported in units of \enquote{per square meter of land/sea} and not in
\enquote{per square meter of grid cell}. After the weighting, the grid-cell
areas can be used to integrate the flux-related variable over any desired
region. In this thesis, \emph{land/sea fraction weighting} is applied to
\ac{GPP} fields given by observational products and models to compute the
corresponding global mean results (see \cref{ch:06:paper_gpp}). The
\emph{trend} preprocessor calculates the linear trend of a variable along an
arbitrary coordinate. The linear trend is defined as the slope of an ordinary
least squares linear regression of the variable against the selected
coordinate. For example, this can be used to calculate the historical trend of
the \ac{GSAT}, which is used as a predictor for \ac{ECS} in several recently
published emergent constraints \autocite{JimenezdelaCuesta2019, Nijsse2020,
  Tokarska2020}.

\begin{figure}[t]
  \centering
  \begin{subfigure}[b]{\SubfigureWidth{}}
    \raisebox{2.5mm}{\includegraphics[width=\columnwidth]{
      ch03_esmvaltool/figs/co2s_map.pdf}}
    \caption{}
    \label{fig:03:co2s:a}
  \end{subfigure}
  ~
  \begin{subfigure}[b]{\SubfigureWidth{}}
    \includegraphics[width=\columnwidth]{
      ch03_esmvaltool/figs/co2s_mlo_cycle.pdf}
    \caption{}
    \label{fig:03:co2s:b}
  \end{subfigure}
  \caption[
    Illustration of the atmospheric \acs{CO2} concentration at the surface
    (\emph{co2}) for \acs{CMIP}6 models.
  ]{
    (a) \acs{CMIP}6 \acf{MMM} of the atmospheric \acs{CO2} concentration at the
    surface (\emph{co2s}) averaged over the months \acf{JJA} of the year 2014
    in the emission-driven historical simulation. The \acs{CMIP}6 \acs{MMM}
    includes the four climate models that provide all data needed for the
    calculation of \emph{co2s} (see legend in (b)). (b) Monthly mean
    \emph{co2s} at Mauna Loa, Hawaii ($19.5 \unit{\degree N}$, $155.6
    \unit{\degree W}$) from 1958 to 2014. The black line shows observations
    from the Mauna Loa Observatory \autocite{Keeling2005}; the remaining lines
    show emission-driven historical simulations from individual \acs{CMIP}6
    models.
  }
  \label{fig:03:co2s}
\end{figure}

In addition to new preprocessor functions, six derivation scripts for the
non-\ac{CMOR} variables \emph{asr}, \emph{co2s}, \emph{et}, \emph{rlntcs},
\emph{rsntcs}, and \emph{uajet} have been implemented into the \ac{ESMValCore}.
Apart from the atmospheric \ac{CO2} concentration at the surface (\emph{co2s}),
which is used in \cref{ch:06:paper_gpp} to derive the predictor variable of the
emergent constraint by \textcite{Wenzel2016}, these variables are not directly
used in this thesis but have been added to reproduce the \ac{MDER} analysis of
\textcite{Wenzel2016a} (\emph{asr} and \emph{uajet}) or have been used in
earlier versions of the studies presented in this thesis (\emph{et} for the
study presented in \cref{ch:06:paper_gpp}; \emph{rlntcs} and \emph{rsntcs} for
the studies presented in \cref{ch:04:papers_ecs_tcr_assessment}).

The absorbed shortwave radiation (\emph{asr}) is defined as the difference
between the incoming \ac{TOA} shortwave radiation and the outgoing \ac{TOA}
shortwave radiation. The atmospheric \ac{CO2} concentration at the surface
(\emph{co2s}) can be calculated from the pressure level--dependent atmospheric
\ac{CO2} concentration and the surface air pressure using interpolation. An
example illustrating \emph{co2s} is given in \cref{fig:03:co2s}.
\Cref{fig:03:co2s:a} shows the \acs{CMIP}6 \ac{MMM} of the global \emph{co2s}
for the months \acf{JJA} of the year 2014 in the emission-driven historical
simulation. As expected, the map shows high atmospheric \ac{CO2} concentrations
over large metropolitan areas with high \ac{CO2} emissions (\eg{} over North
America, the Arabian Peninsula, and East Asia). Moreover, regions with high
photosynthetic activity (\eg{} the boreal forests in the Northern Hemisphere
summer and the tropical rainforests in South America) exhibit smaller \ac{CO2}
concentrations. \Cref{fig:03:co2s:b} shows the monthly mean \emph{co2s} at
Mauna Loa, Hawaii from 1958 to 2014 for observations performed at the Mauna Loa
Observatory \autocite{Keeling2005} (thick black line) and emission-driven
historical simulations from four \acs{CMIP}6 models (remaining colored lines).
Apart from the model CNRM-ESM2-1, the simulated atmospheric \ac{CO2}
concentrations show the expected seasonal cycle present in the Keeling Curve
(see \cref{fig:02:keeling_curve}). All \acp{ESM} correctly simulate the
increase in \emph{co2s} over the years. However, there are some differences in
the absolute values with the model BCC-CSM2-MR showing the largest deviations.

The evapotranspiration (\emph{et}) defined as the sum of evaporation and plant
transpiration can be calculated from the surface latent heat flux and the
latent heat vaporization constant. The longwave/shortwave clear-sky net
\ac{TOA} radiations (\emph{rlntcs}/\emph{rsntcs}) is defined as the difference
between the longwave/shortwave incoming \ac{TOA} radiation assuming clear-sky
and the longwave/shortwave outgoing \ac{TOA} radiation assuming clear-sky.
Finally, the position of the austral jet stream (\emph{uajet}) is given by the
latitude with maximum zonal mean eastward wind speed at $850 \unit{hPa}$ on the
Southern Hemisphere between $80 \unit{\degree S}$ and $30 \unit{\degree S}$.


\section{Contributions to \acs{ESMValTool}}
\label{sec:03:contributions_to_esmvaltool}

In addition to the analyses presented in
\cref{ch:04:papers_ecs_tcr_assessment,ch:05:paper_ecs,ch:06:paper_gpp}, several
more recipes and \acs{CMOR}ization scripts for observational data have been
added to the main \ac{ESMValTool} repository on GitHub. A summary of these is
given in \cref{tab:03:changes_to_esmvaltool}. Moreover, multiple minor changes
and additions have been implemented to improve the code base and/or to fix
bugs. In total, the author of this study has added 88668 lines of code and
removed 47816 lines of code from the \ac{ESMValTool} repository (as of
\TheMonth{}). Parts of these implementations are already published in the
scientific documentation of the \ac{ESMValTool}, which covers large-scale
diagnostics \autocite{Eyring2020}, diagnostics for emergent constraints and
future projections \autocite{Lauer2020}, and diagnostics for extreme events and
regional evaluation \autocite{Weigel2020}.

\begin{table}[t]
  \centering
  \begin{tabulary}{\columnwidth}{L L L}
    \toprule
    Type & Name & Main reference \\
    \midrule
    Recipe & \emph{bock20jgr} (Figures 8 and 10) & \autocite{Bock2020} \\
    & \emph{cox18nature} & \autocite{Cox2018} \\
    & \emph{ecs} & \autocite{Gregory2004} \\
    & \emph{flato13ipcc} (Figure 9.42) & \autocite{Flato2013} \\
    & \emph{meehl20sciadv} & \autocite{Meehl2020} \\
    & \emph{schlund20esd} & \autocite{Schlund2020a} \\
    & \emph{schlund20jgr\_gpp\_abs\_rcp85} & \autocite{Schlund2020} \\
    & \emph{schlund20jgr\_gpp\_change\_1pct} & \autocite{Schlund2020} \\
    & \emph{schlund20jgr\_gpp\_change\_rcp85} & \autocite{Schlund2020} \\
    & \emph{tcr} & \autocite{Gregory2008} \\
    \midrule
    \acs{CMOR}izer & CRU & \autocite{Harris2014} \\
    & CT2019 & \autocite{Jacobson2020} \\
    & GCP & \autocite{Friedlingstein2020} \\
    & HWSD & \autocite{Wieder2014} \\
    & JMA-TRANSCOM & \autocite{Maki2010} \\
    & LAI3g & \autocite{Zhu2013} \\
    & LandFlux-EVAL & \autocite{Mueller2013} \\
    & MLS-Aura & \autocite{Read2015} \\
    & FLUXNET-MTE & \autocite{Jung2011} \\
    & NDP & \autocite{Gibbs2006} \\
    & Scripps-\acs{CO2} & \autocite{Keeling2005} \\
    \bottomrule
  \end{tabulary}
  \caption[
    Summary of new recipes and \acs{CMOR}ization scripts for observational data
    contributed to the \acf{ESMValTool} by the author of this thesis.
  ]{
    Summary of new recipes and \acs{CMOR}ization scripts for observational data
    contributed to the \acf{ESMValTool} by the author of this thesis. The
    corresponding file names of the recipes in the \acs{ESMValTool} repository
    are given by \emph{recipe\_(name\_in\_table).yml}.
  }
  \label{tab:03:changes_to_esmvaltool}
\end{table}

Due to an exhaustive overhaul of the \ac{ESMValTool} Version 2, older recipes
from Version 1 \autocite{Eyring2016a} cannot be used directly in the latest
release. Porting recipes from the old version to the current version requires
substantial changes on the recipes themselves, but also on the corresponding
diagnostic scripts. As of \TheMonth{}, the author of this thesis has ported two
recipes originally written by other authors to the new version of
\ac{ESMValTool}: \emph{recipe\_anav13jclim.yml} and
\emph{recipe\_wenzel16jclim.yml}. \emph{recipe\_anav13jclim.yml} reproduces the
analysis of \textcite{Anav2013}, who evaluate carbon cycle--related variables
for \acp{ESM} of the \acs{CMIP}5 ensemble. This includes climatologies, trends,
and variabilities of important quantities of the terrestrial and oceanic carbon
cycle like carbon fluxes, carbon reservoir sizes, and vegetation distributions.
\emph{recipe\_wenzel16jclim.yml} includes the \ac{MDER} analysis performed by
\textcite{Wenzel2016a} to constrain future changes in the position of the
austral jet stream. Due to its flexible code, the corresponding diagnostic
scripts can be used to apply the \ac{MDER} method to arbitrary target variables
and predictors. Both recipes include analyses that contributed ideas and
methods for the study presented in \cref{ch:06:paper_gpp}, in which the
\ac{MDER} method is expanded with \iac{ML} approach and applied to a carbon
cycle--related target variable (\ac{GPP}).

\begin{figure}[t]
  \centering
  \begin{subfigure}[b]{\SubfigureWidth{}}
    \includegraphics[width=\columnwidth]{ch03_esmvaltool/figs/cox18_psi.pdf}
    \caption{}
    \label{fig:03:cox18:a}
  \end{subfigure}
  ~
  \begin{subfigure}[b]{\SubfigureWidth{}}
    \includegraphics[width=\columnwidth]{ch03_esmvaltool/figs/cox18_ec.pdf}
    \caption{}
    \label{fig:03:cox18:b}
  \end{subfigure}
  \caption[
    Emergent constraint on the \acf{ECS} from global temperature variability.
  ]{
    Emergent constraint on the \acf{ECS} from global temperature variability
    \autocite{Cox2018}. Magenta colors indicate high-sensitivity climate models
    ($\abs*{\lambda} < 1 \unit{W m^{-2} K^{-1}}$), and green colors indicate
    low-sensitivity climate models ($\abs*{\lambda} > 1 \unit{W m^{-2}
      K^{-1}}$). Both panels have been created with the \acs{ESMValTool} using
    \emph{recipe\_cox18nature.yml} and are similar to \textcite{Cox2018}. (a)
    Temperature variability metric $\Psi$ versus time for historical
    simulations from \acs{CMIP}5 models (colored lines) and observations from
    the HadCRUT4 dataset (black points) \autocite{Morice2012}. (b) Emergent
    relationship between \acs{ECS} and the temperature variability metric
    $\Psi$. The dot-dashed black line shows the linear regression across the
    climate model ensemble with its associated uncertainties indicated by the
    dashed black lines (\acl{SPE}; see \cref{eq:02:spe}). Blue lines show the
    observational constraint from HadCRUT4 with its best estimate (dot-dashed
    line) and standard deviation (dashed lines). The letters represent
    individual \acs{CMIP}5 models (see \textcite{Cox2018} for details).
  }
  \label{fig:03:cox18}
\end{figure}

In addition to the ported recipes, several new recipes and diagnostic scripts
have been contributed to the public \ac{ESMValTool} repository (see
\cref{tab:03:changes_to_esmvaltool} for a summary). Some of these include the
analyses performed in other \lcnamecrefs{ch:04:papers_ecs_tcr_assessment} of
this thesis, namely \emph{recipe\_bock20jgr.yml}
(\cref{ch:04:papers_ecs_tcr_assessment}), \emph{recipe\_meehl20sciadv.yml}
(\cref{ch:04:papers_ecs_tcr_assessment}), \emph{recipe\_schlund20esd.yml}
(\cref{ch:05:paper_ecs}), and \emph{recipe\_schlund20jgr\_*.yml}
(\cref{ch:06:paper_gpp}). \emph{recipe\_ecs.yml} and \emph{recipe\_tcr.yml}
contain diagnostic scripts that allow the calculation of the climate metrics
\ac{ECS} and \ac{TCR} for arbitrary model output (see
\cref{sec:02:climate_sensitivity}), which is also used in
\cref{ch:04:papers_ecs_tcr_assessment,ch:05:paper_ecs}. Corresponding plots are
shown in \cref{fig:02:gregory_regression,fig:02:tcr}.
\emph{recipe\_cox18nature.yml} reproduces the analysis of \textcite{Cox2018},
who introduce an emergent constraint in \ac{ECS} based on a global temperature
variability metric $\Psi$. \Cref{fig:03:cox18:a} shows $\Psi$ over the past 85
years for \acs{CMIP}5 models and observations from HadCRUT4
\autocite{Morice2012}. Since high-sensitivity climate models exhibit higher
values of $\Psi$ than low-sensitivity models, an emergent relationship between
\ac{ECS} and $\Psi$ can be established (\cref{fig:03:cox18:b}). With
observations of $\Psi$, \ac{ECS} can eventually be constrained to $\left( 2.8
\pm 0.6 \right) \unit{K}$ ($66 \unit{\%}$ confidence range). The emergent
constraint by \textcite{Cox2018} is, together with 10 other emergent
constraints on \ac{ECS}, analyzed in great detail for the \acs{CMIP}5 and
\acs{CMIP}6 ensemble in \cref{ch:05:paper_ecs}, which also includes a
description of the calculation of $\Psi$ in \cref{subsec:05:cox}.

An important class of recipes in the \ac{ESMValTool} covers international
climate assessments. One such example is \emph{recipe\_flato13ipcc.yml}, which
reproduces large parts of the climate model evaluation performed in Chapter 9
of the \ac{IPCC}'s Fifth \acl{AR} (\acs{AR}5) \autocite{Flato2013}. For this
recipe, the Figure 9.42 of \textcite{Flato2013} has been added, which shows the
relationship between the historical and pre-industrial \ac{GSAT} and \ac{ECS}
(\cref{fig:03:flato13:a}) and the relationship between \ac{TCR} and \ac{ECS}
(\cref{fig:03:flato13:b}) within the \acs{CMIP}5 ensemble.
\Cref{fig:03:flato13:a} illustrates that there is no clear connection between
\ac{ECS} and the historical or pre-industrial \ac{GSAT}, \ie{} models with high
temperatures in the historical or pre-industrial period do not necessarily
exhibit a high equilibrium warming in the future or vice versa. On the other
hand, \ac{TCR} and \ac{ECS} are well correlated within the \acs{CMIP}5 ensemble
as theoretically expected from \cref{eq:02:tcr_vs_ecs}. However, this figure
only shows a linear regression fit between the two variables and not the
expected reciprocal relation. The output of \emph{recipe\_flato13ipcc.yml}
provided useful input for the analysis of climate sensitivity in \acs{CMIP}6
performed in \textcite{Meehl2020}, which is also presented in
\cref{ch:04:papers_ecs_tcr_assessment}.

\begin{figure}[t]
  \centering
  \begin{subfigure}[b]{0.36\columnwidth}
    \includegraphics[width=\columnwidth]{ch03_esmvaltool/figs/flato13_942a.pdf}
    \caption{}
    \label{fig:03:flato13:a}
  \end{subfigure}
  \begin{subfigure}[b]{0.36\columnwidth}
    \includegraphics[width=\columnwidth]{ch03_esmvaltool/figs/flato13_942b.pdf}
    \caption{}
    \label{fig:03:flato13:b}
  \end{subfigure}
  \begin{subfigure}[b]{0.26\columnwidth}
    \raisebox{12.5mm}{\includegraphics[width=\columnwidth]{
      ch03_esmvaltool/figs/flato13_942_legend.pdf}}
  \end{subfigure}
  \caption[
    Relationships between climate metrics within the \acs{CMIP}5 climate model
    ensemble.
  ]{
    Relationships between climate metrics within the \acs{CMIP}5 climate model
    ensemble. Both panels have been created with the \acs{ESMValTool} using
    \emph{recipe\_flato13ipcc.yml} and are similar to \textcite{Flato2013}. (a)
    \Acf{GSAT} versus the \acf{ECS} for \acs{CMIP}5 models for the period
    \range{1961}{1990} (larger symbols) and for pre-industrial control runs
    (smaller symbols). (b) \Acf{TCR} versus \acs{ECS} for \acs{CMIP}5 models.
    The black line shows a linear fit between \acs{TCR} and \acs{ECS}.
  }
  \label{fig:03:flato13}
\end{figure}

The implemented \acs{CMOR}ization scripts (also referred to as
\emph{\acs{CMOR}izers}) allow the processing of arbitrary observational data
with the \ac{ESMValTool}. For this, the raw observational datasets are
reformatted in such a way that they adhere to the \ac{CMOR} standards. Since
the \acs{CMOR}ization scripts are publicly available like any other code in the
\ac{ESMValTool} and \ac{ESMValCore} repositories, the entire scientific
community can use them to \acs{CMOR}ize the observational data. A complete list
of all 11 newly implemented \acs{CMOR}izers including the main reference for
the corresponding observational datasets is shown in
\cref{tab:03:changes_to_esmvaltool}. The majority of these datasets are used
for analyses presented in this thesis (\eg{} see
\cref{tab:05:overview_emergent_constraints,tab:06:predictors}).
