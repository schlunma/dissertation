%=============================================================================%
%                                Dissertation                                 %
%                               Manuel Schlund                                %
%                                  (c) 2020                                   %
%=============================================================================%
%                                Introduction                                 %
%=============================================================================%



\chapter{Introduction}
\label{ch:introduction}


\section{Structure of the thesis}
\label{sec:introduction:structure}

Parts of this thesis have already been published in scientific publications,
which will be clearly stated at the beginning of each chapter.
\Cref{ch:scientific_background} introduces the scientific background for this
thesis. This includes basic principles of Earth System Modeling, sources of
uncertainty in future projections of the climate, relevant definitions and
state-of-the-art methods used to evaluate \ac{ESM} simulations and reduce
associated uncertainties. \Cref{ch:esmvaltool} gives an overview over the
contributions made to the \ac{ESMValTool}, an open-source software for the
analysis of \acp{ESM}. These contributions helped improving the routine
evaluation of \acp{ESM} which is useful for the whole scientific community and
lead to co-authorship in four peer-reviewed studies
\autocite{Righi2020, Eyring2020, Lauer2020, Weigel2020}.
\Cref{ch:assessment_climate_metrics} covers the assessment of policy-relevant
climate metrics like the \acl{EqCS} and the \acl{TCR} in the latest generation
of \acp{ESM}. This work has already been published in two scientific
publications \autocite{Bock2020, Meehl2020}. \Cref{ch:paper_ecs} describes the
evaluation of emergent constraints (a technique to reduce uncertainties in
climate model projections, see \vref{sec:scientific_background:techniques}) on
the climate sensitivity in \acs{CMIP6} models. The contents of this chapter
have been published in \emph{Earth System Dynamics} \autocite{Schlund2020a}.
\Cref{ch:paper_gpp} focuses on a new method to reduce climate model
uncertainties based on \ac{ML}. As an example, the method is applied to the
photosynthesis rate at the end of the \tfst{} century, which is already
published in the \emph{Journal of Geophysical Research: Biogeosciences}
\autocite{Schlund2020}. Finally, \cref{ch:summary_outlook} provides a summary
of the results of this thesis and gives an outlook into possible future works.