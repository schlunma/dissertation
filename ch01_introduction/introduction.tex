%=============================================================================%
%                            Doctoral Dissertation                            %
%                            (c) by Manuel Schlund                            %
%=============================================================================%

%=============================================================================%
% This work is licensed under a
% Creative Commons Attribution 4.0 International License.
%
% You should have received a copy of the license along with this
% work. If not, see <http://creativecommons.org/licenses/by/4.0/>.
%=============================================================================%



\chapter{Introduction}
\label{ch:01:introduction}


\section{Motivation}
\label{sec:01:motivation}

Climate change is one of the greatest challenges for humankind today. The
warming of the climate system is \enquote{unequivocal}, and \enquote{many of
  the observed changes are unprecedented over decades to millennia}
\autocite{IPCC2014}. The changing climate increases the \enquote{likelihood of
  severe, pervasive and irreversible impacts for people, species and
  ecosystems} with \enquote{mostly negative impacts for biodiversity, ecosystem
services and economic development}, and amplifies \enquote{risks for
livelihoods and for food and human security} \autocite{IPCC2014}. Potential
drivers for climate change are all natural and anthropogenic substances and
processes that may alter the Earth's energy budget. The human influence on
the climate system is clear: Ever increasing emissions of \acp{GHG} since the
end of the pre-industrial era largely driven by economic and population
growth led to atmospheric concentrations of \ac{CO2}, \ac{CH4}, and \ac{N2O}
that are \enquote{unprecedented in at least the last 800000 years}
\autocite{IPCC2014}. The effects of these \ac{GHG} emissions and other
anthropogenic drivers have been \enquote{detected throughout the climate
system and are extremely likely to have been the dominant cause of the
observed warming since the mid-\nth{20} century} \autocite{IPCC2014}. The
greenhouse effect is based on the optical properties of the \acp{GHG}: While
the atmosphere is mostly transparent for the incoming solar (shortwave)
radiation, the outgoing infrared (longwave) radiation that is reflected from
the planet's surface is partly absorbed by the corresponding \ac{GHG}
molecules through an excitation of their corresponding vibrational modes and
re-emitted isotropically. This process traps energy near the surface and
leads to a warming of the Earth's surface and the lower atmosphere. The
individual impact of the different drivers of climate change measured with
the so-called \emph{radiative forcing}, which quantifies the change in energy
fluxes caused by changes in these drivers relative to pre-industrial
conditions \autocite{IPCC2013}. The total radiative forcing is positive
(which corresponds to a warming) and its largest contribution is caused by
the increase in the atmospheric concentration of \ac{CO2} due to fossil fuel
emissions since the year 1750 \autocite{IPCC2014}. Apart from their physical
warming effect on the climate, carbon-based \acp{GHG} like \ac{CO2} also
directly influence the global carbon cycle, an important biogeochemical cycle
of the Earth. A crucial flux of the carbon cycle is the \emph{\ac{GPP}}, which
describes the carbon uptake of the terrestrial biosphere due to
photosynthesis. Since this land carbon uptake absorbs about $30 \unit{\%}$ of
the anthropogenic \ac{CO2} emissions in today's climate
\autocite{Friedlingstein2020}, this process substantially slows down global
warming and directly contributes to the magnitude of the climate sensitivity.
Other important anthropogenic drivers of climate change are the emission of
aerosols and land use/land cover changes. Apart from anthropogenic drivers,
there are also natural processes which impact the climate system like changes
in the solar activity or volcanic eruptions. However, there is clear evidence
that these natural drivers alone cannot explain the observed climate change
\autocite{Haustein2017}.

To successfully mitigate the massive impacts of climate change, a first
important step is the understanding of climate change and its accurate
quantification. Extremely valuable tools in this context are climate models,
which allow us to simulate the behavior of the climate system under arbitrary
conditions without having to perform (ethnically questionable) experiments in
the real world. All around the world climate research institutes provide a
variety of different climate models. Many of them participate in the \ac{CMIP},
which was initiated in 1995 by the \ac{WGCM} of the \ac{WCRP} to
\enquote{better understand past, present and future climate changes arising
  from natural, unforced variability or in response to changes in radiative
  forcing in a multi-model context} \autocite{WCRP2020}. The \ac{CMIP} models
provide crucial input for the international climate assessments given by the
\acp{AR} of the \ac{IPCC}. For example, the latest generation of climate models
from the most recent (sixth) phase of \ac{CMIP} \commentcite{Eyring2016}{known
  as \acs{CMIP}6} support the assessment of the upcoming \ac{IPCC} Sixth
\acl{AR} (\acs{AR}6), and their predecessor models from \acs{CMIP}5
\autocite{Taylor2012} have been assessed as part of the Fifth \acl{AR}
(\acs{AR}5) in 2013 \autocite{Flato2013}. Modern-day climate models, which
allow the simulation of biological and chemical processes in addition to the
dynamics of the physical components of the Earth system, are also known as
\emph{\acp{ESM}} and provide the most sophisticated simulations of the Earth's
climate. In this thesis, the terms \enquote{climate model} and
\enquote{\acl{ESM}} are used interchangeably since most modern models
participating in \ac{CMIP} are \acp{ESM} or at least have \ac{ESM} versions.

Simulations that extrapolate the state of the climate system into the future
are called \emph{climate model projections}. These include idealized
simulations with only a prescribed change in the atmospheric \ac{CO2}
concentration (\eg{} an instantaneous doubling of \ac{CO2} or a \ac{CO2}
increase of $1 \unit{\%}$ per year) as well as more realistic projections that
consider different future scenarios (\eg{} a fossil fuel--based future or a
scenario that is based on a sustainable development). In many variables that
are relevant for climate change, multi-model projections from \ac{CMIP} show a
large inter-model range \autocite{Collins2013, Flato2013}. A crucial and
policy-relevant example for this is the climate sensitivity, which refers to
the change in the \ac{GSAT} that results from a change in the radiative
forcing. Common metrics for this are the \emph{\ac{ECS}}, which describes the
equilibrium response of the climate system after a doubling of the atmospheric
\ac{CO2} concentration, and the \emph{\ac{TCR}}, which describes the transient
response of the system to a \ac{CO2} doubling. In \acs{AR}5, both quantities
have been assessed with large ranges of $\rangeunit{1.5}{4.5}{K}$ and
$\rangeunit{1.0}{2.5}{K}$ for \ac{ECS} and \ac{TCR}, respectively
\autocite{Stocker2013}. The corresponding inter-model ranges from the
\acs{CMIP}5 models show similar results \autocite{Flato2013}. For this reason,
a careful statistical evaluation and further refinement of the output of
multi-model climate projections is necessary in order to reduce associated
uncertainties. A state-of-the-art technique for this is the \emph{emergent
  constraints} method, which uses a physically-based inter-model relationship
between an observable quantity of the Earth system and a target variable to
reduce uncertainties in the target variable with observations
\autocite{Allen2002}. An alternative approach is the weighting of climate
models based on their performance (\ie{} the distance of one model to
observational products) and interdependence (\ie{} the distance of one model to
other climate models) \autocite{Knutti2017a}. These techniques form the
baseline for the new analyses and results presented in this thesis, which
partly utilize methods from a new emerging research field in climate sciences:
\ac{AI} and \ac{ML}.


\section{Key Science Questions}
\label{sec:01:key_science_questions}

The aim of this thesis is to reduce uncertainties in multi-model climate
projections with observations by addressing the following three
\namecrefs{enum:01:question_1}:

\begingroup
\crefalias{enumi}{question}
\begin{enumerate}
  \item \label{enum:01:question_1} \KeyScienceQuestionOne{}
  \item \label{enum:01:question_2} \KeyScienceQuestionTwo{}
  \item \label{enum:01:question_3} \KeyScienceQuestionThree{}
\end{enumerate}
\endgroup


\section{Structure of the Thesis}
\label{sec:01:structure}

Parts of this thesis are already published in multiple peer-reviewed
publications (two first-author studies and six co-author studies). A complete
list of these is given on \cpageref{ch:bib:own}. Wherever material from these
studies is presented in this thesis, the pronoun \enquote{we} is used to
increase readability by avoiding the passive voice and to acknowledge all
involved contributors. However, unless stated otherwise, all contents from
these publications (text, figures, and tables) shown in this thesis originate
from the author of this thesis. A detailed list of contributions to these
studies is given in the corresponding \namecrefs{ch:03:esmvaltool}.

This thesis is structured as follows: \Cref{ch:02:scientific_background}
introduces the scientific background. This includes relevant literature that is
used as a baseline for this thesis. \Cref{ch:03:esmvaltool} gives an overview
over the contributions made to the \ac{ESMValTool}, an open-source software for
the analysis of \acp{ESM}. These contributions helped improving the routine
evaluation of \acp{ESM} which is useful for the entire scientific community and
led to co-authorship in four peer-reviewed studies \autocite{Eyring2020,
  Lauer2020, Righi2020, Weigel2020}. \Cref{ch:04:papers_ecs_tcr_assessment}
covers the assessment of climate sensitivity metrics like the \ac{ECS} or
\ac{TCR} in the latest generation of \acp{ESM} from \acs{CMIP}6. This work is
already published in two scientific publications \autocite{Bock2020,
  Meehl2020}. Since the \ac{ECS} and \ac{TCR} are considerably higher in this
new climate model generation, \cref{ch:05:paper_ecs} describes the assessment
of emergent constraints on the \ac{ECS} for these \acs{CMIP}6 models and
compares these to results derived from \acs{CMIP}5 models. The contents of this
\namecref{ch:05:paper_ecs} are published in \emph{Earth System Dynamics}
\autocite{Schlund2020a}. \Cref{ch:06:paper_gpp} focuses on a new method to
reduce uncertainties in multi-dimensional (gridded) multi-model projections of
the future climate with observations based on \ac{ML}. As an example, the
method is applied to \ac{GPP} at the end of the \nth{21} century, which is
already published in the \emph{Journal of Geophysical Research: Biogeosciences}
\autocite{Schlund2020}. Finally, \cref{ch:07:conclusion} provides a summary of
the results of this thesis and gives an outlook of possible future works.
