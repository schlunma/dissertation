%=============================================================================%
%                                Dissertation                                 %
%                               Manuel Schlund                                %
%                                  (c) 2020                                   %
%=============================================================================%
%                                Introduction                                 %
%=============================================================================%



\chapter{Introduction}
\label{ch:01:introduction}


\section{Structure of the thesis}
\label{sec:01:structure}

Parts of this thesis are published in multiple scientific publications (two as
first author six as co-author). If applicable, this is clearly stated at the
beginning of each chapter. \Cref{ch:02:scientific_background} introduces the
scientific background for this thesis. This includes basic principles of Earth
System Modeling, sources of uncertainty in future projections of the climate,
relevant definitions and state-of-the-art methods used to evaluate \ac{ESM}
simulations and reduce associated uncertainties. \Cref{ch:03:esmvaltool} gives
an overview over the contributions made to the \ac{ESMValTool}, an open-source
software for the analysis of \acp{ESM}. These contributions helped improving
the routine evaluation of \acp{ESM} which is useful for the whole scientific
community and lead to co-authorship in four peer-reviewed studies
\autocite{Righi2020, Eyring2020, Lauer2020, Weigel2020}.
\Cref{ch:04:assessment_climate_metrics} covers the assessment of
policy-relevant climate metrics like the \ac{EqCS} and the \ac{TCR} in the
latest generation of \acp{ESM}. This work is already published in two
scientific publications \autocite{Bock2020, Meehl2020}. Since the \ac{EqCS} and
\ac{TCR} are considerably higher in this new climate model generation,
\cref{ch:05:paper_ecs} describes the assessment of emergent constraints (a
technique to reduce uncertainties in climate model projections, see
\vref{sec:02:techniques}) on the \ac{EqCS} for these \acp{ESM}. The contents of
this chapter are published in \emph{Earth System Dynamics}
\autocite{Schlund2020a}. \Cref{ch:06:paper_gpp} focuses on a new method to
reduce climate model uncertainties based on \ac{ML}. As an example, the method
is applied to the photosynthesis rate at the end of the \nth{21} century, which
is already published in the \emph{Journal of Geophysical Research:
Biogeosciences} \autocite{Schlund2020}. Finally, \cref{ch:07:summary_outlook}
provides a summary of the results of this thesis and gives an outlook of
possible future works.