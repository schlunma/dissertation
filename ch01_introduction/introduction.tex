%=============================================================================%
%                                Dissertation                                 %
%                               Manuel Schlund                                %
%                                  (c) 2020                                   %
%=============================================================================%
%                                Introduction                                 %
%=============================================================================%



\chapter{Introduction}
\label{ch:01:introduction}


\section{Motivation}
\label{sec:01:motivation}

- Why is climate change relevant for the society?

- Cite latest warming reports, also other variables (droughts, extreme events,
etc.)

- basic physics of greenhouse gases (vibration modes, etc.)

- definition of radiative forcing

- well-mixed greenhouse gases (water vapor, CO2, CH4) + sources

- long-lived vs. short-lived GHGs?

- aerosols + cooling effect

- natural vs. anthropogenic effects (D\&A) -> clear evidence that climate
change over 21st century is caused by human influence. Website:
(https://globalwarmingindex.org/, Haustein et al., 2017)

- remove description of emergent constraints from section 1.3, since this will
probably be included in this section.

- answer question: what is climate sensitivity? What is \ac{ECS}? What is
\ac{TCR}? What is \ac{CMIP}?

- One sentence on \ac{ML}.

- Mention IPCC assessments, older ones and upcoming AR6

- "climate model" is used as generic term which include \acp{AOGCM} and
\acp{ESM} (see \cref{subsec:02:numerical_climate_modeling}) (subcategories).
Since most modern climate models participating in CMIP are \acp{ESM} (or at
least have \ac{ESM} versions), the terms "climate model" and "ESM" are used
interchangeably in this thesis.


\section{Key Science Questions}
\label{sec:01:research_questions}

This section will present the \textbf{AWESOME} key science questions!!!

\begin{enumerate}
  \item What is the range of climate sensitivity in the latest generation of
  \acp{ESM} compared to previous multi-model ensembles, and do we understand
  the processes that determine this uncertainty range?

  \item Can uncertainties in climate sensitivity be reduced with observations
  using the emergent constraint approach?

  \item Can uncertainties in multi-dimensional (gridded) climate projections be
  reduced with \ac{ML} techniques and observations?
\end{enumerate}


\section{Structure of the Thesis}
\label{sec:01:structure}

Parts of this thesis are already published in multiple peer-reviewed
publications (two first-author studies and six co-author studies). Wherever
material from these studies is presented in this thesis, the pronoun
\enquote{we} is used to increase readability by avoiding the passive voice and
to acknowledge all involved contributors. However, unless stated otherwise, all
contents from these publications (text, figures and tables) shown in this
thesis originate from the author of this thesis. A detailed list of
contributions to these studies (including their full reference) is given in the
corresponding chapters.

This thesis is structured as follows: \Cref{ch:02:scientific_background}
introduces the scientific background. This includes relevant literature that is
used as a baseline for this thesis. \Cref{ch:03:esmvaltool} gives an overview
over the contributions made to the \ac{ESMValTool}, an open-source software for
the analysis of \acp{ESM}. These contributions helped improving the routine
evaluation of \acp{ESM} which is useful for the entire scientific community and
lead to co-authorship in four peer-reviewed studies \autocite{Eyring2020,
  Lauer2020, Righi2020, Weigel2020}. \Cref{ch:04:papers_ecs_tcr_assessment}
covers the assessment of climate sensitivity metrics like the \ac{ECS} or
\ac{TCR} in the latest generation of \acp{ESM} (\acs{CMIP}6). This work is
already published in two scientific publications \autocite{Bock2020,
  Meehl2020}. Since the \ac{ECS} and \ac{TCR} are considerably higher in this
new climate model generation, \cref{ch:05:paper_ecs} describes the assessment
of emergent constraints (a technique to reduce uncertainties in multi-model
climate projections, see \cref{subsec:02:emergent_constraints}) on the \ac{ECS}
for these \acp{ESM}. The contents of this chapter are published in \emph{Earth
  System Dynamics} \autocite{Schlund2020a}. \Cref{ch:06:paper_gpp} focuses on a
new method to reduce uncertainties in multi-dimensional (gridded) multi-model
projections of the future climate with observations based on \ac{ML}. As an
example, the method is applied to the photosynthesis rate at the end of the
\nth{21} century, which is already published in the \emph{Journal of
  Geophysical Research: Biogeosciences} \autocite{Schlund2020}. Finally,
\cref{ch:07:summary_outlook} provides a summary of the results of this thesis
and gives an outlook of possible future works.
