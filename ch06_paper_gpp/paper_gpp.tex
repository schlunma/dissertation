%=============================================================================%
%                                Dissertation                                 %
%                               Manuel Schlund                                %
%                                  (c) 2020                                   %
%=============================================================================%
%                                  Paper GPP                                  %
%=============================================================================%



\chapter{Constraining Future Gross Primary Production with Machine Learning}
\label{ch:06:paper_gpp}

As argued in detail in the previous chapter, single-process-oriented emergent
constraints do not seem to be beneficial in constraining climate projections
from the vastly complex modern-day \acp{ESM} since they might overly simplify
the problem. To overcome this issue, we explore an alternative approach that
expands the \acf{MDER} method with a supervised \acf{ML} algorithm. In contrast
to emergent constraints, this approach considers (1) multiple process-based
predictors, (2) multi-dimensional (gridded) target variables and (3) non-linear
relationships between the predictors and the target variable. Since the new
technique relies on a large number of data points in order to train the \ac{ML}
algorithm, \ac{ECS} as a single scalar metric for each climate model is not an
appropriate target variable. As an alternative, we analyze projections of
future \acf{GPP} in this study, which constitutes the largest flux of the
terrestrial carbon uptake. As described in detail in
\cref{subsec:02:carbon_cycle_perturbations}, the global carbon cycle actively
slows down global warming by removing \ac{CO2} from the atmosphere. In
particular, the terrestrial biosphere absorbs about $30 \unit{\%}$ of the total
anthropogenic \ac{CO2} emission \autocite{Friedlingstein2020}. Thus, accurately
quantifying the future evolution of the terrestrial carbon sink is crucial for
reliable climate change projections.

Due to complex feedback processes connected to the global carbon cycle
(concentration-carbon feedback and climate-carbon feedback; see
\cref{subsec:02:carbon_cycle_perturbations}), the response of the terrestrial
carbon cycle to changes in the atmospheric \ac{CO2} concentration and climate
is a major source of uncertainty in climate projections \autocite{Bodman2013,
  Booth2012, Collins2013}. This applies for example to multi-model projections
of future \ac{GPP} of the \acs{CMIP}5 ensemble. Thus, the aim of this study is
to reduce uncertainties in multi-model projections of future \ac{GPP} in the
\acs{CMIP}5 \acs{RCP}8.5 scenario at the end of the \nth{21} century with a new
\ac{ML}-based weighting technique and appropriate observations. The new
two-step approach utilizes aspects of emergent constraints and the \ac{MDER}
technique in combination with a \ac{ML} algorithm. In the first step, we apply
an existing emergent constraint on \ac{CO2} fertilization \autocite{Wenzel2016}
to constrain the \acp{ESM}' responses to rising atmospheric \ac{CO2}
concentration using observations of the increase of the \ac{CO2} seasonal cycle
amplitude at \ac{KUM} \autocite{Keeling2005}. In a second step, we introduce a
supervised \ac{ML} algorithm based on boosting trees \autocite{Friedman2001} to
learn an empirical spatial relationship that links grid-wise future \ac{GPP} to
historical processes relevant to its simulation under present-day conditions.
In combination with observational products of the predictors, that relationship
can be used to further constrain uncertainties in the projected spatial maps of
\ac{GPP} at the end of the \nth{21} century in the \acs{RCP}8.5 scenario. We
examine both constraining the absolute \ac{GPP} and the fractional change in
\ac{GPP} as two independent approaches and target variables. Unlike univariate
linear regression used in the \ac{MDER} algorithm, the proposed \ac{GBRT}
algorithm is able to handle multiple predictors and copes with non-linearities
in the data. \ac{GBRT} is a well-known and successful tool used for
interpolation, classification and prediction in other fields of data science
and engineering \autocite{Death2007, Elith2008}. In the context of climate
science, \ac{GBRT} was recently applied to identify the key drivers of spatial
variations of the ratio of plant transpiration to total terrestrial
evapotranspiration in \acp{ESM} \autocite{Lian2018}.

This work is already published in \textcite{Schlund2020}. For this paper, the
author of this thesis lead the writing and the analysis and implemented the
code to reproduce this study with all figures and tables using the
\ac{ESMValTool}. \Cref{sec:06:data_and_methods} of this chapter provides an
overview of the data and methods used in this paper. The results are presented
in \cref{sec:06:results} and \cref{sec:06:summary_and_discussion} closes with a
summary and discussion.


\section{Data and Methods}
\label{sec:06:data_and_methods}

TBA.

\begin{figure}[t]
  \centering
  \includegraphics[width=\LargeFigureWidth{}]{ch06_paper_gpp/figs/1.pdf}
  \caption{Schematic illustration of our two-step approach. In step 1 an
    emergent constraint by \textcite{Wenzel2016} is used to constrain the
    global mean fractional change in \acf{GPP} over the 21st century.
    Moreover, this constraint is used to rescale two different gridded target
    variables: absolute \acs{GPP} at the end of the \nth{21} century and
    fractional \acs{GPP} change over the \nth{21} century. In step 2, a
    \acl{ML} model is used to constrain these two target variables (step 2a:
    absolute \acs{GPP}; step 2b: fractional \acs{GPP} change) in two
    independent approaches. \AdaptedFrom{Schlund2020}. \cref{eq:06:xxx}}
  \label{fig:06:schematic_steps}
\end{figure}

\begin{figure}[p]
  \centering
  \includegraphics[width=\LargerFigureWidth{}]{ch06_paper_gpp/figs/2.pdf}
  \caption{Schematic illustration of our \acf{ML} approach to constrain
    projected absolute \acf{GPP} in step 2a. (a) In the training phase of the
    algorithm, the model is fitted to the training data interpolating the
    empirical (non-linear) relationship between two process-oriented
    diagnostics of the past climate $\left\{ x^{(1)}, x^{(2)} \right\}$ and
    (rescaled) future \acs{GPP} (gray surface). The points show the training
    points for the supervised \acs{ML} algorithm, each of them representing a
    single grid cell/month of a single climate model (the different colors
    correspond to different climate models). (b) In the prediction phase,
    observation-based values of the diagnostic (black points) are fed into the
    trained \acs{ML} model to constrain \acs{GPP} for every grid cell/month to
    a value which best agrees with the observations. (c) For an independent
    validation of our method, we use an out-of-sample testing setup based on a
    leave-one-model-out \acl{CV} approach (see
    \cref{subsec:02:model_weighting} for details). The schematic illustration
    of step 2b differs only in the target variable used (fractional \acs{GPP}
    change instead of absolute \acs{GPP}). \AdaptedFrom{Schlund2020}.}
  \label{fig:06:mlr_concept}
\end{figure}

\begin{table}[t]
  \centering
  \begin{tabular}[t]{p{0.27\columnwidth} l p{0.11\columnwidth}
      p{0.3\columnwidth}}
    \toprule
    Predictor & \makecell{Observation‐driven \\ data} & Used time range &
    Physical connection to \acs{GPP} \\
    \midrule
    \predictor{GPP} & \makecell{FLUXNET-MTE \\ \autocite{Jung2011}} &
    \range{1991}{2000} & -- \\
    \midrule
    \predictor{LAI} & \makecell{LAI3g \\ \autocite{Zhu2013}} &
    \range{1982}{2005} & \acs{LAI} is a measure for the number of leaves in a
    grid cell. The photosynthesis rate is highly dependent on the number of
    leaves (and vegetation in general). \\
    \midrule
    \predictor{PR} & \makecell{CRU \\ \autocite{Harris2014}} &
    \range{1901}{2005} & Water is essential for the chemical processes of
    photosynthesis. \\
    \midrule
    \predictor{RSDS} & \makecell{ERA-Interim \\ \autocite{Dee2011}} &
    \range{1979}{2005} & Solar radiation is essential for the chemical
    processes of photosynthesis. \\
    \midrule
    \predictor{T} & \makecell{CRU \\ \autocite{Harris2014}} &
    \range{1901}{2005} & \acs{T} and photosynthesis rate have a common driver
    (incoming solar radiation). \\
    \bottomrule
  \end{tabular}
  \caption{Process-oriented diagnostics (\enquote{predictors} or
    \enquote{features}) used in the \acf{GBRT} model to predict the target
    variables. For step 2a (target variable: absolute \acs{GPP}), all listed
    variables are monthly climatologies of the specified time ranges in the
    historical climate. For step 2b (target variable: fractional \acs{GPP}
    change), the temporal mean over the specified time ranges is calculated for
    all variables. \AdaptedFrom{Schlund2020}.}
  \label{tab:06:predictors}
\end{table}


\subsection{Constraining the \texorpdfstring{\acs{CO2}}{CO2} Fertilization
  Effect (Step 1)}
\label{subsec:06:step_1}

TBA.


\section{Results}
\label{sec:06:results}

TBA.

\begin{figure}[t]
  \centering
  \begin{subfigure}[b]{0.37\columnwidth}
    \includegraphics[width=\columnwidth]{ch06_paper_gpp/figs/3a.pdf}
    \caption{}
    \label{fig:06:step1:a}
  \end{subfigure}
  \begin{subfigure}[b]{0.37\columnwidth}
    \includegraphics[width=\columnwidth]{ch06_paper_gpp/figs/3b.pdf}
    \caption{}
    \label{fig:06:step1:b}
  \end{subfigure}
  \begin{subfigure}[b]{0.24\columnwidth}
    \raisebox{12.9mm}{\includegraphics[width=\columnwidth]{
      ch06_paper_gpp/figs/3c.pdf}}
  \end{subfigure}
  \caption{Emergent relationship between the global mean fractional change in
    \acf{GPP} and the sensitivity of the \acs{CO2} seasonal cycle amplitude to
    rising atmospheric \acs{CO2} concentrations observed at \acf{KUM}. Colored
    points refer to \acs{CMIP}5 models, the orange line and shaded area to the
    linear  regression fit and its corresponding standard prediction error and
    the dashed black lines to the observational constraint. (a) Similar to
    \textcite{Wenzel2016}: The global mean fractional \acs{GPP} change after
    \acs{CO2} doubling in a \acf{1BGC} defined as 10-year mean of the
    \acs{1BGC} run centered at the time of \acs{CO2} doubling relative to the
    10-year mean of pre-industrial control conditions at the beginning of the
    \acs{1BGC} run. The sensitivity of the \acs{CO2} amplitude for the
    \acs{CMIP}5 models is calculated from the years \range{1860}{2005}, the
    corresponding observational value from the years \range{1979}{2019}. The
    constrained global mean \acs{GPP} change after \acs{CO2} doubling in the
    \acs{1BGC} run is $\pmrangeunit{30}{9}{\%}$. (b) Fractional global mean
    \acs{GPP} change over the \nth{21} century calculated from the 10-year
    mean \acs{GPP} at the end of the \nth{21} century (\range{2091}{2100}) in
    the emission-driven fully coupled \acs{RCP}8.5 simulations and the 10-year
    mean \acs{GPP} at the end of the \nth{20} century (\range{1991}{2000}) in
    the emission‐driven fully-coupled historical run (see \cref{eq:06:xxx}).
    In contrast to \textcite{Wenzel2016}, the sensitivity of the \acs{CO2}
    amplitude is calculated from the years \range{1979}{2019} for climate
    models and observations for better comparability (the historical
    \acs{CMIP}5 simulations are extended with the \acs{RCP}8.5 simulations for
    the years \range{2006}{2019}). The constrained global mean \acs{GPP}
    change over the \nth{21} century is $\pmrangeunit{39}{7}{\%}$.
    \AdaptedFrom{Schlund2020}.}
  \label{fig:06:step1}
\end{figure}

\begin{figure}[t]
  \centering
  \begin{subfigure}[b]{\SubfigureWidth{}}
    \raisebox{5.5mm}{\includegraphics[width=\columnwidth]{
      ch06_paper_gpp/figs/4a.pdf}}
    \caption{}
    \label{fig:06:step2a_ml_inference:a}
  \end{subfigure}
  ~
  \begin{subfigure}[b]{\SubfigureWidth{}}
    \includegraphics[width=\columnwidth]{ch06_paper_gpp/figs/4b.pdf}
    \caption{}
    \label{fig:06:step2a_ml_inference:b}
  \end{subfigure}
  \caption{(a) Box plot of the \acf{RMSEP} distributions for six different
    statistical models used to predict future absolute \acf{GPP} in step 2a
    using a leave-one-model-out \acl{CV} approach. The distribution for each
    statistical model contains seven elements (black points; one for each
    climate model used as truth) and is represented in the following way: The
    lower and upper limit of the blue boxes correspond to the $25 \unit{\%}$
    and $75 \unit{\%}$ quantiles, respectively. The central line in the box
    shows the median and the black \enquote{x} shows the mean of the
    distribution. The whiskers outside the box represent the range of the
    distribution. Compared to the \acs{CMIP}5 \acf{MMM} and its corresponding
    rescaled version (r\acs{MMM}), the prediction uncertainty measured by the
    mean \acs{RMSEP} is significantly reduced by up to $48 \unit{\%}$ and $39
    \unit{\%}$, respectively, when using other statistical models. Moreover,
    the non-linear \acf{GBRT} models can slightly reduce the mean \acs{RMSEP}
    compared to the linear \acf{LASSO} models by about $2 \unit{\%}$ for
    \acs{GBRT}-1D (using historical \acs{GPP} as single predictor) and $3
    \unit{\%}$ for the full \acs{GBRT} model (using all predictors). (b)
    Relative global feature importance for the different predictors used in
    the \acs{GBRT} model to predict future absolute \acs{GPP} (step 2a). The
    red bars correspond to positive Pearson correlation coefficients between
    all predictors and the target variable. Due to its strong positive linear
    relationship with the future \acs{GPP}, the historical \acs{GPP} is by far
    the most important predictor in the model. A local feature importance map
    using \acf{LIME} is not shown here because \acs{GPP} is the dominant
    predictor for all grid cells. \AdaptedFrom{Schlund2020}.}
  \label{fig:06:step2a_ml_inference}
\end{figure}

\begin{figure}[t]
  \centering
  \begin{subfigure}[b]{\SubfigureWidth{}}
    \includegraphics[width=\columnwidth]{ch06_paper_gpp/figs/5a.pdf}
    \caption{}
    \label{fig:06:step2a_results:a}
  \end{subfigure}
  \\
  \begin{subfigure}[b]{\SubfigureWidth{}}
    \includegraphics[width=\columnwidth]{ch06_paper_gpp/figs/5b.pdf}
    \caption{}
    \label{fig:06:step2a_results:b}
  \end{subfigure}
  ~
  \begin{subfigure}[b]{\SubfigureWidth{}}
    \includegraphics[width=\columnwidth]{ch06_paper_gpp/figs/5c.pdf}
    \caption{}
    \label{fig:06:step2a_results:c}
  \end{subfigure}
  \caption{(a) Ratio of the rescaled \acs{CMIP}5 ensemble mean of the absolute
    \acf{GPP} at the end of the \nth{21} century (\range{2091}{2100}) in the
    \acs{RCP}8.5 scenario using \cref{eq:06:xxx} (r\acs{MMM}) and its
    unweighted version (\acs{MMM}). The plot shows an almost constant value
    over the whole globe with a mean of $0.92$, which corresponds to the ratio
    of the constrained global mean \acs{GPP} change over the \nth{21} century
    from step 1 ($39 \unit{\%}$) and the \acs{CMIP}5 ensemble mean global mean
    \acs{GPP} change ($43 \unit{\%}$). All values close to $0$ for the dataset
    in the denominator have been masked to avoid divisions by $0$. (b) Bias
    between r\acs{MMM} and our \acf{GBRT} prediction for the end of the
    \nth{21} century. This corresponds to step 2a of our approach. (c) Bias
    between the modeled \acs{GPP} in the \acs{CMIP}5 \acs{MMM} of the
    historical simulation and the FLUXNET-MTE observation‐based estimate of
    \acs{GPP} \autocite{Jung2011} averaged between 1991 and 2000. Over large
    swaths of the globe, the \acs{CMIP}5 ensemble overestimates \acs{GPP} (red
    color). Panels (b) and (c) show similar bias patterns (pattern correlation
    of $R^2 = 0.88$). Thus, the \acs{GBRT} prediction in step 2a is able to
    correct the historical bias of the \acs{CMIP}5 ensemble relative to the
    FLUXNET‐MTE product. \AdaptedFrom{Schlund2020}.}
  \label{fig:06:step2a_results}
\end{figure}

\begin{figure}[t]
  \centering
  \begin{subfigure}[b]{\SubfigureWidth{}}
    \includegraphics[width=\columnwidth]{ch06_paper_gpp/figs/6a.pdf}
    \caption{}
    \label{fig:06:step2a_results_biases:a}
  \end{subfigure}
  ~
  \begin{subfigure}[b]{\SubfigureWidth{}}
    \includegraphics[width=\columnwidth]{ch06_paper_gpp/figs/6b.pdf}
    \caption{}
    \label{fig:06:step2a_results_biases:b}
  \end{subfigure}
  \\
  \begin{subfigure}[b]{\SubfigureWidth{}}
    \includegraphics[width=\columnwidth]{ch06_paper_gpp/figs/6c.pdf}
    \caption{}
    \label{fig:06:step2a_results_biases:c}
  \end{subfigure}
  ~
  \begin{subfigure}[b]{\SubfigureWidth{}}
    \includegraphics[width=\columnwidth]{ch06_paper_gpp/figs/6d.pdf}
    \caption{}
    \label{fig:06:step2a_results_biases:d}
  \end{subfigure}
  \caption{Difference (a) and ratio (b) of the both biases shown in
    \cref{fig:06:step2a_results:b,fig:06:step2a_results:c}. For panel (b), all
    values close to $0$ for the dataset in the denominator have been masked to
    avoid divisions by $0$. Both panels show that our approach is only to
    first-order a bias correction, in which case both plots would only show
    constant values. (c) Comparison of the \acf{GBRT} versus \acs{GBRT}-1D
    projections of future \acf{GPP}. This panel indicates a clear difference
    between the full \acs{GBRT} model using all predictors and the \acs{GBRT}
    model using only historical \acs{GPP} as single predictor. (d) Comparison
    of the \acs{GBRT} versus the \acf{LASSO} projections of future \acs{GPP}.
    This panel shows that there is a clear difference between using the
    non-linear \acs{GBRT} model and the linear \acs{LASSO} model. The results
    of the \acs{LASSO}-1D model are not shown here because they are very
    similar to the results of the \acs{LASSO} model.
    \AdaptedFrom{Schlund2020}.}
  \label{fig:06:step2a_results_biases}
\end{figure}

\begin{figure}[p]
  \centering
  \begin{subfigure}[b]{\SubfigureWidth{}}
    \raisebox{5mm}{\includegraphics[width=\columnwidth]{
      ch06_paper_gpp/figs/7a.pdf}}
    \caption{}
    \label{fig:06:step2b_ml_inference:a}
  \end{subfigure}
  ~
  \begin{subfigure}[b]{\SubfigureWidth{}}
    \includegraphics[width=\columnwidth]{ch06_paper_gpp/figs/7b.pdf}
    \caption{}
    \label{fig:06:step2b_ml_inference:b}
  \end{subfigure}
  \\
  \begin{subfigure}[b]{\SubfigureWidth{}}
    \raisebox{12.5mm}{\includegraphics[width=\columnwidth]{
      ch06_paper_gpp/figs/7c.pdf}}
    \caption{}
    \label{fig:06:step2b_ml_inference:c}
  \end{subfigure}
  ~
  \begin{subfigure}[b]{\SubfigureWidth{}}
    \includegraphics[width=\columnwidth]{ch06_paper_gpp/figs/7d.pdf}
    \caption{}
    \label{fig:06:step2b_ml_inference:d}
  \end{subfigure}
  \caption{(a) Box plot of the \acf{RMSEP} distributions for four different
    statistical models used to predict the fractional change in \acf{GPP} over
    the \nth{21} century (step 2b) using the leave‐one‐model‐out \acl{CV}
    approach. A detailed description of the representation of these
    distributions is given in \cref{fig:06:step2a_ml_inference:a}. The
    \acf{GBRT} algorithm shows the minimal mean and median \acs{RMSEP}.
    Compared to the \acs{CMIP}5 \acf{MMM}, its corresponding rescaled version
    (r\acs{MMM}), and the linear \acf{LASSO} model, the mean \acs{RMSEP} of the
    \acs{GBRT} model is reduced by more than $16 \unit{\%}$, $9 \unit{\%}$, and
    $3 \unit{\%}$, respectively. (b) Relative global feature importance for
    the different features used in the \acs{LASSO} model to predict the
    fractional \acs{GPP} change (step 2b). The feature importance for the
    \acs{LASSO} model is given by the normalized linear coefficients of the
    model. (c) Local feature importance for the \acs{GBRT} model used to
    predict fractional \acs{GPP} change (step 2b) calculated using the
    \acf{LIME} approach \autocite{Ribeiro2016} for the three dominant features
    \acf{T}, \acf{LAI} and \acs{GPP}. The relative influence of these three
    features (ignoring all other features) is color-coded according to the
    triangle in the lower left corner. Over large parts of the globe, \acs{T}
    is the dominant feature. (d) Relative global feature importance for the
    different features used in the \acs{GBRT} model to predict the fractional
    \acs{GPP} change (step 2b). The blue bars for the two global feature
    importance plots (b) and (d) correspond to negative Pearson correlation
    coefficients between all predictors and the target variable.
    \AdaptedFrom{Schlund2020}.}
  \label{fig:06:step2b_ml_inference}
\end{figure}

\begin{figure}[t]
  \centering
  \begin{subfigure}[b]{\SubfigureWidth{}}
    \includegraphics[width=\columnwidth]{ch06_paper_gpp/figs/8a.pdf}
    \caption{}
    \label{fig:06:step2b_results:a}
  \end{subfigure}
  ~
  \begin{subfigure}[b]{\SubfigureWidth{}}
    \includegraphics[width=\columnwidth]{ch06_paper_gpp/figs/8b.pdf}
    \caption{}
    \label{fig:06:step2b_results:b}
  \end{subfigure}
  \\
  \begin{subfigure}[b]{\SubfigureWidth{}}
    \includegraphics[width=\columnwidth]{ch06_paper_gpp/figs/8c.pdf}
    \caption{}
    \label{fig:06:step2b_results:c}
  \end{subfigure}
  ~
  \begin{subfigure}[b]{\SubfigureWidth{}}
    \includegraphics[width=\columnwidth]{ch06_paper_gpp/figs/8d.pdf}
    \caption{}
    \label{fig:06:step2b_results:d}
  \end{subfigure}
  \caption{Fractional change in \acf{GPP} over the \nth{21} century (2100
    versus 2000) in the \acs{RCP}8.5 scenario for different statistical models
    (step 2b): (a) \acs{CMIP}5 \acf{MMM} of the fractional \acs{GPP} change,
    (b) rescaled \acs{CMIP}5 \acl{MMM} (r\acs{MMM}) using \cref{eq:06:xxx}, (c)
    linear \acf{LASSO} model, and (d) \acf{GBRT} model. The geographical
    patterns from the different statistical models are very similar and show a
    higher \acs{GPP} increase in high latitudes and a lower \acs{GPP} closer
    to the equator. \AdaptedFrom{Schlund2020}.}
  \label{fig:06:step2b_results}
\end{figure}

\begin{figure}[t]
  \centering
  \begin{subfigure}[b]{\SubfigureWidth{}}
    \includegraphics[width=\columnwidth]{ch06_paper_gpp/figs/9a.pdf}
    \caption{}
    \label{fig:06:step2b_results_biases:a}
  \end{subfigure}
  \\
  \begin{subfigure}[b]{\SubfigureWidth{}}
    \includegraphics[width=\columnwidth]{ch06_paper_gpp/figs/9b.pdf}
    \caption{}
    \label{fig:06:step2b_results_biases:b}
  \end{subfigure}
  ~
  \begin{subfigure}[b]{\SubfigureWidth{}}
    \includegraphics[width=\columnwidth]{ch06_paper_gpp/figs/9c.pdf}
    \caption{}
    \label{fig:06:step2b_results_biases:c}
  \end{subfigure}
  \caption{(a) Ratio of the rescaled \acs{CMIP}5 ensemble mean of the
    fractional change in \acf{GPP} over the \nth{21} century using
    \cref{eq:06:xxx} (r\acs{MMM}; see \cref{fig:06:step2b_results:b}) and its
    unweighted version (\acs{MMM}; see \cref{fig:06:step2b_results:a}). The
    plot shows an almost constant value over the whole globe with a mean of
    $0.91$, which corresponds to the ratio of the constrained global mean
    \acs{GPP} change over the \nth{21} century from step 1 ($39 \unit{\%}$)
    and the \acs{CMIP}5 ensemble mean global mean GPP change ($43 \unit{\%}$).
    All values close to $0$ for the dataset in the denominator have been
    masked to avoid divisions by $0$. Absolute (b) and relative (c)
    differences between the fractional \acs{GPP} change over the \nth{21}
    century predicted by the \acf{GBRT} model (see
    \cref{fig:06:step2b_results:d}) and r\acs{MMM} (see
    \cref{fig:06:step2b_results:b}). For panel (c), all values close to $0$ for
    the dataset in the denominator have been masked to avoid divisions by $0$.
    Both plots show a good agreement over large parts of the globe
    (corresponding to values of $0$). There are large absolute differences in
    the Sahara region and central Asia. The largest relative differences
    (except for the Sahara and Arabian Peninsula region) appear over South
    America, South Africa, the west coast of Africa, the Middle East, parts of
    Australia and western parts of the United States.
    \AdaptedFrom{Schlund2020}.}
  \label{fig:06:step2b_results_biases}
\end{figure}

\begin{figure}[t]
  \centering
  \begin{subfigure}[b]{\SubfigureWidth{}}
    \includegraphics[width=\columnwidth]{ch06_paper_gpp/figs/10a.pdf}
    \caption{}
    \label{fig:06:comparison_step2a_step2b:a}
  \end{subfigure}
  ~
  \begin{subfigure}[b]{\SubfigureWidth{}}
    \includegraphics[width=\columnwidth]{ch06_paper_gpp/figs/10b.pdf}
    \caption{}
    \label{fig:06:comparison_step2a_step2b:b}
  \end{subfigure}
  \\
  \begin{subfigure}[b]{\SubfigureWidth{}}
    \includegraphics[width=\columnwidth]{ch06_paper_gpp/figs/10c.pdf}
    \caption{}
    \label{fig:06:comparison_step2a_step2b:c}
  \end{subfigure}
  ~
  \begin{subfigure}[b]{\SubfigureWidth{}}
    \includegraphics[width=\columnwidth]{ch06_paper_gpp/figs/10d.pdf}
    \caption{}
    \label{fig:06:comparison_step2a_step2b:d}
  \end{subfigure}
  \caption{Top row: Absolute future \acf{GPP} at the end of the \nth{21}
    century (\range{2091}{2100}) calculated using step 2a (a) and step 2b (b).
    Both approaches give similar results with global averages of $169
    \unit{\GtCyr{}}$ (a) and $175 \unit{\GtCyr{}}$ (b), which are both
    consistent with the global result of step 1 of
    $\pmrangeunit{171}{12}{\GtCyr{}}$. The pattern correlation between both
    approaches is $R^2 = 0.97$. Bottom row: Absolute (c) and relative (d)
    differences between panels (a) and (b). \AdaptedFrom{Schlund2020}.}
  \label{fig:06:comparison_step2a_step2b}
\end{figure}


\section{Summary and Discussion}
\label{sec:06:summary_and_discussion}

TBA.
