%=============================================================================%
%                                Dissertation                                 %
%                               Manuel Schlund                                %
%                                  (c) 2020                                   %
%=============================================================================%
%                                  Paper GPP                                  %
%=============================================================================%



\chapter{Constraining Future \acl{GPP} with Machine Learning}
\label{ch:06:paper_gpp}

- Single-process emergent constraints have disadvantages

To overcome these, we explore an alternative approach that expands the MDER
method with an ML algorithm. In contrast to emergent constraints, this approach
considers (1) multiple process-based predictors, (2) multi-dimensional
(gridded) target variables and (3) non-linear relationships between the
predictors and the target variable. The target variable analyzed is GPP. As
described in detail in section XXX, the carbon cycle slows down global warming
by absorbing 30 \% CO2, thus accurately quantifying the carbon sink is crucial.
GPP is most largest flux of the terrestrial carbon cycle. Due to feedbacks
(section XXX, NAME OF FEEDBACKS), multi-model projectionf of future GPP is
uncertain, especially in CMIP5 RCP 8.5 at the end of the century. Thus, the aim
of the works presented in this chapter is to reduce uncertainties in
multi-model projections of future GPP in the RCP 8.5 scenario at the end of the
21st century with a new ML-based weighting technique and observations. The new
two‐step approach utilizes aspects of emergent constraints and the MDER
technique in combination with a supervised machine learning algorithm. In the
first step, we apply an existing emergent... (see SCHLUND ET AL., 2020).


\section{Step 1: XXXXXXXXXXXXXXXXXX}
\label{sec:06:step_1}

TBA.