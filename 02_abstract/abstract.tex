%=============================================================================%
%                            Doctoral Dissertation                            %
%                            (c) by Manuel Schlund                            %
%=============================================================================%

%=============================================================================%
% This work is licensed under a
% Creative Commons Attribution 4.0 International License.
%
% You should have received a copy of the license along with this
% work. If not, see <http://creativecommons.org/licenses/by/4.0/>.
%=============================================================================%



\chapter{\abstractname}

A precise quantification of climate change is crucial to assess optimal
mitigation and adaptation strategies. \AcfpAbstract{ESM}, which are
state-of-the-art climate models that allow numerical simulations of the complex
physical, biological, and chemical processes of the Earth system, are common
tools to understand and project climate change. Due to the chaotic nature of
the climate system, unknowns in future emission pathways, and uncertainties in
the climate models, projections of the future climate are associated with large
uncertainties. The main focus of this thesis is the analysis of future climate
projections from \acsp{ESM} participating in the \acfAbstract{CMIP} with the
aim to reduce uncertainties in climate projections with observations.

In a first step, climate sensitivity (\ie{} the temperature response of the
climate system to an external forcing) is evaluated in the latest generation of
\acsp{ESM} from \acs{CMIP}6. For the \acfAbstract{ECS}, which is an estimation
of the equilibrium temperature response that follows a doubling of the
atmospheric \acfAbstract{CO2} concentration, a \acfAbstract{MMM} of $3.74
\unit{K}$ and a multi-model range of $\rangeunit{1.8}{5.6}{K}$ are found. These
values are higher than in any previous \acs{CMIP} ensemble before. Moreover, a
third of the analyzed \acs{CMIP}6 models exceed the upper bound of the likely
\acs{ECS} range of $\rangeunit{1.5}{4.5}{K}$ assessed by the \acfAbstract{IPCC}
Fifth \acl{AR} (\acs{AR}5) from 2013. Similarly, the transient response of the
climate system to a doubling of \acs{CO2}, also known as \acfAbstract{TCR},
shows an inter-model range of $\rangeunit{1.3}{3.0}{K}$ with an upper bound
again higher than the likely range assessed in \acs{AR}5 of
$\rangeunit{1.0}{2.5}{K}$. Possible reasons for the increased climate
sensitivity in many \acs{CMIP}6 models are the addition of new prognostic
aerosol schemes that include aerosol-cloud interactions and changes in the
microphysical representation of mixed-phase clouds. The changes in the
mixed-cloud representation reduce the strong negative shortwave cloud phase
change feedback over the Southern Ocean that is present in climate models from
previous \acs{CMIP} generations.

To reduce uncertainties in \acs{ECS} projected by the \acs{CMIP}6 models, 11
published emergent constraints on \acs{ECS} (mostly derived from models
participating in \acs{CMIP}5, the predecessor generation of \acs{CMIP}6) are
systematically analyzed. Emergent constraints are potentially promising
approaches to reduce uncertainties in climate model projections by combining
observations and output from \acsp{ESM}. The focus of this analysis is on
testing if these emergent constraints hold for \acsp{ESM} participating in
\acs{CMIP}6. Since none of the emergent constraints considered here have been
derived from the \acs{CMIP}6 ensemble, the \acs{CMIP}6 models can be used for
cross-checking the emergent constraints on a new model ensemble. The
application of the emergent constraints to \acs{CMIP}6 data shows a decrease in
skill and statistical significance of the emergent relationship for nearly all
constraints, with this decrease being large in many cases. Consequently, the
size of the constrained \acs{ECS} ranges ($66 \unit{\%}$ confidence intervals)
widens by $51 \unit{\%}$ on average in \acs{CMIP}6 compared to \acs{CMIP}5.
This is likely related to the increased multi-model spread of \acs{ECS} in
\acs{CMIP}6, but may in some cases also be due to spurious statistical
relationships or a too-small number of models in the ensemble that the emergent
constraint was originally derived from. The corresponding best estimates of
\acs{ECS} given by the emergent constraints also increase from \acs{CMIP}5 to
\acs{CMIP}6 by $12 \unit{\%}$ on average. This can at least be partly explained
by the increased number of high-\acs{ECS} models in \acs{CMIP}6 without a
corresponding change in the constraint predictors, suggesting the emergence of
new feedback processes rather than changes in the strength of those previously
dominant. The results support previous studies concluding that emergent
constraints should be based on an independently verifiable physical mechanism
and that process-based emergent constraints on \acs{ECS} should rather be
thought of as constraints for the process or feedback they are actually
targeting.

To overcome these issues of single-process-oriented emergent constraints, an
alternative approach based on \acfAbstract{ML} is introduced. Since this new
technique relies on a large number of data points to train the \acs{ML}
algorithm, the scalar climate sensitivity expressed as \acs{ECS} or \acs{TCR}
is not an appropriate target variable. Therefore, \acfAbstract{GPP} as a
process that contributes to climate sensitivity is studied as an alternative.
\acs{GPP} is the largest flux of the terrestrial carbon uptake and slows down
global warming by removing \acs{CO2} from the atmosphere. In this analysis, an
existing emergent constraint on \acs{CO2} fertilization is combined with an
\acs{ML} approach to constrain the spatial variations of multi-model \acs{GPP}
projections. In the first step of the two-step approach, observed changes in
the \acs{CO2} seasonal cycle at \acl{KUM} are used to constrain the global mean
\acs{GPP} at the end of the \nth{21} century (\range{2091}{2100}) in \acl{RCP}
8.5 simulations with \acsp{ESM} participating in \acs{CMIP}5 to
$\pmrangeunit{171}{12}{\GtCyr{}}$, compared to the unconstrained model range
$\rangeunit{156}{247}{\GtCyr{}}$. In a second step, an \acs{ML} model is used
to constrain gridded future absolute \acs{GPP} and gridded fractional \acs{GPP}
change in two independent approaches. For this, observational data is fed into
the \acs{ML} algorithm that has been trained on \acs{CMIP}5 data to learn
relationships between present-day physically relevant diagnostics and the
target variable. In a leave-one-model-out \acl{CV} approach, the \acs{ML} model
shows superior performance to the \acs{CMIP}5 \acs{MMM}. The new approach
predicts a higher \acs{GPP} increase in high latitudes and a lower \acs{GPP}
increase in regions closer to the equator.
