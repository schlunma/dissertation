%=============================================================================%
%                                Dissertation                                 %
%                               Manuel Schlund                                %
%                                  (c) 2020                                   %
%=============================================================================%
%                                  Paper ECS                                  %
%=============================================================================%



\chapter{Evaluation of Emergent Constraints on the \acl{ECS} in \acs{CMIP}6}
\label{ch:05:paper_ecs}

The massive spread in \ac{ECS} within the \acs{CMIP}6 ensemble discussed in the
previous chapter begs the question about the interpretation and credibility of
these results. Since the \acs{CMIP}6 models provide crucial input for the
upcoming international climate assessment of the \ac{IPCC} (\acs{AR}6), this is
not only important for the scientific community, but also directly affects
society and policymakers. An important question in this context could be the
following: Is it reasonable to adapt the assessed \ac{ECS} range from \acs{AR}5
($\range{1.5}{4.5} \unit{K}$) in \acs{AR}6 based on the \acs{CMIP}6 model range
of $\range{1.8}{5.6} \unit{K}$? An argument for this is that due to the many
new and improved processes included to the models, the \acs{CMIP}6 ensemble is
arguably more realistic than its predecessor generations. However, even the
\acs{CMIP}6 models are far from fully simulating all aspects and processes of
the full vastly complex Earth system. This could introduce new biases which
have not been present in earlier \ac{CMIP} models. For example, the apparent
increase in climate sensitivity in the \acs{CMIP}6 models might be wrongly
caused by missing processes that cancel out the effects of the newly included
processes. Thus, it is vital to take other sources of information into account
before answering the question raised above.

One of the most important sources of information for this purpose is
undoubtedly observational data. As presented in
\cref{subsec:02:emergent_constraints}, appropriate observations of the Earth
system can be utilized within the emergent constraints framework to assess the
accuracy of climate model projections and reduce associated uncertainties. For
\ac{ECS}, a large variety of different emergent constraints has been developed
over the years, which we now evaluate on the \acs{CMIP}6 ensemble in order to
assess whether they still hold for these new climate models. In total we assess
eleven emergent constraints on \ac{ECS} (see
\cref{tab:05:overview_emergent_constraints}), out of which the most are related
to cloud feedbacks since these constitute the most important source of
uncertainty for \ac{ECS} \commentcite{Boucher2013, Flato2013}{see also
  \cref{subsec:02:climate_feedbacks}}. Since all of the evaluated emergent
constraints have been derived on the \acs{CMIP}3 or \acs{CMIP}5 ensemble, the
\acs{CMIP}6 models offer an exciting possibility to perform out-of-sample
testing using the previously unseen data.

This analysis is already published in \textcite{Schlund2020a}. For this paper,
the author of this thesis lead the writing and the analysis and implemented the
code to reproduce this analysis with all figures and tables using the
\ac{ESMValTool}. In this chapter, we first introduce the data (climate model
output and observations) and methods used in this study
(\cref{sec:05:data_and_methods}). After that, we present the eleven emergent
constraints on \acs{ECS} and evaluate them on the \acs{CMIP}5 and \acs{CMIP}6
ensemble (\cref{sec:05:comparison_of_emergent_constraints}). Finally, we
provide a discussion (\cref{sec:05:discussion}) and a summary of the results
(\cref{sec:05:summary}).

\begin{table}[p]
  \centering
  \hbadness=3500
  \begin{tabular}{p{0.07\columnwidth} p{0.24\columnwidth} p{0.29\columnwidth}
      p{0.29\columnwidth}}
    \toprule
    Label & Reference & Description of \xaxis{} & Variables and corresponding
    observations \\
    \midrule
    BRI & \textcite{Brient2016} & Sensitivity of shortwave cloud albedo
    to changes in \acf{SST} [$\si{\pctK{}}$] & \makecell{\tabitem\emph{hur}
      (ERA-Interim) \\ \tabitem \emph{rsdt} (CERES-EBAF) \\ \tabitem
      \emph{rsut} (CERES-EBAF) \\ \tabitem \emph{rsutcs} (CERES-EBAF) \\
      \tabitem \emph{ts} (HadISST)} \\
    \midrule
    COX & \textcite{Cox2018} & $\Psi$ (temperature variability metric)
    [$\si{K}$] &
    \tabitem \emph{tas} (HadCRUT4) \\
    \midrule
    LIP & \textcite{Lipat2017} & Extent of Southern Hemisphere Hadley cell
    [$\si{\degree}$] & \tabitem \emph{va} (ERA-Interim) \\
    \midrule
    SHD & \textcite{Sherwood2014} & $D$-index (large-scale lower-tropospheric
    mixing) [$\si{1}$] & \tabitem \emph{wap} (ERA-Interim) \\
    \midrule
    SHL & \textcite{Sherwood2014} & \acs{LTMI} (\acl{LTMI}) [$\si{1}$] &
    \makecell{\tabitem \emph{hur} (ERA-Interim) \\ \tabitem \emph{ta}
      (ERA-Interim) \\ \tabitem \emph{wap} (ERA-Interim)} \\
    \midrule
    SHS & \textcite{Sherwood2014} & $S$-index (small-scale lower-tropospheric
    mixing) [$\si{1}$] & \makecell{\tabitem \emph{hur} (ERA-Interim) \\
      \tabitem \emph{ta} (ERA-Interim) \\ \tabitem \emph{wap} (ERA-Interim)} \\
    \midrule
    SU & \textcite{Su2014} & Error in vertical profile of relative humidity
    [$\si{1}$] & \tabitem \emph{hur} (AIRS, MLS-Aura) \\
    \midrule
    TIH & \textcite{Tian2015} & Tropical mid-tropospheric humidity asymmetry
    index from AIRS [$\si{\%}$] & \tabitem \emph{hus} (AIRS) \\
    \midrule
    TII & \textcite{Tian2015} & Southern \acs{ITCZ} index from GPCP
    [$\si{\mmday{}}$] & \tabitem \emph{pr} (GPCP) \\
    \midrule
    VOL* & \textcite{Volodin2008} & Difference between tropical and southern
    midlatitudinal cloud fraction [$\si{\%}$] & \tabitem \emph{clt} (ISCCP
    D-2) \\
    \midrule
    ZHA* & \textcite{Zhai2015} & Seasonal response of \acl{MBLC} cloud
    fraction to changes in \acs{SST} [$\si{\pctK{}}$] &
    \makecell{\tabitem \emph{cl} (Cloudsat/CALIPSO) \\ \tabitem \emph{tos}
      (AMSRE \ac{SST}) \\ \tabitem \emph{wap} (ERA-Interim)} \\
    \bottomrule
  \end{tabular}
  \caption{Overview of the eleven emergent constraints on the \acf{ECS}
    evaluated in this chapter. Detailed descriptions of the variables used to
    calculate the \xaxis{} values of the emergent constraints and the
    references for the corresponding observations (fourth column) are given in
    \cref{tab:app:a:overview_variables,tab:app:a:observations}, respectively.
    For emergent constraints marked with an asterisk (*), the observational
    value of the original publication is used to calculate the observational
    constraint. For all others, the observational value is computed with the
    \acs{ESMValTool}. \AdaptedFrom{Schlund2020a}.}
  \label{tab:05:overview_emergent_constraints}
\end{table}


\section{Data and Methods}
\label{sec:05:data_and_methods}

In this study we use the output from climate models participating in
\acs{CMIP}5 and \acs{CMIP}6 as shown in
\cref{tab:app:a:cmip5_models,tab:app:a:cmip6_models}, respectively. In
addition, for each emergent constraint at least one observational dataset is
used to calculate the observational constraint. A complete list of these
datasets is given in \cref{tab:05:overview_emergent_constraints} and in
\cref{tab:app:a:observations}. Following other similar studies, we calculate
\ac{ECS} with the Gregory regression method \autocite{Gregory2004}, which is
described in detail in \cref{subsec:02:ecs}. An overview of the eleven emergent
constraints on \ac{ECS} analyzed in this study including the variables required
for their calculations is given in \cref{tab:05:overview_emergent_constraints}
and \cref{sec:05:comparison_of_emergent_constraints}. We chose these particular
emergent constraints since these had already been implemented in the
\ac{ESMValTool} at the time of writing this study, which greatly facilitated
this analysis. For all emergent constraints, we use the historical simulations
of \acs{CMIP}5 and \acs{CMIP}6 in order to ensure maximum agreement with the
observational data. If necessary, the historical simulation of \acs{CMIP}5 is
extended after its final year 2005 with data from the \acs{RCP}8.5 scenario
\autocite{Riahi2011}. Note that we only use data through 2014, during which
time all \ac{RCP} scenarios behave similarly and the choice of the scenario is
not expected to affect the results considerably. Such an extension is not
needed for \acs{CMIP}6 models as their historical simulations cover a longer
time period until 2014. We quantify the skill of the emergent constraints with
three metrics: the coefficient of determination $R^2$ of the linear
relationship, the statistical significance using a two-sided $t$-test based on
the Pearson correlation coefficient $r$ (see \cref{eq:02:t}) and the
constrained range of \ac{ECS} based on the constrained \ac{PDF} (see
\cref{eq:02:pdf_y_given_x0}).


\section{Comparison of Emergent Constraints on \acs{ECS} for \acs{CMIP}5 and
  \acs{CMIP}6}
\label{sec:05:comparison_of_emergent_constraints}

In this section we describe and discuss the eleven emergent constraints on
\ac{ECS} summarized in \cref{tab:05:overview_emergent_constraints} using
\acs{CMIP}5 and \acs{CMIP}6 data
(\cref{subsec:05:bri,subsec:05:cox,subsec:05:lip,subsec:05:shd,subsec:05:shs,%
  subsec:05:shl,subsec:05:su,subsec:05:tih,subsec:05:tii,subsec:05:vol,%
  subsec:05:zha}) and provide a best estimate for \ac{ECS} and statistical
significance of the eleven emergent constraints in
\cref{subsec:05:emergent_constraints_summary}. While most of these emergent
constraints have been derived using data from the \acs{CMIP}5 and/or
\acs{CMIP}3 ensembles, to our knowledge none of them has been evaluated on
the \acs{CMIP}6 ensemble so far. The results for the individual emergent
constraints described in the following are shown in
\cref{fig:05:bri_cox_lip,fig:05:shd_shl_shs,fig:05:su_tih_tii,fig:05:vol_zha}.
The left columns in these figures show the emergent relationships, including
the uncertainty of the linear regressions (blue and orange shaded areas; see
\cref{eq:02:spe}) and the uncertainty in the observations (gray shaded area;
see \cref{eq:02:pdf_obs}). The right columns show the \acp{PDF} of \ac{ECS} in
the original model ensemble (histogram) and the constrained distribution given
by the emergent constraints (blue and orange line; see
\cref{eq:02:pdf_y_given_x0}). \Cref{tab:05:overview_results} shows the
corresponding $66 \unit{\%}$ confidence intervals (\ie{} the $\range{17}{83}
\unit{\%}$ intervals) of \ac{ECS} derived from the \acp{PDF} given by
\cref{eq:02:pdf_y_given_x0} and the $p$-values used to assess the significance
of the emergent relationships.


\subsection{Sensitivity of Shortwave Cloud Albedo to changes in \acl{SST}
  (BRI)}
\label{subsec:05:bri}

In this emergent constraint proposed by \textcite{Brient2016}, \ac{ECS} is
correlated with the \ac{TLC} albedo, \ie{} using the covariance of clouds with
changes in \acp{SST}. Differences in the \ac{TLC} albedo account for more than
half of the variance of \ac{ECS} in the \acs{CMIP}5 ensemble. Following
\textcite{Brient2016}, \ac{TLC} regions are defined as grid points that are in
the driest quartile of $500 \unit{hPa}$ relative humidity of all grid cells
over the ocean between $30 \unit{\degree S}$ and $30 \unit{\degree N}$. The
\ac{TLC} albedo is obtained by calculating the ratio of \ac{TOA} shortwave
cloud radiative forcing and solar insolation averaged over the \ac{TLC} region.
The regression coefficients of de-seasonalized variations of \ac{TLC} shortwave
albedo and \ac{SST} (in $\si{\pctK{}}$) are then used as an emergent constraint
for \ac{ECS}. Here, we use observational data from HadISST for \ac{SST}
\autocite{Rayner2003}, ERA-Interim for $500 \unit{hPa}$ relative humidity
\autocite{Dee2011} and CERES-EBAF \autocite{Loeb2018} for the \ac{TOA}
radiative fluxes over the time period \range{2001}{2005}. In the original
publication, \textcite{Brient2016} use similar observation-based datasets with
the exception of \ac{SST}, where they take ERSST data \autocite{Smith2003} as
reference instead. Our analysis yields a $66 \unit{\%}$ confidence range for
\ac{ECS} of $\pmrangeunit{3.72}{0.59}{K}$ for \acs{CMIP}5 ($R^2 = 0.38$) and
$\pmrangeunit{4.32}{1.07}{K}$ for \acs{CMIP}6, with much lower $R^2 = 0.12$.
The original publication states a best estimate of $4.0 \unit{K}$ with a very
low likelihood of values below $2.3 \unit{K}$ ($90 \unit{\%}$ confidence). The
statistical significance of the emergent relationship drops from $p = 0.0005$
for \acs{CMIP}5 to $p = 0.0355$ for \acs{CMIP}6.

\begin{figure}[p]
  \centering
  \begin{subfigure}[b]{\SmallSubfigureWidth{}}
    \includegraphics[width=\columnwidth]{
      ch05_paper_ecs/figs/scatterplot_merged_training_data_bri.pdf}
    \caption{}
    \label{fig:05:bri_cox_lip:a}
  \end{subfigure}
  ~
  \begin{subfigure}[b]{\SmallSubfigureWidth{}}
    \includegraphics[width=\columnwidth]{
      ch05_paper_ecs/figs/target_distribution_training_data_bri.pdf}
    \caption{}
    \label{fig:05:bri_cox_lip:b}
  \end{subfigure}
  \\
  \begin{subfigure}[b]{\SmallSubfigureWidth{}}
    \includegraphics[width=\columnwidth]{
      ch05_paper_ecs/figs/scatterplot_merged_training_data_cox.pdf}
    \caption{}
    \label{fig:05:bri_cox_lip:c}
  \end{subfigure}
  ~
  \begin{subfigure}[b]{\SmallSubfigureWidth{}}
    \includegraphics[width=\columnwidth]{
      ch05_paper_ecs/figs/target_distribution_training_data_cox.pdf}
    \caption{}
    \label{fig:05:bri_cox_lip:d}
  \end{subfigure}
  \\
  \begin{subfigure}[b]{\SmallSubfigureWidth{}}
    \includegraphics[width=\columnwidth]{
      ch05_paper_ecs/figs/scatterplot_merged_training_data_lip.pdf}
    \caption{}
    \label{fig:05:bri_cox_lip:e}
  \end{subfigure}
  ~
  \begin{subfigure}[b]{\SmallSubfigureWidth{}}
    \includegraphics[width=\columnwidth]{
      ch05_paper_ecs/figs/target_distribution_training_data_lip.pdf}
    \caption{}
    \label{fig:05:bri_cox_lip:f}
  \end{subfigure}
  \caption{Emergent constraints BRI, COX and LIP applied to the \acs{CMIP}5
    (blue) and \acs{CMIP}6 ensemble (orange). (a), (c), (e) Emergent
    relationships (solid lines) for the \acs{CMIP} models (numbers; see
    \cref{tab:app:a:cmip5_models,tab:app:a:cmip6_models} for details) with
    their standard prediction errors (blue and orange shaded areas; see
    \cref{eq:02:spe}). The vertical dashed line corresponds to the
    observational reference with its standard error (gray shaded area). The
    horizontal dashed lines show the best estimates of the constrained
    \acf{ECS} and the colored circles mark the \aclp{MMM} for \acs{CMIP}5
    (blue) and \acs{CMIP}6 (orange). (b), (d), (f) \Aclp{PDF} for \acs{ECS} of
    the emergent constraints (solid lines; see \cref{eq:02:pdf_y_given_x0})
    and the unconstrained model ensembles (histograms). Due to the
    availability of data, the histograms may differ for the different emergent
    constraints. \AdaptedFrom{Schlund2020a}.}
  \label{fig:05:bri_cox_lip}
\end{figure}


\subsection{Temperature Variability Metric (COX)}
\label{subsec:05:cox}

The emergent constraint on \ac{ECS} proposed by \textcite{Cox2018} uses a
temperature variability metric $\Psi$ that is based on the interannual
variation of \ac{GSAT} calculated from its variance (in time) and $1$-year lag
autocorrelation. In contrast to the majority of emergent constraints that focus
on cloud-related processes, this constraint is based on the
fluctuation-dissipation theorem, which relates the long-term response of the
climate system to an external forcing (\ac{ECS}) and short-term variations of
the climate system (climate variability). This arguably places the constraint
on a more solid theoretical foundation, although several questions have been
raised on the robustness of the results to choices made in the analysis
\autocite{Brown2018, PoChedley2018, Rypdal2018}. For example,
\textcite{Annan2020} has shown that the assumed linear relationship between
$\Psi$ and \ac{ECS} does not hold when adding a deep ocean to the model. As
observational data, here we use the HadCRUT4 dataset \autocite{Morice2012} over
the time period \range{1880}{2014}. Under the COX constraint we assess a $66
\unit{\%}$ \ac{ECS} range of $\pmrangeunit{3.03}{0.73}{K}$ for \acs{CMIP}5
($R^2 = 0.31$) and $\pmrangeunit{3.71}{1.09}{K}$ for \acs{CMIP}6 ($R^2 =
0.01$). \textcite{Cox2018} derive a $66 \unit{\%}$ range of
$\pmrangeunit{2.8}{0.6}{K}$ from a different subset of \acs{CMIP}5 models but
the same observations. When moving from \acs{CMIP}5 to \acs{CMIP}6, the
significance of the emergent relation drops massively from $p = 0.0032$ to $p =
0.5415$, respectively.


\subsection{Extent of Southern Hemisphere Hadley Cell (LIP)}
\label{subsec:05:lip}

The results of \textcite{Lipat2017} show that the multi-year average extent of
the Hadley cell correlates with \acs{ECS} in \acs{CMIP}5 models. The Hadley
cell edge is defined as the latitude of the first two grid cells from the
Equator going south where the zonal average $500 \unit{hPa}$ mass stream
function calculated from \ac{DJF} means of the meridional wind field changes
sign from negative to positive. \textcite{Lipat2017} explain this correlation
by tying it to the observed correlation of the interannual variability in
midlatitudinal clouds and their radiative effects with the poleward extent of
the Hadley cell. For the calculation of the emergent constraint, we use
reanalysis data from ERA-Interim \autocite{Dee2011} for the meridional wind
speed over the time period \range{1980}{2005}. Our application of this emergent
constraint gives \ac{ECS} $66 \unit{\%}$ ranges of
$\pmrangeunit{2.97}{0.75}{K}$ for \acs{CMIP}5 ($R^2 = 0.18$) and
$\pmrangeunit{3.75}{1.11}{K}$ for \acs{CMIP}6 ($R^2 < 0.01$). The original
publication does not specify an \ac{ECS} range. For \acs{CMIP}6, the emergent
constraint shows a much lower statistical significance ($p = 0.6791$) than for
\acs{CMIP}5 ($p = 0.0228$).


\subsection{Large-Scale Lower-Tropospheric Mixing (SHD)}
\label{subsec:05:shd}

\textcite{Sherwood2014} propose that the degree of mixing in the lower
troposphere determines the response of boundary layer clouds and humidity to
climate warming, as the associated moisture transport would increase rapidly in
a warmer atmosphere due to the Clausius-Clapeyron relationship. The large-scale
component $D$ of this mixing is defined as the ratio of shallow to deep
overturning. $D$ is calculated from the vertical velocities averaged over two
height regions: $850$ and $700 \unit{hPa}$ for shallow overturning and $600$,
$500$ and $400 \unit{hPa}$ for deep overturning. Both quantities are averaged
over parts of the tropical ocean region away from the regions of highest
\ac{SST} and strongest mid-level ascent, specifically the region $\range{30
  \unit{\degree S}}{30 \unit{\degree N}}$, $\range{160 \unit{\degree W}}{30
  \unit{\degree E}}$, wherever air is ascending at low levels. As
observation-based data, we use vertical velocities from ERA-Interim
\autocite{Dee2011} over the time period \range{1989}{1998} similar to the
original publication. We derive \ac{ECS} $66 \unit{\%}$ confidence ranges of
$\pmrangeunit{3.65}{0.64}{K}$ for \acs{CMIP}5 ($R^2 = 0.28$) and
$\pmrangeunit{3.77}{1.06}{K}$ for \acs{CMIP}6 ($R^2 = 0.03$).
\textcite{Sherwood2014} do not give a best estimate for \ac{ECS} based on the
large-scale component of mixing $D$ or its small-scale counterpart $S$ (see
\cref{subsec:05:shs}) but instead for the sum of $D + S$ only (see
\cref{subsec:05:shl}). The regression shows a much lower significance for
\acs{CMIP}6 ($p = 0.2805$) than for \acs{CMIP}5 ($p = 0.0037$).

\begin{figure}[!t]
  \centering
  \begin{subfigure}[b]{\SubfigureWidth{}}
    \includegraphics[width=\columnwidth]{
      ch05_paper_ecs/figs/scatterplot_merged_training_data_shd.pdf}
    \caption{}
    \label{fig:05:shd_shl_shs:a}
  \end{subfigure}
  ~
  \begin{subfigure}[b]{\SubfigureWidth{}}
    \includegraphics[width=\columnwidth]{
      ch05_paper_ecs/figs/target_distribution_training_data_shd.pdf}
    \caption{}
    \label{fig:05:shd_shl_shs:b}
  \end{subfigure}
  \\
  \begin{subfigure}[b]{\SubfigureWidth{}}
    \includegraphics[width=\columnwidth]{
      ch05_paper_ecs/figs/scatterplot_merged_training_data_shl.pdf}
    \caption{}
    \label{fig:05:shd_shl_shs:c}
  \end{subfigure}
  ~
  \begin{subfigure}[b]{\SubfigureWidth{}}
    \includegraphics[width=\columnwidth]{
      ch05_paper_ecs/figs/target_distribution_training_data_shl.pdf}
    \caption{}
    \label{fig:05:shd_shl_shs:d}
  \end{subfigure}
  \\
  \begin{subfigure}[b]{\SubfigureWidth{}}
    \includegraphics[width=\columnwidth]{
      ch05_paper_ecs/figs/scatterplot_merged_training_data_shs.pdf}
    \caption{}
    \label{fig:05:shd_shl_shs:e}
  \end{subfigure}
  ~
  \begin{subfigure}[b]{\SubfigureWidth{}}
    \includegraphics[width=\columnwidth]{
      ch05_paper_ecs/figs/target_distribution_training_data_shs.pdf}
    \caption{}
    \label{fig:05:shd_shl_shs:f}
  \end{subfigure}
  \caption{As in \cref{fig:05:bri_cox_lip} but for the emergent constraints
    SHD, SHL and SHS. \AdaptedFrom{Schlund2020a}.}
  \label{fig:05:shd_shl_shs}
\end{figure}


\subsection{Small-Scale Lower-Tropospheric Mixing (SHS)}
\label{subsec:05:shs}

The small-scale mixing $S$ \autocite{Sherwood2014} is calculated from the
differences in relative humidity and temperature between $700$ and $850
\unit{hPa}$. The differences are averaged over all grid cells within the upper
quartile of the annual mean $500 \unit{hPa}$ ascent rate (within ascending
regions) in the tropics. The tropics are defined as the region between $30
\unit{\degree S}$ and $30 \unit{\degree N}$. In the \ac{CFMIP} models
\autocite{Webb2017}, for which convective tendencies are available, upward
moisture transport by parameterized convection is shown to increase more
rapidly with warming for higher values of $S$. We use reanalysis data from
ERA-Interim \autocite{Dee2011} for temperature and relative humidity to
calculate the observational constraint (\range{1989}{1998}). Our analysis shows
a $66 \unit{\%}$ range of \ac{ECS} of $\pmrangeunit{3.07}{0.73}{K}$ for
\acs{CMIP}5 ($R^2 = 0.13$) and $\pmrangeunit{3.48}{1.07}{K}$ for \acs{CMIP}6
($R^2 = 0.12$). The correlation of $S$ and \acs{ECS} shows a slightly higher
significance in the \acs{CMIP}6 ensemble ($p = 0.0396$) than in the \acs{CMIP}5
ensemble ($p = 0.0647$). The SHS constraint is one of the two analyzed emergent
constraints (ZHA being the other exception) that shows a higher statistical
significance for the \acs{CMIP}6 than for the \acs{CMIP}5 ensemble.


\subsection{Lower-Tropospheric Mixing Index (SHL)}
\label{subsec:05:shl}

The \ac{LTMI} formulated by \textcite{Sherwood2014} is defined as the sum of
the small-scale mixing $S$ (see \cref{subsec:05:shs}) and the large-scale
mixing $D$ (see \cref{subsec:05:shd}), which are supposed to capture
complementary components of the total mixing phenomenon.
\textcite{Sherwood2014} argue that the increase in dehydration depends on
initial mixing linking it to cloud feedbacks and thus also to \ac{ECS}. For
this constraint, we derive an \ac{ECS} $66 \unit{\%}$ confidence range of
$\pmrangeunit{3.42}{0.65}{K}$ for \acs{CMIP}5 ($R^2 = 0.41$) and
$\pmrangeunit{3.67}{1.06}{K}$ for \acs{CMIP}6 ($R^2 = 0.16$).
\textcite{Sherwood2014} give a best estimate of about $4 \unit{K}$ with a lower
limit of $3 \unit{K}$. Similar to both other constraints by
\textcite{Sherwood2014}, SHD and SHS, the statistical significance of the SHL
emergent relation decreases in \acs{CMIP}6 ($p = 0.0138$) compared to
\acs{CMIP}5 ($p = 0.0002$).


\subsection{Error in Vertical Profile of Relative Humidity (SU)}
\label{subsec:05:su}

\begin{figure}[!t]
  \centering
  \begin{subfigure}[b]{\SubfigureWidth{}}
    \includegraphics[width=\columnwidth]{
      ch05_paper_ecs/figs/scatterplot_merged_training_data_su.pdf}
    \caption{}
    \label{fig:05:su_tih_tii:a}
  \end{subfigure}
  ~
  \begin{subfigure}[b]{\SubfigureWidth{}}
    \includegraphics[width=\columnwidth]{
      ch05_paper_ecs/figs/target_distribution_training_data_su.pdf}
    \caption{}
    \label{fig:05:su_tih_tii:b}
  \end{subfigure}
  \\
  \begin{subfigure}[b]{\SubfigureWidth{}}
    \includegraphics[width=\columnwidth]{
      ch05_paper_ecs/figs/scatterplot_merged_training_data_tih.pdf}
    \caption{}
    \label{fig:05:su_tih_tii:c}
  \end{subfigure}
  ~
  \begin{subfigure}[b]{\SubfigureWidth{}}
    \includegraphics[width=\columnwidth]{
      ch05_paper_ecs/figs/target_distribution_training_data_tih.pdf}
    \caption{}
    \label{fig:05:su_tih_tii:d}
  \end{subfigure}
  \\
  \begin{subfigure}[b]{\SubfigureWidth{}}
    \includegraphics[width=\columnwidth]{
      ch05_paper_ecs/figs/scatterplot_merged_training_data_tii.pdf}
    \caption{}
    \label{fig:05:su_tih_tii:e}
  \end{subfigure}
  ~
  \begin{subfigure}[b]{\SubfigureWidth{}}
    \includegraphics[width=\columnwidth]{
      ch05_paper_ecs/figs/target_distribution_training_data_tii.pdf}
    \caption{}
    \label{fig:05:su_tih_tii:f}
  \end{subfigure}
  \caption{As in \cref{fig:05:bri_cox_lip} but for the emergent constraints
    SU, TIH and TII. \AdaptedFrom{Schlund2020a}.}
  \label{fig:05:su_tih_tii}
\end{figure}

Another emergent constraint on \ac{ECS} that targets uncertainties in cloud
feedbacks is proposed by \textcite{Su2014}. They show that changes in the
Hadley circulation are physically connected to changes in tropical clouds and
thus \ac{ECS}. Consequently, the inter-model spread in the change of the Hadley
circulation in an ensemble of climate models is well correlated with the
corresponding changes in the \ac{TOA} \ac{CRE}. Moreover, \textcite{Su2014}
find a correlation between a model's \ac{ECS} and its ability to represent the
present-day Hadley circulation. The latter is calculated from the tropical
($\range{45 \unit{\degree S}}{40 \unit{\degree N}}$) zonal-mean vertical
profiles of relative humidity from the surface to $100 \unit{hPa}$. These
profiles are then used to define the \xaxis{} of the SU constraint by
calculating a performance metric based on the slope of the linear regression
between a climate model's relative humidity profile and the corresponding
observational reference. Similarly to the original publication, we use humidity
observations from AIRS \autocite{Aumann2003} for pressure levels greater than
$300 \unit{hPa}$ and MLS-Aura data \autocite{Beer2006} for pressure levels of
less than $300 \unit{hPa}$. Our analysis yields a constrained $66 \unit{\%}$
range of \ac{ECS} of $\pmrangeunit{3.30}{0.88}{K}$ for \acs{CMIP}5 ($R^2 =
0.08$) and $\pmrangeunit{3.77}{1.35}{K}$ for \acs{CMIP}6 ($R^2 = 0.05$). The
original publication gives a best estimate of $4 \unit{K}$ with a lower limit
of $3 \unit{K}$. \Cref{fig:05:su_tih_tii:a} shows that in addition to the low
$R^2$ values, the emergent relationship shows different slopes for \acs{CMIP}5
and \acs{CMIP}6. For \acs{CMIP}5, the expected positive correlation is found,
while for \acs{CMIP}6, a negative correlation is found. This suggests that the
constraint is not working (anymore) when applied to the \acs{CMIP}6 data.
Consequently, the SU constraint shows a weaker statistical significance in
\acs{CMIP}6 ($p = 0.1935$) than in \acs{CMIP}5 ($p = 0.1676$). The SU
constraint is related to an emergent constraint on \ac{ECS} proposed by
\textcite{Fasullo2012}, who correlate \range{May}{August} zonal-mean relative
humidity against \ac{ECS}. In contrast to \textcite{Su2014}, they do not use
the entire tropics, but identify two distinct regions with the largest
correlation.


\subsection{Tropical Mid-Tropospheric Humidity Asymmetry Index (TIH)}
\label{subsec:05:tih}

\textcite{Tian2015} finds a link between mid-tropospheric humidity over the
tropical Pacific and simulated moisture, precipitation, clouds, large-scale
circulation, and thus \ac{ECS} in \acs{CMIP}3 and \acs{CMIP}5 models. The study
explains this link with the similarity of mid-tropospheric humidity and
precipitation patterns as both are related to the \ac{ITCZ}. The proposed
tropical mid-tropospheric humidity asymmetry index to constrain \ac{ECS} is
defined as relative bias (in $\si{\%}$) in simulated annual mean $500
\unit{hPa}$ specific humidity averaged over the Southern Hemisphere tropical
Pacific ($\range{30 \unit{\degree S}}{0 \unit{\degree N}}$, $\range{80
  \unit{\degree W}}{120 \unit{\degree E}}$) minus the bias averaged over the
Northern Hemisphere tropical Pacific ($\range{0 \unit{\degree N}}{20
  \unit{\degree N}}$, $\range{80 \unit{\degree W}}{120 \unit{\degree E}}$) when
compared with observations. Similar to the SU constraint, the index proposed by
\textcite{Tian2015} seems to be related to the emergent constraint by
\textcite{Fasullo2012}, who find correlations between relative humidity of the
middle and upper troposphere and \ac{ECS}. Here, we use humidity observations
from AIRS \autocite{Aumann2003} over the time period \range{2003}{2005} as the
reference dataset. We assess a $66 \unit{\%}$ \ac{ECS} range of
$\pmrangeunit{3.88}{0.75}{K}$ for \acs{CMIP}5 ($R^2 = 0.24$) and
$\pmrangeunit{4.15}{1.10}{K}$ for \acs{CMIP}6 ($R^2 = 0.06$).
\textcite{Tian2015} specifies a best estimate of $4.0 \unit{K}$. The
significance of the emergent relationship drops massively from $p = 0.0089$ in
\acs{CMIP}5 to $p = 0.1348$ in \acs{CMIP}6.


\subsection{Southern \acs{ITCZ} index from (TII)}
\label{subsec:05:tii}

In addition to the humidity index, \textcite{Tian2015} proposes an emergent
constraint on \ac{ECS} based on the southern \ac{ITCZ} index
\autocite{Bellucci2010, Hirota2011}. This index is defined as the
climatological annual mean precipitation bias averaged over the southeastern
Pacific ($\range{30 \unit{\degree S}}{0 \unit{\degree N}}$, $\range{150
  \unit{\degree W}}{100 \unit{\degree W}}$). The southern \ac{ITCZ} index is
calculated in $\si{\mmday{}}$ and dominated by the so-called double-\ac{ITCZ},
a common problem in many \acs{CMIP}5 climate models. \textcite{Tian2015} finds
a link between double-\ac{ITCZ} bias and simulated moisture, precipitation,
clouds, and large-scale circulation in \acs{CMIP}3 and \acs{CMIP}5 models. He
argues that this could explain the link between the double-\acs{ITCZ} bias and
\ac{ECS}. As reference data, we use observed precipitation data for the years
\range{1986}{2005} from GPCP \autocite{Adler2003}. We calculate an \ac{ECS} $66
\unit{\%}$ confidence range of $\pmrangeunit{3.87}{0.67}{K}$ for \acs{CMIP}5
($R^2 = 0.33$) and $\pmrangeunit{3.84}{1.09}{K}$ for \acs{CMIP}6 ($R^2 <
0.01$). \textcite{Tian2015} specifies a best estimate of $4.0 \unit{K}$. The
emergent relationship shows a much lower statistical significance in
\acs{CMIP}6 ($p = 0.8236$) than in \acs{CMIP}5 ($p = 0.0013$).


\subsection{Difference Between Tropical and Southern Midlatitudinal Cloud
  Fraction (VOL)}
\label{subsec:05:vol}

The study by \textcite{Volodin2008} aims at constraining \ac{ECS} based on the
geographical distribution of clouds in climate models. Since this early
emergent constraint has originally been trained on \acs{CMIP}3 models, both
\acs{CMIP}5 and \acs{CMIP}6 are out-of-sample tests for it.
\textcite{Volodin2008} shows that high \ac{ECS} models tend to simulate a
higher total cloud cover over the southern midlatitudes and a lower total cloud
cover over the tropics (relative to the \ac{MMM}). This can be used to
establish an emergent relationship between the \ac{ECS} and the difference in
tropical total cloud cover ($\range{28 \unit{\degree S}}{28 \unit{\degree N}}$)
and the southern midlatitudinal total cloud cover ($\range{56 \unit{\degree
    S}}{36 \unit{\degree S}}$). Analogous to the original study, we use the
ISCCP-D2 data \autocite{Rossow1991} as observational reference. For the VOL
constraint, we calculate a constrained $66 \unit{\%}$ range of \ac{ECS} of
$\pmrangeunit{3.74}{0.64}{K}$ for \acs{CMIP}5 ($R^2 = 0.38$) and
$\pmrangeunit{4.21}{1.04}{K}$ for \acs{CMIP}6 ($R^2 = 0.18$), whereas the
original publication gives a range of $\pmrangeunit{3.6}{0.4}{K}$ (standard
deviation) for a climate model ensemble of \acs{CMIP}3 models. The emergent
constraint by \textcite{Volodin2008} shows a lower significance in the
\acs{CMIP}6 ensemble ($p = 0.0056$) than in the \acs{CMIP}5 ensemble ($p =
0.0004$).

\begin{figure}[t]
  \centering
  \begin{subfigure}[b]{\SubfigureWidth{}}
    \includegraphics[width=\columnwidth]{
      ch05_paper_ecs/figs/scatterplot_merged_training_data_vol.pdf}
    \caption{}
    \label{fig:05:vol_zha:a}
  \end{subfigure}
  ~
  \begin{subfigure}[b]{\SubfigureWidth{}}
    \includegraphics[width=\columnwidth]{
      ch05_paper_ecs/figs/target_distribution_training_data_vol.pdf}
    \caption{}
    \label{fig:05:vol_zha:b}
  \end{subfigure}
  \\
  \begin{subfigure}[b]{\SubfigureWidth{}}
    \includegraphics[width=\columnwidth]{
      ch05_paper_ecs/figs/scatterplot_merged_training_data_zha.pdf}
    \caption{}
    \label{fig:05:vol_zha:c}
  \end{subfigure}
  ~
  \begin{subfigure}[b]{\SubfigureWidth{}}
    \includegraphics[width=\columnwidth]{
      ch05_paper_ecs/figs/target_distribution_training_data_zha.pdf}
    \caption{}
    \label{fig:05:vol_zha:d}
  \end{subfigure}
  \caption{As in \cref{fig:05:bri_cox_lip} but for the emergent constraints
    VOL and ZHA. \AdaptedFrom{Schlund2020a}.}
  \label{fig:05:vol_zha}
\end{figure}


\subsection{Response of Marine Boundary Layer Cloud Fraction to changes in
  \acl{SST} (ZHA)}
\label{subsec:05:zha}

\begin{figure}[t]
  \centering
  \includegraphics[width=\LargeFigureWidth{}]{ch05_paper_ecs/figs/zha.pdf}
  \caption{Emergent relationship ZHA \autocite{Zhai2015} for different subsets
    of \acs{CMIP}5 models. Blue circles show the 15 \acs{CMIP}5 models used in
    the original publication (except for CESM1-CAM5). The solid blue line and
    blue shaded area show the emergent relationships evaluated on these models
    including the uncertainty range. In our study, we added eleven more
    \acs{CMIP}5 models (red circles). The corresponding emergent relationship
    that considers all available \acs{CMIP}5 models is shown in red colors.
    This relationship shows a considerably lower coefficient of determination
    ($R^2$) and higher $p$-value than the relationship using the original
    subset of \acs{CMIP}5 models. The vertical dashed line and shaded area
    correspond to the observational reference and the horizontal dashed lines
    show the corresponding constraints on the \acf{ECS} using this observation.
    \AdaptedFrom{Schlund2020a}.}
  \label{fig:05:zha}
\end{figure}

\textcite{Zhai2015} focus on the variations of \acp{MBLC}, which largely
contribute to the shortwave cloud feedback and thus to the uncertainty in
modeled \ac{ECS}. Their central quantity is the response of the \ac{MBLC}
fraction to changes in the \ac{SST} in subtropical oceanic subsidence regions
for both hemispheres ($\range{20 \unit{\degree}}{40 \unit{\degree}}$). On short
(seasonal) and long (centennial under a forcing) time scales, this quantity is
well correlated with \ac{ECS} among an ensemble of \acs{CMIP}3 and \acs{CMIP}5
models. Together with observations of cloud fraction from Cloudsat/CALIPSO
\autocite{Mace2009}, \ac{SST} from AMSRE \ac{SST} \autocite{AMSRE2011} and
vertical velocity from ERA-Interim \autocite{Dee2011}, the seasonal response of
\ac{MBLC} fraction to changes in \ac{SST} forms an emergent constraint on
\ac{ECS}. We assess a $66 \unit{\%}$ \ac{ECS} range of
$\pmrangeunit{3.35}{0.74}{K}$ for \acs{CMIP}5 ($R^2 = 0.05$) and
$\pmrangeunit{3.79}{0.67}{K}$ for \acs{CMIP}6 ($R^2 = 0.62$). In their original
publication, \textcite{Zhai2015} find an \ac{ECS} range of
$\pmrangeunit{3.90}{0.45}{K}$ (standard deviation) for a combination of
\acs{CMIP}3 and \acs{CMIP}5 models. In terms of statistical significance, the
results of the ZHA constraints are somewhat surprising: although \acs{CMIP}5
data (in combination with \acs{CMIP}3 data) has successfully been used in the
original publication, our approach finds that the statistical significance of
the emergent relationship is much higher in the unseen \acs{CMIP}6 ensemble ($p
< 0.0001$) than in the previously available \acs{CMIP}5 ensemble ($p =
0.2567$). The ZHA constraint is the only emergent constraint analyzed here that
shows this extreme behavior (only one other constraint, SHS, shows a slightly
higher significance in \acs{CMIP}6; all other constraints show lower
significances in \acs{CMIP}6). The reason for the erratic skill in \acs{CMIP}5
is the set of climate models used. For our analysis, we use eleven additional
\acs{CMIP}5 models that have not been used in the original publication
(ACCESS1-0, ACCESS1-3, bcc-csm1-1, bcc-csm1-1-m, CCSM4, GFDL-ESM2G, GFDL-ESM2M,
IPSL-CM5A-MR, IPSL-CM5B-LR, MPI-ESM-MR and MPI-ESM-P). Due to a lack of
publicly available data, the model CESM1-CAM5 that is used in the original
publication is not included in our analysis. The effect of choosing different
subsets of \acs{CMIP}5 models on the emergent relationship is illustrated in
\cref{fig:05:zha}. Using the original \acs{CMIP}5 models from the original
publication gives a considerably higher correlation ($R^2 = 0.38$) than using
all available \acs{CMIP}5 models ($R^2 = 0.05$). This result shows a strong
dependency of this emergent constraint on the subset of climate models used.
Nonetheless, the performance on \acs{CMIP}6 models is, surprisingly, the best
of all the constraints and much better than on either subset of \acs{CMIP}5
models.


\subsection{Constrained \acs{ECS} ranges and statistical significance of the
  eleven emergent constraints}
\label{subsec:05:emergent_constraints_summary}

\begin{table}[!b]
  \centering
  \begin{tabular}{p{0.06\columnwidth} p{0.34\columnwidth} y{0.12} y{0.12}
      y{0.1} y{0.1}}
    \toprule
    Label & \ecreshead{\acs{ECS}}{(original publication)} &
    \ecreshead{\acs{ECS} [$\si{K}$]}{(\acs{CMIP}5)} & \ecreshead{\acs{ECS}
      [$\si{K}$]}{(\acs{CMIP}6)} & \ecreshead{$p$}{(\acs{CMIP}5)} &
    \ecreshead{$p$}{(\acs{CMIP}6)} \\
    \midrule
    BRI & most likely $4.0 \unit{K}$, $< 2.30 \unit{K}$ very unlikely ($90
    \unit{\%}$ confidence) & 3.72
    \pm 0.59 & 4.32 \pm 1.07 & 0.0005 & 0.0355 \\
    COX & $\pmrangeunit{2.8}{0.6}{K}$ & 3.03 \pm 0.73 & 3.71 \pm 1.09 & 0.0032
    & 0.5415 \\
    LIP & no best estimate given & 2.97 \pm 0.75 & 3.75 \pm 1.11 & 0.0228 &
    0.6791 \\
    SHD & none --- see SHL & 3.65 \pm 0.64 & 3.77 \pm 1.06 & 0.0037 & 0.2805 \\
    SHL & most likely $4 \unit{K}$, lower limit $3 \unit{K}$ & 3.42 \pm 0.65 &
    3.67 \pm 1.06 & 0.0002 & 0.0138 \\
    SHS & none --- see SHL & 3.07 \pm 0.73 & 3.48 \pm 1.07 & 0.0647 & 0.0396 \\
    SU & most likely $4 \unit{K}$, lower limit $3 \unit{K}$ & 3.30 \pm 0.88 &
    3.77 \pm 1.35 & 0.1676 & 0.1935 \\
    TIH & most likely $4.0 \unit{K}$ & 3.88 \pm 0.75 & 4.15 \pm 1.10 & 0.0089
    & 0.1348 \\
    TII & most likely $4.0 \unit{K}$ & 3.87 \pm 0.67 & 3.84 \pm 1.09 & 0.0013
    & 0.8236 \\
    VOL & $\pmrangeunit{3.6}{0.4}{K}$ (\stddev{}) & 3.74 \pm 0.64 & 4.21 \pm
    1.04 & 0.0004 & 0.0056 \\
    ZHA & $\pmrangeunit{3.90}{0.45}{K}$ (\stddev{}) & 3.35 \pm 0.74 & 3.79 \pm
    0.67 & 0.2567 & < 0.0001 \\
    \bottomrule
  \end{tabular}
  \caption{Overview of the constrained \acf{ECS} ranges and $p$-values for all
    eleven analyzed emergent constraints. If not further specified, the
    uncertainty ranges correspond to the $66 \unit{\%}$ confidence intervals
    ($\range{17}{83} \unit{\%}$). For \acs{CMIP}5 and \acs{CMIP}6, these are
    evaluated from the \acl{PDF} given by \cref{eq:02:pdf_y_given_x0} (see also
    the right columns of \cref{fig:05:bri_cox_lip,fig:05:shd_shl_shs,%
      fig:05:su_tih_tii,fig:05:vol_zha}). Note that even though \acs{CMIP}5
    models have been used for some constraints in the original publications,
    the constrained ranges in the second and third column might differ due to
    the use of a different subset of climate models (in this study, we use
    output from all \acs{CMIP} models that is publicly available; see
    \cref{tab:app:a:cmip5_emergent_constraints_part1,%
      tab:app:a:cmip5_emergent_constraints_part2,%
      tab:app:a:cmip6_emergent_constraints_part1,%
      tab:app:a:cmip6_emergent_constraints_part2} for details). The $p$-values
    describing the significance of the emergent relationships are defined as
    the probability to obtain an absolute correlation coefficient $\abs*{r}$
    or higher under the null hypothesis that the true underlying correlation
    coefficient between the predictor and \acs{ECS} is zero. Smaller
    $p$-values point to higher significance and vice versa (see
    \cref{subsec:02:emergent_constraints} for details).
    \AdaptedFrom{Schlund2020a}.}
  \label{tab:05:overview_results}
\end{table}

In most cases, the emergent relationships (left columns of
\cref{fig:05:bri_cox_lip,fig:05:shd_shl_shs,fig:05:su_tih_tii,fig:05:vol_zha})
show the same sign of the slope (as expected from the theory) for \acs{CMIP}5
and \acs{CMIP}6, with the SU constraint being the only exception. However, the
coefficient of determination ($R^2$) is lower for \acs{CMIP}6 compared to
\acs{CMIP}5 for all one constraint: ZHA. The probability distributions of the
constrained \ac{ECS} that we obtain (right columns of
\cref{fig:05:bri_cox_lip,fig:05:shd_shl_shs,fig:05:su_tih_tii,fig:05:vol_zha})
give similar results: except for the ZHA constraint, the constraint on the
\acs{CMIP}6 ensemble is weaker, \ie{} the constrained \acp{PDF} derived from
the \acs{CMIP}6 ensemble are broader than their respective \acs{CMIP}5
counterparts. As shown in \cref{tab:05:overview_results} and
\cref{fig:05:overview_results}, for \acs{CMIP}5, the range of the best (maximum
likelihood) estimates for \acs{ECS} is $2.97 \unit{K}$ to $3.88 \unit{K}$,
while the corresponding \acs{CMIP}6 best estimates are higher for almost every
tested emergent constraint (TII being the only exception), resulting in a range
of best estimates of $3.48 \unit{K}$ to $4.32 \unit{K}$. Using the arithmetic
mean of all analyzed emergent constraints, this results in a mean increase of
the \ac{ECS} best estimate of $12 \unit{\%}$ in \acs{CMIP}6 compared to
\acs{CMIP}5. Similarly, the size of the $66 \unit{\%}$ \ac{ECS} ranges
($\range{17}{83} \unit{\%}$ confidence) shows values of $1.16 \unit{K}$ to
$1.75 \unit{K}$ in \acs{CMIP}5 and $1.32 \unit{K}$ to $2.70 \unit{K}$ in
\acs{CMIP}6, resulting in an increase of $51 \unit{\%}$ averaged over all
emergent constraints.

\begin{figure}[t]
  \centering
  \includegraphics[width=\FigureWidth{}]{
    ch05_paper_ecs/figs/constrained_ranges_summary.pdf}
  \caption{Overview of the constrained \acf{ECS} ranges given by the eleven
    analyzed emergent constraints using the \acs{CMIP}5 (blue) and \acs{CMIP}6
    (orange) ensemble. The bars correspond to the $66 \unit{\%}$ confidence
    intervals ($\range{17}{83} \unit{\%}$) calculated from the constrained
    \acl{PDF} of \acs{ECS} following \cref{eq:02:pdf_y_given_x0}. The vertical
    black lines illustrate the corresponding best estimates of the
    distributions. For almost all emergent constraints, the constrained best
    estimate and range of \ac{ECS} is higher in the \acs{CMIP}6 ensemble.}
  \label{fig:05:overview_results}
\end{figure}

In summary, the $R^2$ of the emergent relationships and the constrained range
of \ac{ECS} each depend strongly on the climate model ensemble used, even
though a physical explanation is given for each emergent constraint that is
thought to be valid for every climate model ensemble. The same behavior is
found for the statistical significance of the emergent relationships using the
null hypothesis that there is no correlation between the predictor and
\ac{ECS}. Except for the ZHA and SHS constraints, every emergent relationship
investigated shows a lower statistical significance (\ie{} a higher $p$- value)
in the \acs{CMIP}6 ensemble than in the \acs{CMIP}5 ensemble. If a conventional
significance test of $p < 0.05$ was applied, eight of the eleven constraints
would pass this test on \acs{CMIP}5 model data but only five (BRI, SHL, SHS,
VOL and ZHA) would pass this test on \acs{CMIP}6. This is still much better
than would be expected purely by chance. Hence, there is still skill in at
least a few of the constraints, but it is much lower than suggested in nearly
all of the initial studies.


\section{Discussion}
\label{sec:05:discussion}

As shown in the previous sections, most emergent relationships show smaller
coefficients of determination when evaluated on the new \acs{CMIP}6 ensemble
compared to the \acs{CMIP}5 ensemble. In this section, we discuss possible
reasons for, and implications of, these differences. As reported by
\textcite{Caldwell2014}, the large amount of data provided by modern \acp{ESM}
can generate spurious correlations of variables between past climate and
\ac{ECS} just by chance, especially when only a small number of climate models
is considered. This would cause the performance of the emergent constraint to
be reduced on out-of-sample data (like the new \acs{CMIP}6 ensemble), since the
emergent relationship appeared just by chance and not because of a physically
based mechanism.

A further reason for the weaker emergent relationships in \acs{CMIP}6 may be
the increased complexity of the participating \acp{ESM}. Each emergent
constraint approach is based on the assumption that a single observable process
or physical aspect in the current climate dominates the uncertainty in
\ac{ECS}. Some emergent constraints such as ZHA and BRI relate changes in cloud
properties (ZHA: low-level cloud fraction; BRI: cloud reflectivity) on seasonal
or interannual time scales driven by changes in \ac{SST} to \ac{ECS}. This
means that it has to be implicitly assumed that the observable changes in these
properties on seasonal or interannual time scales are related to those
occurring as a result of climate forcing in a way successfully captured by the
\acp{ESM}. While this assumption seems to make sense, we do not know whether
the \acp{ESM} cover all relevant processes of the real Earth system. For
example, it may be possible that there exist processes that are unimportant in
the \acp{ESM} (and hence are not captured by the emergent constraints) but are
actually important in reality. This lack of relevant processes may lead to an
overconfident constraint. Thus, the more complex \acp{ESM} of the \acs{CMIP}6
ensemble are more likely to capture relevant processes of the true climate,
which leads to weaker emergent relationships. On the other hand, emergent
constraints on the less complex \acs{CMIP}3 and \acs{CMIP}5 ensemble may be
overconfident.

For \acs{CMIP}6 models, \textcite{Zelinka2020} show that cloud feedbacks and
thus \ac{ECS} in high-sensitivity models are to some extent associated with
changes in clouds over the Southern Ocean, while in \acs{CMIP}3 and \acs{CMIP}5
the uncertainty in cloud feedbacks is dominated by clouds in the subtropical
subsidence regions. One might speculate that a possible reason for this might
be an improved simulation of clouds over the Southern Ocean in some models
\autocite{BodasSalcedo2019, Gettelman2019}, as shown for some pre-\acs{CMIP}6
model versions evaluated by \textcite{Lauer2018}. The findings of
\textcite{Zelinka2020} can also at least partly explain the larger inter-model
spread in climate sensitivity due to a greater diversity of cloud feedbacks,
which also results in a weaker emergent constraint compared with \acs{CMIP}5
models, as most of them constrained low-cloud feedbacks. They find that on
average, the shortwave low-cloud feedback is larger in \acs{CMIP}6 than in
\acs{CMIP}5, which they primarily relate to changes in the representation of
clouds. As a possible explanation, \textcite{Zelinka2020} give an increase in
mean-state supercooled liquid water (\ie{} increase in the cloud water liquid
fraction) in mixed-phase clouds resulting in less pronounced increases in
low-level cloud cover and water content with warmer \acp{SST} particularly in
midlatitudes.

Our findings suggest that the process-oriented emergent constraints (\ie{} all
of the emergent constraints investigated here except COX) are only successful
in constraining \ac{ECS} as long as the uncertainty in \ac{ECS} is dominated by
the same process or feedback. In the \acs{CMIP}5 ensemble, cloud feedback is
the main contributor to the spread in \ac{ECS} with low-level clouds in
tropical subsidence regions dominating the spread in cloud feedback
\autocite{Ceppi2017}. If any other process or feedback is biased (or missing)
in the ensemble as a whole, then these process-oriented emergent constraints
will be biased in their estimates of \ac{ECS}. The appearance of diverse new
feedback processes in \acs{CMIP}6 could explain the reduced skill when applied
to \acs{CMIP}6 data, and a tendency for these to be positive would explain the
upward shift in the model \ac{ECS} distribution that is not captured by the
\acs{CMIP}5-trained constraints. Process-oriented emergent constraints are
therefore perhaps best thought of as constraints on the processes that they
target, rather than constraints on \ac{ECS}.

Emergent constraints that use global temperature change as a way of
constraining \ac{ECS} could in principle overcome this problem. If one feedback
is biased in an ensemble the constraint might still work as both global
temperature change and \ac{ECS} might similarly reflect the sum of all
feedbacks. Emergent constraints of this kind include the tropical temperature
during the Last Glacial Maximum \autocite{Hargreaves2012}, tropical temperature
anomalies during the mid-Pliocene Warm Period \autocite{Hargreaves2016} and
post-1970s warming \autocite{JimenezdelaCuesta2019}. This seems to be supported
by the findings of \textcite{Tokarska2020}, who test an emergent constraint for
the \ac{TCR} based on recent global warming trends on the \acs{CMIP}5 and
\acs{CMIP}6 ensembles with similar results for both model ensembles.
Nevertheless, these temperature-based estimates are sensitive to assumptions
about forcings and unforced decadal temperature variations, which could also be
incorrect, as could model-predicted relationships between feedbacks on short
and long time scales that are implicit in most such measures.


\section{Summary}
\label{sec:05:summary}

This paper assesses eleven different emergent constraints on \ac{ECS}, of which
most are directly or indirectly related to cloud feedbacks, by applying them to
results from \acp{ESM} contributing to \acs{CMIP}5 and \acs{CMIP}6. Of
particular interest are the results from \acs{CMIP}6, since all analyzed
emergent constraints have been published prior to the availability of
\acs{CMIP}6 data. In summary, the best estimate of \ac{ECS} averaged over all
emergent constraints increases by $12 \unit{\%}$ when moving from relationships
trained on \acs{CMIP}5 to those trained on \acs{CMIP}6. Some increase is
predicted by almost every constraint we analyze (with TII as the only
exception) and can be at least partly explained by the increased multi-model
mean \ac{ECS} of \acs{CMIP}6, which has not been accompanied by systematic
changes in the constraint variables that could explain this increase, leading
to regression fits with higher intercept values at observed constraint values.
This is also illustrated by the \acs{CMIP}5 and \acs{CMIP}6 \acp{MMM} in the
left columns of
\cref{fig:05:bri_cox_lip,fig:05:shd_shl_shs,fig:05:su_tih_tii,fig:05:vol_zha}
(colored circles), in which the connecting line between the \acs{CMIP}5 and
\acs{CMIP}6 \ac{MMM} is not parallel to the \acs{CMIP}5 emergent relationships
for all emergent constraints.

Our results also show that, except for ZHA, all emergent relationships are
weaker (in terms of the coefficient of determination $R^2$) in \acs{CMIP}6
compared to \acs{CMIP}5, which means that the corresponding emergent
relationships are able to explain less of the \ac{ECS} variation simulated by
the newer \acs{CMIP}6 models than by those of \acs{CMIP}5. This is also
demonstrated by the statistical significance, which is lower in \acs{CMIP}6
than in \acs{CMIP}5 for all emergent constraint except for ZHA and SHS. As
described in \cref{subsec:02:emergent_constraints}, our test for statistical
significance uses the null hypothesis that there is no correlation between the
predictor variable of the emergent constraint and \ac{ECS}. Further evidence of
the decreased performance of the emergent constraints in \acs{CMIP}6 is given
by the size of the constrained \acs{ECS} ranges, which widens by $51 \unit{\%}$
in \acs{CMIP}6 compared to \acs{CMIP}5 on average. Moreover, our more detailed
analysis of the ZHA constraint (see \cref{fig:05:zha}) shows that this emergent
constraint is very sensitive to outliers and the subset of the climate model
ensemble used to fit the emergent relationship. Such a behavior might not be
unique to the ZHA constraint but could apply to other emergent constraints as
well. This in turn suggests that the number of climate models commonly used for
emergent constraints might be too low, leading to non-robust relationships.

Our analysis makes a number of simplifying assumptions common to other studies,
such as model independence, discussed in detail in
\cref{subsec:02:ecs,subsec:02:emergent_constraints}. These assumptions affect
the significance of emergent relationships and the \acp{PDF} of \ac{ECS} based
on a constraint. However, they do not affect our main conclusions here, which
concern the change in performance on \acs{CMIP}6 relative to \acs{CMIP}5 and
the implications for robustness and future use of emergent constraints.

\ac{ECS} is the product of the complex interactions of the many components and
feedbacks. Thus, constraining \ac{ECS} with a single physical process might
overly simplify this problem. Such single process-oriented emergent constraints
therefore do not seem to be helpful in constraining \ac{ECS} but should
probably rather be thought of as constraints for the process or feedback they
are actually targeting (if that can be clearly identified). With increasing
computational resources available to climate science, more and more detailed
process interactions can be taken into account in a modern \ac{ESM}. In
contrast, the predecessor versions \acs{CMIP}3 and \acs{CMIP}5 are less complex
with simpler atmospheric process representations, so constraining uncertainties
of a single dominant process may allow for an apparently more successful
constraint of \ac{ECS} than would be achieved in more complex models. As a
conclusion, we argue that to constrain \ac{ECS} in a more robust way, it might
be beneficial to apply multivariate approaches that are able to consider
multiple (different) relevant physical processes and feedbacks at once and thus
are able to get a broader picture of the complex reality. A possible approach
for this is given by \textcite{Bretherton2020}, who combine the information
from multiple emergent constraints on \ac{ECS} using a multivariate Gaussian
and multiple linear regression. For the \acs{CMIP}3 and \acs{CMIP}5 ensembles,
they find an increased best estimate relative to the unweighted ensemble mean
similar to the participating individual emergent constraints, but with lower
uncertainty range. Moreover, new machine learning techniques are a promising
avenue forward for such multivariate approaches and for constraining
uncertainties in multi-model projections \commentcite{Schlund2020}{see
  \cref{ch:06:paper_gpp}} with the aim of further improving climate modeling
and analysis \autocite{Reichstein2019}.
