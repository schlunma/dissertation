%=============================================================================%
%                                Dissertation                                 %
%                               Manuel Schlund                                %
%                                  (c) 2020                                   %
%=============================================================================%
%                                  Paper ECS                                  %
%=============================================================================%



\chapter{Evaluation of Emergent Constraints on the \acl{ECS} in \acs{CMIP}6}
\label{ch:05:paper_ecs}

The massive spread in \ac{ECS} within the \acs{CMIP}6 ensemble discussed in the
previous chapter begs the question about the interpretation and credibility of
these results. Since the \acs{CMIP}6 models provide crucial input for the
upcoming international climate assessment of the \ac{IPCC} (\acs{AR}6), this is
not only important for the scientific community, but also directly affects
society and policymakers. An important question in this context could be the
following: Is it reasonable to adapt the assessed \ac{ECS} range from \acs{AR}5
($\range{1.5}{4.5} \unit{K}$) in \acs{AR}6 based on the \acs{CMIP}6 model range
of $\range{1.8}{5.6} \unit{K}$? An argument for this is that due to the many
new and improved processes included to the models, the \acs{CMIP}6 ensemble is
arguably more realistic than its predecessor generations. However, even the
\acs{CMIP}6 models are far from fully simulating all aspects and processes of
the full vastly complex Earth system. This could introduce new biases which
have not been present in earlier \ac{CMIP} models. For example, the apparent
increase in climate sensitivity in the \acs{CMIP}6 models might be wrongly
caused by missing processes that cancel out the effects of the newly included
processes. Thus, it is vital to take other sources of information into account
before answering the question raised above.

One of the most important sources of information for this purpose is
undoubtedly observational data. As presented in
\cref{subsec:02:emergent_constraints}, appropriate observations of the Earth
system can be utilized within the emergent constraints framework to assess the
accuracy of climate model projections and reduce associated uncertainties. For
\ac{ECS}, a large variety of different emergent constraints has been developed
over the years, which we now evaluate on the \acs{CMIP}6 ensemble in order to
assess whether they still hold for these new climate models. In total we assess
eleven emergent constraints on \ac{ECS}, out of which the most are related to
cloud feedbacks since these constitute the most important source of uncertainty
for \ac{ECS} \commentcite{Boucher2013, Flato2013}{see also
  \cref{subsec:02:climate_feedbacks}}. A complete list of these emergent
constraints, which are selected by their availability in the \ac{ESMValTool},
is given in \cref{tab:05:overview_emergent_constraints}. Since all of the
evaluated emergent constraints have been derived on the \acs{CMIP}3 or
\ac{CMIP}5 ensemble, the \ac{CMIP}6 models offer an exciting possibility to
perform out-of-sample testing using the previously unseen data.

\begin{table}[!t]
  \centering
  \hbadness=3500
  \begin{tabular}{p{0.07\columnwidth} p{0.24\columnwidth} p{0.29\columnwidth}
      p{0.29\columnwidth}}
    \toprule
    Label & Reference & Description of \xaxis{} & Variables and corresponding
    observations \\
    \midrule
    BRI & \textcite{Brient2016} & Sensitivity of shortwave cloud albedo
    to changes in \acf{SST} [$\si{\pctK{}}$] & \makecell{\tabitem\emph{hur}
      (ERA-Interim) \\ \tabitem \emph{rsdt} (CERES-EBAF) \\ \tabitem
      \emph{rsut} (CERES-EBAF) \\ \tabitem \emph{rsutcs} (CERES-EBAF) \\
      \tabitem \emph{ts} (HadISST)} \\
    \midrule
    COX & \textcite{Cox2018} & $\Psi$ (temperature variability metric)
    [$\si{K}$] &
    \tabitem \emph{tas} (HadCRUT4) \\
    \midrule
    LIP & \textcite{Lipat2017} & Extent of Southern Hemisphere Hadley cell
    [$\si{\degree}$] & \tabitem \emph{va} (ERA-Interim) \\
    \midrule
    SHD & \textcite{Sherwood2014} & $D$-index (large-scale lower-tropospheric
    mixing) [$\si{1}$] & \tabitem \emph{wap} (ERA-Interim) \\
    \midrule
    SHL & \textcite{Sherwood2014} & \acs{LTMI} (\acl{LTMI}) [$\si{1}$] &
    \makecell{\tabitem \emph{hur} (ERA-Interim) \\ \tabitem \emph{ta}
      (ERA-Interim) \\ \tabitem \emph{wap} (ERA-Interim)} \\
    \midrule
    SHS & \textcite{Sherwood2014} & $S$-index (small-scale lower-tropospheric
    mixing) [$\si{1}$] & \makecell{\tabitem \emph{hur} (ERA-Interim) \\
      \tabitem \emph{ta} (ERA-Interim) \\ \tabitem \emph{wap} (ERA-Interim)} \\
    \midrule
    SU & \textcite{Su2014} & Error in vertical profile of relative humidity
    [$\si{1}$] & \tabitem \emph{hur} (AIRS, MLS-Aura) \\
    \midrule
    TIH & \textcite{Tian2015} & Tropical mid-tropospheric humidity asymmetry
    index from AIRS [$\si{\%}$] & \tabitem \emph{hus} (AIRS) \\
    \midrule
    TII & \textcite{Tian2015} & Southern ITCZ index from GPCP
    [$\si{\mmday{}}$] & \tabitem \emph{pr} (GPCP) \\
    \midrule
    VOL* & \textcite{Volodin2008} & Difference between tropical and southern
    midlatitudinal cloud fraction [$\si{\%}$] & \tabitem \emph{clt} (ISCCP
    D-2) \\
    \midrule
    ZHA* & \textcite{Zhai2015} & Seasonal response of marine boundary layer
    cloud fraction to changes in \acs{SST} [$\si{\pctK{}}$] &
    \makecell{\tabitem \emph{cl} (CloudSat/CALIPSO) \\ \tabitem \emph{tos}
      (AMSRE SST) \\ \tabitem \emph{wap} (ERA-Interim)} \\
    \bottomrule
  \end{tabular}
  \caption{Overview of the eleven emergent constraints on the \acf{ECS}
    evaluated in this chapter. Detailed descriptions of the variables used to
    calculate the \xaxis{} values of the emergent constraints and the
    references for the corresponding observations (fourth column) are given in
    \cref{tab:app:a:overview_variables,tab:app:a:observations}, respectively.
    For emergent constraints marked with an asterisk (*), the observational
    value of the original publication is used to calculate the observational
    constraint. For all others, the observational value is computed with the
    \acs{ESMValTool}. \AdaptedFrom{Schlund2020a}.}
  \label{tab:05:overview_emergent_constraints}
\end{table}

This analysis is already published in \textcite{Schlund2020a}. For this paper,
the author of this thesis lead the writing and the analysis and implemented the
code to reproduce this analysis with all figures and tables using the
\ac{ESMValTool}. In this chapter, we first introduce the data (climate model
output and observations) and methods used in this study
(\cref{sec:05:data_and_methods}). After that, we present the eleven emergent
constraints on \acs{ECS} and evaluate them on the \acs{CMIP}5 and \acs{CMIP}6
ensemble (\cref{sec:05:comparison_of_emergent_constraints}). Finally, we
provide a discussion (\cref{sec:05:discussion}) and a summary of the results
(\cref{sec:05:summary}).


\section{Data and Methods}
\label{sec:05:data_and_methods}

In this study we use the output from climate models participating in
\acs{CMIP}5 and \acs{CMIP}6, shown in \cref{sec:app:cmip_models_in_chapter_5}
in \cref{tab:app:a:cmip5_models,tab:app:a:cmip6_models}, respectively.

We quantify the skill of the emergent constraints with three metrics: the
coefficient of determination $R^2$ of the linear relationship, the statistical
significance using a two-sided $t$-test based on the Pearson correlation
coefficient $r$ (see \cref{eq:02:t}) and the constrained range of \ac{ECS}
based on the constrained \ac{PDF} (see \cref{eq:02:pdf_y_given_x0}).


\section{Comparison of Emergent Constraints on \acs{ECS} for \acs{CMIP}5 and
  \acs{CMIP}6}
\label{sec:05:comparison_of_emergent_constraints}


\subsection{COX}
\label{subsec:05:cox}

\begin{figure}[p]
  \centering
  \begin{subfigure}[b]{\SubfigureWidth{}}
    \includegraphics[width=\columnwidth]{
      ch05_paper_ecs/figs/scatterplot_merged_training_data_BRI.pdf}
    \caption{}
    \label{fig:05:bri_cox_lip:a}
  \end{subfigure}
  ~
  \begin{subfigure}[b]{\SubfigureWidth{}}
    \includegraphics[width=\columnwidth]{
      ch05_paper_ecs/figs/target_distribution_training_data_BRI.pdf}
    \caption{}
    \label{fig:05:bri_cox_lip:b}
  \end{subfigure}
  \\
  \begin{subfigure}[b]{\SubfigureWidth{}}
    \includegraphics[width=\columnwidth]{
      ch05_paper_ecs/figs/scatterplot_merged_training_data_COX.pdf}
    \caption{}
    \label{fig:05:bri_cox_lip:c}
  \end{subfigure}
  ~
  \begin{subfigure}[b]{\SubfigureWidth{}}
    \includegraphics[width=\columnwidth]{
  ch05_paper_ecs/figs/target_distribution_training_data_COX.pdf}
    \caption{}
    \label{fig:05:bri_cox_lip:d}
  \end{subfigure}
  \\
  \begin{subfigure}[b]{\SubfigureWidth{}}
    \includegraphics[width=\columnwidth]{
      ch05_paper_ecs/figs/scatterplot_merged_training_data_LIP.pdf}
    \caption{}
    \label{fig:05:bri_cox_lip:e}
  \end{subfigure}
  ~
  \begin{subfigure}[b]{\SubfigureWidth{}}
    \includegraphics[width=\columnwidth]{
  ch05_paper_ecs/figs/target_distribution_training_data_LIP.pdf}
    \caption{}
    \label{fig:05:bri_cox_lip:f}
  \end{subfigure}
  \caption{xD. \AdaptedFrom{Schlund2020a}.}
  \label{fig:05:bri_cox_lip}
\end{figure}

\subsection{SHL}
\label{subsec:05:shl}

This is the \ac{LTMI}!!


\section{Discussion}
\label{sec:05:discussion}


\section{Summary}
\label{sec:05:summary}
