%=============================================================================%
%                                Dissertation                                 %
%                               Manuel Schlund                                %
%                                  (c) 2020                                   %
%=============================================================================%
%                                  Paper ECS                                  %
%=============================================================================%



\chapter{Evaluation of Emergent Constraints on the \acl{ECS} in \acs{CMIP}6}
\label{ch:05:paper_ecs}

The massive spread in \ac{ECS} within the \acs{CMIP}6 ensemble discussed in the
previous chapter begs the question about the interpretation and credibility of
these results. Since the \acs{CMIP}6 models provide crucial input for the
upcoming international climate assessment of the \ac{IPCC} (\acs{AR}6), this is
not only important for the scientific community, but also directly affects
society and policymakers. An important question in this context could be the
following: Is it reasonable to adapt the assessed \ac{ECS} range from \acs{AR}5
($\range{1.5}{4.5} \unit{K}$) in \acs{AR}6 based on the \acs{CMIP}6 model
range? An argument for this is that due to the many new and improved processes
included to the models, the \acs{CMIP}6 ensemble is arguably more realistic
than its predecessor generations. However, even the \acs{CMIP}6 models are far
from fully simulating all aspects and processes of the vastly complex Earth
system. For this reason, the apparent increase in climate sensitivity in the
\acs{CMIP}6 models might be wrongly caused by missing processes that cancel out
the effects of the newly included processes, leading to a biased \ac{ECS} even
though the models are more realistic. Thus, it is vital to take other sources
of information into account before answering the question raised above.

One of the most important sources of information for this purpose is
undoubtedly observational data. As presented in
\cref{subsec:02:emergent_constraints}, emergent constraints are a beneficial
tool to assess the accuracy of climate model projections and reduce associated
uncertainties with appropriate observations of the Earth system. For \ac{ECS},
a large variety of different emergent constraints has been developed over the
years, which we now evaluate on the new \acs{CMIP}6 ensemble. In total we
assess eleven emergent constraints (selected by their availability in the
\ac{ESMValTool}), out of which the most are related to cloud feedbacks, since
these constitute the most important source of uncertainty for \ac{ECS}
\commentcite{Boucher2013, Flato2013}{see also
  \cref{subsec:02:climate_feedbacks}}. Since all of the evaluated emergent
constraints have been derived on the \acs{CMIP}3 or \ac{CMIP}5 ensemble, the
\ac{CMIP}6 models offer an exciting possibility to perform out-of-sample
testing using the previously unseen data. We quantify the skill of the emergent
constraints with three metrics: the coefficient of determination $R^2$ of the
linear relationship, the statistical significance using a two-sided $t$-test
based on the Pearson correlation coefficient $r$ (see \cref{eq:02:t}) and the
constrained range of \ac{ECS} based on the constrained \ac{PDF} (see
\cref{eq:02:pdf_y_given_x0}). This analysis has already been published in
\textcite{Schlund2020a}. For this paper, the author of this thesis lead the
writing and the analysis and contributed the code to reproduce this analysis
including all figures and tables with the \ac{ESMValTool}. In this chapter, we
first introduce the methods and data (climate model output and observations)
used in this study (\cref{sec:05:methods_and_data}). After that, we present the
eleven emergent constraints on \acs{ECS} and evaluate them on the \acs{CMIP}5
and \acs{CMIP}6 ensemble (\cref{sec:05:comparison_of_emergent_constraints}).
Finally, we provide a discussion (\cref{sec:05:discussion}) and a summary of
the results (\cref{sec:05:summary}).


\section{Methods and Data}
\label{sec:05:methods_and_data}

In this study we use the output from climate models participating in
\acs{CMIP}5 and \acs{CMIP}6, shown in \cref{sec:app:cmip_models_in_chapter_5}
in \cref{tab:app:a:cmip5_models,tab:app:a:cmip6_models}, respectively.


\section{Comparison of Emergent Constraints on \acs{ECS} for \acs{CMIP}5 and
  \acs{CMIP}6}
\label{sec:05:comparison_of_emergent_constraints}


\subsection{COX}
\label{subsec:05:cox}

\begin{figure}[p]
  \centering
  \begin{subfigure}[b]{\SubfigureWidth{}}
    \includegraphics[width=\columnwidth]{
      ch05_paper_ecs/figs/scatterplot_merged_training_data_BRI.pdf}
    \caption{}
    \label{fig:05:bri_cox_lip:a}
  \end{subfigure}
  ~
  \begin{subfigure}[b]{\SubfigureWidth{}}
    \includegraphics[width=\columnwidth]{
      ch05_paper_ecs/figs/target_distribution_training_data_BRI.pdf}
    \caption{}
    \label{fig:05:bri_cox_lip:b}
  \end{subfigure}
  \\
  \begin{subfigure}[b]{\SubfigureWidth{}}
    \includegraphics[width=\columnwidth]{
      ch05_paper_ecs/figs/scatterplot_merged_training_data_COX.pdf}
    \caption{}
    \label{fig:05:bri_cox_lip:c}
  \end{subfigure}
  ~
  \begin{subfigure}[b]{\SubfigureWidth{}}
    \includegraphics[width=\columnwidth]{
  ch05_paper_ecs/figs/target_distribution_training_data_COX.pdf}
    \caption{}
    \label{fig:05:bri_cox_lip:d}
  \end{subfigure}
  \\
  \begin{subfigure}[b]{\SubfigureWidth{}}
    \includegraphics[width=\columnwidth]{
      ch05_paper_ecs/figs/scatterplot_merged_training_data_LIP.pdf}
    \caption{}
    \label{fig:05:bri_cox_lip:e}
  \end{subfigure}
  ~
  \begin{subfigure}[b]{\SubfigureWidth{}}
    \includegraphics[width=\columnwidth]{
  ch05_paper_ecs/figs/target_distribution_training_data_LIP.pdf}
    \caption{}
    \label{fig:05:bri_cox_lip:f}
  \end{subfigure}
  \caption{xD. \AdaptedFrom{Schlund2020a}.}
  \label{fig:05:bri_cox_lip}
\end{figure}


\section{Discussion}
\label{sec:05:discussion}


\section{Summary}
\label{sec:05:summary}
