%=============================================================================%
%                                Dissertation                                 %
%                               Manuel Schlund                                %
%                                  (c) 2020                                   %
%=============================================================================%
%                                  Paper ECS                                  %
%=============================================================================%



\chapter{Evaluation of Emergent Constraints on the \acl{ECS} in \acs{CMIP}6}
\label{ch:05:paper_ecs}

The massive spread in \ac{ECS} within the \acs{CMIP}6 ensemble discussed in the
previous chapter begs the question about the interpretation and credibility of
these results. Since the \acs{CMIP}6 models provide crucial input for the
upcoming international climate assessment of the \ac{IPCC} (\acs{AR}6), this is
not only important for the scientific community, but also directly affects
society and policymakers. An important question in this context could be the
following: Is it reasonable to adapt the assessed \ac{ECS} range from \acs{AR}5
($\range{1.5}{4.5} \unit{K}$) in \acs{AR}6 based on the \acs{CMIP}6 model range
of $\range{1.8}{5.6} \unit{K}$? An argument for this is that due to the many
new and improved processes included to the models, the \acs{CMIP}6 ensemble is
arguably more realistic than its predecessor generations. However, even the
\acs{CMIP}6 models are far from fully simulating all aspects and processes of
the full vastly complex Earth system. This could introduce new biases which
have not been present in earlier \ac{CMIP} models. For example, the apparent
increase in climate sensitivity in the \acs{CMIP}6 models might be wrongly
caused by missing processes that cancel out the effects of the newly included
processes. Thus, it is vital to take other sources of information into account
before answering the question raised above.

One of the most important sources of information for this purpose is
undoubtedly observational data. As presented in
\cref{subsec:02:emergent_constraints}, appropriate observations of the Earth
system can be utilized within the emergent constraints framework to assess the
accuracy of climate model projections and reduce associated uncertainties. For
\ac{ECS}, a large variety of different emergent constraints has been developed
over the years, which we now evaluate on the \acs{CMIP}6 ensemble in order to
assess whether they still hold for these new climate models. In total we assess
eleven emergent constraints on \ac{ECS} (see
\cref{tab:05:overview_emergent_constraints}), out of which the most are related
to cloud feedbacks since these constitute the most important source of
uncertainty for \ac{ECS} \commentcite{Boucher2013, Flato2013}{see also
  \cref{subsec:02:climate_feedbacks}}. Since all of the evaluated emergent
constraints have been derived on the \acs{CMIP}3 or \acs{CMIP}5 ensemble, the
\acs{CMIP}6 models offer an exciting possibility to perform out-of-sample
testing using the previously unseen data.

This analysis is already published in \textcite{Schlund2020a}. For this paper,
the author of this thesis lead the writing and the analysis and implemented the
code to reproduce this analysis with all figures and tables using the
\ac{ESMValTool}. In this chapter, we first introduce the data (climate model
output and observations) and methods used in this study
(\cref{sec:05:data_and_methods}). After that, we present the eleven emergent
constraints on \acs{ECS} and evaluate them on the \acs{CMIP}5 and \acs{CMIP}6
ensemble (\cref{sec:05:comparison_of_emergent_constraints}). Finally, we
provide a discussion (\cref{sec:05:discussion}) and a summary of the results
(\cref{sec:05:summary}).

\begin{table}[p]
  \centering
  \hbadness=3500
  \begin{tabular}{p{0.07\columnwidth} p{0.24\columnwidth} p{0.29\columnwidth}
      p{0.29\columnwidth}}
    \toprule
    Label & Reference & Description of \xaxis{} & Variables and corresponding
    observations \\
    \midrule
    BRI & \textcite{Brient2016} & Sensitivity of shortwave cloud albedo
    to changes in \acf{SST} [$\si{\pctK{}}$] & \makecell{\tabitem\emph{hur}
      (ERA-Interim) \\ \tabitem \emph{rsdt} (CERES-EBAF) \\ \tabitem
      \emph{rsut} (CERES-EBAF) \\ \tabitem \emph{rsutcs} (CERES-EBAF) \\
      \tabitem \emph{ts} (HadISST)} \\
    \midrule
    COX & \textcite{Cox2018} & $\Psi$ (temperature variability metric)
    [$\si{K}$] &
    \tabitem \emph{tas} (HadCRUT4) \\
    \midrule
    LIP & \textcite{Lipat2017} & Extent of Southern Hemisphere Hadley cell
    [$\si{\degree}$] & \tabitem \emph{va} (ERA-Interim) \\
    \midrule
    SHD & \textcite{Sherwood2014} & $D$-index (large-scale lower-tropospheric
    mixing) [$\si{1}$] & \tabitem \emph{wap} (ERA-Interim) \\
    \midrule
    SHL & \textcite{Sherwood2014} & \acs{LTMI} (\acl{LTMI}) [$\si{1}$] &
    \makecell{\tabitem \emph{hur} (ERA-Interim) \\ \tabitem \emph{ta}
      (ERA-Interim) \\ \tabitem \emph{wap} (ERA-Interim)} \\
    \midrule
    SHS & \textcite{Sherwood2014} & $S$-index (small-scale lower-tropospheric
    mixing) [$\si{1}$] & \makecell{\tabitem \emph{hur} (ERA-Interim) \\
      \tabitem \emph{ta} (ERA-Interim) \\ \tabitem \emph{wap} (ERA-Interim)} \\
    \midrule
    SU & \textcite{Su2014} & Error in vertical profile of relative humidity
    [$\si{1}$] & \tabitem \emph{hur} (AIRS, MLS-Aura) \\
    \midrule
    TIH & \textcite{Tian2015} & Tropical mid-tropospheric humidity asymmetry
    index from AIRS [$\si{\%}$] & \tabitem \emph{hus} (AIRS) \\
    \midrule
    TII & \textcite{Tian2015} & Southern ITCZ index from GPCP
    [$\si{\mmday{}}$] & \tabitem \emph{pr} (GPCP) \\
    \midrule
    VOL* & \textcite{Volodin2008} & Difference between tropical and southern
    midlatitudinal cloud fraction [$\si{\%}$] & \tabitem \emph{clt} (ISCCP
    D-2) \\
    \midrule
    ZHA* & \textcite{Zhai2015} & Seasonal response of \acl{MBLC} cloud
    fraction to changes in \acs{SST} [$\si{\pctK{}}$] &
    \makecell{\tabitem \emph{cl} (CloudSat/CALIPSO) \\ \tabitem \emph{tos}
      (AMSRE SST) \\ \tabitem \emph{wap} (ERA-Interim)} \\
    \bottomrule
  \end{tabular}
  \caption{Overview of the eleven emergent constraints on the \acf{ECS}
    evaluated in this chapter. Detailed descriptions of the variables used to
    calculate the \xaxis{} values of the emergent constraints and the
    references for the corresponding observations (fourth column) are given in
    \cref{tab:app:a:overview_variables,tab:app:a:observations}, respectively.
    For emergent constraints marked with an asterisk (*), the observational
    value of the original publication is used to calculate the observational
    constraint. For all others, the observational value is computed with the
    \acs{ESMValTool}. \AdaptedFrom{Schlund2020a}.}
  \label{tab:05:overview_emergent_constraints}
\end{table}


\section{Data and Methods}
\label{sec:05:data_and_methods}

In this study we use the output from climate models participating in
\acs{CMIP}5 and \acs{CMIP}6 as shown in
\cref{tab:app:a:cmip5_models,tab:app:a:cmip6_models}, respectively. In
addition, for each emergent constraint at least one observational dataset is
used to calculate the observational constraint. A complete list of these
datasets is given in \cref{tab:05:overview_emergent_constraints} and in
\cref{tab:app:a:observations}. Following other similar studies, we calculate
\ac{ECS} with the Gregory regression method \autocite{Gregory2004}, which is
described in detail in \cref{subsec:02:ecs}. An overview of the eleven emergent
constraints on \ac{ECS} analyzed in this study including the variables required
for their calculations is given in \cref{tab:05:overview_emergent_constraints}
and \cref{sec:05:comparison_of_emergent_constraints}. We chose these particular
emergent constraints since these were already implemented in the
\ac{ESMValTool} at the time of writing this study, which greatly facilitated
this analysis. For all emergent constraints, we use the historical simulations
of \acs{CMIP}5 and \acs{CMIP}6 in order to ensure maximum agreement with the
observational data. If necessary, the historical simulation of \acs{CMIP}5 is
extended after its final year 2005 with data from the \acs{RCP}8.5 scenario
\autocite{Riahi2011}. Note that we only use data through 2014, during which
time all \ac{RCP} scenarios behave similarly and the choice of the scenario is
not expected to affect the results considerably. Such an extension is not
needed for \acs{CMIP}6 models as their historical simulations cover a longer
time period until 2014. We quantify the skill of the emergent constraints with
three metrics: the coefficient of determination $R^2$ of the linear
relationship, the statistical significance using a two-sided $t$-test based on
the Pearson correlation coefficient $r$ (see \cref{eq:02:t}) and the
constrained range of \ac{ECS} based on the constrained \ac{PDF} (see
\cref{eq:02:pdf_y_given_x0}).


\section{Comparison of Emergent Constraints on \acs{ECS} for \acs{CMIP}5 and
  \acs{CMIP}6}
\label{sec:05:comparison_of_emergent_constraints}

In this section we describe and discuss the eleven emergent constraints on
\ac{ECS} summarized in \cref{tab:05:overview_emergent_constraints} using
\acs{CMIP}5 and \acs{CMIP}6 data
(\cref{subsec:05:bri,subsec:05:cox,subsec:05:lip,subsec:05:shd,subsec:05:shs,%
  subsec:05:shl,subsec:05:su,subsec:05:tih,subsec:05:tii,subsec:05:vol,%
  subsec:05:zha}) and provide a best estimate for \ac{ECS} and statistical
significance of the eleven emergent constraints in
\cref{subsec:05:emergent_constraints_summary}. While most of these emergent
constraints have been derived using data from the \acs{CMIP}5 and/or
\acs{CMIP}3 ensembles, to our knowledge none of them has been evaluated on
the \acs{CMIP}6 ensemble so far. The results for the individual emergent
constraints described in the following are shown in
\cref{fig:05:bri_cox_lip,fig:05:shd_shl_shs,fig:05:su_tih_tii,fig:05:vol_zha}.
The left columns in these figures show the emergent relationships, including
the uncertainty of the linear regressions (blue and orange shaded areas; see
\cref{eq:02:spe}) and the uncertainty in the observations (gray shaded area;
see \cref{eq:02:pdf_obs}). The right columns show the \acp{PDF} of \ac{ECS} in
the original model ensemble (histogram) and the constrained distribution given
by the emergent constraints (blue and orange line; see
\cref{eq:02:pdf_y_given_x0}). \Cref{tab:05:overview_results} shows the
corresponding $66 \unit{\%}$ confidence intervals (\ie{} the $\range{17}{83}
\unit{\%}$ intervals) of \ac{ECS} derived from the \acp{PDF} given by
\cref{eq:02:pdf_y_given_x0} and the $p$-values used to assess the significance
of the emergent relationships.


\subsection{Sensitivity of Shortwave Cloud Albedo to changes in \acl{SST}
  (BRI)}
\label{subsec:05:bri}

In this emergent constraint proposed by \textcite{Brient2016}, \ac{ECS} is
correlated with the \ac{TLC} albedo, \ie{} using the covariance of clouds with
changes in \acp{SST}. Differences in the \ac{TLC} albedo account for more than
half of the variance of \ac{ECS} in the \acs{CMIP}5 ensemble. Following
\textcite{Brient2016}, \ac{TLC} regions are defined as grid points that are in
the driest quartile of $500 \unit{hPa}$ relative humidity of all grid cells
over the ocean between $30 \unit{\degree S}$ and $30 \unit{\degree N}$. The
\ac{TLC} albedo is obtained by calculating the ratio of \ac{TOA} shortwave
cloud radiative forcing and solar insolation averaged over the \ac{TLC} region.
The regression coefficients of de-seasonalized variations of \ac{TLC} shortwave
albedo and \ac{SST} (in $\si{\pctK{}}$) are then used as an emergent constraint
for \ac{ECS}. Here, we use observational data from HadISST for \ac{SST}
\autocite{Rayner2003}, ERA-Interim for $500 \unit{hPa}$ relative humidity
\autocite{Dee2011} and CERES-EBAF \autocite{Loeb2018} for the \ac{TOA}
radiative fluxes over the time period \range{2001}{2005}. In the original
publication, \textcite{Brient2016} use similar observation-based datasets with
the exception of \ac{SST}, where they take ERSST data \autocite{Smith2003} as
reference instead. Our analysis yields a $66 \unit{\%}$ confidence range for
\ac{ECS} of $\pmrangeunit{3.72}{0.59}{K}$ for \acs{CMIP}5 ($R^2 = 0.38$) and
$\pmrangeunit{4.32}{1.07}{K}$ for \acs{CMIP}6, with much lower $R^2 = 0.12$.
The original publication states a best estimate of $4.0 \unit{K}$ with a very
low likelihood of values below $2.3 \unit{K}$ ($90 \unit{\%}$ confidence). The
statistical significance of the emergent relationship drops from $p = 0.0005$
for \acs{CMIP}5 to $p = 0.0355$ for \acs{CMIP}6.

\begin{figure}[p]
  \centering
  \begin{subfigure}[b]{\SmallSubfigureWidth{}}
    \includegraphics[width=\columnwidth]{
      ch05_paper_ecs/figs/scatterplot_merged_training_data_BRI.pdf}
    \caption{}
    \label{fig:05:bri_cox_lip:a}
  \end{subfigure}
  ~
  \begin{subfigure}[b]{\SmallSubfigureWidth{}}
    \includegraphics[width=\columnwidth]{
      ch05_paper_ecs/figs/target_distribution_training_data_BRI.pdf}
    \caption{}
    \label{fig:05:bri_cox_lip:b}
  \end{subfigure}
  \\
  \begin{subfigure}[b]{\SmallSubfigureWidth{}}
    \includegraphics[width=\columnwidth]{
      ch05_paper_ecs/figs/scatterplot_merged_training_data_COX.pdf}
    \caption{}
    \label{fig:05:bri_cox_lip:c}
  \end{subfigure}
  ~
  \begin{subfigure}[b]{\SmallSubfigureWidth{}}
    \includegraphics[width=\columnwidth]{
      ch05_paper_ecs/figs/target_distribution_training_data_COX.pdf}
    \caption{}
    \label{fig:05:bri_cox_lip:d}
  \end{subfigure}
  \\
  \begin{subfigure}[b]{\SmallSubfigureWidth{}}
    \includegraphics[width=\columnwidth]{
      ch05_paper_ecs/figs/scatterplot_merged_training_data_LIP.pdf}
    \caption{}
    \label{fig:05:bri_cox_lip:e}
  \end{subfigure}
  ~
  \begin{subfigure}[b]{\SmallSubfigureWidth{}}
    \includegraphics[width=\columnwidth]{
      ch05_paper_ecs/figs/target_distribution_training_data_LIP.pdf}
    \caption{}
    \label{fig:05:bri_cox_lip:f}
  \end{subfigure}
  \caption{Emergent constraints BRI, COX and LIP applied to the \acs{CMIP}5
    (blue) and \acs{CMIP}6 ensemble (orange). (a), (c), (e) Emergent
    relationships (solid lines) for the \acs{CMIP} models (numbers; see
    \cref{tab:app:a:cmip5_models,tab:app:a:cmip6_models} for details) with
    their standard prediction errors (blue and orange shaded areas; see
    \cref{eq:02:spe}). The vertical dashed line corresponds to the
    observational reference with its standard error (gray shaded area). The
    horizontal dashed lines show the best estimates of the constrained
    \acf{ECS} and the colored points mark the \aclp{MMM} for \acs{CMIP}5
    (blue) and \acs{CMIP}6 (orange). (b), (d), (f) \Aclp{PDF} for \acs{ECS} of
    the emergent constraints (solid lines; see \cref{eq:02:pdf_y_given_x0})
    and the unconstrained model ensembles (histograms). Due to the
    availability of data, the histograms may differ for the different emergent
    constraints. \AdaptedFrom{Schlund2020a}.}
  \label{fig:05:bri_cox_lip}
\end{figure}


\subsection{Temperature Variability Metric (COX)}
\label{subsec:05:cox}

The emergent constraint on \ac{ECS} proposed by \textcite{Cox2018} uses a
temperature variability metric $\Psi$ that is based on the interannual
variation of \ac{GSAT} calculated from its variance (in time) and $1$-year lag
autocorrelation. In contrast to the majority of emergent constraints that focus
on cloud-related processes, this constraint is based on the
fluctuation-dissipation theorem, which relates the long-term response of the
climate system to an external forcing (\ac{ECS}) and short-term variations of
the climate system (climate variability). This arguably places the constraint
on a more solid theoretical foundation, although several questions have been
raised on the robustness of the results to choices made in the analysis
\autocite{Brown2018, PoChedley2018, Rypdal2018}. For example,
\textcite{Annan2020} has shown that the assumed linear relationship between
$\Psi$ and \ac{ECS} does not hold when adding a deep ocean to the model. As
observational data, here we use the HadCRUT4 dataset \autocite{Morice2012} over
the time period \range{1880}{2014}. Under the COX constraint we assess a $66
\unit{\%}$ \ac{ECS} range of $\pmrangeunit{3.03}{0.73}{K}$ for \acs{CMIP}5
($R^2 = 0.31$) and $\pmrangeunit{3.71}{1.09}{K}$ for \acs{CMIP}6 ($R^2 =
0.01$). \textcite{Cox2018} derive a $66 \unit{\%}$ range of
$\pmrangeunit{2.8}{0.6}{K}$ from a different subset of \acs{CMIP}5 models but
the same observations. When moving from \acs{CMIP}5 to \acs{CMIP}6, the
significance of the emergent relation drops massively from $p = 0.0032$ to $p =
0.5415$, respectively.


\subsection{Extent of Southern Hemisphere Hadley Cell (LIP)}
\label{subsec:05:lip}

The results of \textcite{Lipat2017} show that the multi-year average extent of
the Hadley cell correlates with \acs{ECS} in \acs{CMIP}5 models. The Hadley
cell edge is defined as the latitude of the first two grid cells from the
Equator going south where the zonal average $500 \unit{hPa}$ mass stream
function calculated from \ac{DJF} means of the meridional wind field changes
sign from negative to positive. \textcite{Lipat2017} explain this correlation
by tying it to the observed correlation of the interannual variability in
midlatitudinal clouds and their radiative effects with the poleward extent of
the Hadley cell. For the calculation of the emergent constraint, we use
reanalysis data from ERA-Interim \autocite{Dee2011} for the meridional wind
speed over the time period \range{1980}{2005}. Our application of this emergent
constraint gives \ac{ECS} $66 \unit{\%}$ ranges of
$\pmrangeunit{2.97}{0.75}{K}$ for \acs{CMIP}5 ($R^2 = 0.18$) and
$\pmrangeunit{3.75}{1.11}{K}$ for \acs{CMIP}6 ($R^2 < 0.01$). The original
publication does not specify an \ac{ECS} range. For \acs{CMIP}6, the emergent
constraint shows a much lower statistical significance ($p = 0.6791$) than for
\acs{CMIP}5 ($p = 0.0228$).

\subsection{Large-Scale Lower-Tropospheric Mixing (SHD)}
\label{subsec:05:shd}

\textcite{Sherwood2014} propose that the degree of mixing in the lower
troposphere determines the response of boundary layer clouds and humidity to
climate warming, as the associated moisture transport would increase rapidly in
a warmer atmosphere due to the Clausius-Clapeyron relationship. The large-scale
component $D$ of this mixing is defined as the ratio of shallow to deep
overturning. $D$ is calculated from the vertical velocities averaged over two
height regions: $850$ and $700 \unit{hPa}$ for shallow overturning and $600$,
$500$ and $400 \unit{hPa}$ for deep overturning. Both quantities are averaged
over parts of the tropical ocean region away from the regions of highest
\ac{SST} and strongest mid-level ascent, specifically the region $\range{30
  \unit{\degree S}}{30 \unit{\degree N}}$, $\range{160 \unit{\degree W}}{30
  \unit{\degree E}}$, wherever air is ascending at low levels. As
observation-based data, we use vertical velocities from ERA-Interim
\autocite{Dee2011} over the time period \range{1989}{1998} similar to the
original publication. We derive \ac{ECS} $66 \unit{\%}$ confidence ranges of
$\pmrangeunit{3.65}{0.64}{K}$ for \acs{CMIP}5 ($R^2 = 0.28$) and
$\pmrangeunit{3.77}{1.06}{K}$ for \acs{CMIP}6 ($R^2 = 0.03$).
\textcite{Sherwood2014} do not give a best estimate for \ac{ECS} based on the
large-scale component of mixing $D$ or its small-scale counterpart $S$ (see
\cref{subsec:05:shs}) but instead for the sum of $D + S$ only (see
\cref{subsec:05:shl}). The regression shows a much lower significance for
\acs{CMIP}6 ($p = 0.2805$) than for \acs{CMIP}5 ($p = 0.0037$).


\begin{figure}[p]
  \centering
  \begin{subfigure}[b]{\SubfigureWidth{}}
    \includegraphics[width=\columnwidth]{
      ch05_paper_ecs/figs/scatterplot_merged_training_data_SHD.pdf}
    \caption{}
    \label{fig:05:shd_shl_shs:a}
  \end{subfigure}
  ~
  \begin{subfigure}[b]{\SubfigureWidth{}}
    \includegraphics[width=\columnwidth]{
      ch05_paper_ecs/figs/target_distribution_training_data_SHD.pdf}
    \caption{}
    \label{fig:05:shd_shl_shs:b}
  \end{subfigure}
  \\
  \begin{subfigure}[b]{\SubfigureWidth{}}
    \includegraphics[width=\columnwidth]{
      ch05_paper_ecs/figs/scatterplot_merged_training_data_SHL.pdf}
    \caption{}
    \label{fig:05:shd_shl_shs:c}
  \end{subfigure}
  ~
  \begin{subfigure}[b]{\SubfigureWidth{}}
    \includegraphics[width=\columnwidth]{
      ch05_paper_ecs/figs/target_distribution_training_data_SHL.pdf}
    \caption{}
    \label{fig:05:shd_shl_shs:d}
  \end{subfigure}
  \\
  \begin{subfigure}[b]{\SubfigureWidth{}}
    \includegraphics[width=\columnwidth]{
      ch05_paper_ecs/figs/scatterplot_merged_training_data_SHS.pdf}
    \caption{}
    \label{fig:05:shd_shl_shs:e}
  \end{subfigure}
  ~
  \begin{subfigure}[b]{\SubfigureWidth{}}
    \includegraphics[width=\columnwidth]{
      ch05_paper_ecs/figs/target_distribution_training_data_SHS.pdf}
    \caption{}
    \label{fig:05:shd_shl_shs:f}
  \end{subfigure}
  \caption{As in \cref{fig:05:bri_cox_lip} but for the emergent constraints
    SHD, SHL and SHS. \AdaptedFrom{Schlund2020a}.}
  \label{fig:05:shd_shl_shs}
\end{figure}


\subsection{Small-Scale Lower-Tropospheric Mixing (SHS)}
\label{subsec:05:shs}


\subsection{Lower-Tropospheric Mixing Index (SHL)}
\label{subsec:05:shl}

I like the \ac{LTMI}!!


\subsection{Error in Vertical Profile of Relative Humidity (SU)}
\label{subsec:05:su}


\subsection{Tropical Mid-Tropospheric Humidity Asymmetry Index (TIH)}
\label{subsec:05:tih}


\subsection{Southern \acs{ITCZ} index from (TII)}
\label{subsec:05:tii}

I like the \ac{ITCZ}!!


\subsection{Difference Between Tropical and Southern Midlatitudinal Cloud
  Fraction (VOL)}
\label{subsec:05:vol}


\subsection{Response of Marine Boundary Layer Cloud Fraction to changes in
  \acl{SST} (ZHA)}
\label{subsec:05:zha}

I like the \ac{MBLC}!!


\subsection{Constrained \acs{ECS} ranges and statistical significance of the
  eleven emergent constraints}
\label{subsec:05:emergent_constraints_summary}

\begin{figure}[p]
  \centering
  \begin{subfigure}[b]{\SubfigureWidth{}}
    \includegraphics[width=\columnwidth]{
      ch05_paper_ecs/figs/scatterplot_merged_training_data_SU.pdf}
    \caption{}
    \label{fig:05:su_tih_tii:a}
  \end{subfigure}
  ~
  \begin{subfigure}[b]{\SubfigureWidth{}}
    \includegraphics[width=\columnwidth]{
      ch05_paper_ecs/figs/target_distribution_training_data_SU.pdf}
    \caption{}
    \label{fig:05:su_tih_tii:b}
  \end{subfigure}
  \\
  \begin{subfigure}[b]{\SubfigureWidth{}}
    \includegraphics[width=\columnwidth]{
      ch05_paper_ecs/figs/scatterplot_merged_training_data_TIH.pdf}
    \caption{}
    \label{fig:05:su_tih_tii:c}
  \end{subfigure}
  ~
  \begin{subfigure}[b]{\SubfigureWidth{}}
    \includegraphics[width=\columnwidth]{
      ch05_paper_ecs/figs/target_distribution_training_data_TIH.pdf}
    \caption{}
    \label{fig:05:su_tih_tii:d}
  \end{subfigure}
  \\
  \begin{subfigure}[b]{\SubfigureWidth{}}
    \includegraphics[width=\columnwidth]{
      ch05_paper_ecs/figs/scatterplot_merged_training_data_TII.pdf}
    \caption{}
    \label{fig:05:su_tih_tii:e}
  \end{subfigure}
  ~
  \begin{subfigure}[b]{\SubfigureWidth{}}
    \includegraphics[width=\columnwidth]{
      ch05_paper_ecs/figs/target_distribution_training_data_TII.pdf}
    \caption{}
    \label{fig:05:su_tih_tii:f}
  \end{subfigure}
  \caption{As in \cref{fig:05:bri_cox_lip} but for the emergent constraints
    SU, TIH and TII. \AdaptedFrom{Schlund2020a}.}
  \label{fig:05:su_tih_tii}
\end{figure}

\begin{figure}[p]
  \centering
  \begin{subfigure}[b]{\SubfigureWidth{}}
    \includegraphics[width=\columnwidth]{
      ch05_paper_ecs/figs/scatterplot_merged_training_data_VOL.pdf}
    \caption{}
    \label{fig:05:vol_zha:a}
  \end{subfigure}
  ~
  \begin{subfigure}[b]{\SubfigureWidth{}}
    \includegraphics[width=\columnwidth]{
      ch05_paper_ecs/figs/target_distribution_training_data_VOL.pdf}
    \caption{}
    \label{fig:05:vol_zha:b}
  \end{subfigure}
  \\
  \begin{subfigure}[b]{\SubfigureWidth{}}
    \includegraphics[width=\columnwidth]{
      ch05_paper_ecs/figs/scatterplot_merged_training_data_ZHA.pdf}
    \caption{}
    \label{fig:05:vol_zha:c}
  \end{subfigure}
  ~
  \begin{subfigure}[b]{\SubfigureWidth{}}
    \includegraphics[width=\columnwidth]{
      ch05_paper_ecs/figs/target_distribution_training_data_ZHA.pdf}
    \caption{}
    \label{fig:05:vol_zha:d}
  \end{subfigure}
  \caption{As in \cref{fig:05:bri_cox_lip} but for the emergent constraints
    VOL and ZHA. \AdaptedFrom{Schlund2020a}.}
  \label{fig:05:vol_zha}
\end{figure}

\begin{table}[t]
  \centering
  \begin{tabular}{p{0.06\columnwidth} p{0.34\columnwidth} y{0.12} y{0.12}
      y{0.1} y{0.1}}
    \toprule
    Label & \ecreshead{\acs{ECS}}{(original publication)} &
    \ecreshead{\acs{ECS} [$\si{K}$]}{(\acs{CMIP}5)} & \ecreshead{\acs{ECS}
      [$\si{K}$]}{(\acs{CMIP}6)} & \ecreshead{$p$}{(\acs{CMIP}5)} &
    \ecreshead{$p$}{(\acs{CMIP}6)} \\
    \midrule
    BRI & most likely $4.0 \unit{K}$, $< 2.30 \unit{K}$ very unlikely ($90
    \unit{\%}$ confidence) & 3.72
    \pm 0.59 & 4.32 \pm 1.07 & 0.0005 & 0.0355 \\
    COX & $\pmrangeunit{2.8}{0.6}{K}$ & 3.03 \pm 0.73 & 3.71 \pm 1.09 & 0.0032
    & 0.5415 \\
    LIP & no best estimate given & 2.97 \pm 0.75 & 3.75 \pm 1.11 & 0.0228 &
    0.6791 \\
    SHD & none --- see SHL & 3.65 \pm 0.64 & 3.77 \pm 1.06 & 0.0037 & 0.2805 \\
    SHL & most likely $4 \unit{K}$, lower limit $3 \unit{K}$ & 3.42 \pm 0.65 &
    3.67 \pm 1.06 & 0.0002 & 0.0138 \\
    SHS & none --- see SHL & 3.07 \pm 0.73 & 3.48 \pm 1.07 & 0.0647 & 0.0396 \\
    SU & most likely $4 \unit{K}$, lower limit $3 \unit{K}$ & 3.30 \pm 0.88 &
    3.77 \pm 1.35 & 0.1676 & 0.1935 \\
    TIH & most likely $4.0 \unit{K}$ & 3.88 \pm 0.75 & 4.15 \pm 1.10 & 0.0089
    & 0.1348 \\
    TII & most likely $4.0 \unit{K}$ & 3.87 \pm 0.67 & 3.84 \pm 1.09 & 0.0013
    & 0.8236 \\
    VOL & $\pmrangeunit{3.6}{0.4}{K}$ (\stddev{}) & 3.74 \pm 0.64 & 4.21 \pm
    1.04 & 0.0004 & 0.0056 \\
    ZHA & $\pmrangeunit{3.90}{0.45}{K}$ (\stddev{}) & 3.35 \pm 0.74 & 3.79 \pm
    0.67 & 0.2567 & < 0.0001 \\
    \bottomrule
  \end{tabular}
  \caption{Overview of the constrained \acf{ECS} ranges and $p$-values for all
    eleven analyzed emergent constraints. If not further specified, the
    uncertainty ranges correspond to the $66 \unit{\%}$ confidence intervals
    ($\range{17}{83} \unit{\%}$). For \acs{CMIP}5 and \acs{CMIP}6, these are
    evaluated from the \acl{PDF} given by \cref{eq:02:pdf_y_given_x0} (see also
    the right columns of \cref{fig:05:bri_cox_lip,fig:05:shd_shl_shs,%
      fig:05:su_tih_tii,fig:05:vol_zha}). Note that even though \acs{CMIP}5
    models were used for some constraints in the original publications, the
    constrained ranges in the second and third column might differ due to the
    use of a different subset of climate models (in this study, we use output
    from all \acs{CMIP} models that is publicly available; see
    \cref{tab:app:a:cmip5_emergent_constraints_part1,%
      tab:app:a:cmip5_emergent_constraints_part2,%
      tab:app:a:cmip6_emergent_constraints_part1,%
      tab:app:a:cmip6_emergent_constraints_part2} for details). The $p$-values
    describing the significance of the emergent relationships are defined as
    the probability to obtain an absolute correlation coefficient $\abs*{r}$ or
    higher under the null hypothesis that the true underlying correlation
    coefficient between the predictor and \acs{ECS} is zero. Smaller $p$-values
    point to higher significance and vice versa (see
    \cref{subsec:02:emergent_constraints} for details).
    \AdaptedFrom{Schlund2020a}.}
  \label{tab:05:overview_results}
\end{table}


\section{Discussion}
\label{sec:05:discussion}


\section{Summary}
\label{sec:05:summary}
